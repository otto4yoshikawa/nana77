\vspace{0.6cm}
\begin{shiika}山ふところに香煙みちて初薬師
\hfill{\rensuji*{1}・\rensuji*{0}・\rensuji*{0}}\end{shiika}
\vspace{0.6cm}
\begin{shiika}初護摩の煙いただき肩かるし
\hfill{\rensuji*{1}・\rensuji*{0}・\rensuji*{0}}\end{shiika}
\vspace{0.6cm}
\begin{shiika}紅梅のふふみしことも友へ書く
\hfill{\rensuji*{1}・\rensuji*{0}・\rensuji*{0}}\end{shiika}
\vspace{0.6cm}
\begin{shiika}大茶盛廻す茶碗に和気あふれ
\hfill{\rensuji*{1}・\rensuji*{0}・\rensuji*{0}}\end{shiika}
\vspace{0.6cm}
\begin{shiika}寒木瓜の紅流れそう雨つづく
\hfill{\rensuji*{1}・\rensuji*{0}・\rensuji*{0}}\end{shiika}
\vspace{0.6cm}
\begin{shiika}春寒し故なく心のとがる今日
\hfill{\rensuji*{1}・\rensuji*{0}・\rensuji*{0}}\end{shiika}
\vspace{0.6cm}
\begin{shiika}契約のとれてマフラー忘れ去ぬ
\hfill{\rensuji*{1}・\rensuji*{0}・\rensuji*{0}}\end{shiika}
\vspace{0.6cm}
\begin{shiika}雪ごもり写経の日々と紙便り
\hfill{\rensuji*{1}・\rensuji*{0}・\rensuji*{0}}\end{shiika}
\vspace{0.6cm}
\begin{shiika}春風や繰り上げ帰国のよき知らせ
\hfill{\rensuji*{1}・\rensuji*{0}・\rensuji*{0}}\end{shiika}
\vspace{0.6cm}
\begin{shiika}引き越しの迫り咲きつぐ春の彩
\hfill{\rensuji*{1}・\rensuji*{0}・\rensuji*{0}}\end{shiika}
\vspace{0.6cm}
\begin{shiika}転宅の別れの集ひ鰆すし
\hfill{\rensuji*{1}・\rensuji*{0}・\rensuji*{0}}\end{shiika}
\vspace{0.6cm}
\begin{shiika}すましたる貴婦人めける柴木蓮
\hfill{\rensuji*{1}・\rensuji*{0}・\rensuji*{0}}\end{shiika}
\vspace{0.6cm}
\begin{shiika}昼顔や島にたづねる古き墓
\hfill{\rensuji*{1}・\rensuji*{0}・\rensuji*{0}}\end{shiika}
\vspace{0.6cm}
\begin{shiika}夕明りのこる卯波や島に泊つ
\hfill{\rensuji*{1}・\rensuji*{0}・\rensuji*{0}}\end{shiika}
\vspace{0.6cm}
\begin{shiika}城下町一望にほふ栗の花
\hfill{\rensuji*{1}・\rensuji*{0}・\rensuji*{0}}\end{shiika}
\vspace{0.6cm}
\begin{shiika}お天主へ石垣高し松の花
\hfill{\rensuji*{1}・\rensuji*{0}・\rensuji*{0}}\end{shiika}
\vspace{0.6cm}
\begin{shiika}天主閣仰ぐ茶店の藤こぼる
\hfill{\rensuji*{1}・\rensuji*{0}・\rensuji*{0}}\end{shiika}
\vspace{0.6cm}
\begin{shiika}紫陽花の彩拡げゆく遊歩道
\hfill{\rensuji*{1}・\rensuji*{0}・\rensuji*{0}}\end{shiika}
\vspace{0.6cm}
\begin{shiika}夏三つ葉雨の小やみに摘む留守居
\hfill{\rensuji*{1}・\rensuji*{0}・\rensuji*{0}}\end{shiika}
\vspace{0.6cm}
\begin{shiika}母も娘もショートカットにさくらんぼ
\hfill{\rensuji*{1}・\rensuji*{0}・\rensuji*{0}}\end{shiika}
\vspace{0.6cm}
\begin{shiika}窓開き大向日葵に見つめらる
\hfill{\rensuji*{1}・\rensuji*{0}・\rensuji*{0}}\end{shiika}
\vspace{0.6cm}
\begin{shiika}驕りても向日葵は好き美くしき
\hfill{\rensuji*{1}・\rensuji*{0}・\rensuji*{0}}\end{shiika}
\vspace{0.6cm}
\begin{shiika}留守居して一人に惜しき風凉し
\hfill{\rensuji*{1}・\rensuji*{0}・\rensuji*{0}}\end{shiika}
\vspace{0.6cm}
\begin{shiika}水撒きて陶狸うれしき顔となる
\hfill{\rensuji*{1}・\rensuji*{0}・\rensuji*{0}}\end{shiika}
\vspace{0.6cm}
\begin{shiika}思ひきり水撒き散らす重きもの
\hfill{\rensuji*{1}・\rensuji*{0}・\rensuji*{0}}\end{shiika}
\vspace{0.6cm}
\begin{shiika}賞め言葉裏に返さず花クローバ
\hfill{\rensuji*{1}・\rensuji*{0}・\rensuji*{0}}\end{shiika}
\vspace{0.6cm}
\begin{shiika}水撒きて木々と話をする留守居
\hfill{\rensuji*{1}・\rensuji*{0}・\rensuji*{0}}\end{shiika}
\vspace{0.6cm}
\begin{shiika}白粉花空家となりし垣に満つ
\hfill{\rensuji*{1}・\rensuji*{0}・\rensuji*{0}}\end{shiika}
\vspace{0.6cm}
\begin{shiika}病葉のこの量踏みて医に通ふ
\hfill{\rensuji*{1}・\rensuji*{0}・\rensuji*{0}}\end{shiika}
\vspace{0.6cm}
\begin{shiika}鳶舞ふ高野の夏の深き空
\hfill{\rensuji*{1}・\rensuji*{0}・\rensuji*{0}}\end{shiika}
\vspace{0.6cm}
\begin{shiika}野猿乗り夏の河原の若者等
\hfill{\rensuji*{1}・\rensuji*{0}・\rensuji*{0}}\end{shiika}
\vspace{0.6cm}
\begin{shiika}グラヂオラス店の娘明るく迎へくれ
\hfill{\rensuji*{1}・\rensuji*{0}・\rensuji*{0}}\end{shiika}
\vspace{0.6cm}
\begin{shiika}ポンポンダリヤ活けて村営コーヒー館
\hfill{\rensuji*{1}・\rensuji*{0}・\rensuji*{0}}\end{shiika}
\vspace{0.6cm}
\begin{shiika}漁火に想ひそれぞれ宿浴衣
\hfill{\rensuji*{1}・\rensuji*{0}・\rensuji*{0}}\end{shiika}
\vspace{0.6cm}
\begin{shiika}盆列車着席までを送らるる
\hfill{\rensuji*{1}・\rensuji*{0}・\rensuji*{0}}\end{shiika}
\vspace{0.6cm}
\begin{shiika}伝説の湖ははるかに芒原
\hfill{\rensuji*{1}・\rensuji*{0}・\rensuji*{0}}\end{shiika}
\vspace{0.6cm}
\begin{shiika}湖も山もみるみる消えて霧の海
\hfill{\rensuji*{1}・\rensuji*{0}・\rensuji*{0}}\end{shiika}
\vspace{0.6cm}
\begin{shiika}山の霧流れて速し湖生る
\hfill{\rensuji*{1}・\rensuji*{0}・\rensuji*{0}}\end{shiika}
\vspace{0.6cm}
\begin{shiika}のぼり来て賽の河原の細芒
\hfill{\rensuji*{1}・\rensuji*{0}・\rensuji*{0}}\end{shiika}
\vspace{0.6cm}
\begin{shiika}旅に訪ふドラマ舞台の町も秋
\hfill{\rensuji*{1}・\rensuji*{0}・\rensuji*{0}}\end{shiika}
\vspace{0.6cm}
\begin{shiika}久の出会ひ杖目じるしと言ふも秋
\hfill{\rensuji*{1}・\rensuji*{0}・\rensuji*{0}}\end{shiika}
\vspace{0.6cm}
\begin{shiika}秋釣の成果に夕餉賑へり
\hfill{\rensuji*{1}・\rensuji*{0}・\rensuji*{0}}\end{shiika}
\vspace{0.6cm}
\begin{shiika}秋雨のやまず留守居の夕仕度
\hfill{\rensuji*{1}・\rensuji*{0}・\rensuji*{0}}\end{shiika}
\vspace{0.6cm}
\begin{shiika}コスモスの身丈を埋めてはるか冨士
\hfill{\rensuji*{1}・\rensuji*{0}・\rensuji*{0}}\end{shiika}
\vspace{0.6cm}
\begin{shiika}湧き水の秋澄む池に冨士の影
\hfill{\rensuji*{1}・\rensuji*{0}・\rensuji*{0}}\end{shiika}
\vspace{0.6cm}
\begin{shiika}天高し誕生釈迦の細き指
\hfill{\rensuji*{1}・\rensuji*{0}・\rensuji*{0}}\end{shiika}
\vspace{0.6cm}
\begin{shiika}落葉かき風に根気の作務の僧
\hfill{\rensuji*{1}・\rensuji*{0}・\rensuji*{0}}\end{shiika}
\vspace{0.6cm}
\begin{shiika}柿届く家なき故郷の友も老ひ
\hfill{\rensuji*{1}・\rensuji*{0}・\rensuji*{0}}\end{shiika}
\vspace{0.6cm}
\begin{shiika}郷言葉の電話果なし老夜長
\hfill{\rensuji*{1}・\rensuji*{0}・\rensuji*{0}}\end{shiika}
\vspace{0.6cm}
\begin{shiika}命延ぶ泉いただき峯を越す
\hfill{\rensuji*{1}・\rensuji*{0}・\rensuji*{0}}\end{shiika}
\vspace{0.6cm}
\begin{shiika}野仏の膝にさい銭紅葉散る
\hfill{\rensuji*{1}・\rensuji*{0}・\rensuji*{0}}\end{shiika}
\vspace{0.6cm}
\begin{shiika}冬濤の音きヽ紀伊の朝茶粥
\hfill{\rensuji*{1}・\rensuji*{0}・\rensuji*{0}}\end{shiika}
\vspace{0.6cm}
\begin{shiika}娘が立てし枕屏風に安眠して
\hfill{\rensuji*{1}・\rensuji*{0}・\rensuji*{0}}\end{shiika}
\vspace{0.6cm}
\begin{shiika}晩菊に名残水やり旅に出る
\hfill{\rensuji*{1}・\rensuji*{0}・\rensuji*{0}}\end{shiika}
\vspace{0.6cm}
\begin{shiika}報恩講善女となりてしる粉賜ぶ
\hfill{\rensuji*{1}・\rensuji*{0}・\rensuji*{0}}\end{shiika}
\vspace{0.6cm}
\begin{shiika}花車たがへず来たり年用意
\hfill{\rensuji*{1}・\rensuji*{0}・\rensuji*{0}}\end{shiika}
\vspace{0.6cm}
\begin{shiika}心ゆくまで謡ひけり年忘れ
\hfill{\rensuji*{1}・\rensuji*{0}・\rensuji*{0}}\end{shiika}
\vspace{0.6cm}
\begin{shiika}娘の忌日となりて年経る小つもごり
\hfill{\rensuji*{1}・\rensuji*{0}・\rensuji*{0}}\end{shiika}
\vspace{0.6cm}
\begin{shiika}旅立ちを止めて眺むる強吹雪
\hfill{\rensuji*{2}・\rensuji*{0}・\rensuji*{0}}\end{shiika}
\vspace{0.6cm}
\begin{shiika}おくれ咲く紅山茶花の雪化粧
\hfill{\rensuji*{2}・\rensuji*{0}・\rensuji*{0}}\end{shiika}
\vspace{0.6cm}
\begin{shiika}潮の香をはこび来る風春近し
\hfill{\rensuji*{2}・\rensuji*{0}・\rensuji*{0}}\end{shiika}
\vspace{0.6cm}
\begin{shiika}水温みあひる天国てふ川辺
\hfill{\rensuji*{2}・\rensuji*{0}・\rensuji*{0}}\end{shiika}
\vspace{0.6cm}
\begin{shiika}指圧効きかろき足もと蕗のとう
\hfill{\rensuji*{2}・\rensuji*{0}・\rensuji*{0}}\end{shiika}
\vspace{0.6cm}
\begin{shiika}桃ふふみ声出し笑ふと嬰便り
\hfill{\rensuji*{2}・\rensuji*{0}・\rensuji*{0}}\end{shiika}
\vspace{0.6cm}
\begin{shiika}初雛に招かれ・
\hfill{\rensuji*{2}・\rensuji*{0}・\rensuji*{0}}\end{shiika}
\vspace{0.6cm}
\begin{shiika}亡母の忌や弟としのぶ春炬燵
\hfill{\rensuji*{2}・\rensuji*{0}・\rensuji*{0}}\end{shiika}
\vspace{0.6cm}
\begin{shiika}高々と辛夷咲きみつ城跡園
\hfill{\rensuji*{2}・\rensuji*{0}・\rensuji*{0}}\end{shiika}
\vspace{0.6cm}
\begin{shiika}もてなさる小さき土鍋に土筆煮て
\hfill{\rensuji*{2}・\rensuji*{0}・\rensuji*{0}}\end{shiika}
\vspace{0.6cm}
\begin{shiika}こんがりと焼味噌蕗のとうほのと
\hfill{\rensuji*{2}・\rensuji*{0}・\rensuji*{0}}\end{shiika}
\vspace{0.6cm}
\begin{shiika}蕗摘みて老の自慢のちらしずし
\hfill{\rensuji*{2}・\rensuji*{0}・\rensuji*{0}}\end{shiika}
\vspace{0.6cm}
\begin{shiika}一心の白夕闇にほのと浮く
\hfill{\rensuji*{2}・\rensuji*{0}・\rensuji*{0}}\end{shiika}
\vspace{0.6cm}
\begin{shiika}陶狸の背出で入る鳥の巣づくりか
\hfill{\rensuji*{2}・\rensuji*{0}・\rensuji*{0}}\end{shiika}
\vspace{0.6cm}
\begin{shiika}葉桜や友のギブスはまだ除れず
\hfill{\rensuji*{2}・\rensuji*{0}・\rensuji*{0}}\end{shiika}
\vspace{0.6cm}
\begin{shiika}露座観音見おろす里の柿若葉
\hfill{\rensuji*{2}・\rensuji*{0}・\rensuji*{0}}\end{shiika}
\vspace{0.6cm}
\begin{shiika}柿若葉光る白壁つづく里
\hfill{\rensuji*{2}・\rensuji*{0}・\rensuji*{0}}\end{shiika}
\vspace{0.6cm}
\begin{shiika}風薫る河童出そうな筑後川
\hfill{\rensuji*{2}・\rensuji*{0}・\rensuji*{0}}\end{shiika}
\vspace{0.6cm}
\begin{shiika}老鴬に迎えられけり峡の宿
\hfill{\rensuji*{2}・\rensuji*{0}・\rensuji*{0}}\end{shiika}
\vspace{0.6cm}
\begin{shiika}鱚一尾釣りて得意の帰宅ベル
\hfill{\rensuji*{2}・\rensuji*{0}・\rensuji*{0}}\end{shiika}
\vspace{0.6cm}
\begin{shiika}釣りし鱚ほめて一箸づつ廻し
\hfill{\rensuji*{2}・\rensuji*{0}・\rensuji*{0}}\end{shiika}
\vspace{0.6cm}
\begin{shiika}ご協力と酢い甘夏を嫁出し来
\hfill{\rensuji*{2}・\rensuji*{0}・\rensuji*{0}}\end{shiika}
\vspace{0.6cm}
\begin{shiika}紫陽花や登山電車は幾曲がり
\hfill{\rensuji*{2}・\rensuji*{0}・\rensuji*{0}}\end{shiika}
\vspace{0.6cm}
\begin{shiika}お世辞とも思ひつつ買ふ夏帽子
\hfill{\rensuji*{2}・\rensuji*{0}・\rensuji*{0}}\end{shiika}
\vspace{0.6cm}
\begin{shiika}夏帽子鏡の顔はヤヤすまし
\hfill{\rensuji*{2}・\rensuji*{0}・\rensuji*{0}}\end{shiika}
\vspace{0.6cm}
\begin{shiika}のびて寝る猫のかたへに端居して
\hfill{\rensuji*{2}・\rensuji*{0}・\rensuji*{0}}\end{shiika}
\vspace{0.6cm}
\begin{shiika}待つ荷物おそし木樺はしぼみ初む
\hfill{\rensuji*{2}・\rensuji*{0}・\rensuji*{0}}\end{shiika}
\vspace{0.6cm}
\begin{shiika}鎌倉の御寺凉やか友葬る
\hfill{\rensuji*{2}・\rensuji*{0}・\rensuji*{0}}\end{shiika}
\vspace{0.6cm}
\begin{shiika}母として慕はれ甥とビールくむ
\hfill{\rensuji*{2}・\rensuji*{0}・\rensuji*{0}}\end{shiika}
\vspace{0.6cm}
\begin{shiika}風鈴や父母知らぬ甥よき父に
\hfill{\rensuji*{2}・\rensuji*{0}・\rensuji*{0}}\end{shiika}
\vspace{0.6cm}
\begin{shiika}五・
\hfill{\rensuji*{2}・\rensuji*{0}・\rensuji*{0}}\end{shiika}
\vspace{0.6cm}
\begin{shiika}巨寺にみちのくらしき萩まつり
\hfill{\rensuji*{2}・\rensuji*{0}・\rensuji*{0}}\end{shiika}
\vspace{0.6cm}
\begin{shiika}雨上がり紅たわヽなるりんご園
\hfill{\rensuji*{2}・\rensuji*{0}・\rensuji*{0}}\end{shiika}
\vspace{0.6cm}
\begin{shiika}子に孫にりんご送りて津軽旅
\hfill{\rensuji*{2}・\rensuji*{0}・\rensuji*{0}}\end{shiika}
\vspace{0.6cm}
\begin{shiika}台風もよしといで湯にやり過ごし
\hfill{\rensuji*{2}・\rensuji*{0}・\rensuji*{0}}\end{shiika}
\vspace{0.6cm}
\begin{shiika}久に来し皇居のお濠曼珠沙華
\hfill{\rensuji*{2}・\rensuji*{0}・\rensuji*{0}}\end{shiika}
\vspace{0.6cm}
\begin{shiika}コスモスの風に流せるほどの些事
\hfill{\rensuji*{2}・\rensuji*{0}・\rensuji*{0}}\end{shiika}
\vspace{0.6cm}
\begin{shiika}ただ声をききたく夜長の遠電話
\hfill{\rensuji*{2}・\rensuji*{0}・\rensuji*{0}}\end{shiika}
\vspace{0.6cm}
\begin{shiika}バスを待つこわれベンチに秋の蝶
\hfill{\rensuji*{2}・\rensuji*{0}・\rensuji*{0}}\end{shiika}
\vspace{0.6cm}
\begin{shiika}茫々の芒の中や美人塚
\hfill{\rensuji*{2}・\rensuji*{0}・\rensuji*{0}}\end{shiika}
\vspace{0.6cm}
\begin{shiika}神在月とガイド熱あり出雲路よ
\hfill{\rensuji*{2}・\rensuji*{0}・\rensuji*{0}}\end{shiika}
\vspace{0.6cm}
\begin{shiika}濃紅葉座禅堂の扉はかたく閉じ
\hfill{\rensuji*{2}・\rensuji*{0}・\rensuji*{0}}\end{shiika}
\vspace{0.6cm}
\begin{shiika}寄進瓦に筆持つひまも紅葉散る
\hfill{\rensuji*{2}・\rensuji*{0}・\rensuji*{0}}\end{shiika}
\vspace{0.6cm}
\begin{shiika}庭小春鳩来て犬が少し吠え
\hfill{\rensuji*{2}・\rensuji*{0}・\rensuji*{0}}\end{shiika}
\vspace{0.6cm}
\begin{shiika}晩菊や顔見ぬ電話言ひ過ぎし
\hfill{\rensuji*{2}・\rensuji*{0}・\rensuji*{0}}\end{shiika}
\vspace{0.6cm}
\begin{shiika}枯木してはるか冨士見る道となる
\hfill{\rensuji*{2}・\rensuji*{0}・\rensuji*{0}}\end{shiika}
\vspace{0.6cm}
\begin{shiika}数の子の歯音うれしや八・
\hfill{\rensuji*{3}・\rensuji*{0}・\rensuji*{0}}\end{shiika}
\vspace{0.6cm}
\begin{shiika}初詣極楽寺てふ名にひかれ
\hfill{\rensuji*{3}・\rensuji*{0}・\rensuji*{0}}\end{shiika}
\vspace{0.6cm}
\begin{shiika}初旅や全き冨士に真向へり
\hfill{\rensuji*{3}・\rensuji*{0}・\rensuji*{0}}\end{shiika}
\vspace{0.6cm}
\begin{shiika}立春の陽に勇気湧きトレーニング
\hfill{\rensuji*{3}・\rensuji*{0}・\rensuji*{0}}\end{shiika}
\vspace{0.6cm}
\begin{shiika}足鍛え眠り覚めたる山のぼる
\hfill{\rensuji*{3}・\rensuji*{0}・\rensuji*{0}}\end{shiika}
\vspace{0.6cm}
\begin{shiika}人波に流されてみる梅まつり
\hfill{\rensuji*{3}・\rensuji*{0}・\rensuji*{0}}\end{shiika}
\vspace{0.6cm}
\begin{shiika}指呼の山みるみるかくす春吹雪
\hfill{\rensuji*{3}・\rensuji*{0}・\rensuji*{0}}\end{shiika}
\vspace{0.6cm}
\begin{shiika}舞へ狂へいで湯ごもりの春吹雪
\hfill{\rensuji*{3}・\rensuji*{0}・\rensuji*{0}}\end{shiika}
\vspace{0.6cm}
\begin{shiika}ほの酔ひや孫つぎくれしお白酒
\hfill{\rensuji*{3}・\rensuji*{0}・\rensuji*{0}}\end{shiika}
\vspace{0.6cm}
\begin{shiika}ひなの前老も交りて撮る今宵
\hfill{\rensuji*{3}・\rensuji*{0}・\rensuji*{0}}\end{shiika}
\vspace{0.6cm}
\begin{shiika}梅林へ少しの坂も手を引かれ
\hfill{\rensuji*{3}・\rensuji*{0}・\rensuji*{0}}\end{shiika}
\vspace{0.6cm}
\begin{shiika}白梅の古木に希ふ吾が余生
\hfill{\rensuji*{3}・\rensuji*{0}・\rensuji*{0}}\end{shiika}
\vspace{0.6cm}
\begin{shiika}湖見ゆる観音堂の大桜
\hfill{\rensuji*{3}・\rensuji*{0}・\rensuji*{0}}\end{shiika}
\vspace{0.6cm}
\begin{shiika}芽柳の日々に大ゆれ風青し
\hfill{\rensuji*{3}・\rensuji*{0}・\rensuji*{0}}\end{shiika}
\vspace{0.6cm}
\begin{shiika}花散るや石州瓦の光る村
\hfill{\rensuji*{3}・\rensuji*{0}・\rensuji*{0}}\end{shiika}
\vspace{0.6cm}
\begin{shiika}初蝶や癒えて佇つ庭彩ふえて
\hfill{\rensuji*{3}・\rensuji*{0}・\rensuji*{0}}\end{shiika}
\vspace{0.6cm}
\begin{shiika}初蝶やふっつり切れし思ひごと
\hfill{\rensuji*{3}・\rensuji*{0}・\rensuji*{0}}\end{shiika}
\vspace{0.6cm}
\begin{shiika}新茶賜ぶ少年今は病院長
\hfill{\rensuji*{3}・\rensuji*{0}・\rensuji*{0}}\end{shiika}
\vspace{0.6cm}
\begin{shiika}芍薬や三度の転居共にして
\hfill{\rensuji*{3}・\rensuji*{0}・\rensuji*{0}}\end{shiika}
\vspace{0.6cm}
\begin{shiika}染め止めて白髪軽し青葉風
\hfill{\rensuji*{3}・\rensuji*{0}・\rensuji*{0}}\end{shiika}
\vspace{0.6cm}
\begin{shiika}年令らしく白髪でおしゃれ夏帽子
\hfill{\rensuji*{3}・\rensuji*{0}・\rensuji*{0}}\end{shiika}
\vspace{0.6cm}
\begin{shiika}釣り土産べらとはうれし瀬戸育ち
\hfill{\rensuji*{3}・\rensuji*{0}・\rensuji*{0}}\end{shiika}
\vspace{0.6cm}
\begin{shiika}早苗田の日毎濃くなる療の窓
\hfill{\rensuji*{3}・\rensuji*{0}・\rensuji*{0}}\end{shiika}
\vspace{0.6cm}
\begin{shiika}山の湖万緑の中遠くあり
\hfill{\rensuji*{3}・\rensuji*{0}・\rensuji*{0}}\end{shiika}
\vspace{0.6cm}
\begin{shiika}山間の夏霧深き駅に着く
\hfill{\rensuji*{3}・\rensuji*{0}・\rensuji*{0}}\end{shiika}
\vspace{0.6cm}
\begin{shiika}立葵彩を揃えて山の駅
\hfill{\rensuji*{3}・\rensuji*{0}・\rensuji*{0}}\end{shiika}
\vspace{0.6cm}
\begin{shiika}薬草湯の香りのこりて宿浴衣
\hfill{\rensuji*{3}・\rensuji*{0}・\rensuji*{0}}\end{shiika}
\vspace{0.6cm}
\begin{shiika}大寸の宿衣たぐりて岩魚膳
\hfill{\rensuji*{3}・\rensuji*{0}・\rensuji*{0}}\end{shiika}
\vspace{0.6cm}

\begin{shiika}億の土地我がもの顔に青すすき
\hfill{\rensuji*{3}・\rensuji*{0}・\rensuji*{0}}\end{shiika}
\vspace{0.6cm}
\begin{shiika}通院の道は川沿ひ月見草
\hfill{\rensuji*{3}・\rensuji*{0}・\rensuji*{0}}\end{shiika}
\vspace{0.6cm}
\begin{shiika}時計おそし独り留守居の小粒ぶどう
\hfill{\rensuji*{3}・\rensuji*{0}・\rensuji*{0}}\end{shiika}
\vspace{0.6cm}
\begin{shiika}秋暑しビルの掃除夫見上ぐ窓
\hfill{\rensuji*{3}・\rensuji*{0}・\rensuji*{0}}\end{shiika}
\vspace{0.6cm}
\begin{shiika}保養所のヴェランダ踊りの列を見る
\hfill{\rensuji*{3}・\rensuji*{0}・\rensuji*{0}}\end{shiika}
\vspace{0.6cm}
\begin{shiika}踊りうちわよべの土産と保養友
\hfill{\rensuji*{3}・\rensuji*{0}・\rensuji*{0}}\end{shiika}
\vspace{0.6cm}
\begin{shiika}秋の湖哀話流して遊覧船
\hfill{\rensuji*{3}・\rensuji*{0}・\rensuji*{0}}\end{shiika}
\vspace{0.6cm}
\begin{shiika}温泉の町にお湯かけ地蔵秋うらら
\hfill{\rensuji*{3}・\rensuji*{0}・\rensuji*{0}}\end{shiika}
\vspace{0.6cm}
\begin{shiika}敬老日ほの酔はされて若返る
\hfill{\rensuji*{3}・\rensuji*{0}・\rensuji*{0}}\end{shiika}
\vspace{0.6cm}
\begin{shiika}誰が家ぞ芒刈られて地鎮祭
\hfill{\rensuji*{3}・\rensuji*{0}・\rensuji*{0}}\end{shiika}
\vspace{0.6cm}
\begin{shiika}秋場所の終り落ちつき夕支度
\hfill{\rensuji*{3}・\rensuji*{0}・\rensuji*{0}}\end{shiika}
\vspace{0.6cm}
\begin{shiika}ゆかしさに秋七草の寺巡り
\hfill{\rensuji*{3}・\rensuji*{0}・\rensuji*{0}}\end{shiika}
\vspace{0.6cm}
\begin{shiika}尊氏も正成も美男菊衣
\hfill{\rensuji*{3}・\rensuji*{0}・\rensuji*{0}}\end{shiika}
\vspace{0.6cm}
\begin{shiika}天高し八・
\hfill{\rensuji*{3}・\rensuji*{0}・\rensuji*{0}}\end{shiika}
\vspace{0.6cm}
\begin{shiika}穂芒の波うねうねと芒山
\hfill{\rensuji*{3}・\rensuji*{0}・\rensuji*{0}}\end{shiika}
\vspace{0.6cm}
\begin{shiika}秋茄子を嫁にすすめて共笑ひ
\hfill{\rensuji*{3}・\rensuji*{0}・\rensuji*{0}}\end{shiika}
\vspace{0.6cm}
\begin{shiika}神有りの出雲の湖はかもめ舞ふ
\hfill{\rensuji*{3}・\rensuji*{0}・\rensuji*{0}}\end{shiika}
\vspace{0.6cm}
\begin{shiika}宍道湖の大橋たもと柳散る
\hfill{\rensuji*{3}・\rensuji*{0}・\rensuji*{0}}\end{shiika}
\vspace{0.6cm}
\begin{shiika}宍道湖の秋の入日に出合ひけり
\hfill{\rensuji*{3}・\rensuji*{0}・\rensuji*{0}}\end{shiika}
\vspace{0.6cm}
\begin{shiika}名菓舗の近くに石焼芋の声
\hfill{\rensuji*{3}・\rensuji*{0}・\rensuji*{0}}\end{shiika}
\vspace{0.6cm}
\begin{shiika}鳴き砂を踏めば聞えし秋の声
\hfill{\rensuji*{3}・\rensuji*{0}・\rensuji*{0}}\end{shiika}
\vspace{0.6cm}
\begin{shiika}白髪を少しのぞかせ冬帽子
\hfill{\rensuji*{3}・\rensuji*{0}・\rensuji*{0}}\end{shiika}
\vspace{0.6cm}
\begin{shiika}もう一度鏡をのぞく冬帽子
\hfill{\rensuji*{3}・\rensuji*{0}・\rensuji*{0}}\end{shiika}
\vspace{0.6cm}
\begin{shiika}久に会ふ少しおしゃれに冬帽子
\hfill{\rensuji*{3}・\rensuji*{0}・\rensuji*{0}}\end{shiika}
\vspace{0.6cm}
\begin{shiika}諦めもした犬癒えて冬ぬくし
\hfill{\rensuji*{3}・\rensuji*{0}・\rensuji*{0}}\end{shiika}
\vspace{0.6cm}
\begin{shiika}独言ならずチロとの話始め
\hfill{\rensuji*{3}・\rensuji*{0}・\rensuji*{0}}\end{shiika}
\vspace{0.6cm}
\begin{shiika}愛犬のチロも淑気の尾をふれり
\hfill{\rensuji*{4}・\rensuji*{0}・\rensuji*{0}}\end{shiika}
\vspace{0.6cm}
\begin{shiika}年の夜吾より古き茶棚拭く
\hfill{\rensuji*{3}・\rensuji*{0}・\rensuji*{0}}\end{shiika}
\vspace{0.6cm}
\begin{shiika}立春大吉吾より古き茶棚拭く
\hfill{\rensuji*{3}・\rensuji*{0}・\rensuji*{0}}\end{shiika}
\vspace{0.6cm}
\begin{shiika}名水へ凍ての渓路手をひかれ
\hfill{\rensuji*{4}・\rensuji*{0}・\rensuji*{0}}\end{shiika}
\vspace{0.6cm}
\begin{shiika}謡初帯山小さく装ふ同志
\hfill{\rensuji*{4}・\rensuji*{0}・\rensuji*{0}}\end{shiika}
\vspace{0.6cm}
\begin{shiika}謡初足のねぢりを許し合ひ
\hfill{\rensuji*{4}・\rensuji*{0}・\rensuji*{0}}\end{shiika}
\vspace{0.6cm}
\begin{shiika}保養所で看る東京の雪ニュース
\hfill{\rensuji*{4}・\rensuji*{0}・\rensuji*{0}}\end{shiika}
\vspace{0.6cm}
\begin{shiika}お返しを気にする老や冬いちご
\hfill{\rensuji*{4}・\rensuji*{0}・\rensuji*{0}}\end{shiika}
\vspace{0.6cm}
\begin{shiika}大山ははるか田に群る白鳥かな
\hfill{\rensuji*{4}・\rensuji*{0}・\rensuji*{0}}\end{shiika}
\vspace{0.6cm}
\begin{shiika}旅帰り待ちくれ紅梅咲き満つる
\hfill{\rensuji*{4}・\rensuji*{0}・\rensuji*{0}}\end{shiika}
\vspace{0.6cm}
\begin{shiika}紅梅や吾が色にせむと言ひし亡友
\hfill{\rensuji*{4}・\rensuji*{0}・\rensuji*{0}}\end{shiika}
\vspace{0.6cm}
\begin{shiika}梅の闇逢ふ日約せし友逝きぬ
\hfill{\rensuji*{4}・\rensuji*{0}・\rensuji*{0}}\end{shiika}
\vspace{0.6cm}
\begin{shiika}旅はずむ卒業進学祝ぎ二つ
\hfill{\rensuji*{4}・\rensuji*{0}・\rensuji*{0}}\end{shiika}
\vspace{0.6cm}
\begin{shiika}たまさかの母と息子の旅春の虹
\hfill{\rensuji*{4}・\rensuji*{0}・\rensuji*{0}}\end{shiika}
\vspace{0.6cm}
\begin{shiika}春眠の・
\hfill{\rensuji*{4}・\rensuji*{0}・\rensuji*{0}}\end{shiika}
\vspace{0.6cm}
\begin{shiika}春セーター鏡に肩のうすきこと
\hfill{\rensuji*{4}・\rensuji*{0}・\rensuji*{0}}\end{shiika}
\vspace{0.6cm}
\begin{shiika}美くしく老いたきものよ柴木蓮
\hfill{\rensuji*{4}・\rensuji*{0}・\rensuji*{0}}\end{shiika}
\vspace{0.6cm}
\begin{shiika}シクラメン茶の間笑ひ溢れさす
\hfill{\rensuji*{4}・\rensuji*{0}・\rensuji*{0}}\end{shiika}
\vspace{0.6cm}
\begin{shiika}ふる里はすみれたんぽぽ墓の径
\hfill{\rensuji*{4}・\rensuji*{0}・\rensuji*{0}}\end{shiika}
\vspace{0.6cm}
\begin{shiika}桃の花さら前かけの辻地蔵
\hfill{\rensuji*{4}・\rensuji*{0}・\rensuji*{0}}\end{shiika}
\vspace{0.6cm}
\begin{shiika}お遍路の憩なる礎石大伽藍
\hfill{\rensuji*{4}・\rensuji*{0}・\rensuji*{0}}\end{shiika}
\vspace{0.6cm}
\begin{shiika}菜の花を手いつぱい摘み日毎漬け
\hfill{\rensuji*{4}・\rensuji*{0}・\rensuji*{0}}\end{shiika}
\vspace{0.6cm}
\begin{shiika}日々摘めど菜の花畑の黄は濃ゆく
\hfill{\rensuji*{4}・\rensuji*{0}・\rensuji*{0}}\end{shiika}
\vspace{0.6cm}
\begin{shiika}花杏真白従妹に甘え気味
\hfill{\rensuji*{4}・\rensuji*{0}・\rensuji*{0}}\end{shiika}
\vspace{0.6cm}
\begin{shiika}芍薬の蕾ふくらむ庭の日々
\hfill{\rensuji*{4}・\rensuji*{0}・\rensuji*{0}}\end{shiika}
\vspace{0.6cm}
\begin{shiika}発つ朝にうす紅ほのと花水木
\hfill{\rensuji*{4}・\rensuji*{0}・\rensuji*{0}}\end{shiika}
\vspace{0.6cm}
\begin{shiika}いそいそと半袖えらび旅立てり
\hfill{\rensuji*{4}・\rensuji*{0}・\rensuji*{0}}\end{shiika}
\vspace{0.6cm}
\begin{shiika}山迫る車窓次々藤の花
\hfill{\rensuji*{4}・\rensuji*{0}・\rensuji*{0}}\end{shiika}
\vspace{0.6cm}
\begin{shiika}若葉風亡妹の友とめぐり逢ひ
\hfill{\rensuji*{4}・\rensuji*{0}・\rensuji*{0}}\end{shiika}
\vspace{0.6cm}
\begin{shiika}短か夜や亡妹の友と泊つ出雲
\hfill{\rensuji*{4}・\rensuji*{0}・\rensuji*{0}}\end{shiika}
\vspace{0.6cm}
\begin{shiika}ビール酌むかちんとグラス若やぎて
\hfill{\rensuji*{4}・\rensuji*{0}・\rensuji*{0}}\end{shiika}
\vspace{0.6cm}
\begin{shiika}ビール酌むドラマのように共鳴し
\hfill{\rensuji*{4}・\rensuji*{0}・\rensuji*{0}}\end{shiika}
\vspace{0.6cm}
\begin{shiika}ビール乾し少し多弁に刻忘る
\hfill{\rensuji*{4}・\rensuji*{0}・\rensuji*{0}}\end{shiika}
\vspace{0.6cm}
\begin{shiika}向日葵が君臨空地の草いくさ
\hfill{\rensuji*{4}・\rensuji*{0}・\rensuji*{0}}\end{shiika}
\vspace{0.6cm}
\begin{shiika}木樺咲く一日の花の教えごと
\hfill{\rensuji*{4}・\rensuji*{0}・\rensuji*{0}}\end{shiika}
\vspace{0.6cm}
\begin{shiika}垣根ばら互の無事を老犬と
\hfill{\rensuji*{4}・\rensuji*{0}・\rensuji*{0}}\end{shiika}
\vspace{0.6cm}
\begin{shiika}夕仕度水の出細き大暑かな
\hfill{\rensuji*{4}・\rensuji*{0}・\rensuji*{0}}\end{shiika}
\vspace{0.6cm}
\begin{shiika}開け放つ窓に早起き木樺かな
\hfill{\rensuji*{4}・\rensuji*{0}・\rensuji*{0}}\end{shiika}
\vspace{0.6cm}
\begin{shiika}酌みもして婿の気配り凉しき餉
\hfill{\rensuji*{4}・\rensuji*{0}・\rensuji*{0}}\end{shiika}
\vspace{0.6cm}
\begin{shiika}倒産の去りゆく一家百日紅
\hfill{\rensuji*{4}・\rensuji*{0}・\rensuji*{0}}\end{shiika}
\vspace{0.6cm}
\begin{shiika}一言がちくりと秋の草に棘
\hfill{\rensuji*{4}・\rensuji*{0}・\rensuji*{0}}\end{shiika}
\vspace{0.6cm}
\begin{shiika}遠冨士の景ある売地草茂る
\hfill{\rensuji*{4}・\rensuji*{0}・\rensuji*{0}}\end{shiika}
\vspace{0.6cm}
\begin{shiika}芝生踏む素足に伝ふ今朝の秋
\hfill{\rensuji*{4}・\rensuji*{0}・\rensuji*{0}}\end{shiika}
\vspace{0.6cm}
\begin{shiika}新凉や試歩の芝生に笑み交す
\hfill{\rensuji*{4}・\rensuji*{0}・\rensuji*{0}}\end{shiika}
\vspace{0.6cm}
\begin{shiika}高階に寝て眺め居り雲の峰
\hfill{\rensuji*{4}・\rensuji*{0}・\rensuji*{0}}\end{shiika}
\vspace{0.6cm}
\begin{shiika}霧にまだ眠る町並試歩はげむ
\hfill{\rensuji*{4}・\rensuji*{0}・\rensuji*{0}}\end{shiika}
\vspace{0.6cm}
\begin{shiika}夏霧の深し湯の町まだ覚めず
\hfill{\rensuji*{4}・\rensuji*{0}・\rensuji*{0}}\end{shiika}
\vspace{0.6cm}
\begin{shiika}回廊に沿ふ白萩に清めらる
\hfill{\rensuji*{4}・\rensuji*{0}・\rensuji*{0}}\end{shiika}
\vspace{0.6cm}
\begin{shiika}水攻めの城跡や蓮の実の大粒
\hfill{\rensuji*{4}・\rensuji*{0}・\rensuji*{0}}\end{shiika}
\vspace{0.6cm}
\begin{shiika}苗木より三年無花果三つ熟れる
\hfill{\rensuji*{4}・\rensuji*{0}・\rensuji*{0}}\end{shiika}
\vspace{0.6cm}
\begin{shiika}長生きに想ひいろいろ敬老日
\hfill{\rensuji*{4}・\rensuji*{0}・\rensuji*{0}}\end{shiika}
\vspace{0.6cm}
\begin{shiika}秋灯下親しきものは虫眼鏡
\hfill{\rensuji*{4}・\rensuji*{0}・\rensuji*{0}}\end{shiika}
\vspace{0.6cm}
\begin{shiika}保養所の昼餉にぎやか大秋刀魚
\hfill{\rensuji*{4}・\rensuji*{0}・\rensuji*{0}}\end{shiika}
\vspace{0.6cm}
\begin{shiika}露芝生試歩の目標果し得て
\hfill{\rensuji*{4}・\rensuji*{0}・\rensuji*{0}}\end{shiika}
\vspace{0.6cm}
\begin{shiika}秋日和木椅子に一病話し合ふ
\hfill{\rensuji*{4}・\rensuji*{0}・\rensuji*{0}}\end{shiika}
\vspace{0.6cm}
\begin{shiika}シャッターを頼む一会や寺紅葉
\hfill{\rensuji*{4}・\rensuji*{0}・\rensuji*{0}}\end{shiika}
\vspace{0.6cm}
\begin{shiika}庭園灯淡きに和せぬ木犀の香
\hfill{\rensuji*{4}・\rensuji*{0}・\rensuji*{0}}\end{shiika}
\vspace{0.6cm}
\begin{shiika}実梅の香まこと顔して嘘をきく
\hfill{\rensuji*{4}・\rensuji*{0}・\rensuji*{0}}\end{shiika}
\vspace{0.6cm}
\begin{shiika}夜の仏間大蜘蛛打ちて逃がしけり
\hfill{\rensuji*{4}・\rensuji*{0}・\rensuji*{0}}\end{shiika}
\vspace{0.6cm}
\begin{shiika}耳遠く独りもよしと新茶汲む
\hfill{\rensuji*{4}・\rensuji*{0}・\rensuji*{0}}\end{shiika}
\vspace{0.6cm}
\begin{shiika}魂迎ふやがては迎えらるる吾
\hfill{\rensuji*{4}・\rensuji*{0}・\rensuji*{0}}\end{shiika}
\vspace{0.6cm}
\begin{shiika}帰省子に一夜越し方きかれけり
\hfill{\rensuji*{4}・\rensuji*{0}・\rensuji*{0}}\end{shiika}
\vspace{0.6cm}
\begin{shiika}山荘の冨士見ゆ窓に姫りんご
\hfill{\rensuji*{4}・\rensuji*{0}・\rensuji*{0}}\end{shiika}
\vspace{0.6cm}
\begin{shiika}夜霧匂ふ同郷なりし荘の主
\hfill{\rensuji*{4}・\rensuji*{0}・\rensuji*{0}}\end{shiika}
\vspace{0.6cm}
\begin{shiika}天高し無傷の紺を飛機が割る
\hfill{\rensuji*{4}・\rensuji*{0}・\rensuji*{0}}\end{shiika}
\vspace{0.6cm}
\begin{shiika}セーターの赤を鏡に問ふ八・
\hfill{\rensuji*{4}・\rensuji*{0}・\rensuji*{0}}\end{shiika}
\vspace{0.6cm}
\begin{shiika}声高や桜紅葉の女子校道
\hfill{\rensuji*{4}・\rensuji*{0}・\rensuji*{0}}\end{shiika}
\vspace{0.6cm}
\begin{shiika}迎えられ娘の柚子風呂の香りかな
\hfill{\rensuji*{4}・\rensuji*{0}・\rensuji*{0}}\end{shiika}
\vspace{0.6cm}
\begin{shiika}いさかひが笑ひに母と娘の冬至
\hfill{\rensuji*{4}・\rensuji*{0}・\rensuji*{0}}\end{shiika}
\vspace{0.6cm}
\begin{shiika}年用意母と娘の声いづれとも
\hfill{\rensuji*{4}・\rensuji*{0}・\rensuji*{0}}\end{shiika}
\vspace{0.6cm}
\begin{shiika}部屋に冷ゆ胸像の夫に独り言
\hfill{\rensuji*{4}・\rensuji*{0}・\rensuji*{0}}\end{shiika}
\vspace{0.6cm}
\begin{shiika}行く年へ刻む時計に息つめて
\hfill{\rensuji*{4}・\rensuji*{0}・\rensuji*{0}}\end{shiika}
\vspace{0.6cm}
\begin{shiika}我が城と正月飾り四畳半
\hfill{\rensuji*{5}・\rensuji*{0}・\rensuji*{0}}\end{shiika}
\vspace{0.6cm}
\begin{shiika}繰るほどに夢ふくらみ来初暦
\hfill{\rensuji*{5}・\rensuji*{0}・\rensuji*{0}}\end{shiika}
\vspace{0.6cm}
\begin{shiika}二日早帰る子送る母の背
\hfill{\rensuji*{5}・\rensuji*{0}・\rensuji*{0}}\end{shiika}
\vspace{0.6cm}
\begin{shiika}好物で老犬はげます寒の入
\hfill{\rensuji*{5}・\rensuji*{0}・\rensuji*{0}}\end{shiika}
\vspace{0.6cm}

\begin{shiika}居候の老に朝毎寒玉子
\hfill{\rensuji*{5}・\rensuji*{0}・\rensuji*{0}}\end{shiika}
\vspace{0.6cm}
\begin{shiika}老犬と共に留守居す梅日和
\hfill{\rensuji*{5}・\rensuji*{0}・\rensuji*{0}}\end{shiika}
\vspace{0.6cm}
\begin{shiika}老犬の背に紅梅の一片が
\hfill{\rensuji*{5}・\rensuji*{0}・\rensuji*{0}}\end{shiika}
\vspace{0.6cm}
\begin{shiika}一跳ねに広がる水輪水ぬるむ
\hfill{\rensuji*{5}・\rensuji*{0}・\rensuji*{0}}\end{shiika}
\vspace{0.6cm}
\begin{shiika}春立ちぬ川面は白き雲浮かべ
\hfill{\rensuji*{5}・\rensuji*{0}・\rensuji*{0}}\end{shiika}
\vspace{0.6cm}
\begin{shiika}白き雲浮かべ川面は春立ちぬ
\hfill{\rensuji*{5}・\rensuji*{0}・\rensuji*{0}}\end{shiika}
\vspace{0.6cm}
\begin{shiika}倖せは歯音にありし年の豆
\hfill{\rensuji*{5}・\rensuji*{0}・\rensuji*{0}}\end{shiika}
\vspace{0.6cm}
\begin{shiika}今日よりはチロ居ぬ生活春寒し
\hfill{\rensuji*{5}・\rensuji*{0}・\rensuji*{0}}\end{shiika}
\vspace{0.6cm}
\begin{shiika}姫こぶし一輪樹下にチロは死す
\hfill{\rensuji*{5}・\rensuji*{0}・\rensuji*{0}}\end{shiika}
\vspace{0.6cm}
\begin{shiika}春嵐おさまる朝にチロは死す
\hfill{\rensuji*{5}・\rensuji*{0}・\rensuji*{0}}\end{shiika}
\vspace{0.6cm}
\begin{shiika}春寒しピンクの布に巻く屍
\hfill{\rensuji*{5}・\rensuji*{0}・\rensuji*{0}}\end{shiika}
\vspace{0.6cm}
\begin{shiika}窓開けばおやつ待つチロ無き余寒
\hfill{\rensuji*{5}・\rensuji*{0}・\rensuji*{0}}\end{shiika}
\vspace{0.6cm}
\begin{shiika}従姉妹どち幼な呼びして桃の郷
\hfill{\rensuji*{5}・\rensuji*{0}・\rensuji*{0}}\end{shiika}
\vspace{0.6cm}
\begin{shiika}故里や摘みてたちまち木の芽和え
\hfill{\rensuji*{5}・\rensuji*{0}・\rensuji*{0}}\end{shiika}
\vspace{0.6cm}
\begin{shiika}故里はお遍路の鈴あわあわと
\hfill{\rensuji*{5}・\rensuji*{0}・\rensuji*{0}}\end{shiika}
\vspace{0.6cm}
\begin{shiika}朧夜や骨までしゃぶる瀬戸の味
\hfill{\rensuji*{5}・\rensuji*{0}・\rensuji*{0}}\end{shiika}
\vspace{0.6cm}
\begin{shiika}短夜やはらから集ふ郷言葉
\hfill{\rensuji*{5}・\rensuji*{0}・\rensuji*{0}}\end{shiika}
\vspace{0.6cm}
\begin{shiika}老鴬に迎え送られ札所寺
\hfill{\rensuji*{5}・\rensuji*{0}・\rensuji*{0}}\end{shiika}
\vspace{0.6cm}
\begin{shiika}仁王門くぐりて見上ぐ余花やさし
\hfill{\rensuji*{5}・\rensuji*{0}・\rensuji*{0}}\end{shiika}
\vspace{0.6cm}
\begin{shiika}牡丹や余生つぎこむ花づくり
\hfill{\rensuji*{5}・\rensuji*{0}・\rensuji*{0}}\end{shiika}
\vspace{0.6cm}
\begin{shiika}新背広卒業の子を見上げけり
\hfill{\rensuji*{5}・\rensuji*{0}・\rensuji*{0}}\end{shiika}
\vspace{0.6cm}
\begin{shiika}祝背広就職といふ巣立かな
\hfill{\rensuji*{5}・\rensuji*{0}・\rensuji*{0}}\end{shiika}
\vspace{0.6cm}
\begin{shiika}就職は別れの一つ鳥雲に
\hfill{\rensuji*{5}・\rensuji*{0}・\rensuji*{0}}\end{shiika}
\vspace{0.6cm}
\begin{shiika}散華とも霊園しとど花吹雪
\hfill{\rensuji*{5}・\rensuji*{0}・\rensuji*{0}}\end{shiika}
\vspace{0.6cm}
\begin{shiika}咲き競ひし源平桃も葉となりぬ
\hfill{\rensuji*{5}・\rensuji*{0}・\rensuji*{0}}\end{shiika}
\vspace{0.6cm}
\begin{shiika}藤娘出そう藤房ととのへり
\hfill{\rensuji*{5}・\rensuji*{0}・\rensuji*{0}}\end{shiika}
\vspace{0.6cm}
\begin{shiika}三代の旅信濃路を青葉風
\hfill{\rensuji*{5}・\rensuji*{0}・\rensuji*{0}}\end{shiika}
\vspace{0.6cm}
\begin{shiika}大手まり真白湯の香の中にゆれ
\hfill{\rensuji*{5}・\rensuji*{0}・\rensuji*{0}}\end{shiika}
\vspace{0.6cm}
\begin{shiika}まじり気のなきみどり嶺よ露天風呂
\hfill{\rensuji*{5}・\rensuji*{0}・\rensuji*{0}}\end{shiika}
\vspace{0.6cm}
\begin{shiika}峯八分疲れは軽し藤の花
\hfill{\rensuji*{5}・\rensuji*{0}・\rensuji*{0}}\end{shiika}
\vspace{0.6cm}
\begin{shiika}からみ合ひ花房乱る深山藤
\hfill{\rensuji*{5}・\rensuji*{0}・\rensuji*{0}}\end{shiika}
\vspace{0.6cm}
\begin{shiika}子に植えし桜桃熟るる少女有美
\hfill{\rensuji*{5}・\rensuji*{0}・\rensuji*{0}}\end{shiika}
\vspace{0.6cm}
\begin{shiika}遍路憩ふ礎石千年語りつぐ
\hfill{\rensuji*{5}・\rensuji*{0}・\rensuji*{0}}\end{shiika}
\vspace{0.6cm}
\begin{shiika}点滴の紫班をさする梅雨の窓
\hfill{\rensuji*{5}・\rensuji*{0}・\rensuji*{0}}\end{shiika}
\vspace{0.6cm}
\begin{shiika}明易すや退院といふ別れかな
\hfill{\rensuji*{5}・\rensuji*{0}・\rensuji*{0}}\end{shiika}
\vspace{0.6cm}
\begin{shiika}濃紫陽花点滴の染みうすれゆく
\hfill{\rensuji*{5}・\rensuji*{0}・\rensuji*{0}}\end{shiika}
\vspace{0.6cm}
\begin{shiika}錠剤をならべ数えて夕薄暑
\hfill{\rensuji*{5}・\rensuji*{0}・\rensuji*{0}}\end{shiika}
\vspace{0.6cm}
\begin{shiika}負け相撲少し頭痛の戻り梅雨
\hfill{\rensuji*{5}・\rensuji*{0}・\rensuji*{0}}\end{shiika}
\vspace{0.6cm}
\begin{shiika}連れだちていそいそ母娘浴衣買ひ
\hfill{\rensuji*{5}・\rensuji*{0}・\rensuji*{0}}\end{shiika}
\vspace{0.6cm}
\begin{shiika}連れだちて母娘の購む派手浴衣
\hfill{\rensuji*{5}・\rensuji*{0}・\rensuji*{0}}\end{shiika}
\vspace{0.6cm}
\begin{shiika}浴衣茶会立居気になる娘を送る
\hfill{\rensuji*{5}・\rensuji*{0}・\rensuji*{0}}\end{shiika}
\vspace{0.6cm}
\begin{shiika}月下美人迎へ車で御対面
\hfill{\rensuji*{5}・\rensuji*{0}・\rensuji*{0}}\end{shiika}
\vspace{0.6cm}
\begin{shiika}月下美人息を弛めず咲き拡ぐ
\hfill{\rensuji*{5}・\rensuji*{0}・\rensuji*{0}}\end{shiika}
\vspace{0.6cm}
\begin{shiika}手伝ひ娘不満あるげに水を打つ
\hfill{\rensuji*{5}・\rensuji*{0}・\rensuji*{0}}\end{shiika}
\vspace{0.6cm}
\begin{shiika}咲きましたとて嫁が見す鷺草鉢
\hfill{\rensuji*{5}・\rensuji*{0}・\rensuji*{0}}\end{shiika}
\vspace{0.6cm}
\begin{shiika}鷺草の飛びさる舞ひよう目離せず
\hfill{\rensuji*{5}・\rensuji*{0}・\rensuji*{0}}\end{shiika}
\vspace{0.6cm}
\begin{shiika}水撒けば陶狸がうれし涙する
\hfill{\rensuji*{5}・\rensuji*{0}・\rensuji*{0}}\end{shiika}
\vspace{0.6cm}
\begin{shiika}これはまあ皿をはみ出る初秋刀魚
\hfill{\rensuji*{5}・\rensuji*{0}・\rensuji*{0}}\end{shiika}
\vspace{0.6cm}
\begin{shiika}倉裡裏の鬼灯赤し妻若し
\hfill{\rensuji*{5}・\rensuji*{0}・\rensuji*{0}}\end{shiika}
\vspace{0.6cm}
\begin{shiika}猫難の子雀放つ秋彼岸
\hfill{\rensuji*{5}・\rensuji*{0}・\rensuji*{0}}\end{shiika}
\vspace{0.6cm}
\begin{shiika}雀獲りしかり猫抱く秋彼岸
\hfill{\rensuji*{5}・\rensuji*{0}・\rensuji*{0}}\end{shiika}
\vspace{0.6cm}
\begin{shiika}映る影流るる音も水の秋
\hfill{\rensuji*{5}・\rensuji*{0}・\rensuji*{0}}\end{shiika}
\vspace{0.6cm}
\begin{shiika}秋晴やいそいそ釣に碁敵と
\hfill{\rensuji*{5}・\rensuji*{0}・\rensuji*{0}}\end{shiika}
\vspace{0.6cm}
\begin{shiika}秋晴や碁敵はまた釣がたき
\hfill{\rensuji*{5}・\rensuji*{0}・\rensuji*{0}}\end{shiika}
\vspace{0.6cm}
\begin{shiika}釣りし沙魚はねる厨にはや碁音
\hfill{\rensuji*{5}・\rensuji*{0}・\rensuji*{0}}\end{shiika}
\vspace{0.6cm}
\begin{shiika}雁渡る双手で握手する別れ
\hfill{\rensuji*{5}・\rensuji*{0}・\rensuji*{0}}\end{shiika}
\vspace{0.6cm}
\begin{shiika}口釜へ増ゆる孫との日向ぼこ
\hfill{\rensuji*{5}・\rensuji*{0}・\rensuji*{0}}\end{shiika}
\vspace{0.6cm}
\begin{shiika}柿送る案内電話の郷言葉
\hfill{\rensuji*{5}・\rensuji*{0}・\rensuji*{0}}\end{shiika}
\vspace{0.6cm}
\begin{shiika}柳散る入日に染まる湖のほとり
\hfill{\rensuji*{5}・\rensuji*{0}・\rensuji*{0}}\end{shiika}
\vspace{0.6cm}
\begin{shiika}五指ほぐすなだむ節おし今朝の秋
\hfill{\rensuji*{5}・\rensuji*{0}・\rensuji*{0}}\end{shiika}
\vspace{0.6cm}
\begin{shiika}夜逃げとや閉ざせる窓に満月光
\hfill{\rensuji*{5}・\rensuji*{0}・\rensuji*{0}}\end{shiika}
\vspace{0.6cm}
\begin{shiika}人恋ふかに垣越し延び来青き蔦
\hfill{\rensuji*{5}・\rensuji*{0}・\rensuji*{0}}\end{shiika}
\vspace{0.6cm}
\begin{shiika}猫舌は母似亡母恋ふ湯豆腐鍋
\hfill{\rensuji*{5}・\rensuji*{0}・\rensuji*{0}}\end{shiika}
\vspace{0.6cm}
\begin{shiika}物言はず一日留守居の師走呆け
\hfill{\rensuji*{5}・\rensuji*{0}・\rensuji*{0}}\end{shiika}
\vspace{0.6cm}
\begin{shiika}冬日向売れぬ空地は猫のもの
\hfill{\rensuji*{5}・\rensuji*{0}・\rensuji*{0}}\end{shiika}
\vspace{0.6cm}
\begin{shiika}カレンダーも庭も山茶花日々惜しむ
\hfill{\rensuji*{5}・\rensuji*{0}・\rensuji*{0}}\end{shiika}
\vspace{0.6cm}
\begin{shiika}柚子ほめてつい佇ち話いただけり
\hfill{\rensuji*{5}・\rensuji*{0}・\rensuji*{0}}\end{shiika}
\vspace{0.6cm}
\begin{shiika}留守居して米研ぐ窓に寒宵月
\hfill{\rensuji*{5}・\rensuji*{0}・\rensuji*{0}}\end{shiika}
\vspace{0.6cm}
\begin{shiika}大晴れや蒲団干す家干せぬ家
\hfill{\rensuji*{5}・\rensuji*{0}・\rensuji*{0}}\end{shiika}
\vspace{0.6cm}
\begin{shiika}爪切りて指美しや賀状書く
\hfill{\rensuji*{5}・\rensuji*{0}・\rensuji*{0}}\end{shiika}
\vspace{0.6cm}
\begin{shiika}吹き溜る枯葉の中の紅一葉
\hfill{\rensuji*{5}・\rensuji*{0}・\rensuji*{0}}\end{shiika}
\vspace{0.6cm}
\begin{shiika}宵戎押さへ揉まれて娘はきげん
\hfill{\rensuji*{6}・\rensuji*{0}・\rensuji*{0}}\end{shiika}
\vspace{0.6cm}
\begin{shiika}ただいまの娘の声弾む宵戎
\hfill{\rensuji*{6}・\rensuji*{0}・\rensuji*{0}}\end{shiika}
\vspace{0.6cm}
\begin{shiika}初釜へ晴着見送る母も美し
\hfill{\rensuji*{6}・\rensuji*{0}・\rensuji*{0}}\end{shiika}
\vspace{0.6cm}
\begin{shiika}はよ来ませ郷言うれし初電話
\hfill{\rensuji*{6}・\rensuji*{0}・\rensuji*{0}}\end{shiika}
\vspace{0.6cm}
\begin{shiika}寒玉子盛りあがる黄身老もまた
\hfill{\rensuji*{6}・\rensuji*{0}・\rensuji*{0}}\end{shiika}
\vspace{0.6cm}
\begin{shiika}春寒やもう夢でしか逢へぬ人
\hfill{\rensuji*{6}・\rensuji*{0}・\rensuji*{0}}\end{shiika}
\vspace{0.6cm}
\begin{shiika}頑張れよ愛犬館も初日さす
\hfill{\rensuji*{6}・\rensuji*{0}・\rensuji*{0}}\end{shiika}
\vspace{0.6cm}
\begin{shiika}受験子に買ふ知恵袋文殊さま
\hfill{\rensuji*{6}・\rensuji*{0}・\rensuji*{0}}\end{shiika}
\vspace{0.6cm}
\begin{shiika}春寒し起ち居いちいち声あげて
\hfill{\rensuji*{6}・\rensuji*{0}・\rensuji*{0}}\end{shiika}
\vspace{0.6cm}
\begin{shiika}中古車群旗はたはたと春を呼ぶ
\hfill{\rensuji*{6}・\rensuji*{0}・\rensuji*{0}}\end{shiika}
\vspace{0.6cm}
\begin{shiika}猫柳活ける娘もまたつやつやし
\hfill{\rensuji*{6}・\rensuji*{0}・\rensuji*{0}}\end{shiika}
\vspace{0.6cm}
\begin{shiika}花葉挿しふと京の友思ひけり
\hfill{\rensuji*{6}・\rensuji*{0}・\rensuji*{0}}\end{shiika}
\vspace{0.6cm}
\begin{shiika}再会や土を割り出る花芽たち
\hfill{\rensuji*{6}・\rensuji*{0}・\rensuji*{0}}\end{shiika}
\vspace{0.6cm}
\begin{shiika}分葱和へおふくろ味の老自慢
\hfill{\rensuji*{6}・\rensuji*{0}・\rensuji*{0}}\end{shiika}
\vspace{0.6cm}
\begin{shiika}名もゆかし若草豆腐のうすみどり
\hfill{\rensuji*{6}・\rensuji*{0}・\rensuji*{0}}\end{shiika}
\vspace{0.6cm}
\begin{shiika}点心に一口ほどのたらの芽よ
\hfill{\rensuji*{6}・\rensuji*{0}・\rensuji*{0}}\end{shiika}
\vspace{0.6cm}
\begin{shiika}茄子胡瓜畑銀座と故里便り
\hfill{\rensuji*{6}・\rensuji*{0}・\rensuji*{0}}\end{shiika}
\vspace{0.6cm}
\begin{shiika}額の花一人で居たき時もあり
\hfill{\rensuji*{6}・\rensuji*{0}・\rensuji*{0}}\end{shiika}
\vspace{0.6cm}
\begin{shiika}夏帽子のぞく白髪も好しとして
\hfill{\rensuji*{6}・\rensuji*{0}・\rensuji*{0}}\end{shiika}
\vspace{0.6cm}
\begin{shiika}夏帽子年齢をきかれて逆に問ひ
\hfill{\rensuji*{6}・\rensuji*{0}・\rensuji*{0}}\end{shiika}
\vspace{0.6cm}
\begin{shiika}山梔子の真白につらき雨つづく
\hfill{\rensuji*{6}・\rensuji*{0}・\rensuji*{0}}\end{shiika}
\vspace{0.6cm}
\begin{shiika}青葉風入れてもきれぬ愚痴話
\hfill{\rensuji*{6}・\rensuji*{0}・\rensuji*{0}}\end{shiika}
\vspace{0.6cm}
\begin{shiika}言ひたきをたたむくちなし真白なる
\hfill{\rensuji*{6}・\rensuji*{0}・\rensuji*{0}}\end{shiika}
\vspace{0.6cm}
\begin{shiika}辻地蔵朝取りトマトにお眼細く
\hfill{\rensuji*{6}・\rensuji*{0}・\rensuji*{0}}\end{shiika}
\vspace{0.6cm}
\begin{shiika}暑に耐える白前掛の辻地蔵
\hfill{\rensuji*{6}・\rensuji*{0}・\rensuji*{0}}\end{shiika}
\vspace{0.6cm}
\begin{shiika}青田風通し一睡の浄土かな
\hfill{\rensuji*{6}・\rensuji*{0}・\rensuji*{0}}\end{shiika}
\vspace{0.6cm}
\begin{shiika}喉走る名水冷えの心太
\hfill{\rensuji*{6}・\rensuji*{0}・\rensuji*{0}}\end{shiika}
\vspace{0.6cm}
\begin{shiika}空暗し呼べば遠退く夕立雲
\hfill{\rensuji*{6}・\rensuji*{0}・\rensuji*{0}}\end{shiika}
\vspace{0.6cm}

\begin{shiika}今日も亦他所夕立とそれにけり
\hfill{\rensuji*{6}・\rensuji*{0}・\rensuji*{0}}\end{shiika}
\vspace{0.6cm}
\begin{shiika}花合歓や渓の音きく温泉の窓
\hfill{\rensuji*{6}・\rensuji*{0}・\rensuji*{0}}\end{shiika}
\vspace{0.6cm}
\begin{shiika}含羞草いで湯泊りの老四人
\hfill{\rensuji*{6}・\rensuji*{0}・\rensuji*{0}}\end{shiika}
\vspace{0.6cm}
\begin{shiika}故里は金比羅歌舞伎花の山
\hfill{\rensuji*{6}・\rensuji*{0}・\rensuji*{0}}\end{shiika}
\vspace{0.6cm}
\begin{shiika}岐れ道ミモザ盛りの島巡り
\hfill{\rensuji*{6}・\rensuji*{0}・\rensuji*{0}}\end{shiika}
\vspace{0.6cm}
\begin{shiika}一言の棘のいたみや夏薊
\hfill{\rensuji*{6}・\rensuji*{0}・\rensuji*{0}}\end{shiika}
\vspace{0.6cm}
\begin{shiika}一言の棘に猛暑の雲みあぐ
\hfill{\rensuji*{6}・\rensuji*{0}・\rensuji*{0}}\end{shiika}
\vspace{0.6cm}
\begin{shiika}風鈴や窓辺に母と娘の笑顔
\hfill{\rensuji*{6}・\rensuji*{0}・\rensuji*{0}}\end{shiika}
\vspace{0.6cm}
\begin{shiika}昼寝覚めまだ侍り猫伸びきって
\hfill{\rensuji*{6}・\rensuji*{0}・\rensuji*{0}}\end{shiika}
\vspace{0.6cm}
\begin{shiika}シルバーホーム笑ち会釈して廊凉し
\hfill{\rensuji*{6}・\rensuji*{0}・\rensuji*{0}}\end{shiika}
\vspace{0.6cm}
\begin{shiika}お元気ねきれいに食べし夏料理
\hfill{\rensuji*{6}・\rensuji*{0}・\rensuji*{0}}\end{shiika}
\vspace{0.6cm}
\begin{shiika}西瓜割漢につづく娘が果す
\hfill{\rensuji*{6}・\rensuji*{0}・\rensuji*{0}}\end{shiika}
\vspace{0.6cm}
\begin{shiika}踊の輪みるみる三重に炭坑節
\hfill{\rensuji*{6}・\rensuji*{0}・\rensuji*{0}}\end{shiika}
\vspace{0.6cm}
\begin{shiika}高階に眼覚めてわっと雲の峰
\hfill{\rensuji*{6}・\rensuji*{0}・\rensuji*{0}}\end{shiika}
\vspace{0.6cm}
\begin{shiika}熱帯夜慣れて別れのなにとなう
\hfill{\rensuji*{6}・\rensuji*{0}・\rensuji*{0}}\end{shiika}
\vspace{0.6cm}
\begin{shiika}朝凉や肩まで掛けてふと淋し
\hfill{\rensuji*{6}・\rensuji*{0}・\rensuji*{0}}\end{shiika}
\vspace{0.6cm}
\begin{shiika}雲の峰息子は太平洋の空ならん
\hfill{\rensuji*{6}・\rensuji*{0}・\rensuji*{0}}\end{shiika}
\vspace{0.6cm}
\begin{shiika}満月や仰ぎし友はいま筑紫
\hfill{\rensuji*{6}・\rensuji*{0}・\rensuji*{0}}\end{shiika}
\vspace{0.6cm}
\begin{shiika}月白やせり上り待つ大舞台
\hfill{\rensuji*{6}・\rensuji*{0}・\rensuji*{0}}\end{shiika}
\vspace{0.6cm}
\begin{shiika}手折り来て芒挿しくれホーム友
\hfill{\rensuji*{6}・\rensuji*{0}・\rensuji*{0}}\end{shiika}
\vspace{0.6cm}
\begin{shiika}敬老日過ぎて忘れを詫ぶ息子かな
\hfill{\rensuji*{6}・\rensuji*{0}・\rensuji*{0}}\end{shiika}
\vspace{0.6cm}
\begin{shiika}夕木槿一日思案し言ふまじと
\hfill{\rensuji*{6}・\rensuji*{0}・\rensuji*{0}}\end{shiika}
\vspace{0.6cm}
\begin{shiika}傷つけしことに気附かず青芒
\hfill{\rensuji*{6}・\rensuji*{0}・\rensuji*{0}}\end{shiika}
\vspace{0.6cm}
\begin{shiika}押し分けも背伸びもなくて草の花
\hfill{\rensuji*{6}・\rensuji*{0}・\rensuji*{0}}\end{shiika}
\vspace{0.6cm}
\begin{shiika}侘びて住むごと庭隅の時鳥草
\hfill{\rensuji*{6}・\rensuji*{0}・\rensuji*{0}}\end{shiika}
\vspace{0.6cm}
\begin{shiika}住むは誰隣の芒刈られけり
\hfill{\rensuji*{6}・\rensuji*{0}・\rensuji*{0}}\end{shiika}
\vspace{0.6cm}
\begin{shiika}息子に目立ちきし白きもの柿をむく
\hfill{\rensuji*{6}・\rensuji*{0}・\rensuji*{0}}\end{shiika}
\vspace{0.6cm}
\begin{shiika}高階に泊つ霧ぬれの大夜景
\hfill{\rensuji*{6}・\rensuji*{0}・\rensuji*{0}}\end{shiika}
\vspace{0.6cm}
\begin{shiika}秋灯に左傾ぎの寿百の字
\hfill{\rensuji*{6}・\rensuji*{0}・\rensuji*{0}}\end{shiika}
\vspace{0.6cm}
\begin{shiika}ふる里や菜飯に小芋の煮ころがし
\hfill{\rensuji*{6}・\rensuji*{0}・\rensuji*{0}}\end{shiika}
\vspace{0.6cm}
\begin{shiika}大根抜く厨に待つはおろしがね
\hfill{\rensuji*{6}・\rensuji*{0}・\rensuji*{0}}\end{shiika}
\vspace{0.6cm}
\begin{shiika}木あがりの茄子見落さず芥子漬
\hfill{\rensuji*{6}・\rensuji*{0}・\rensuji*{0}}\end{shiika}
\vspace{0.6cm}
\begin{shiika}木あがりの茄子と思へぬ芥子漬
\hfill{\rensuji*{6}・\rensuji*{0}・\rensuji*{0}}\end{shiika}
\vspace{0.6cm}
\begin{shiika}そつと出る夫追ふ妻や露の畑
\hfill{\rensuji*{6}・\rensuji*{0}・\rensuji*{0}}\end{shiika}
\vspace{0.6cm}
\begin{shiika}医と寺の娘が幼な友木の葉髪
\hfill{\rensuji*{6}・\rensuji*{0}・\rensuji*{0}}\end{shiika}
\vspace{0.6cm}
\begin{shiika}秋風や札所の寺の大礎石
\hfill{\rensuji*{6}・\rensuji*{0}・\rensuji*{0}}\end{shiika}
\vspace{0.6cm}
\begin{shiika}木犀匂ふ金銀並びし故里の庭
\hfill{\rensuji*{6}・\rensuji*{0}・\rensuji*{0}}\end{shiika}
\vspace{0.6cm}
\begin{shiika}着ぶくれて椅子のくぼみに孫自慢
\hfill{\rensuji*{6}・\rensuji*{0}・\rensuji*{0}}\end{shiika}
\vspace{0.6cm}
\begin{shiika}ほほえみで答ふ遠耳冬すみれ
\hfill{\rensuji*{6}・\rensuji*{0}・\rensuji*{0}}\end{shiika}
\vspace{0.6cm}
\begin{shiika}言ふだけを言ふてコートの忘れ物
\hfill{\rensuji*{6}・\rensuji*{0}・\rensuji*{0}}\end{shiika}
\vspace{0.6cm}
\begin{shiika}爪切りて指美くしく賀状書く
\hfill{\rensuji*{6}・\rensuji*{0}・\rensuji*{0}}\end{shiika}
\vspace{0.6cm}
\begin{shiika}保養所の握手の別れ紅葉散る
\hfill{\rensuji*{6}・\rensuji*{0}・\rensuji*{0}}\end{shiika}
\vspace{0.6cm}
\begin{shiika}晩菊にそとさよならをしばし旅
\hfill{\rensuji*{6}・\rensuji*{0}・\rensuji*{0}}\end{shiika}
\vspace{0.6cm}
\begin{shiika}物忘れめつきり増えて年の暮
\hfill{\rensuji*{6}・\rensuji*{0}・\rensuji*{0}}\end{shiika}
\vspace{0.6cm}
\begin{shiika}晩菊の一本供花とし剪りにけり
\hfill{\rensuji*{6}・\rensuji*{0}・\rensuji*{0}}\end{shiika}
\vspace{0.6cm}
\begin{shiika}補聴器を切りて一人の冬の夜
\hfill{\rensuji*{6}・\rensuji*{0}・\rensuji*{0}}\end{shiika}
\vspace{0.6cm}
\begin{shiika}ほんのりと米寿の頬に屠蘇の紅
\hfill{\rensuji*{7}・\rensuji*{0}・\rensuji*{0}}\end{shiika}
\vspace{0.6cm}
\begin{shiika}倖せは初夢もなき深眠り
\hfill{\rensuji*{7}・\rensuji*{0}・\rensuji*{0}}\end{shiika}
\vspace{0.6cm}
\begin{shiika}住連飾りドアーにかけて・
\hfill{\rensuji*{7}・\rensuji*{0}・\rensuji*{0}}\end{shiika}
\vspace{0.6cm}
\begin{shiika}開かんと冬薔薇秘めし力かな
\hfill{\rensuji*{7}・\rensuji*{0}・\rensuji*{0}}\end{shiika}
\vspace{0.6cm}
\begin{shiika}梅一輪いちりん日々を留守居して
\hfill{\rensuji*{7}・\rensuji*{0}・\rensuji*{0}}\end{shiika}
\vspace{0.6cm}
\begin{shiika}倖せや日々の留守居に梅一輪
\hfill{\rensuji*{7}・\rensuji*{0}・\rensuji*{0}}\end{shiika}
\vspace{0.6cm}
\begin{shiika}紅梅や白磁揃ひの朝餉の膳
\hfill{\rensuji*{7}・\rensuji*{0}・\rensuji*{0}}\end{shiika}
\vspace{0.6cm}
\begin{shiika}話す日々米寿祝の冬ばらに
\hfill{\rensuji*{7}・\rensuji*{0}・\rensuji*{0}}\end{shiika}
\vspace{0.6cm}
\begin{shiika}毛糸解く編み直されぬ過去てふもの
\hfill{\rensuji*{7}・\rensuji*{0}・\rensuji*{0}}\end{shiika}
\vspace{0.6cm}
\begin{shiika}春寒し幼なに戻るおないどし
\hfill{\rensuji*{7}・\rensuji*{0}・\rensuji*{0}}\end{shiika}
\vspace{0.6cm}
\begin{shiika}空地占め空の青吸ひ犬ふぐり
\hfill{\rensuji*{7}・\rensuji*{0}・\rensuji*{0}}\end{shiika}
\vspace{0.6cm}
\begin{shiika}椀に浮くさみどりを吸い春一番
\hfill{\rensuji*{7}・\rensuji*{0}・\rensuji*{0}}\end{shiika}
\vspace{0.6cm}
\begin{shiika}朝桜夢のあと追ふ思慕の人
\hfill{\rensuji*{7}・\rensuji*{0}・\rensuji*{0}}\end{shiika}
\vspace{0.6cm}
\begin{shiika}聞くだけで事情を愚痴の春炬燵
\hfill{\rensuji*{7}・\rensuji*{0}・\rensuji*{0}}\end{shiika}
\vspace{0.6cm}
\begin{shiika}躓きて掌をつくところ土筆んぼ
\hfill{\rensuji*{7}・\rensuji*{0}・\rensuji*{0}}\end{shiika}
\vspace{0.6cm}
\begin{shiika}躓きて土筆三本折りて詫ぶ
\hfill{\rensuji*{7}・\rensuji*{0}・\rensuji*{0}}\end{shiika}
\vspace{0.6cm}
\begin{shiika}雪柳白壁拒み闇寄せず
\hfill{\rensuji*{7}・\rensuji*{0}・\rensuji*{0}}\end{shiika}
\vspace{0.6cm}
\begin{shiika}白壁の汚れはじらふ雪柳
\hfill{\rensuji*{7}・\rensuji*{0}・\rensuji*{0}}\end{shiika}
\vspace{0.6cm}
\begin{shiika}ワインの栓ぼんに拍手や夜はおぼろ
\hfill{\rensuji*{7}・\rensuji*{0}・\rensuji*{0}}\end{shiika}
\vspace{0.6cm}
\begin{shiika}花は葉に母の素直は息子の憂ひ
\hfill{\rensuji*{7}・\rensuji*{0}・\rensuji*{0}}\end{shiika}
\vspace{0.6cm}
\begin{shiika}応えなく平寝落ちしよ花疲れ
\hfill{\rensuji*{7}・\rensuji*{0}・\rensuji*{0}}\end{shiika}
\vspace{0.6cm}
\begin{shiika}落ち椿さつさと主掃きにけり
\hfill{\rensuji*{7}・\rensuji*{0}・\rensuji*{0}}\end{shiika}
\vspace{0.6cm}
\begin{shiika}兄弟が初鯉のぼり揚げにけり
\hfill{\rensuji*{7}・\rensuji*{0}・\rensuji*{0}}\end{shiika}
\vspace{0.6cm}
\begin{shiika}母の日に娘二人の遠電話
\hfill{\rensuji*{7}・\rensuji*{0}・\rensuji*{0}}\end{shiika}
\vspace{0.6cm}
\begin{shiika}母の日や六・
\hfill{\rensuji*{7}・\rensuji*{0}・\rensuji*{0}}\end{shiika}
\vspace{0.6cm}
\begin{shiika}岐れ道えらべば険し果の余花
\hfill{\rensuji*{7}・\rensuji*{0}・\rensuji*{0}}\end{shiika}
\vspace{0.6cm}
\begin{shiika}試歩のばす思ひたがわず藤の花
\hfill{\rensuji*{7}・\rensuji*{0}・\rensuji*{0}}\end{shiika}
\vspace{0.6cm}
\begin{shiika}絵タイルの道若やぎて地球の日
\hfill{\rensuji*{7}・\rensuji*{0}・\rensuji*{0}}\end{shiika}
\vspace{0.6cm}
\begin{shiika}高きほど大揺れてをり夾竹桃
\hfill{\rensuji*{7}・\rensuji*{0}・\rensuji*{0}}\end{shiika}
\vspace{0.6cm}
\begin{shiika}雑草の茂りたくまし子もたくまし
\hfill{\rensuji*{7}・\rensuji*{0}・\rensuji*{0}}\end{shiika}
\vspace{0.6cm}
\begin{shiika}草いくさ陣地広げし青芒
\hfill{\rensuji*{7}・\rensuji*{0}・\rensuji*{0}}\end{shiika}
\vspace{0.6cm}
\begin{shiika}葉を研ぎて陣地広げむ青芒
\hfill{\rensuji*{7}・\rensuji*{0}・\rensuji*{0}}\end{shiika}
\vspace{0.6cm}
\begin{shiika}職退くも余生と言へぬ梅青し
\hfill{\rensuji*{7}・\rensuji*{0}・\rensuji*{0}}\end{shiika}
\vspace{0.6cm}
\begin{shiika}娘名で忌の案内状梅雨じめり
\hfill{\rensuji*{7}・\rensuji*{0}・\rensuji*{0}}\end{shiika}
\vspace{0.6cm}
\begin{shiika}海の風山の風入れ夏座敷
\hfill{\rensuji*{7}・\rensuji*{0}・\rensuji*{0}}\end{shiika}
\vspace{0.6cm}
\begin{shiika}夕木槿汚れなき白閉じにけり
\hfill{\rensuji*{7}・\rensuji*{0}・\rensuji*{0}}\end{shiika}
\vspace{0.6cm}
\begin{shiika}春秋を裾にひろげて讃岐冨士
\hfill{\rensuji*{7}・\rensuji*{0}・\rensuji*{0}}\end{shiika}
\vspace{0.6cm}
\begin{shiika}はいはいと重ねてさびし含羞草
\hfill{\rensuji*{7}・\rensuji*{0}・\rensuji*{0}}\end{shiika}
\vspace{0.6cm}
\begin{shiika}眠り草ねむらぬ葉あり反抗期
\hfill{\rensuji*{7}・\rensuji*{0}・\rensuji*{0}}\end{shiika}
\vspace{0.6cm}
\begin{shiika}装ひし遠き日のあり薄衣
\hfill{\rensuji*{7}・\rensuji*{0}・\rensuji*{0}}\end{shiika}
\vspace{0.6cm}
\begin{shiika}咲き満つもなほあわあわと花みずき
\hfill{\rensuji*{7}・\rensuji*{0}・\rensuji*{0}}\end{shiika}
\vspace{0.6cm}
\begin{shiika}花水木乙女の恋の物語
\hfill{\rensuji*{7}・\rensuji*{0}・\rensuji*{0}}\end{shiika}
\vspace{0.6cm}
\begin{shiika}故郷発つ朝採りトマト重すぎて
\hfill{\rensuji*{7}・\rensuji*{0}・\rensuji*{0}}\end{shiika}
\vspace{0.6cm}
\begin{shiika}傷つけしこと気付かずや青芒
\hfill{\rensuji*{7}・\rensuji*{0}・\rensuji*{0}}\end{shiika}
\vspace{0.6cm}
\begin{shiika}やさしくも棘ある言葉夏薊
\hfill{\rensuji*{7}・\rensuji*{0}・\rensuji*{0}}\end{shiika}
\vspace{0.6cm}
\begin{shiika}夏痩せを知らずに生きて米寿かな
\hfill{\rensuji*{7}・\rensuji*{0}・\rensuji*{0}}\end{shiika}
\vspace{0.6cm}
\begin{shiika}掌中の珠とはこれよ白桃むく
\hfill{\rensuji*{7}・\rensuji*{0}・\rensuji*{0}}\end{shiika}
\vspace{0.6cm}
\begin{shiika}無花果を鳥につつかれ犬叱る
\hfill{\rensuji*{7}・\rensuji*{0}・\rensuji*{0}}\end{shiika}
\vspace{0.6cm}
\begin{shiika}新凉や又取り出して読む佳信
\hfill{\rensuji*{7}・\rensuji*{0}・\rensuji*{0}}\end{shiika}
\vspace{0.6cm}
\begin{shiika}爽やかや返書のペンのよくすべり
\hfill{\rensuji*{7}・\rensuji*{0}・\rensuji*{0}}\end{shiika}
\vspace{0.6cm}
\begin{shiika}鳥わたる返書に三色ボールペン
\hfill{\rensuji*{7}・\rensuji*{0}・\rensuji*{0}}\end{shiika}
\vspace{0.6cm}
\begin{shiika}露けしや二人の友の新佛
\hfill{\rensuji*{7}・\rensuji*{0}・\rensuji*{0}}\end{shiika}
\vspace{0.6cm}
\begin{shiika}コスモスに手をふる急行待避駅
\hfill{\rensuji*{7}・\rensuji*{0}・\rensuji*{0}}\end{shiika}
\vspace{0.6cm}
\begin{shiika}秋夕焼こつくりさんの道標
\hfill{\rensuji*{7}・\rensuji*{0}・\rensuji*{0}}\end{shiika}
\vspace{0.6cm}
\begin{shiika}出ぬ電話そうか今宵は月の句座
\hfill{\rensuji*{7}・\rensuji*{0}・\rensuji*{0}}\end{shiika}
\vspace{0.6cm}
\begin{shiika}家の味継ぎて伝えて祭ずし
\hfill{\rensuji*{7}・\rensuji*{0}・\rensuji*{0}}\end{shiika}
\vspace{0.6cm}
\begin{shiika}貰ふなら遠慮はすまじ秋茄子
\hfill{\rensuji*{7}・\rensuji*{0}・\rensuji*{0}}\end{shiika}
\vspace{0.6cm}
\begin{shiika}栗むくや消えぬ弟の国訛
\hfill{\rensuji*{7}・\rensuji*{0}・\rensuji*{0}}\end{shiika}
\vspace{0.6cm}
\begin{shiika}故郷もつ倖せしかと柿をむく
\hfill{\rensuji*{7}・\rensuji*{0}・\rensuji*{0}}\end{shiika}
\vspace{0.6cm}
\begin{shiika}文化の日遠き明治の今日生れ
\hfill{\rensuji*{7}・\rensuji*{0}・\rensuji*{0}}\end{shiika}
\vspace{0.6cm}
\begin{shiika}透きとおる秋や少年ハーモニカ吹く
\hfill{\rensuji*{7}・\rensuji*{0}・\rensuji*{0}}\end{shiika}
\vspace{0.6cm}
\begin{shiika}鰯雲告げたき人は遠く住み
\hfill{\rensuji*{7}・\rensuji*{0}・\rensuji*{0}}\end{shiika}
\vspace{0.6cm}
\begin{shiika}いま倖障子をよぎる鳥の影
\hfill{\rensuji*{7}・\rensuji*{0}・\rensuji*{0}}\end{shiika}
\vspace{0.6cm}
\begin{shiika}山茶花や豆腐屋を待つ留守居役
\hfill{\rensuji*{7}・\rensuji*{0}・\rensuji*{0}}\end{shiika}
\vspace{0.6cm}
\begin{shiika}冬桜口紅うすくひく米寿 
\hfill{\rensuji*{7}・\rensuji*{0}・\rensuji*{0}}\end{shiika}
\vspace{0.6cm}
\begin{shiika}騙されてをれば事なし枯尾花
\hfill{\rensuji*{7}・\rensuji*{0}・\rensuji*{0}}\end{shiika}
\vspace{0.6cm}
\begin{shiika}梅ケ枝の終の一葉の散る別れ
\hfill{\rensuji*{7}・\rensuji*{0}・\rensuji*{0}}\end{shiika}
\vspace{0.6cm}

平成八年と九年の原本を喪失した。句だけはのこっていたので
allに載せてある。
\endinput
\begin{shiika}いつまでも御元気でねてふ賀状の数 
\hfill{\rensuji*{8}・\rensuji*{0}・\rensuji*{0}}\end{shiika}
\vspace{0.6cm}
\begin{shiika}退職と一筆添へし賀状かな     
\hfill{\rensuji*{8}・\rensuji*{0}・\rensuji*{0}}\end{shiika}
\vspace{0.6cm}
\begin{shiika}初入日三六六の一を呑み
\hfill{\rensuji*{8}・\rensuji*{0}・\rensuji*{0}}\end{shiika}
\vspace{0.6cm}
\begin{shiika}ページくる吾が音寒し影寒し
\hfill{\rensuji*{8}・\rensuji*{0}・\rensuji*{0}}\end{shiika}
\vspace{0.6cm}
\begin{shiika}小豆粥老ひてすこやか姉弟
\hfill{\rensuji*{8}・\rensuji*{0}・\rensuji*{0}}\end{shiika}
\vspace{0.6cm}
\begin{shiika}春寒し言はでききをり二度話    
\hfill{\rensuji*{8}・\rensuji*{0}・\rensuji*{0}}\end{shiika}
\vspace{0.6cm}
\begin{shiika}鳥は雲に二度行くスーパー買いわすれ 
\hfill{\rensuji*{8}・\rensuji*{0}・\rensuji*{0}}\end{shiika}
\vspace{0.6cm}
\begin{shiika}梅二月八・
\hfill{\rensuji*{8}・\rensuji*{0}・\rensuji*{0}}\end{shiika}
\vspace{0.6cm}
\begin{shiika}娘等去にてかろき疲れに窓の梅   
\hfill{\rensuji*{8}・\rensuji*{0}・\rensuji*{0}}\end{shiika}
\vspace{0.6cm}
\begin{shiika}よきことを知らす娘の声梅紅し   
\hfill{\rensuji*{8}・\rensuji*{0}・\rensuji*{0}}\end{shiika}
\vspace{0.6cm}
\begin{shiika}芽吹く庭健かと木々に呼びかけて  
\hfill{\rensuji*{8}・\rensuji*{0}・\rensuji*{0}}\end{shiika}
\vspace{0.6cm}
\begin{shiika}鳥雲に謝しつつ辛き車椅子     
\hfill{\rensuji*{8}・\rensuji*{0}・\rensuji*{0}}\end{shiika}
\vspace{0.6cm}
\begin{shiika}鶯やに車椅子停めくれ息子よ    
\hfill{\rensuji*{8}・\rensuji*{0}・\rensuji*{0}}\end{shiika}
\vspace{0.6cm}
\begin{shiika}とてせめて電話は春の声    
\hfill{\rensuji*{8}・\rensuji*{0}・\rensuji*{0}}\end{shiika}
\vspace{0.6cm}
\begin{shiika}春彼岸弟訪ひくれ仏顔に
\hfill{\rensuji*{8}・\rensuji*{0}・\rensuji*{0}}\end{shiika}
\vspace{0.6cm}
\begin{shiika}岬うらら成果一尾の小半日     
\hfill{\rensuji*{8}・\rensuji*{0}・\rensuji*{0}}\end{shiika}
\vspace{0.6cm}
\begin{shiika}春の夕餉釣りし一尾を母の前    
\hfill{\rensuji*{8}・\rensuji*{0}・\rensuji*{0}}\end{shiika}
\vspace{0.6cm}
\begin{shiika}快気とはかくもうれしき春の朝   
\hfill{\rensuji*{8}・\rensuji*{0}・\rensuji*{0}}\end{shiika}
\vspace{0.6cm}
\begin{shiika}春光やを拝み浴びをり癒え兆    
\hfill{\rensuji*{8}・\rensuji*{0}・\rensuji*{0}}\end{shiika}
\vspace{0.6cm}
\begin{shiika}径端の小さき笑顔犬ふぐり     
\hfill{\rensuji*{8}・\rensuji*{0}・\rensuji*{0}}\end{shiika}
\vspace{0.6cm}
\begin{shiika}鯉のぼりたーかく揚げて待つ帰国
\hfill{\rensuji*{8}・\rensuji*{0}・\rensuji*{0}}\end{shiika}
\vspace{0.6cm}
\begin{shiika}日本を知らぬ児を待つ武者飾り   
\hfill{\rensuji*{8}・\rensuji*{0}・\rensuji*{0}}\end{shiika}
\vspace{0.6cm}
\begin{shiika}薔薇咲かせ迎え明るき指圧院    
\hfill{\rensuji*{8}・\rensuji*{0}・\rensuji*{0}}\end{shiika}
\vspace{0.6cm}
\begin{shiika}土産地蕗香りひろげて国言葉    
\hfill{\rensuji*{8}・\rensuji*{0}・\rensuji*{0}}\end{shiika}
\vspace{0.6cm}
\begin{shiika}木の芽雨偲び草とて届く茶器    
\hfill{\rensuji*{8}・\rensuji*{0}・\rensuji*{0}}\end{shiika}
\vspace{0.6cm}
\begin{shiika}片隅に生きる幸せ額の花      
\hfill{\rensuji*{8}・\rensuji*{0}・\rensuji*{0}}\end{shiika}
\vspace{0.6cm}
\begin{shiika}新茶くみほめ言葉待つ母の顔    
\hfill{\rensuji*{8}・\rensuji*{0}・\rensuji*{0}}\end{shiika}
\vspace{0.6cm}
\begin{shiika}草茂る逆らはぬこと牙につきて   
\hfill{\rensuji*{8}・\rensuji*{0}・\rensuji*{0}}\end{shiika}
\vspace{0.6cm}
\begin{shiika}明易やドイツ転勤ききしより    
\hfill{\rensuji*{8}・\rensuji*{0}・\rensuji*{0}}\end{shiika}
\vspace{0.6cm}
\begin{shiika}泰山木朽ちてすがれる花かなし   
\hfill{\rensuji*{8}・\rensuji*{0}・\rensuji*{0}}\end{shiika}
\vspace{0.6cm}
\begin{shiika}朝涼やからっぽ頭にからっ腹    
\hfill{\rensuji*{8}・\rensuji*{0}・\rensuji*{0}}\end{shiika}
\vspace{0.6cm}
\begin{shiika}いざ昼寝今日はいづこへ夢の旅   
\hfill{\rensuji*{8}・\rensuji*{0}・\rensuji*{0}}\end{shiika}
\vspace{0.6cm}
\begin{shiika}夕涼し肌になじみし藍の服     
\hfill{\rensuji*{8}・\rensuji*{0}・\rensuji*{0}}\end{shiika}
\vspace{0.6cm}
\begin{shiika}暑からむ遅れて浴びる百視線    
\hfill{\rensuji*{8}・\rensuji*{0}・\rensuji*{0}}\end{shiika}
\vspace{0.6cm}
\begin{shiika}端居して出世無縁の長寿眉     
\hfill{\rensuji*{8}・\rensuji*{0}・\rensuji*{0}}\end{shiika}
\vspace{0.6cm}
\begin{shiika}暑に耐えし頬なでてみる今朝の風  
\hfill{\rensuji*{8}・\rensuji*{0}・\rensuji*{0}}\end{shiika}
\vspace{0.6cm}
\begin{shiika}秋暑し訪問販売二度のブザー    
\hfill{\rensuji*{8}・\rensuji*{0}・\rensuji*{0}}\end{shiika}
\vspace{0.6cm}
\begin{shiika}夜々うれし子の友に賜ぶ古梅酒   
\hfill{\rensuji*{8}・\rensuji*{0}・\rensuji*{0}}\end{shiika}
\vspace{0.6cm}
\begin{shiika}花火見に橋へ子が押す車椅子    
\hfill{\rensuji*{8}・\rensuji*{0}・\rensuji*{0}}\end{shiika}
\vspace{0.6cm}
\begin{shiika}癒へてつくる迎え送りの盆団子   
\hfill{\rensuji*{8}・\rensuji*{0}・\rensuji*{0}}\end{shiika}
\vspace{0.6cm}
\begin{shiika}白萩や見知らぬ同志笑みかわし   
\hfill{\rensuji*{8}・\rensuji*{0}・\rensuji*{0}}\end{shiika}
\vspace{0.6cm}
\begin{shiika}寺育ち白曼珠沙華燃え知らず    
\hfill{\rensuji*{8}・\rensuji*{0}・\rensuji*{0}}\end{shiika}
\vspace{0.6cm}
\begin{shiika}風やさしコスモスやさし車椅子   
\hfill{\rensuji*{8}・\rensuji*{0}・\rensuji*{0}}\end{shiika}
\vspace{0.6cm}
\begin{shiika}思はざる花つけにけり秋の草    
\hfill{\rensuji*{8}・\rensuji*{0}・\rensuji*{0}}\end{shiika}
\vspace{0.6cm}
\begin{shiika}秋冷ゆる友の情の京しるこ     
\hfill{\rensuji*{8}・\rensuji*{0}・\rensuji*{0}}\end{shiika}
\vspace{0.6cm}
\begin{shiika}故里や出会ふたれかれ野菊晴    
\hfill{\rensuji*{8}・\rensuji*{0}・\rensuji*{0}}\end{shiika}
\vspace{0.6cm}
\begin{shiika}栗むきつ老ひて姉弟郷言葉     
\hfill{\rensuji*{8}・\rensuji*{0}・\rensuji*{0}}\end{shiika}
\vspace{0.6cm}
\begin{shiika}風のまま吾も白髪穂亡や
\hfill{\rensuji*{8}・\rensuji*{0}・\rensuji*{0}}\end{shiika}
\vspace{0.6cm}
\begin{shiika}花は実に色増す石榴日々親し    
\hfill{\rensuji*{8}・\rensuji*{0}・\rensuji*{0}}\end{shiika}
\vspace{0.6cm}
\begin{shiika}急げともあわてるなとも虫の鳴く  
\hfill{\rensuji*{8}・\rensuji*{0}・\rensuji*{0}}\end{shiika}
\vspace{0.6cm}
\begin{shiika}天高し卒寿見上ぐる明治晴     
\hfill{\rensuji*{8}・\rensuji*{0}・\rensuji*{0}}\end{shiika}
\vspace{0.6cm}
\begin{shiika}秋深き豆煮る母のひとり言
\hfill{\rensuji*{8}・\rensuji*{0}・\rensuji*{0}}\end{shiika}
\vspace{0.6cm}
\begin{shiika}冬に入る病上手に附き合わす   
\hfill{\rensuji*{8}・\rensuji*{0}・\rensuji*{0}}\end{shiika}
\vspace{0.6cm}
\begin{shiika}いつまでも娘は子こたつの母苦言  
\hfill{\rensuji*{8}・\rensuji*{0}・\rensuji*{0}}\end{shiika}
\vspace{0.6cm}
\begin{shiika}よろこびにふとある怖さ夕紅葉   
\hfill{\rensuji*{8}・\rensuji*{0}・\rensuji*{0}}\end{shiika}
\vspace{0.6cm}
\begin{shiika}熟柿つるっと食べばふるさと近く来る
\hfill{\rensuji*{8}・\rensuji*{0}・\rensuji*{0}}\end{shiika}
\vspace{0.6cm}
\begin{shiika}枝桜紅葉に告ぐ別れ
\hfill{\rensuji*{8}・\rensuji*{0}・\rensuji*{0}}\end{shiika}
\vspace{0.6cm}
\begin{shiika}落葉掃きつい長くなる隣同志   
\hfill{\rensuji*{8}・\rensuji*{0}・\rensuji*{0}}\end{shiika}
\vspace{0.6cm}
\begin{shiika}やがてこの娘が孫の嫁冬いちご  
\hfill{\rensuji*{8}・\rensuji*{0}・\rensuji*{0}}\end{shiika}
\vspace{0.6cm}
\begin{shiika}雲を割る冬日や老のねがふこと  
\hfill{\rensuji*{8}・\rensuji*{0}・\rensuji*{0}}\end{shiika}
\vspace{0.6cm}
\begin{shiika}お元旦老母くり返すありがたや  
\hfill{\rensuji*{9}・\rensuji*{0}・\rensuji*{0}}\end{shiika}
\vspace{0.6cm}
\begin{shiika}しわのなき黒豆に老母初お箸
\hfill{\rensuji*{9}・\rensuji*{0}・\rensuji*{0}}\end{shiika}
\vspace{0.6cm}
\begin{shiika}初写真嫁孫の笑み三代
\hfill{\rensuji*{9}・\rensuji*{0}・\rensuji*{0}}\end{shiika}
\vspace{0.6cm}
\begin{shiika}愛犬と話す日日あり寒日和
\hfill{\rensuji*{9}・\rensuji*{0}・\rensuji*{0}}\end{shiika}
\vspace{0.6cm}
\begin{shiika}翔ばたいて大きなおまへ初からす 
\hfill{\rensuji*{9}・\rensuji*{0}・\rensuji*{0}}\end{shiika}
\vspace{0.6cm}
\begin{shiika}五・
\hfill{\rensuji*{9}・\rensuji*{0}・\rensuji*{0}}\end{shiika}
\vspace{0.6cm}
\begin{shiika}孫嫁のもうすぐ二人梅紅し
\hfill{\rensuji*{9}・\rensuji*{0}・\rensuji*{0}}\end{shiika}
\vspace{0.6cm}
\begin{shiika}お化粧で他人顔なり春写真    
\hfill{\rensuji*{9}・\rensuji*{0}・\rensuji*{0}}\end{shiika}
\vspace{0.6cm}
\begin{shiika}春障子四畳半の城明るし
\hfill{\rensuji*{9}・\rensuji*{0}・\rensuji*{0}}\end{shiika}
\vspace{0.6cm}
\begin{shiika}下萌に煎餅分ける愛犬に
\hfill{\rensuji*{9}・\rensuji*{0}・\rensuji*{0}}\end{shiika}
\vspace{0.6cm}
\begin{shiika}春耕をまぶしく見をりホーム窓
\hfill{\rensuji*{9}・\rensuji*{0}・\rensuji*{0}}\end{shiika}
\vspace{0.6cm}
\begin{shiika}啓窒やシルバーホームの預け解け
\hfill{\rensuji*{9}・\rensuji*{0}・\rensuji*{0}}\end{shiika}
\vspace{0.6cm}
\begin{shiika}春暁の正夢なれや初ひ孫
\hfill{\rensuji*{9}・\rensuji*{0}・\rensuji*{0}}\end{shiika}
\vspace{0.6cm}
\begin{shiika}向ひ合うパ・
\hfill{\rensuji*{9}・\rensuji*{0}・\rensuji*{0}}\end{shiika}
\vspace{0.6cm}
\begin{shiika}おばさんと呼びくれ三人桜餅
\hfill{\rensuji*{9}・\rensuji*{0}・\rensuji*{0}}\end{shiika}
\vspace{0.6cm}
\begin{shiika}浮雲に名付けあそびや春の風
\hfill{\rensuji*{9}・\rensuji*{0}・\rensuji*{0}}\end{shiika}
\vspace{0.6cm}
\begin{shiika}こちら向くラッパ水仙こんにちは
\hfill{\rensuji*{9}・\rensuji*{0}・\rensuji*{0}}\end{shiika}
\vspace{0.6cm}
\begin{shiika}花衣車椅子にも湧くはずみ
\hfill{\rensuji*{9}・\rensuji*{0}・\rensuji*{0}}\end{shiika}
\vspace{0.6cm}
\begin{shiika}思い桜樹齢二百を恋う卒寿
\hfill{\rensuji*{9}・\rensuji*{0}・\rensuji*{0}}\end{shiika}
\vspace{0.6cm}
\begin{shiika}花の雨ワインケーキの香に和む
\hfill{\rensuji*{9}・\rensuji*{0}・\rensuji*{0}}\end{shiika}
\vspace{0.6cm}
\begin{shiika}初咲きの大勺や句や婚の朝    
\hfill{\rensuji*{9}・\rensuji*{0}・\rensuji*{0}}\end{shiika}
\vspace{0.6cm}
\begin{shiika}桜湯のぱーつとひらけり控室
\hfill{\rensuji*{9}・\rensuji*{0}・\rensuji*{0}}\end{shiika}
\vspace{0.6cm}
\begin{shiika}純白の花嫁孫となる五月
\hfill{\rensuji*{9}・\rensuji*{0}・\rensuji*{0}}\end{shiika}
\vspace{0.6cm}
\begin{shiika}柿若葉秘仏開扉めぐり会い
\hfill{\rensuji*{9}・\rensuji*{0}・\rensuji*{0}}\end{shiika}
\vspace{0.6cm}
\begin{shiika}来し道の険しさ言はず余花仰ぐ
\hfill{\rensuji*{9}・\rensuji*{0}・\rensuji*{0}}\end{shiika}
\vspace{0.6cm}
\begin{shiika}御幣上る薫風にのる上棟歌
\hfill{\rensuji*{9}・\rensuji*{0}・\rensuji*{0}}\end{shiika}
\vspace{0.6cm}
\begin{shiika}目つむりて青汁ぐっとばら真紅
\hfill{\rensuji*{9}・\rensuji*{0}・\rensuji*{0}}\end{shiika}
\vspace{0.6cm}
\begin{shiika}痛いとは生ける証しか梅雨の膝
\hfill{\rensuji*{9}・\rensuji*{0}・\rensuji*{0}}\end{shiika}
\vspace{0.6cm}
\begin{shiika}梅雨鏡拭けば亡母にとれほどに
\hfill{\rensuji*{9}・\rensuji*{0}・\rensuji*{0}}\end{shiika}
\vspace{0.6cm}
\begin{shiika}都忘れ咲かせ老いけり京遠く
\hfill{\rensuji*{9}・\rensuji*{0}・\rensuji*{0}}\end{shiika}
\vspace{0.6cm}
\begin{shiika}今年また梅酒たまわる命かな
\hfill{\rensuji*{9}・\rensuji*{0}・\rensuji*{0}}\end{shiika}
\vspace{0.6cm}
\begin{shiika}子つばめの翔つを見送る車椅子
\hfill{\rensuji*{9}・\rensuji*{0}・\rensuji*{0}}\end{shiika}
\vspace{0.6cm}
\begin{shiika}ナイターに興じる老母の片辺して
\hfill{\rensuji*{9}・\rensuji*{0}・\rensuji*{0}}\end{shiika}
\vspace{0.6cm}
\begin{shiika}白髪といていのちあるもの髪洗ふ
\hfill{\rensuji*{9}・\rensuji*{0}・\rensuji*{0}}\end{shiika}
\vspace{0.6cm}
\begin{shiika}ぎょうさんな娘の悲鳴蜘蛛の糸
\hfill{\rensuji*{9}・\rensuji*{0}・\rensuji*{0}}\end{shiika}
\vspace{0.6cm}
\begin{shiika}郷ばなしつきずやさしき団扇かぜ
\hfill{\rensuji*{9}・\rensuji*{0}・\rensuji*{0}}\end{shiika}
\vspace{0.6cm}
\begin{shiika}夏服の派手を鏡に息子の土産
\hfill{\rensuji*{9}・\rensuji*{0}・\rensuji*{0}}\end{shiika}
\vspace{0.6cm}
\begin{shiika}きれし夢惜しや貴船のはも料理
\hfill{\rensuji*{9}・\rensuji*{0}・\rensuji*{0}}\end{shiika}
\vspace{0.6cm}
\begin{shiika}迎はるる仏とならで魂迎ふ
\hfill{\rensuji*{9}・\rensuji*{0}・\rensuji*{0}}\end{shiika}
\vspace{0.6cm}
\begin{shiika}仏めく盆僧の額黒光り
\hfill{\rensuji*{9}・\rensuji*{0}・\rensuji*{0}}\end{shiika}
\vspace{0.6cm}
\begin{shiika}赤とんぼヘルパーと唄う車椅子
\hfill{\rensuji*{9}・\rensuji*{0}・\rensuji*{0}}\end{shiika}
\vspace{0.6cm}
\begin{shiika}星月夜シルバーホーム消灯はやき
\hfill{\rensuji*{9}・\rensuji*{0}・\rensuji*{0}}\end{shiika}
\vspace{0.6cm}
\begin{shiika}誰似かと爽やかろんぎ初曽孫
\hfill{\rensuji*{9}・\rensuji*{0}・\rensuji*{0}}\end{shiika}
\vspace{0.6cm}
\begin{shiika}白桔梗時には欲しい母小言
\hfill{\rensuji*{9}・\rensuji*{0}・\rensuji*{0}}\end{shiika}
\vspace{0.6cm}
\begin{shiika}おきし手を又も引きよす枝豆を
\hfill{\rensuji*{9}・\rensuji*{0}・\rensuji*{0}}\end{shiika}
--------------------------------------
\vspace{0.6cm}
\begin{shiika}野仏の笑ひ在せり曼珠沙華
\hfill{\rensuji*{48}・\rensuji*{0}・\rensuji*{0}}\end{shiika}
\vspace{0.6cm}
\begin{shiika}日を浴びてままごとの子や草紅葉
\hfill{\rensuji*{48}・\rensuji*{0}・\rensuji*{0}}\end{shiika}
\vspace{0.6cm}
\begin{shiika}顔見世の名残を夢に見しも去年
\hfill{\rensuji*{48}・\rensuji*{0}・\rensuji*{0}}\end{shiika}
\vspace{0.6cm}
\begin{shiika}髪結ひて寝ず娘は待つ初詣
\hfill{\rensuji*{49}・\rensuji*{0}・\rensuji*{0}}\end{shiika}
\vspace{0.6cm}
\begin{shiika}猫の恋根笹の乱れ昨日今日
\hfill{\rensuji*{49}・\rensuji*{0}・\rensuji*{0}}\end{shiika}
\vspace{0.6cm}
\begin{shiika}山の色幾重の果の雪解光
\hfill{\rensuji*{49}・\rensuji*{0}・\rensuji*{0}}\end{shiika}
\vspace{0.6cm}
\begin{shiika}陵の薄陽の濠も水草生ふ
\hfill{\rensuji*{49}・\rensuji*{0}・\rensuji*{0}}\end{shiika}
\vspace{0.6cm}
\begin{shiika}娘の縁談又もこわれぬ春の雪
\hfill{\rensuji*{49}・\rensuji*{0}・\rensuji*{0}}\end{shiika}
\vspace{0.6cm}
\begin{shiika}花過ぎぬいづこともなき旅心
\hfill{\rensuji*{49}・\rensuji*{0}・\rensuji*{0}}\end{shiika}
\vspace{0.6cm}
\begin{shiika}山裾の雨に煙れる桐の花
\hfill{\rensuji*{49}・\rensuji*{0}・\rensuji*{0}}\end{shiika}
\vspace{0.6cm}
\begin{shiika}夜神東の明りに映ゆる銀杏黄葉
\hfill{\rensuji*{49}・\rensuji*{0}・\rensuji*{0}}\end{shiika}
\vspace{0.6cm}
\begin{shiika}野仏の顔かくすまで草の花
\hfill{\rensuji*{49}・\rensuji*{0}・\rensuji*{0}}\end{shiika}
\vspace{0.6cm}
\begin{shiika}置炬燵向ふ人なきあで蒲団
\hfill{\rensuji*{49}・\rensuji*{0}・\rensuji*{0}}\end{shiika}
\vspace{0.6cm}
\begin{shiika}年用意丹波男の荷は売れ早き
\hfill{\rensuji*{49}・\rensuji*{0}・\rensuji*{0}}\end{shiika}
\vspace{0.6cm}
\begin{shiika}友待つに暮色刻々粉雪舞ふ
\hfill{\rensuji*{50}・\rensuji*{0}・\rensuji*{0}}\end{shiika}
\vspace{0.6cm}
\begin{shiika}風ぬくき末黒野烏群をなし
\hfill{\rensuji*{50}・\rensuji*{0}・\rensuji*{0}}\end{shiika}
\vspace{0.6cm}
\begin{shiika}化粧水掌に冷えのなし春隣
\hfill{\rensuji*{50}・\rensuji*{0}・\rensuji*{0}}\end{shiika}
\vspace{0.6cm}
\begin{shiika}綿菓子も売れて野崎の花曇
\hfill{\rensuji*{50}・\rensuji*{0}・\rensuji*{0}}\end{shiika}
\vspace{0.6cm}
\begin{shiika}花曇年甲斐もなき物忘れ
\hfill{\rensuji*{50}・\rensuji*{0}・\rensuji*{0}}\end{shiika}
\vspace{0.6cm}
\begin{shiika}若やぎて夏来る歌口ずさむ
\hfill{\rensuji*{50}・\rensuji*{0}・\rensuji*{0}}\end{shiika}
\vspace{0.6cm}
\begin{shiika}梅雨曇出入せはしき軒雀
\hfill{\rensuji*{50}・\rensuji*{0}・\rensuji*{0}}\end{shiika}
\vspace{0.6cm}
\begin{shiika}花葵露地の家々箱咲きに
\hfill{\rensuji*{50}・\rensuji*{0}・\rensuji*{0}}\end{shiika}
\vspace{0.6cm}
\begin{shiika}あらはなるちくり根洗ひ大夕立
\hfill{\rensuji*{50}・\rensuji*{0}・\rensuji*{0}}\end{shiika}
\vspace{0.6cm}
\begin{shiika}看る夜の心もとなき星の飛ぶ
\hfill{\rensuji*{50}・\rensuji*{0}・\rensuji*{0}}\end{shiika}
\vspace{0.6cm}
\begin{shiika}子等去りぬ礎石にならぶ蝉の殻
\hfill{\rensuji*{50}・\rensuji*{0}・\rensuji*{0}}\end{shiika}
\vspace{0.6cm}
\begin{shiika}大月夜唐招提寺の庭に彳つ
\hfill{\rensuji*{50}・\rensuji*{0}・\rensuji*{0}}\end{shiika}
\vspace{0.6cm}
\begin{shiika}色鳥や朝の湖の小桟橋
\hfill{\rensuji*{50}・\rensuji*{0}・\rensuji*{0}}\end{shiika}
\vspace{0.6cm}
\begin{shiika}秋惜しむほほ紅少こしさしてみむ
\hfill{\rensuji*{50}・\rensuji*{0}・\rensuji*{0}}\end{shiika}
\vspace{0.6cm}
\begin{shiika}新鮮と我から言ひて冬菜売
\hfill{\rensuji*{50}・\rensuji*{0}・\rensuji*{0}}\end{shiika}
\vspace{0.6cm}
\begin{shiika}独り居の朝茶の香り笹に来る
\hfill{\rensuji*{51}・\rensuji*{0}・\rensuji*{0}}\end{shiika}
\vspace{0.6cm}
\begin{shiika}家長の座に心しまりて大福茶
\hfill{\rensuji*{51}・\rensuji*{0}・\rensuji*{0}}\end{shiika}
\vspace{0.6cm}
\begin{shiika}新らしき命を呼びて野火勢ふ
\hfill{\rensuji*{51}・\rensuji*{0}・\rensuji*{0}}\end{shiika}
\vspace{0.6cm}
\begin{shiika}春泥の径つき寺の小門あり
\hfill{\rensuji*{51}・\rensuji*{0}・\rensuji*{0}}\end{shiika}
\vspace{0.6cm}
\begin{shiika}黄帽子水筒どの児の靴も春の泥
\hfill{\rensuji*{51}・\rensuji*{0}・\rensuji*{0}}\end{shiika}
\vspace{0.6cm}
\begin{shiika}花の奥雨に煙れる塔のあり
\hfill{\rensuji*{51}・\rensuji*{0}・\rensuji*{0}}\end{shiika}
\vspace{0.6cm}
\begin{shiika}老鶯や御手の茶壺のかたむける
\hfill{\rensuji*{51}・\rensuji*{0}・\rensuji*{0}}\end{shiika}
\vspace{0.6cm}
\begin{shiika}老鴬に唐松林行きにゆく
\hfill{\rensuji*{51}・\rensuji*{0}・\rensuji*{0}}\end{shiika}
\vspace{0.6cm}
\begin{shiika}湖見ゆる古戦場道落し文
\hfill{\rensuji*{51}・\rensuji*{0}・\rensuji*{0}}\end{shiika}
\vspace{0.6cm}
\begin{shiika}病妹の欲りし日とあり梨供ふ
\hfill{\rensuji*{51}・\rensuji*{0}・\rensuji*{0}}\end{shiika}
\vspace{0.6cm}
\begin{shiika}鐘楼に屋根草のびて露ふかし
\hfill{\rensuji*{51}・\rensuji*{0}・\rensuji*{0}}\end{shiika}
\vspace{0.6cm}
\begin{shiika}四つ手網死魚の乾けり秋の声
\hfill{\rensuji*{51}・\rensuji*{0}・\rensuji*{0}}\end{shiika}
\vspace{0.6cm}
\begin{shiika}晩菊のうつろいはじむ白きより
\hfill{\rensuji*{51}・\rensuji*{0}・\rensuji*{0}}\end{shiika}
\vspace{0.6cm}
\begin{shiika}晩菊やなほ美くしき謡の師
\hfill{\rensuji*{51}・\rensuji*{0}・\rensuji*{0}}\end{shiika}
\vspace{0.6cm}
\begin{shiika}秋冷ゆる赤きストビラ散る舗道
\hfill{\rensuji*{51}・\rensuji*{0}・\rensuji*{0}}\end{shiika}
\vspace{0.6cm}
\begin{shiika}綿虫の籬越え来て雨を呼ぶ
\hfill{\rensuji*{51}・\rensuji*{0}・\rensuji*{0}}\end{shiika}