\fbox{編者のコメント}

母ふみ子は昭和二十五年から、阪急京都線 相川駅前で文房具
の店を始めた。その後雑誌 書籍も扱うようになった。二十年頑張ったころは
店員に任せて旅行できる余裕ができた。

旅行は、高松女学校のクラスメート、京都女専のクラスメート、文具商の
組合からの誘いだった。

寝起きは 相川北通りの家で 家の半分は貸していた。

昭和十九年に長柄から強制疎開で 相川に来た当時は、母。姉三人、私 そして 
居候が三人、女中さんの大所帯だったが、姉達はかたづき、私は東京に就職
で、母は一人暮らしになった。

私の東京での就職に関しては、母は行動範囲が増えるといって、賛成してくれた。
昭和五十年頃は 私はソフトウエア会社に勤めて、妻と子供二人で、世田谷の
マンション暮らしだった。
