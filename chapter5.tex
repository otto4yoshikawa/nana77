\begin{shiika}端居して出世無縁の長寿眉 \hfill{199607} \end{shiika}   

この句は四国の故郷で読む
故郷は香川県高松市国分 で、従弟の村上勝美宅を宿としていた。
\\そこで村上勝美氏の眉を読んだ句。京鹿子の特選賞となり、
数ページの誉め言葉があった。\\端居の季語は夏である。

\vspace{5mm}
\begin{shiika}初入日三六六の一を呑み 199601 \end{shiika} 

三六六は閏年からくる。1996年は閏年だった。ひねった句。
\vspace{5mm}
\begin{shiika}朧夜や骨までしゃぶる瀬戸の味 19930400\end{shiika} 

四国高松で従弟の村上久夫さんに 鯛の兜煮 をご馳走
になった。
\\骨までしゃぶる は京鹿子の海道主宰から
手紙で「骨までしゃぶる 全く感心いたしました 故郷は
よいもの 良いところ。故郷のあるものは倖せですね と
\vspace{5mm}
\begin{shiika}啓窒やシルバーホームの預け解け 1997/03\end{shiika}

1997年2月に。私と喜美子と清子さんの3人で 
ドイツ ヂュッセルドルフの郷生のマンションに10日間泊った。
その間 母を湘南台の老人ホームに預けた。その帰国が
丁度3月上旬だったので。
\vspace{5mm}

\begin{shiika}春暁の正夢なれや初ひ孫 1997/03\end{shiika}

清子さんが千里を懐妊したとの知らせをめでて。
