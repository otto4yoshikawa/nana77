\begin{multicols}{2}
a
\\野仏の笑ひ在せり曼珠沙華\hfill{19730900}
\\日を浴びてままごとの子や草紅葉\hfill{19731000}
\\顔見世の名残を夢に見しも去年\hfill{19731200}
\\髪結ひて寝ず娘は待つ初詣\hfill{19740100}
\\猫の恋根笹の乱れ昨日今日\hfill{19740200}
\\山の色幾重の果の雪解光\hfill{19740200}
\\陵の薄陽の濠も水草生ふ\hfill{19740300}
\\娘の縁談又もこわれぬ春の雪\hfill{19740300}
\\花過ぎぬいづこともなき旅心\hfill{19740400}
\\山裾の雨に煙れる桐の花\hfill{19740500}
\\夜神東の明りに映ゆる銀杏黄葉\hfill{19741100}
\\野仏の顔かくすまで草の花\hfill{19740900}
\\置炬燵向ふ人なきあで蒲団\hfill{19741100}
\\年用意丹波男の荷は売れ早き\hfill{19741200}
\\友待つに暮色刻々粉雪舞ふ\hfill{19750100}
\\風ぬくき末黒野烏群をなし\hfill{19750200}
\\化粧水掌に冷えのなし春隣\hfill{19750300}
\\綿菓子も売れて野崎の花曇\hfill{19750400}
\\花曇年甲斐もなき物忘れ\hfill{19750400}
\\若やぎて夏来る歌口ずさむ\hfill{19750500}
\\梅雨曇出入せはしき軒雀\hfill{19750600}
\\花葵露地の家々箱咲きに\hfill{19750600}
\\あらはなるちくり根洗ひ大夕立\hfill{19750700}
\\看る夜の心もとなき星の飛ぶ\hfill{19750826}
\\子等去りぬ礎石にならぶ蝉の殻\hfill{19750800}
\\大月夜唐招提寺の庭に彳つ\hfill{197508}
\\色鳥や朝の湖の小桟橋\hfill{19751000}
\\秋惜しむほほ紅少こしさしてみむ\hfill{19751000}
\\新鮮と我から言ひて冬菜売\hfill{19751200}
\\独り居の朝茶の香り笹に来る\hfill{19760100}
\\家長の座に心しまりて大福茶\hfill{19760100}
\\新らしき命を呼びて野火勢ふ\hfill{19760200}
\\春泥の径つき寺の小門あり\hfill{19760300}
\\黄帽子水筒どの児の靴も春の泥\hfill{19760300}
\\花の奥雨に煙れる塔のあり\hfill{19760400}
\\老鶯や御手の茶壺のかたむける\hfill{19760517}
\\老鴬に唐松林行きにゆく\hfill{19760516}
\\湖見ゆる古戦場道落し文\hfill{19760700}
\\病妹の欲りし日とあり梨供ふ\hfill{19760900}
\\鐘楼に屋根草のびて露ふかし\hfill{19761017}
\\四つ手網死魚の乾けり秋の声\hfill{19761017}
\\晩菊のうつろいはじむ白きより\hfill{19761100}
\\晩菊やなほ美くしき謡の師\hfill{19761100}
\\秋冷ゆる赤きストビラ散る舗道\hfill{19761100}
\\綿虫の籬越え来て雨を呼ぶ\hfill{19761100}
\\蛤の潮のしたたり出船待つ\hfill{19770305}
\\河原なる飛球の行方風光る\hfill{19770300}
\\吉野山春蘭の店は客呼ばず\hfill{19770405}
\\花弁ゆれ奥より出でし虻の貌\hfill{19770400}
\\燕の子黄ならびの嘴花のごと\hfill{19770500}
\\木苺や山の佛の唇あせて\hfill{19770625}
\\寝冷え子のうつろの瞳絵本散る\hfill{19770700}
\\蜜豆に唇さみし嘘を言ふ\hfill{19770700}
\\湖の色北より深み秋きざす\hfill{19770800}
\\竹生島真向ふ宿の洗鯉\hfill{19770800}
\\登るほど尾花は細し高野道\hfill{19770900}
\\行けど行けど穂芒波や夕茜\hfill{19770900}
\\天高し隠岐の草原牛肥えて\hfill{19770900}
\\霊場の鐘にも和さずけらつつき\hfill{19771000}
\\下枝より褪せて小庭の実むらさき\hfill{19771000}
\\庭雀床払ひせしふとん干す\hfill{19771200}
\\白寿祝ぐ願いをこめて羽根蒲団\hfill{19771200}
\\若水や心新らたに栓開く\hfill{19780100}
\\句友の訃夜を沈丁の香のせまり\hfill{19780300}
\\春潮に群れ飛ぶかもめ水尾追ひて\hfill{19780300}
\\門かたく喪の家ひそと花ゆすら\hfill{19780400}
\\潮騒の丘の花冷学徒眠る\hfill{19780300}
\\城跡の古井戸涸れず苔の花\hfill{19780605}
\\桑の実に郷愁ありて札所径\hfill{19780600}
\\焼香待つ黒幕裾の蟻地獄\hfill{19780700}
\\葉鶏頭一筋町の故郷晴れ\hfill{19781000}
\\結願の杖納め得し鵙日和\hfill{19781000}
\\花売の残す菊の香路地の朝\hfill{19781200}
\\口ませし孫の電話や冬すみれ\hfill{19781200}
\\曼珠沙華島の陵人稀に\hfill{19780900}
\\出張のしげかれ疾かれ牡蠣土産\hfill{19781000}
\\寄れば逃ぐ子に獅子舞の昂りて\hfill{19781000}
\\寒餅を切る夜のまど  とろり\hfill{19781000}
\\旅立ちの鏡に向ふ夏帽子\hfill{19781000}
\\久々の子に浴衣着せ今宵酌む\hfill{19781000}
\\草の花名を問ひ問はれ三輪の径\hfill{19781000}
\\三代が屠蘇なみなみと三つの盃\hfill{19790100}
\\冬萠や繃帯の足歩を試す\hfill{19790100}
\\昂りぬ沈丁の雨音もなく\hfill{19790300}
\\啓執や旅誘ひの友便り\hfill{19790300}
\\花の下城址碑ひそと休暇村\hfill{19790420}
\\山の温泉は音なく春蚊早出でし\hfill{19790420}
\\草餅に門前町の賑へる\hfill{19790600}
\\実生栗初花咲けり吾も健\hfill{19790600}
\\冷奴遠き旅より帰り酌む\hfill{19790600}
\\落ちるまま実梅の匂ひ城のみち\hfill{19790716}
\\城の灯のうるみ郡上の踊更く\hfill{19790823}
\\新秋や欄間彫る町木の香り\hfill{19790824}
\\谷底は見えずバス行く山の霧\hfill{19790824}
\\高原の駅コスモスの色極め\hfill{19790824}
\\結願の梵鐘ひびく峯の秋\hfill{19791200}
\\太りゆく大根今日も抜き惜しみ\hfill{19791200}
\\実むらさき実生をたのむ土かぶせ\hfill{19791200}
\\青木の実名知らぬ鳥も枝くぐり\hfill{19791200}
\\心地よき帯のしまりや謡ひ初め\hfill{19800100}
\\新年の交す汽笛に群れ鴎\hfill{19800101}
\\通夜の冷え遺作のばら絵明るきも\hfill{19800000}
\\出棺す白梅こぼる砂踏みて\hfill{19800000}
\\雨戸くる朝なあさなを蕗育つ\hfill{19800400}
\\菜園の菊菜色よし久の子に\hfill{19800400}
\\青葉して忌ごもる友と病める友\hfill{19800500}
\\明易し潮騒近き島の宿\hfill{19800531}
\\島の雷止みて翼船ましぐら\hfill{19800601}
\\梅雨嵐し離れ病む子をただ祈る\hfill{19800600}
\\見送られ見返る薄暮白あやめ\hfill{19800600}
\\健やかな孫の寝息やプール焼け\hfill{19800800}
\\草引きて草の匂ひの手枕寝\hfill{19800800}
\\水引の紅ぬれづめに水車\hfill{19800900}
\\みのり田の道登校のペダル踏む\hfill{19800900}
\\温泉涼し重き一事を成しとげて\hfill{19800900}
\\退院の友いきいきと派手浴衣\hfill{19800717}
\\ダム澄める揺れ映りいる合歓の花\hfill{19800802}
\\露天湯の一灯淡く月見草\hfill{19800803}
\\霊峰の碧に真向ひ秋ざくら\hfill{19800804}
\\先急ぎつつ仰ぎゆく峯紅葉\hfill{19801102}
\\しみじみと語らな白菊活けて待つ\hfill{19801102}
\\遠き旅はなやぎ帰り菊を焚く\hfill{19801102}
\\枯菊を焚きつつしばし物思ひ\hfill{19801102}
\\鉄橋を渡れば小駅片時雨\hfill{19801200}
\\黄の翅の止り色増す実むらさき\hfill{19801100}
\\天高し施肥よく効きし畑の色\hfill{19801100}
\\七草の数揃はねど畑の菜を\hfill{19810100}
\\一望に漁港おさめて梅の丘\hfill{19810130}
\\春炬燵尽きぬ話の果は伏し\hfill{19810300}
\\春の冷え別れて一人立つ小駅\hfill{19810399}
\\争ひてふと空しかり梅の闇\hfill{19810300}
\\合格の祝袋は字も太く\hfill{19810300}
\\摘みし蕗独りの厨たのしかり\hfill{19810400}
\\散る桜庭の胸像ただ黙し\hfill{19810400}
\\武具飾る子は父となり遠くあり\hfill{19810500}
\\解禁の夕べたまはる吉野鮎\hfill{19810500}
\\釣りし鮒川に戻して春の風\hfill{19810400}
\\冨士聳ゆ裾野の町の鯉のぼり\hfill{19810500}
\\滝水をコップに汲みて喉しまる\hfill{19810700}
\\御詠歌の流れへいそぐ地蔵盆\hfill{19810800}
\\枝豆に酌みて不意なる遠き客\hfill{19810900}
\\釣る夫の片辺に妻の秋日傘\hfill{19811000}
\\武家屋敷崩れ土塀に石蕗盛り\hfill{19811022}
\\草子里時雨れる朝の大き虹\hfill{19811024}
\\わだかまり解けて減りゆく盛みかん\hfill{19811000}
\\噂消え火事場に茂る泡立草\hfill{19811100}
\\売地札草にかくれて秋暮るる\hfill{19811100}
\\栗おこわ我が誕生は頃もよく\hfill{19811100}
\\霜よけにレタス生々玉巻ける\hfill{19811100}
\\供華の菊剪りためらひぬ眠り蝶\hfill{19811100}
\\落葉炊く煙の中に思ふこと \hfill{19811100}
\\新らしく菊きり供え旅に出る \hfill{19811100}
\\踏み惜しみつつ鎌倉の銀杏黄葉\hfill{19811124}
\\ウインドに背まるく映る師走町\hfill{19811200}
\\晦日そば孫の食べざま頼もしく\hfill{19811200}
\\窓の梅ほころびゆくをみるしじま\hfill{19820200}
\\散り梅のかかり濯ぎのもの乾く\hfill{19820200}
\\春遠しこもれる叔母に京の菓子\hfill{19820200}
\\受験生泊めて祈りを同心に\hfill{19820300}
\\日脚伸ぶ中洲に群れる鳥の白\hfill{19820300}
\\蕗の薹焼みその香の朝厨\hfill{19820300}
\\散る花の流れゆくあり踏まるあり\hfill{19820407}
\\天主より振る手呼ぶ声花の中\hfill{19820400}
\\葱坊主垣越しの子はよくしゃべる\hfill{19820500}
\\耳遠く笑顔で応ふ木の芽雨\hfill{19820500}
\\草餅にふと道変へて娘に急ぐ\hfill{19820500}
\\直ぐ消ゆる足跡砂に五月旅\hfill{19820511}
\\風光る砂丘を踏めば若返る\hfill{19820511}
\\石段のあえぎに著莪の花やさし\hfill{19820512}
\\単線の停車は長し青田風\hfill{19820600}
\\花栗の香に堂守の鍵開く\hfill{19820700}
\\老鴬や堂守力こめて説く\hfill{19820629}
\\知床の大雪渓に昼の月\hfill{19820629}
\\雪渓を映し知床五湖寂と \hfill{19820629}
\\えぞかんぞう岬はるかは異国なる\hfill{19820629}
\\昆布乾すさいはての島明易し \hfill{19820629}
\\獅子独活の花眼の限り能取岬\hfill{19820706}
\\鷺草の鷺二羽となる娘に甘え\hfill{19820706}
\\魂迎ふ一人となりて古家守る\hfill{19820800}
\\手ごなしで土をかぶせる秋の種\hfill{19820629}
\\豪雷にいさかふ妹弟抱き合ふ\hfill{19820629}
\\亡娘ノート紙魚生きている悲しさよ\hfill{19820629}
\\秋立ちぬ束ねてさせり亡母の櫛\hfill{19820629}
\\晩菊の咲くや明日より他人の庭\hfill{19821000}
\\引き越しの荷隅にかばふ冬すみれ\hfill{19821000}
\\秋そゞろ引越荷物嵩む部屋\hfill{19821000}
\\秋風も他人もやさし移り住み\hfill{19821100}
\\見捨てかね新居に挿せり倒れ菊\hfill{19821100}
\\寛ぎて見る山荘の紅葉濃し\hfill{19821100}
\\乗りおくれくやしき顔に冬の月\hfill{19821100}
\\寒椿にぶる起ち居のすべもなく\hfill{19821200}
\\友呼ばむ一人に余る日向ぼこ\hfill{19821200}
\\転宅の迫りし庭の実むらさき\hfill{19821000}
\\移り住む名残の菊香衰えず\hfill{19821000}
\\玉砂利に歩の乱れなし神の留守\hfill{19821000}
\\大役の初旅冨士が雲間より\hfill{19830103}
\\梅日和白壁光る村一望\hfill{19830200}
\\しつけとる春立つ朝の装ひに\hfill{19830300}
\\水ぬるむ就職決り紅さす娘\hfill{19830300}
\\桜餅娘の訪ひくれし小半日\hfill{19830300}
\\目口なき紙の雛や掌になじむ\hfill{19830300}
\\裏の家の雨に堪へ咲く八重桜\hfill{19830400}
\\友の情雨に摘みきしわらび飯\hfill{19830400}
\\忌に集るしのぶ日がなを花の雨\hfill{19830400}
\\楠公通の大楠学校庭に移し植え\hfill{19830400}
\\除り去らる囀り包む街の樹が\hfill{19830400}
\\読むも憂し眺むも憂しや花の雨\hfill{19830400}
\\集ればお国訛よよもぎ餅\hfill{19830400}
\\秩父路につづく芽桑の夕映えて\hfill{19830407}
\\万緑や一言神に願一つ\hfill{19830521}
\\田植機の若者帽子に赤い花\hfill{19830521}
\\桜桃たわわの国へ喜寿の旅\hfill{19830611}
\\杖たよる友出迎へに梅雨はげし\hfill{19830700}
\\朝涼し咲きつぐ花を供華日記\hfill{19830700}
\\引き越して来たる浜木綿咲き安堵\hfill{19830700}
\\娘三人訪ひくれ風鈴よく鳴れり\hfill{19830700}
\\一族の年長となり魂まつる\hfill{19830800}
\\動かぬ灯動く灯一望盆の果\hfill{19830800}
\\洗ひ髪立つベランダの風は秋\hfill{19830800}
\\蕎麦三日食べてさわやか信濃旅\hfill{19830904}
\\色鳥や岳に真向ふ湖の宿\hfill{19830900}
\\大き鳥湖上を舞ひて夏去れり\hfill{19830900}
\\庭紅葉もえて謡に力声\hfill{19831100}
\\謡ひ果て山荘黄葉をのこし暮る\hfill{19831100}
\\翅やすむ蝶もむらさき式部の実\hfill{19831100}
\\独り居のよき日淋し日菊挿して\hfill{19831100}
\\疎く住み安けき日々や杜鵤草\hfill{19831100}
\\屑金魚育ち掬ひし児も少年\hfill{19831100}
\\案内三日京の紅葉に酔ひ疲る\hfill{19831100}
\\照紅葉京一望の峯の寺\hfill{19831100}
\\山荘の集ひに菜飯冬ぬくし\hfill{19831207}
\\冬入日竹叢透し荘なごむ\hfill{19831207}
\\一とせを会ひ得ぬ人の賀状増し\hfill{19840100}
\\しきたりをつづけて独り屠蘇機嫌\hfill{19840100}
\\トンネルを抜ける度雪深くなり\hfill{19840102}
\\ただいまと灯せば応ふ室の花\hfill{19840200}
\\ちゃん呼びで遠き日戻る木の葉髪\hfill{19840200}
\\春寒やぱったり出会ひ出ぬ名前\hfill{19840200}
\\争ひも夢よ首塚土筆の芽\hfill{19840300}
\\老夫婦夜をぼつぼつとひなあられ\hfill{19840303}
\\雪解風由布岳さして大鴉\hfill{19840305}
\\土を割る花芽それぞれ色ありて\hfill{19840300}
\\によきによきと花芽ラッシュの庭の土\hfill{19840300}
\\花苺児にしやがみ見す芯の粒\hfill{19840400}
\\朝毎の独りに足りる庭苺\hfill{19840500}
\\団地住みテレビの上の兜の威\hfill{19840500}
\\ホース先そらせばそこも青蛙\hfill{19840700}
\\花南天隣初嬰の襁褓干す\hfill{19840700}
\\待ちつつも一人を凉しと思ふ日も\hfill{19840800}
\\庭茂り払ふ枝にもある生命\hfill{19840800}
\\孫の名をとりちがえ呼ぶ盆家族\hfill{19840800}
\\夏萩に誰みくじ結ふ禁よそに\hfill{19840800}
\\忌ごもりの友訪ひて汨つ戻り梅雨\hfill{19840700}
\\夏書終へ東塔西塔仰ぐ朝\hfill{19840600}
\\空と無の多き夏書や朝鴉\hfill{19840600}
\\りんどうや標高識のたつ小駅\hfill{19840900}
\\高原列車おそしとゆれる花すすき\hfill{19840900}
\\紫の小波たてり松虫草\hfill{19840900}
\\思はざる遠冨士すゝきの小窓より\hfill{19840900}
\\朝風に彩をひろげてのうぜん花\hfill{19840700}
\\風凉し天主の床の黒光り\hfill{19840700}
\\俳聖殿忍者屋敷も蝉しぐれ\hfill{19840900}
\\秋凉し絵とき説法に笑ひあり\hfill{19840917}
\\水軍の洞の跡や秋の潮\hfill{19840917}
\\青い眼の手ぶりに見入る踊の輪\hfill{19840800}
\\諷刺歌踊りの櫓は高調し\hfill{19840800}
\\送り火やもとの一人に戻る夜\hfill{19840800}
\\帰省子の言葉大人ひふと淋し\hfill{19840800}
\\若者となるは別れか鳥雲に\hfill{19840800}
\\夏霧の湧きて流れて山の湖\hfill{19840700}
\\山茶花の垣咲き始めぬ謡声\hfill{19841100}
\\冬の雲まこと知らせぬ人見舞ふ\hfill{19840000}
\\年忘れ流す憂さなきワインの香\hfill{19841200}
\\賀状書く亡母の字に似る母の年令\hfill{19841200}
\\寄せ鍋の沸々はずむ故郷ことば\hfill{19841200}
\\するつと食ぶ熟柿に郷愁そぞろ湧く\hfill{19841200}
\\吾が誕生秋刀魚で祝ひ心足る\hfill{19841100}
\\初冨士や大東京の隅に住み\hfill{19850100}
\\林立の煙突冨士に初煙\hfill{19850100}
\\初仕事裾野の町の白煙\hfill{19850100}
\\移し植え三年の梅に初つぼみ\hfill{19850200}
\\陽を集め日毎ふくらむ木瓜の花\hfill{19850200}
\\蘭匂ふ独りの部屋に惜しき程\hfill{19850300}
\\逆縁の香たく背なに春空し\hfill{19850200}
\\春や憂し着かえし裾の静電気\hfill{19850400}
\\割れ込まれ句心とぎれぬ春炬燵\hfill{19850300}
\\初蕨(わらび)雨に持ちくれ留守の扉に\hfill{19850400}
\\名にひかれ植え初花をひめ辛夷\hfill{19850400}
\\天主より眺むる花の城下町\hfill{19850421}
\\階高し一打の鐘に花の散る\hfill{19850421}
\\老鴬に耳あそばせて喜寿の足\hfill{19850509}
\\蝸牛わがもの顔に城跡の碑\hfill{19850509}
\\ぷちぷちと峠に摘めり夏わらび\hfill{19850618}
\\木苺の酢っぱ甘さや渓流に\hfill{19850617}
\\塗りかへて狭庭の客に青蛙\hfill{19850600}
\\花ざくろ觸れて硬しや朱の色\hfill{19850600}
\\御名のごと清らに生きて蓮花\hfill{19850600}
\\たまはりし紫式部さわ咲けど\hfill{19850800}
\\短夜や句机ならぶ夢の切れ\hfill{19850800}
\\夜濯ぎて一日終りぬ恙なく\hfill{19850800}
\\働けることの幸玉の汗\hfill{19850800}
\\言ふだけで気のすむ愚痴に団扇風\hfill{19850800}
\\階暑し団地こつこつセールスマン\hfill{19850900}
\\梅雨しめる記帳簿将軍旧居訪ひ\hfill{19850625}
\\苔の花将軍愛馬の小さき塚\hfill{19850625}
\\将軍旧居もちの花\hfill{19850625}
\\意を通し過ぎし淋しさ夏の蝶\hfill{19850625}
\\小駅の時計おそしと思ふ時雨来て\hfill{19851119}
\\名もゆかしこほろぎ橋の渓紅葉\hfill{19851120}
\\冬の雷一発のみや能登に泊つ\hfill{19851120}
\\冬ぬくし見舞ひし友にもてなされ\hfill{19851200}
\\謡声白山茶花の垣流れ\hfill{19851200}
\\小説の終りのごとく落葉散る\hfill{19851200}
\\愛語りし腰掛石や昼ちちろ\hfill{19850000}
\\曼茶羅に政子のむかし秋そぞろ\hfill{19850000}
\\露けくて墨のうすれしいわれ書\hfill{19850000}
\\輪飾りの小さきをかけ団地の扉\hfill{19860100}
\\寒木瓜の紅を深めて雨上る\hfill{19860100}
\\盆梅や鉢の木謡ひたき夜なり\hfill{19860100}
\\成人の日の背広着し子を見上ぐ\hfill{19860200}
\\試験子の窓に憂きほど春深雪\hfill{19860300}
\\弔ひて無口の帰り春吹雪\hfill{19860200}
\\ことなげに抜歯をされて春寒し\hfill{19860300}
\\白梅や三百年を語る幹\hfill{19860300}
\\ゆずり合ひつヽ空うばひ梅盛る\hfill{19860300}
\\春時雨急げば合はす鍵の鈴\hfill{19860300}
\\土を割る花芽それぞれ色ありて\hfill{19860300}
\\書き終えてほつと紅茶の浅き春\hfill{19860300}
\\庭隅に鈴蘭匂ひ旅ごころ\hfill{19860400}
\\屋根草もうすき緑に御寺春\hfill{19860400}
\\枝うつるりす生き生きと新樹光\hfill{19860400}
\\散るものは散らして扇塚の春\hfill{19860400}
\\明日に咲く牡丹見よと泊めくれし\hfill{19860500}
\\牡丹の今開かむと息づかひ\hfill{19860500}
\\身も心青く染まりぬ宮若葉\hfill{19860500}
\\山越ゆるあの辺野崎か花曇\hfill{19860400}
\\バスの窓遠見を塞ぐ栗の花\hfill{19860613}
\\蛇の衣板一枚の城跡文\hfill{19860614}
\\アイスクリーム売の熱弁落城譜\hfill{19860614}
\\蔦青し城見ゆ坂のオランダ塀\hfill{19860615}
\\青葉冷え天主の跡の落城譜\hfill{19860615}
\\踊太鼓すぐそこにきき足を病む\hfill{19860800}
\\山男めきひげ面の帰省孫\hfill{19860800}
\\癒ゆること信じてきけり蝉の声\hfill{19860800}
\\癒ゆきざししかと凉しき今朝の風\hfill{19860900}
\\亡母の櫛ふとさしてみる盆支度\hfill{19860800}
\\杖に頼る試歩の足もと萩こぼる\hfill{19860900}
\\寝団扇にうちわどころの故郷のこと\hfill{19860900}
\\去ぬ燕便りとたよりすれちがひ\hfill{19860900}
\\鰯雲交しておかむ生き形見\hfill{19861000}
\\風に雲に秋の深みを知る夕べ\hfill{19861000}
\\カタカナ語事典にいどむ老夜長\hfill{19861000}
\\菊の香や来し方遠し五十年忌\hfill{19860900}
\\雲を割り冬陽美し退職す\hfill{19861100}
\\むなしさも煙としたり菊を焚く\hfill{19861100}
\\年用意心のこもる故郷の荷\hfill{19861200}
\\満目の紅葉それぞれちがふ色\hfill{19861115}
\\静かなりいで湯娘と在り去年今年\hfill{19870101}
\\たまさかの晴着に帯と初芝居\hfill{19870100}
\\シテ謡ひ修めし安堵室の梅\hfill{19870100}
\\誰が為と笑はれもして初鏡\hfill{19870100}
\\梅白し陽ざしの居間の笑ひ声\hfill{19870200}
\\男子校女子校つづき芽ふく道\hfill{19870200}
\\庭の陽を占めて寒木瓜紅の濃し\hfill{19870200}
\\火廼要慎祀符の墨字に春ぼこり\hfill{19870300}
\\今日は憂し今日は美くし木の芽雨\hfill{19870300}
\\春愁を恥じて陶狸の腹を撫ず\hfill{19870300}
\\名桜につきぬ名残の里を去る\hfill{198870419}
\\山裾の梨の花園に白昼夢\hfill{19870415}
\\花クローバ終の棲家の地鎮祭\hfill{19870500}
\\松の花傘寿を集ふ公の庭\hfill{19870513}
\\文学館出でてまぶしき若葉光\hfill{19870513}
\\目礼がことばよ通院路の茂り\hfill{19870600}
\\青葉雨千人塚の匂ひ濃し\hfill{19870527}
\\土産店菖蒲と競ふ肥後名所\hfill{19870428}
\\五月晴阿蘇の寝釈迦に帰途祈り\hfill{19870529}
\\夏草に五百羅漢のかくれんぼ\hfill{19870709}
\\夏草にあそびつ羅漢の泣き笑ひ\hfill{19870709}
\\自転車で五日の旅の戻り梅雨\hfill{19870700}
\\初咲きの桔梗と供華に朝づとめ\hfill{19870800}
\\夜濯ぎの干場思はず下手な歌\hfill{19870800}
\\八階に住みて音なき遠花火\hfill{19870800}
\\早発ちてさかさ冨士みむ秋の湖\hfill{19870915}
\\霧晴れて小波が消すさかさ冨士\hfill{19870915}
\\文学碑たてる峠に秋の冨士\hfill{19870915}
\\花すゝき駅近かそうで遠かりし\hfill{19870904}
\\招くごとコスモス揺るる無人駅\hfill{19870904}
\\誰も来ずくつろぐ時の菊日和\hfill{19871100}
\\老夜長旅に集めし箸袋\hfill{19871100}
\\とっておきのワインもてなす良夜かな\hfill{19871000}
\\南洲を語る白髪月の部屋\hfill{19871000}
\\紅葉濃し峠二つを越えし温泉\hfill{19871119}
\\隣より争ひ声や秋の暮\hfill{19871100}
\\石蕗さかり先は稲荷の鳥居径\hfill{19871100}
\\海知らぬ犬を毎朝冬の浜\hfill{19871200}
\\新らしき木の香の中に賀状書く\hfill{19871200}
\\看とりつつ句帳かた辺に長き夜\hfill{19871000}
\\看とり女にある秋晴や特選句\hfill{19871000}
\\祭太鼓看とりの窓に遠くきく\hfill{19871000}
\\安眠なき看とりの夜々に虫親し\hfill{19871000}
\\愛語りし腰掛石や昼ちちろ\hfill{19871000}
\\露けしや墨のうすれしいわれ書\hfill{19871000}
\\曼茶羅に政子の昔秋そぞろ\hfill{19871000}
\\寒青空娘は頬染めて婚約を\hfill{19880100}
\\梅二月婚約成りし娘のまぶし\hfill{19880200}
\\婚近き娘と春いちご分ちあい\hfill{19880300}
\\列車徐行深雪のここに友住ふ\hfill{19880200}
\\たまわりし手造り味噌に蕗のとう\hfill{19880200}
\\枯芝にねてにらまるゝはらみ猫\hfill{19880200}
\\春寒や三日もつづく探しもの\hfill{19880200}
\\春灯失せものこゝに出て笑ふ\hfill{19880200}
\\椿落つ今日も名知らぬ鳥の来て\hfill{19880300}
\\ゆかし名ばかり揃えて盆梅展\hfill{19880200}
\\春潮に水尾ひく連絡船(ふね)のあと幾日\hfill{19880300}
\\終航の間近かき名残瀬戸の春\hfill{19880300}
\\花菜漬土産に訪ひくれ京言葉\hfill{19880300}
\\手染めとて淡き春着の京言葉\hfill{19880300}
\\花冷えて鬼女の棲みける巨き岩\hfill{19880423}
\\恐ろしき昔語りや花の里\hfill{19880423}
\\杉古りて黒塚ひそと花曇る\hfill{19880423}
\\若やぎて傘寿の集ひ牡丹園\hfill{19880516}
\\声低く僧が餅売る牡丹寺\hfill{19880516}
\\手をとりて笑む道祖神若葉光\hfill{19880516}
\\花の雨眠る山湖を去りがたく\hfill{19880517}
\\老鴬や奥へとたずね政子墓所\hfill{19880601}
\\旧姓で呼びあふ荘の明易し鎌倉荘)\hfill{19880601}
\\まぐなぎを払ひ百体地蔵訪ふ\hfill{19880600}
\\探ねゆく流れ涼しき渓いで湯(太閤の湯)\hfill{19880700}
\\カンナ燃えひしめきあえる養鶏舎\hfill{19880700}
\\雲走り峯にこま草這ひて咲く\hfill{19880700}
\\浜木綿にしばらくのこる夕茜\hfill{19880700}
\\故里の植田にうつす己が影\hfill{19880800}
\\錦飾る故郷ならずも茄子の花\hfill{19880800}
\\甚平着て今日も碁敵待つ\hfill{19880800}
\\叔父跡地ひまわり咲かす家五軒\hfill{19880800}
\\朝顔や一家は北に赴任して\hfill{19880800}
\\秋蝶が惜しむ別れの前よぎる\hfill{19880900}
\\見送りの垣根アベリア咲きこぼる\hfill{19880900}
\\滝二つ遠見の台に小手かざし\hfill{19880900}
\\穂すすきのみるみる刈られゆく売地\hfill{19880900}
\\吾が暮し覗いて聞いて青芒\hfill{19880900}
\\秋と思ふホームに目立つ黒い靴\hfill{19880900}
\\爽かや事終へて発つ旅の朝\hfill{19880900}
\\大秋晴善光寺平一望に\hfill{19880900}
\\歌声をのせて寄せ来る芒波\hfill{19880900}
\\コスモスのゆれる川沿ひ遊歩道\hfill{19880900}
\\母となる娘に寄す思ひ冬ぬくし\hfill{19881100}
\\実南天紅し娘は母となる\hfill{19881100}
\\晩菊や終止符打たん独り住み\hfill{19881100}
\\息子と同居決めむ独りの湯豆腐鍋\hfill{19881100}
\\トンネルを出て越前の雪景色\hfill{19881200}
\\仏壇を買ひに越路へ雪清し\hfill{19881200}
\\山ふところに香煙みちて初薬師\hfill{19890102}
\\初護摩の煙いただき肩かるし\hfill{19890102}
\\紅梅のふふみしことも友へ書く\hfill{19890000}
\\大茶盛廻す茶碗に和気あふれ\hfill{19890100}
\\寒木瓜の紅流れそう雨つづく\hfill{19890200}
\\春寒し故なく心のとがる今日\hfill{19890200}
\\契約のとれてマフラー忘れ去ぬ\hfill{19890200}
\\雪ごもり写経の日々と紙便り\hfill{19890200}
\\春風や繰り上げ帰国のよき知らせ\hfill{19890200}
\\引き越しの迫り咲きつぐ春の彩\hfill{19890300}
\\転宅の別れの集ひ鰆すし\hfill{19890300}
\\すましたる貴婦人めける柴木蓮\hfill{19890400}
\\昼顔や島にたづねる古き墓\hfill{19890430}
\\夕明りのこる卯波や島に泊つ\hfill{19890430}
\\城下町一望にほふ栗の花\hfill{19890425}
\\お天主へ石垣高し松の花\hfill{19890425}
\\天主閣仰ぐ茶店の藤こぼる\hfill{19890425}
\\紫陽花の彩拡げゆく遊歩道\hfill{19890500}
\\夏三つ葉雨の小やみに摘む留守居\hfill{19890600}
\\母も娘もショートカットにさくらんぼ\hfill{19890600}
\\窓開き大向日葵に見つめらる\hfill{19890700}
\\驕りても向日葵は好き美くしき\hfill{19890700}
\\留守居して一人に惜しき風凉し\hfill{19890700}
\\水撒きて陶狸うれしき顔となる\hfill{19890700}
\\思ひきり水撒き散らす重きもの\hfill{19890700}
\\賞め言葉裏に返さず花クローバ\hfill{19890700}
\\水撒きて木々と話をする留守居\hfill{19890800}
\\白粉花空家となりし垣に満つ\hfill{19890800}
\\病葉のこの量踏みて医に通ふ\hfill{19890800}
\\鳶舞ふ高野の夏の深き空\hfill{19890700}
\\野猿乗り夏の河原の若者等\hfill{19890700}
\\グラヂオラス店の娘明るく迎へくれ\hfill{19890700}
\\ポンポンダリヤ活けて村営コーヒー館\hfill{19890700}
\\漁火に想ひそれぞれ宿浴衣\hfill{19890800}
\\盆列車着席までを送らるる\hfill{19890800}
\\伝説の湖ははるかに芒原\hfill{19890900}
\\湖も山もみるみる消えて霧の海\hfill{19890900}
\\山の霧流れて速し湖生る\hfill{19890900}
\\のぼり来て賽の河原の細芒\hfill{19890900}
\\旅に訪ふドラマ舞台の町も秋\hfill{19890900}
\\久の出会ひ杖目じるしと言ふも秋\hfill{19890900}
\\秋釣の成果に夕餉賑へり\hfill{19891000}
\\秋雨のやまず留守居の夕仕度\hfill{19891000}
\\コスモスの身丈を埋めてはるか冨士\hfill{19891000}
\\湧き水の秋澄む池に冨士の影\hfill{19891000}
\\天高し誕生釈迦の細き指\hfill{19891029}
\\落葉かき風に根気の作務の僧\hfill{19891029}
\\柿届く家なき故郷の友も老ひ\hfill{19891100}
\\郷言葉の電話果なし老夜長\hfill{19891100}
\\命延ぶ泉いただき峯を越す\hfill{19891106}
\\野仏の膝にさい銭紅葉散る\hfill{19891106}
\\冬濤の音きヽ紀伊の朝茶粥\hfill{19891200}
\\娘が立てし枕屏風に安眠して\hfill{19891200}
\\晩菊に名残水やり旅に出る\hfill{19891200}
\\報恩講善女となりてしる粉賜ぶ\hfill{19891029}
\\花車たがへず来たり年用意\hfill{19891200}
\\心ゆくまで謡ひけり年忘れ\hfill{19891200}
\\娘の忌日となりて年経る小つもごり\hfill{19891200}
\\旅立ちを止めて眺むる強吹雪\hfill{19900100}
\\おくれ咲く紅山茶花の雪化粧\hfill{19900100}
\\潮の香をはこび来る風春近し\hfill{19900200}
\\水温みあひる天国てふ川辺\hfill{19900200}
\\指圧効きかろき足もと蕗のとう\hfill{19900200}
\\桃ふふみ声出し笑ふと嬰便り\hfill{19900300}
\\初雛に招かれ曾孫しかと抱く\hfill{19900300}
\\亡母の忌や弟としのぶ春炬燵\hfill{19900300}
\\高々と辛夷咲きみつ城跡園\hfill{19900300}
\\もてなさる小さき土鍋に土筆煮て\hfill{19900300}
\\こんがりと焼味噌蕗のとうほのと\hfill{19900300}
\\蕗摘みて老の自慢のちらしずし\hfill{19900400}
\\一心の白夕闇にほのと浮く\hfill{19900400}
\\陶狸の背出で入る鳥の巣づくりか\hfill{19900400}
\\葉桜や友のギブスはまだ除れず\hfill{19900400}
\\露座観音見おろす里の柿若葉\hfill{19900500}
\\柿若葉光る白壁つづく里\hfill{19900500}
\\風薫る河童出そうな筑後川\hfill{19900500}
\\老鴬に迎えられけり峡の宿\hfill{19900500}
\\鱚一尾釣りて得意の帰宅ベル\hfill{19900600}
\\釣りし鱚ほめて一箸づつ廻し\hfill{19900600}
\\ご協力と酢い甘夏を嫁出し来\hfill{19900600}
\\紫陽花や登山電車は幾曲がり\hfill{19900600}
\\お世辞とも思ひつつ買ふ夏帽子\hfill{19900700}
\\夏帽子鏡の顔はヤヤすまし\hfill{19900700}
\\のびて寝る猫のかたへに端居して\hfill{19900700}
\\待つ荷物おそし木樺はしぼみ初む\hfill{19900700}
\\鎌倉の御寺凉やか友葬る\hfill{19900700}
\\母として慕はれ甥とビールくむ\hfill{19900800}
\\風鈴や父母知らぬ甥よき父に\hfill{19900800}
\\五十年忌修すあの日も秋暑く\hfill{19900800}
\\巨寺にみちのくらしき萩まつり\hfill{19900900}
\\雨上がり紅たわヽなるりんご園\hfill{19900900}
\\子に孫にりんご送りて津軽旅\hfill{19900900}
\\台風もよしといで湯にやり過ごし\hfill{19900900}
\\久に来し皇居のお濠曼珠沙華\hfill{19901000}
\\コスモスの風に流せるほどの些事\hfill{19901000}
\\ただ声をききたく夜長の遠電話\hfill{19901000}
\\バスを待つこわれベンチに秋の蝶\hfill{19901000}
\\茫々の芒の中や美人塚\hfill{19901110}
\\神在月とガイド熱あり出雲路よ\hfill{19901110}
\\濃紅葉座禅堂の扉はかたく閉じ\hfill{19901119}
\\寄進瓦に筆持つひまも紅葉散る\hfill{19901119}
\\庭小春鳩来て犬が少し吠え\hfill{19901200}
\\晩菊や顔見ぬ電話言ひ過ぎし\hfill{19901200}
\\枯木してはるか冨士見る道となる\hfill{19901200}
\\数の子の歯音うれしや八十路三つ\hfill{19910101}
\\初詣極楽寺てふ名にひかれ\hfill{19910102}
\\初旅や全き冨士に真向へり\hfill{19910100}
\\立春の陽に勇気湧きトレーニング\hfill{19910200}
\\足鍛え眠り覚めたる山のぼる\hfill{19910200}
\\人波に流されてみる梅まつり\hfill{19910200}
\\指呼の山みるみるかくす春吹雪\hfill{19910219}
\\舞へ狂へいで湯ごもりの春吹雪\hfill{19910219}
\\ほの酔ひや孫つぎくれしお白酒\hfill{19910300}
\\ひなの前老も交りて撮る今宵\hfill{19910300}
\\梅林へ少しの坂も手を引かれ\hfill{19910310}
\\白梅の古木に希ふ吾が余生\hfill{19910310}
\\湖見ゆる観音堂の大桜\hfill{19910400}
\\芽柳の日々に大ゆれ風青し\hfill{19910400}
\\花散るや石州瓦の光る村\hfill{19910400}
\\初蝶や癒えて佇つ庭彩ふえて\hfill{19910400}
\\初蝶やふっつり切れし思ひごと\hfill{19910400}
\\新茶賜ぶ少年今は病院長\hfill{19910500}
\\芍薬や三度の転居共にして\hfill{19910500}
\\染め止めて白髪軽し青葉風\hfill{19910600}
\\年令らしく白髪でおしゃれ夏帽子\hfill{19910600}
\\釣り土産べらとはうれし瀬戸育ち\hfill{19910600}
\\早苗田の日毎濃くなる療の窓\hfill{19910600}
\\山の湖万緑の中遠くあり\hfill{19910600}
\\山間の夏霧深き駅に着く\hfill{19910700}
\\立葵彩を揃えて山の駅\hfill{19910700}
\\薬草湯の香りのこりて宿浴衣\hfill{19910700}
\\大寸の宿衣たぐりて岩魚膳\hfill{19910700}
\\億の土地我がもの顔に青すすき\hfill{19910800}
\\通院の道は川沿ひ月見草\hfill{19910800}
\\時計おそし独り留守居の小粒ぶどう\hfill{19910800}
\\秋暑しビルの掃除夫見上ぐ窓\hfill{19910800}
\\保養所のヴェランダ踊りの列を見る\hfill{19910800}
\\踊りうちわよべの土産と保養友\hfill{19910800}
\\秋の湖哀話流して遊覧船\hfill{19910900}
\\温泉の町にお湯かけ地蔵秋うらら\hfill{19910900}
\\敬老日ほの酔はされて若返る\hfill{19910900}
\\誰が家ぞ芒刈られて地鎮祭\hfill{19910900}
\\秋場所の終り落ちつき夕支度\hfill{19910900}
\\ゆかしさに秋七草の寺巡り\hfill{19910900}
\\尊氏も正成も美男菊衣\hfill{19911000}
\\天高し八十路二人が峯に彳つ\hfill{19911000}
\\穂芒の波うねうねと芒山\hfill{19911000}
\\秋茄子を嫁にすすめて共笑ひ\hfill{19911000}
\\神有りの出雲の湖はかもめ舞ふ\hfill{19911100}
\\宍道湖の大橋たもと柳散る\hfill{19911100}
\\宍道湖の秋の入日に出合ひけり\hfill{19911100}
\\名菓舗の近くに石焼芋の声\hfill{19911100}
\\鳴き砂を踏めば聞えし秋の声\hfill{19911100}
\\白髪を少しのぞかせ冬帽子\hfill{19911200}
\\もう一度鏡をのぞく冬帽子\hfill{19911200}
\\久に会ふ少しおしゃれに冬帽子\hfill{19911200}
\\諦めもした犬癒えて冬ぬくし\hfill{19911200}
\\独言ならずチロとの話始め\hfill{19911200}
\\愛犬のチロも淑気の尾をふれり\hfill{19920100}
\\年の夜吾より古き茶棚拭く\hfill{19911200}
\\立春大吉吾より古き茶棚拭く\hfill{19911200}
\\名水へ凍ての渓路手をひかれ\hfill{19920103}
\\謡初帯山小さく装ふ同志\hfill{19920100}
\\謡初足のねぢりを許し合ひ\hfill{19920100}
\\保養所で看る東京の雪ニュース\hfill{19920200}
\\お返しを気にする老や冬いちご\hfill{19920200}
\\大山ははるか田に群る白鳥かな\hfill{19920200}
\\旅帰り待ちくれ紅梅咲き満つる\hfill{19920200}
\\紅梅や吾が色にせむと言ひし亡友\hfill{19920200}
\\梅の闇逢ふ日約せし友逝きぬ\hfill{19920200}
\\旅はずむ卒業進学祝ぎ二つ\hfill{19920300}
\\たまさかの母と息子の旅春の虹\hfill{19920300}
\\春眠の十指ほぐしつ今日へ覚む\hfill{19920300}
\\春セーター鏡に肩のうすきこと\hfill{19920300}
\\美くしく老いたきものよ柴木蓮\hfill{19920400}
\\シクラメン茶の間笑ひ溢れさす\hfill{19920400}
\\ふる里はすみれたんぽぽ墓の径\hfill{19920400}
\\桃の花さら前かけの辻地蔵\hfill{19920400}
\\お遍路の憩なる礎石大伽藍\hfill{19920400}
\\菜の花を手いつぱい摘み日毎漬け\hfill{19920400}
\\日々摘めど菜の花畑の黄は濃ゆく\hfill{19920400}
\\花杏真白従妹に甘え気味\hfill{19920400}
\\芍薬の蕾ふくらむ庭の日々\hfill{19920500}
\\発つ朝にうす紅ほのと花水木\hfill{19920500}
\\いそいそと半袖えらび旅立てり\hfill{19920500}
\\山迫る車窓次々藤の花\hfill{19920500}
\\若葉風亡妹の友とめぐり逢ひ\hfill{19920500}
\\短か夜や亡妹の友と泊つ出雲\hfill{19920500}
\\ビール酌むかちんとグラス若やぎて\hfill{19920600}
\\ビール酌むドラマのように共鳴し\hfill{19920600}
\\ビール乾し少し多弁に刻忘る\hfill{19920600}
\\向日葵が君臨空地の草いくさ\hfill{19920700}
\\木樺咲く一日の花の教えごと\hfill{19920700}
\\垣根ばら互の無事を老犬と\hfill{19920700}
\\夕仕度水の出細き大暑かな\hfill{19920700}
\\開け放つ窓に早起き木樺かな\hfill{19920700}
\\酌みもして婿の気配り凉しき餉\hfill{19920700}
\\倒産の去りゆく一家百日紅\hfill{19920800}
\\一言がちくりと秋の草に棘\hfill{19920800}
\\遠冨士の景ある売地草茂る\hfill{19920800}
\\芝生踏む素足に伝ふ今朝の秋\hfill{19920800}
\\新凉や試歩の芝生に笑み交す\hfill{19920800}
\\高階に寝て眺め居り雲の峰\hfill{19920800}
\\霧にまだ眠る町並試歩はげむ\hfill{19920900}
\\夏霧の深し湯の町まだ覚めず\hfill{19920900}
\\回廊に沿ふ白萩に清めらる\hfill{19920900}
\\水攻めの城跡や蓮の実の大粒\hfill{19920900}
\\苗木より三年無花果三つ熟れる\hfill{19920900}
\\長生きに想ひいろいろ敬老日\hfill{19920900}
\\秋灯下親しきものは虫眼鏡\hfill{19921000}
\\保養所の昼餉にぎやか大秋刀魚\hfill{19921000}
\\露芝生試歩の目標果し得て\hfill{19921000}
\\秋日和木椅子に一病話し合ふ\hfill{19921000}
\\シャッターを頼む一会や寺紅葉\hfill{19921000}
\\庭園灯淡きに和せぬ木犀の香\hfill{19921000}
\\実梅の香まこと顔して嘘をきく\hfill{19920700}
\\夜の仏間大蜘蛛打ちて逃がしけり\hfill{19920700}
\\耳遠く独りもよしと新茶汲む\hfill{19920700}
\\魂迎ふやがては迎えらるる吾\hfill{19920700}
\\帰省子に一夜越し方きかれけり\hfill{19920700}
\\山荘の冨士見ゆ窓に姫りんご\hfill{19921100}
\\夜霧匂ふ同郷なりし荘の主\hfill{19921100}
\\天高し無傷の紺を飛機が割る\hfill{19921100}
\\セーターの赤を鏡に問ふ八十路\hfill{19921100}
\\声高や桜紅葉の女子校道\hfill{19921100}
\\迎えられ娘の柚子風呂の香りかな\hfill{19921200}
\\いさかひが笑ひに母と娘の冬至\hfill{19921200}
\\年用意母と娘の声いづれとも\hfill{19921200}
\\部屋に冷ゆ胸像の夫に独り言\hfill{19921200}
\\行く年へ刻む時計に息つめて\hfill{19921200}
\\我が城と正月飾り四畳半\hfill{19930100}
\\繰るほどに夢ふくらみ来初暦\hfill{19930100}
\\二日早帰る子送る母の背\hfill{19930100}
\\好物で老犬はげます寒の入\hfill{19930100}
\\居候の老に朝毎寒玉子\hfill{19930100}
\\老犬と共に留守居す梅日和\hfill{19930200}
\\老犬の背に紅梅の一片が\hfill{19930200}
\\一跳ねに広がる水輪水ぬるむ\hfill{19930200}
\\春立ちぬ川面は白き雲浮かべ\hfill{19930200}
\\白き雲浮かべ川面は春立ちぬ\hfill{19930200}
\\倖せは歯音にありし年の豆\hfill{19930200}
\\今日よりはチロ居ぬ生活春寒し\hfill{19930330}
\\姫こぶし一輪樹下にチロは死す\hfill{19930330}
\\春嵐おさまる朝にチロは死す\hfill{19930330}
\\春寒しピンクの布に巻く屍\hfill{19930330}
\\窓開けばおやつ待つチロ無き余寒\hfill{19930330}
\\従姉妹どち幼な呼びして桃の郷\hfill{19930400}
\\故里や摘みてたちまち木の芽和え\hfill{19930400}
\\故里はお遍路の鈴あわあわと\hfill{19930400}
\\朧夜や骨までしゃぶる瀬戸の味\hfill{19930400}
\\短夜やはらから集ふ郷言葉\hfill{19930400}
\\老鴬に迎え送られ札所寺\hfill{19930400}
\\仁王門くぐりて見上ぐ余花やさし\hfill{19930400}
\\牡丹や余生つぎこむ花づくり\hfill{19930400}
\\新背広卒業の子を見上げけり\hfill{19930400}
\\祝背広就職といふ巣立かな\hfill{19930400}
\\就職は別れの一つ鳥雲に\hfill{19930400}
\\散華とも霊園しとど花吹雪\hfill{19930400}
\\咲き競ひし源平桃も葉となりぬ\hfill{19930500}
\\藤娘出そう藤房ととのへり\hfill{19930500}
\\三代の旅信濃路を青葉風\hfill{19930500}
\\大手まり真白湯の香の中にゆれ\hfill{19930500}
\\まじり気のなきみどり嶺よ露天風呂\hfill{19930500}
\\峯八分疲れは軽し藤の花\hfill{19930500}
\\からみ合ひ花房乱る深山藤\hfill{19930500}
\\子に植えし桜桃熟るる少女有美\hfill{19930400}
\\遍路憩ふ礎石千年語りつぐ\hfill{19930400}
\\点滴の紫班をさする梅雨の窓\hfill{19930605}
\\明易すや退院といふ別れかな\hfill{19930513}
\\濃紫陽花点滴の染みうすれゆく\hfill{19930513}
\\錠剤をならべ数えて夕薄暑\hfill{19930700}
\\負け相撲少し頭痛の戻り梅雨\hfill{19930700}
\\連れだちていそいそ母娘浴衣買ひ\hfill{19930700}
\\連れだちて母娘の購む派手浴衣\hfill{19930700}
\\浴衣茶会立居気になる娘を送る\hfill{19930700}
\\月下美人迎へ車で御対面\hfill{19930700}
\\月下美人息を弛めず咲き拡ぐ\hfill{19930700}
\\手伝ひ娘不満あるげに水を打つ\hfill{19930700}
\\咲きましたとて嫁が見す鷺草鉢\hfill{19930800}
\\鷺草の飛びさる舞ひよう目離せず\hfill{19930800}
\\水撒けば陶狸がうれし涙する\hfill{19930800}
\\これはまあ皿をはみ出る初秋刀魚\hfill{19930900}
\\倉裡裏の鬼灯赤し妻若し\hfill{19930900}
\\猫難の子雀放つ秋彼岸\hfill{19930900}
\\雀獲りしかり猫抱く秋彼岸\hfill{19930900}
\\映る影流るる音も水の秋\hfill{19931000}
\\秋晴やいそいそ釣に碁敵と\hfill{19931000}
\\秋晴や碁敵はまた釣がたき\hfill{19931000}
\\釣りし沙魚はねる厨にはや碁音\hfill{19931000}
\\雁渡る双手で握手する別れ\hfill{19931000}
\\口釜へ増ゆる孫との日向ぼこ\hfill{19931000}
\\柿送る案内電話の郷言葉\hfill{19931000}
\\柳散る入日に染まる湖のほとり\hfill{19931100}
\\五指ほぐすなだむ節おし今朝の秋\hfill{19931100}
\\夜逃げとや閉ざせる窓に満月光\hfill{19931000}
\\人恋ふかに垣越し延び来青き蔦\hfill{19931000}
\\猫舌は母似亡母恋ふ湯豆腐鍋\hfill{19931200}
\\物言はず一日留守居の師走呆け\hfill{19931200}
\\冬日向売れぬ空地は猫のもの\hfill{19931200}
\\カレンダーも庭も山茶花日々惜しむ\hfill{19931200}
\\柚子ほめてつい佇ち話いただけり\hfill{19931200}
\\留守居して米研ぐ窓に寒宵月\hfill{19931200}
\\大晴れや蒲団干す家干せぬ家\hfill{19931200}
\\爪切りて指美しや賀状書く\hfill{19931200}
\\吹き溜る枯葉の中の紅一葉\hfill{19931200}
\\宵戎押さへ揉まれて娘はきげん\hfill{19940100}
\\ただいまの娘の声弾む宵戎\hfill{19940100}
\\初釜へ晴着見送る母も美し\hfill{19940100}
\\はよ来ませ郷言うれし初電話\hfill{19940100}
\\寒玉子盛りあがる黄身老もまた\hfill{19940100}
\\春寒やもう夢でしか逢へぬ人\hfill{19940109}
\\頑張れよ愛犬館も初日さす\hfill{19940100}
\\受験子に買ふ知恵袋文殊さま\hfill{19940116}
\\春寒し起ち居いちいち声あげて\hfill{19940200}
\\中古車群旗はたはたと春を呼ぶ\hfill{19940200}
\\猫柳活ける娘もまたつやつやし\hfill{19940200}
\\花葉挿しふと京の友思ひけり\hfill{19940200}
\\再会や土を割り出る花芽たち\hfill{19940300}
\\分葱和へおふくろ味の老自慢\hfill{19940300}
\\名もゆかし若草豆腐のうすみどり\hfill{19940300}
\\点心に一口ほどのたらの芽よ\hfill{19940300}
\\茄子胡瓜畑銀座と故里便り\hfill{19940600}
\\額の花一人で居たき時もあり\hfill{19940600}
\\夏帽子のぞく白髪も好しとして\hfill{19940600}
\\夏帽子年齢をきかれて逆に問ひ\hfill{19940600}
\\山梔子の真白につらき雨つづく\hfill{19940600}
\\青葉風入れてもきれぬ愚痴話\hfill{19940600}
\\言ひたきをたたむくちなし真白なる\hfill{19940600}
\\辻地蔵朝取りトマトにお眼細く\hfill{19940700}
\\暑に耐える白前掛の辻地蔵\hfill{19940700}
\\青田風通し一睡の浄土かな\hfill{19940700}
\\喉走る名水冷えの心太\hfill{19940700}
\\空暗し呼べば遠退く夕立雲\hfill{19940700}
\\今日も亦他所夕立とそれにけり\hfill{19940700}
\\花合歓や渓の音きく温泉の窓\hfill{19940700}
\\含羞草いで湯泊りの老四人\hfill{19940700}
\\故里は金比羅歌舞伎花の山\hfill{19940400}
\\岐れ道ミモザ盛りの島巡り\hfill{19940400}
\\一言の棘のいたみや夏薊\hfill{19940700}
\\一言の棘に猛暑の雲みあぐ\hfill{19940700}
\\風鈴や窓辺に母と娘の笑顔\hfill{19940700}
\\昼寝覚めまだ侍り猫伸びきって\hfill{19940700}
\\シルバーホーム笑ち会釈して廊凉し\hfill{19940800}
\\お元気ねきれいに食べし夏料理\hfill{19940800}
\\西瓜割漢につづく娘が果す\hfill{19940800}
\\踊の輪みるみる三重に炭坑節\hfill{19940800}
\\高階に眼覚めてわっと雲の峰\hfill{19940800}
\\熱帯夜慣れて別れのなにとなう\hfill{19940800}
\\朝凉や肩まで掛けてふと淋し\hfill{19940800}
\\雲の峰息子は太平洋の空ならん\hfill{19940800}
\\満月や仰ぎし友はいま筑紫\hfill{19940900}
\\月白やせり上り待つ大舞台\hfill{19940900}
\\手折り来て芒挿しくれホーム友\hfill{19940900}
\\敬老日過ぎて忘れを詫ぶ息子かな\hfill{19940900}
\\夕木槿一日思案し言ふまじと\hfill{19940900}
\\傷つけしことに気附かず青芒\hfill{19940900}
\\押し分けも背伸びもなくて草の花\hfill{19940900}
\\侘びて住むごと庭隅の時鳥草\hfill{19941000}
\\住むは誰隣の芒刈られけり\hfill{19941000}
\\息子に目立ちきし白きもの柿をむく\hfill{19941000}
\\高階に泊つ霧ぬれの大夜景\hfill{19941000}
\\秋灯に左傾ぎの寿百の字\hfill{19941000}
\\ふる里や菜飯に小芋の煮ころがし\hfill{19941100}
\\大根抜く厨に待つはおろしがね\hfill{19941100}
\\木あがりの茄子見落さず芥子漬\hfill{19941100}
\\木あがりの茄子と思へぬ芥子漬\hfill{19941100}
\\そつと出る夫追ふ妻や露の畑\hfill{19941100}
\\医と寺の娘が幼な友木の葉髪\hfill{19941100}
\\秋風や札所の寺の大礎石\hfill{19941100}
\\木犀匂ふ金銀並びし故里の庭\hfill{19941100}
\\着ぶくれて椅子のくぼみに孫自慢\hfill{19941200}
\\ほほえみで答ふ遠耳冬すみれ\hfill{19941200}
\\言ふだけを言ふてコートの忘れ物\hfill{19941200}
\\爪切りて指美くしく賀状書く\hfill{19941200}
\\保養所の握手の別れ紅葉散る\hfill{19941200}
\\晩菊にそとさよならをしばし旅\hfill{19941200}
\\物忘れめつきり増えて年の暮\hfill{19941200}
\\晩菊の一本供花とし剪りにけり\hfill{19941200}
\\補聴器を切りて一人の冬の夜\hfill{19941200}
\\ほんのりと米寿の頬に屠蘇の紅\hfill{19950100}
\\倖せは初夢もなき深眠り\hfill{19950100}
\\住連飾りドアーにかけて十二階\hfill{19950100}
\\開かんと冬薔薇秘めし力かな\hfill{19950100}
\\梅一輪いちりん日々を留守居して\hfill{19950200}
\\倖せや日々の留守居に梅一輪\hfill{19950200}
\\紅梅や白磁揃ひの朝餉の膳\hfill{19950200}
\\話す日々米寿祝の冬ばらに\hfill{199502000}
\\毛糸解く編み直されぬ過去てふもの\hfill{19950200}
\\春寒し幼なに戻るおないどし\hfill{19950200}
\\空地占め空の青吸ひ犬ふぐり\hfill{19950300}
\\椀に浮くさみどりを吸い春一番\hfill{19950300}
\\朝桜夢のあと追ふ思慕の人\hfill{19950300}
\\聞くだけで事情を愚痴の春炬燵\hfill{19950300}
\\躓きて掌をつくところ土筆んぼ\hfill{19950300}
\\躓きて土筆三本折りて詫ぶ\hfill{19950300}
\\雪柳白壁拒み闇寄せず\hfill{19950400}
\\白壁の汚れはじらふ雪柳\hfill{19950400}
\\ワインの栓ぼんに拍手や夜はおぼろ\hfill{19950400}
\\花は葉に母の素直は息子の憂ひ\hfill{19950400}
\\応えなく平寝落ちしよ花疲れ\hfill{19950400}
\\落ち椿さつさと主掃きにけり\hfill{19950400}
\\兄弟が初鯉のぼり揚げにけり\hfill{19950500}
\\母の日に娘二人の遠電話\hfill{19950500}
\\母の日や六十年を母の道\hfill{19950500}
\\岐れ道えらべば険し果の余花\hfill{19950500}
\\試歩のばす思ひたがわず藤の花\hfill{19950500}
\\絵タイルの道若やぎて地球の日\hfill{19950500}
\\高きほど大揺れてをり夾竹桃\hfill{19950600}
\\雑草の茂りたくまし子もたくまし\hfill{19950600}
\\草いくさ陣地広げし青芒\hfill{19950600}
\\葉を研ぎて陣地広げむ青芒\hfill{19950600}
\\職退くも余生と言へぬ梅青し\hfill{19950600}
\\娘名で忌の案内状梅雨じめり\hfill{19950600}
\\海の風山の風入れ夏座敷\hfill{19950700}
\\夕木槿汚れなき白閉じにけり\hfill{19950700}
\\春秋を裾にひろげて讃岐冨士\hfill{19950700}
\\はいはいと重ねてさびし含羞草\hfill{19950700}
\\眠り草ねむらぬ葉あり反抗期\hfill{19950700}
\\装ひし遠き日のあり薄衣\hfill{19950700}
\\咲き満つもなほあわあわと花みずき\hfill{19950700}
\\花水木乙女の恋の物語\hfill{19950700}
\\故郷発つ朝採りトマト重すぎて\hfill{19950700}
\\傷つけしこと気付かずや青芒\hfill{19950800}
\\やさしくも棘ある言葉夏薊\hfill{19950800}
\\夏痩せを知らずに生きて米寿かな\hfill{19950800}
\\掌中の珠とはこれよ白桃むく\hfill{19950800}
\\無花果を鳥につつかれ犬叱る\hfill{19950800}
\\新凉や又取り出して読む佳信\hfill{19950800}
\\爽やかや返書のペンのよくすべり\hfill{19950800}
\\鳥わたる返書に三色ボールペン\hfill{19950800}
\\露けしや二人の友の新佛\hfill{19950800}
\\コスモスに手をふる急行待避駅\hfill{19951000}
\\秋夕焼こつくりさんの道標\hfill{19951000}
\\出ぬ電話そうか今宵は月の句座\hfill{19951000}
\\家の味継ぎて伝えて祭ずし\hfill{19951000}
\\貰ふなら遠慮はすまじ秋茄子\hfill{19951000}
\\栗むくや消えぬ弟の国訛\hfill{19951100}
\\故郷もつ倖せしかと柿をむく\hfill{19951100}
\\文化の日遠き明治の今日生れ\hfill{19951100}
\\透きとおる秋や少年ハーモニカ吹く\hfill{19951100}
\\鰯雲告げたき人は遠く住み\hfill{19951100}
\\いま倖障子をよぎる鳥の影\hfill{19951200}
\\山茶花や豆腐屋を待つ留守居役\hfill{19951200}
\\冬桜口紅うすくひく米寿 \hfill{19951200}
\\騙されてをれば事なし枯尾花\hfill{19951200}
\\梅ケ枝の終の一葉の散る別れ\hfill{19951200}
\\いつまでも御元気でねてふ賀状の数 \hfill{199601}
\\退職と一筆添へし賀状かな     \hfill{199601}
\\初入日三六六の一を呑み\hfill{199601}
\\ページくる吾が音寒し影寒し\hfill{199601}
\\小豆粥老ひてすこやか姉弟\hfill{199601}
\\春寒し言はでききをり二度話    \hfill{199602}
\\鳥は雲に二度行くスーパー買いわすれ\hfill{199602}
\\梅二月八十路わきまふ笑顔よき   \hfill{199602}
\\娘等去にてかろき疲れに窓の梅   \hfill{199602}
\\よきことを知らす娘の声梅紅し   \hfill{199602}
\\芽吹く庭健かと木々に呼びかけて  \hfill{199603}
\\鳥雲に謝しつつ辛き車椅子     \hfill{199603}
\\鶯やに車椅子停めくれ息子よ    \hfill{199603}
\\とてせめて電話は春の声\hfill{199603}
\\春彼岸弟訪ひくれ仏顔に      \hfill{199603}
\\岬うらら成果一尾の小半日     \hfill{199604}
\\春の夕餉釣りし一尾を母の前    \hfill{199604}
\\快気とはかくもうれしき春の朝   \hfill{199604}
\\春光やを拝み浴びをり癒え兆    \hfill{199604}
\\径端の小さき笑顔犬ふぐり     \hfill{199604}
\\鯉のぼりたーかく揚げて待つ帰国  \hfill{199605}
\\日本を知らぬ児を待つ武者飾り   \hfill{199605}
\\薔薇咲かせ迎え明るき指圧院    \hfill{199605}
\\土産地蕗香りひろげて国言葉    \hfill{199605}
\\木の芽雨偲び草とて届く茶器    \hfill{199605}
\\片隅に生きる幸せ額の花      \hfill{199606}
\\新茶くみほめ言葉待つ母の顔    \hfill{199606}
\\草茂る逆らはぬこと牙につきて   \hfill{199606}
\\明易やドイツ転勤ききしより    \hfill{199606}
\\泰山木朽ちてすがれる花かなし   \hfill{199606}
\\朝涼やからっぽ頭にからっ腹    \hfill{199607}
\\いざ昼寝今日はいづこへ夢の旅   \hfill{199607}
\\夕涼し肌になじみし藍の服     \hfill{199607}
\\暑からむ遅れて浴びる百視線    \hfill{199607}
\\端居して出世無縁の長寿眉     \hfill{199607}
\\暑に耐えし頬なでてみる今朝の風  \hfill{199608}
\\秋暑し訪問販売二度のブザー    \hfill{199608}
\\夜々うれし子の友に賜ぶ古梅酒   \hfill{199608}
\\花火見に橋へ子が押す車椅子    \hfill{199608}
\\癒へてつくる迎え送りの盆団子   \hfill{199608}
\\白萩や見知らぬ同志笑みかわし   \hfill{199609}
\\寺育ち白曼珠沙華燃え知らず    \hfill{199609}
\\風やさしコスモスやさし車椅子   \hfill{199609}
\\思はざる花つけにけり秋の草    \hfill{199609}
\\秋冷ゆる友の情の京しるこ     \hfill{199609}
\\故里や出会ふたれかれ野菊晴    \hfill{199610}
\\栗むきつ老ひて姉弟郷言葉     \hfill{199610}
\\風のまま吾も白髪穂亡や\hfill{199610}
\\花は実に色増す石榴日々親し    \hfill{199610}
\\急げともあわてるなとも虫の鳴く  \hfill{199610}
\\天高し卒寿見上ぐる明治晴     \hfill{199611}
\\秋深き豆煮る母のひとり言\hfill{199611}
\\冬に入る病上手に附き合わす   \hfill{199611}
\\いつまでも娘は子こたつの母苦言  \hfill{199611}
\\よろこびにふとある怖さ夕紅葉   \hfill{199611}
\\熟柿つるっと食べばふるさと近く来る\hfill{199612}
\\枝桜紅葉に告ぐ別れ\hfill{199612}
\\落葉掃きつい長くなる隣同志   \hfill{199612}
\\やがてこの娘が孫の嫁冬いちご  \hfill{199612}
\\雲を割る冬日や老のねがふこと  \hfill{199612}
\\お元旦老母くり返すありがたや  \hfill{199701}
\\しわのなき黒豆に老母初お箸\hfill{199701}
\\初写真嫁孫の笑み三代\hfill{199701}
\\愛犬と話す日日あり寒日和\hfill{199701}
\\翔ばたいて大きなおまへ初からす \hfill{199701 }
\\五十年忌白梅古りし月日かな   \hfill{199702 }
\\孫嫁のもうすぐ二人梅紅し\hfill{199702}
\\お化粧で他人顔なり春写真    \hfill{199702 }
\\春障子四畳半の城明るし\hfill{199702}
\\下萌に煎餅分ける愛犬に\hfill{199702}
\\春耕をまぶしく見をりホーム窓\hfill{199703}
\\啓窒やシルバーホームの預け解け\hfill{199703}
\\春暁の正夢なれや初ひ孫\hfill{199703}
\\向ひ合うパソコン句帖春炬燵\hfill{199703 }
\\おばさんと呼びくれ三人桜餅\hfill{199703}
\\浮雲に名付けあそびや春の風\hfill{199704}
\\こちら向くラッパ水仙こんにちは\hfill{199704}
\\花衣車椅子にも湧くはずみ\hfill{199704}
\\思い桜樹齢二百を恋う卒寿\hfill{199704}
\\花の雨ワインケーキの香に和む\hfill{199704}
\\初咲きの大勺や句や婚の朝    \hfill{199705}
\\桜湯のぱーつとひらけり控室\hfill{199705}
\\純白の花嫁孫となる五月\hfill{199705}
\\柿若葉秘仏開扉めぐり会い\hfill{199705}
\\来し道の険しさ言はず余花仰ぐ\hfill{199705}
\\御幣上る薫風にのる上棟歌\hfill{199706}
\\目つむりて青汁ぐっとばら真紅\hfill{199706}
\\痛いとは生ける証しか梅雨の膝\hfill{199706}
\\梅雨鏡拭けば亡母にとれほどに\hfill{199706}
\\都忘れ咲かせ老いけり京遠く\hfill{199706}
\\今年また梅酒たまわる命かな\hfill{199707}
\\子つばめの翔つを見送る車椅子\hfill{199707}
\\ナイターに興じる老母の片辺して\hfill{199707}
\\白髪といていのちあるもの髪洗ふ\hfill{199707}
\\ぎょうさんな娘の悲鳴蜘蛛の糸\hfill{199707}
\\郷ばなしつきずやさしき団扇かぜ\hfill{199708}
\\夏服の派手を鏡に息子の土産\hfill{199708}
\\きれし夢惜しや貴船のはも料理\hfill{199708}
\\迎はるる仏とならで魂迎ふ\hfill{199708}
\\仏めく盆僧の額黒光り\hfill{199708}
\\赤とんぼヘルパーと唄う車椅子\hfill{199709}
\\星月夜シルバーホーム消灯はやき\hfill{199709}
\\誰似かと爽やかろんぎ初曽孫\hfill{199709}
\\白桔梗時には欲しい母小言\hfill{19970900}
\\おきし手を又も引きよす枝豆を\hfill{19970900}
\end{multicols}
