西川さん 増田さん と淡路島健和荘泊り 灘水仙郷 若人も森など巡る。\\
帰途乗船場にて浅利貝を買う。
\begin{shiika}蛤の潮のしたたり出船待つ
\hfill{\rensuji*{52}・\rensuji*{3}・\rensuji*{0}}\end{shiika}
\vspace{0.6cm}
東横線多摩川鉄橋通過
\begin{shiika}河原なる飛球の行方風光る
\hfill{\rensuji*{52}・\rensuji*{3}・\rensuji*{0}}\end{shiika}
\vspace{0.6cm}
小田さんの案内で山下さんと三人で吉野山へ
\begin{shiika}吉野山春蘭の店は客呼ばず
\hfill{\rensuji*{52}・\rensuji*{4}・\rensuji*{5}}\end{shiika}
\vspace{0.6cm}
相川の畑にて
\begin{shiika}花弁ゆれ奥より出でし虻の貌
\hfill{\rensuji*{52}・\rensuji*{4}・\rensuji*{0}}\end{shiika}
\vspace{0.6cm}
相川の店二階の軒先に燕巣をつくる
\begin{shiika}燕の子黄ならびの嘴花のごと
\hfill{\rensuji*{52}・\rensuji*{5}・\rensuji*{0}}\end{shiika}
\vspace{0.6cm}
あわくら荘に青山さん 西川さん 増田さん と。自然林のほうへ
\begin{shiika}木苺や山の佛の唇あせて
\hfill{\rensuji*{52}・\rensuji*{6}・\rensuji*{25}}\end{shiika}
\vspace{0.6cm}
整くんが寝冷えしていた時
\begin{shiika}寝冷え子のうつろの瞳絵本散る
\hfill{\rensuji*{52}・\rensuji*{7}・\rensuji*{0}}\end{shiika}
\vspace{0.6cm}
「蜜豆」ふとこんなこともあったかな
\begin{shiika}蜜豆に唇さみし嘘を言ふ
\hfill{\rensuji*{52}・\rensuji*{7}・\rensuji*{0}}\end{shiika}
\vspace{0.6cm}
%------------------------------------
一家の旅今津 海津大崎 竹生島 つづら荘泊り\\
八月も終わりに近い つづら荘の前の湖辺にて得た句
\begin{shiika}湖の色北より深み秋きざす
\hfill{双適入選\rensuji*{52}・\rensuji*{8}・\rensuji*{0}}\end{shiika}
\begin{shiika}竹生島真向ふ宿の洗鯉
\hfill{\rensuji*{52}・\rensuji*{8}・\rensuji*{0}}\end{shiika}
\vspace{0.6cm}
高野山登山ケーブルカーの窓より芒を眺めて
\begin{shiika}登るほど尾花は細し高野道
\hfill{\rensuji*{52}・\rensuji*{9}・\rensuji*{0}}\end{shiika}
\vspace{0.6cm}
芒むらの眺めはあちこちに得られた。それに秋吉台の景を重ねて
\begin{shiika}行けど行けど穂芒波や夕茜
\hfill{\rensuji*{52}・\rensuji*{9}・\rensuji*{0}}\end{shiika}
\vspace{0.6cm}
\begin{shiika}天高し隠岐の草原牛肥えて
\hfill{\rensuji*{52}・\rensuji*{9}・\rensuji*{0}}\end{shiika}
\begin{shiika}霊場の鐘にも和さずけらつつき
\hfill{\rensuji*{52}・\rensuji*{10}・\rensuji*{0}}\end{shiika}
%----------------------------
\vspace{0.6cm}
小田から頂戴した紫しきぶが大きくなって美しい実をたくさんに。
\begin{shiika}下枝より褪せて小庭の実むらさき
\hfill{\rensuji*{52}・\rensuji*{10}・\rensuji*{0}}\end{shiika}
\vspace{0.6cm}
相川の家で お謡の小川先生御母堂白寿祝い
\begin{shiika}庭雀床払ひせしふとん干す
\hfill{\rensuji*{52}・\rensuji*{12}・\rensuji*{0}}\end{shiika}
\begin{shiika}白寿祝ぐ願いをこめて羽根蒲団
\hfill{\rensuji*{52}・\rensuji*{12}・\rensuji*{0}}\end{shiika}
相川の家元旦の水。若水を汲むにはあらねど。
\begin{shiika}若水や心新らたに栓開く
\hfill{\rensuji*{53}・\rensuji*{1}・\rensuji*{0}}\end{shiika}
\vspace{0.6cm}
小田澄子さんの御親類 句友 藤田みや様の訃。
\begin{shiika}句友の訃夜を沈丁の香のせまり
\hfill{\rensuji*{53}・\rensuji*{3}・\rensuji*{0}}\end{shiika}
%-----------------------
\vspace{0.6cm}
淡路島への船中よりの景を思い出して
\begin{shiika}春潮に群れ飛ぶかもめ水尾追ひて
\hfill{\rensuji*{53}・\rensuji*{3}・\rensuji*{0}}\end{shiika}
\vspace{0.6cm}
大森先生御他界 城陽大森家を訪ねる
\\中を開かない門のうちには花ゆらす
\vspace{0.6cm}
\begin{shiika}門かたく喪の家ひそと花ゆすら
\hfill{\rensuji*{53}・\rensuji*{4}・\rensuji*{0}}\end{shiika}
\vspace{0.6cm}
\begin{shiika}潮騒の丘の花冷学徒眠る
\hfill{\rensuji*{53}・\rensuji*{5}・\rensuji*{0}}\end{shiika}
\vspace{0.6cm}
小森田 美佐さんと淡路島行く
\begin{shiika}城跡の古井戸涸れず苔の花
\hfill{\rensuji*{53}・\rensuji*{6}・\rensuji*{5}}\end{shiika}
\vspace{0.6cm}
四国八十八ケ所札どころ巡拝
\begin{shiika}桑の実に郷愁ありて札所径
\hfill{\rensuji*{53}・\rensuji*{6}・\rensuji*{0}}\end{shiika}
\vspace{0.6cm}
相川蒔田家の告別式だったか
\begin{shiika}焼香待つ黒幕裾の蟻地獄
\hfill{\rensuji*{53}・\rensuji*{7}・\rensuji*{0}}\end{shiika}
\vspace{0.6cm}
%-------------------------------------------
八十八ケ所霊場巡り(文友会) 最終回さぬき路\\杖は本当に持ち帰り
\begin{shiika}葉鶏頭一筋町の故郷晴れ
\hfill{\rensuji*{53}・\rensuji*{10}・\rensuji*{0}}\end{shiika}
\begin{shiika}結願の杖納め得し鵙日和
\hfill{\rensuji*{53}・\rensuji*{10}・\rensuji*{0}}\end{shiika}
\vspace{0.6cm}
相川風景 よく花屋さん狭い路にも立ち入る
\begin{shiika}花売の残す菊の香路地の朝
\hfill{\rensuji*{53}・\rensuji*{12}・\rensuji*{0}}\end{shiika}
\vspace{0.6cm}
郷生の電話だったかなー
\begin{shiika}口ませし孫の電話や冬すみれ
\hfill{\rensuji*{53}・\rensuji*{12}・\rensuji*{0}}\end{shiika}
\vspace{0.6cm}
クラス会佐渡
\begin{shiika}曼珠沙華島の陵人稀に
\hfill{\rensuji*{53}・\rensuji*{9}・\rensuji*{0}}\end{shiika}
\vspace{0.6cm}
一善広島より出張大阪に来て泊る
\begin{shiika}出張のしげかれ疾かれ牡蠣土産
\hfill{\rensuji*{53}・\rensuji*{10}・\rensuji*{0}}\end{shiika}
\begin{shiika}寄れば逃ぐ子に獅子舞の昂りて
\hfill{\rensuji*{53}・\rensuji*{0}・\rensuji*{0}}\end{shiika}
\begin{shiika}寒餅を切る夜のまど?文とろり
\hfill{\rensuji*{53}・\rensuji*{10}・\rensuji*{0}}\end{shiika}
\begin{shiika}旅立ちの鏡に向ふ夏帽子
\hfill{\rensuji*{53}・\rensuji*{10}・\rensuji*{0}}\end{shiika}
\begin{shiika}久々の子に浴衣着せ今宵酌む
\hfill{\rensuji*{53}・\rensuji*{10}・\rensuji*{0}}\end{shiika}
\begin{shiika}菜の花名を問ひ問はれ三輪の径
\hfill{\rensuji*{53}・\rensuji*{10}・\rensuji*{0}}\end{shiika}
\vspace{0.6cm}
%----------------------------------------------
元旦のお祝い
\begin{shiika}三代が屠蘇なみなみと三つの盃
\hfill{\rensuji*{54}・\rensuji*{1}・\rensuji*{1}}\end{shiika}
\vspace{0.6cm}
年末相川の店より北通りの家へ帰宅の途中走り出た猫に足元狂い捻挫して
佐古整形院で治療
\begin{shiika}冬萠や繃帯の足歩を試す
\hfill{\rensuji*{54}・\rensuji*{1}・\rensuji*{0}}\end{shiika}
\vspace{0.6cm}

楽しんで相川の家えは沈丁花を挿し木いた。\\
すくすく成長したかと思うと突然枯れもした。私はその香りがあまり
好きでなかった、気になる匂ひだから何とか句材にした。
\begin{shiika}昂りぬ沈丁の雨音もなく
\hfill{\rensuji*{54}・\rensuji*{3}・\rensuji*{0}}\end{shiika}
\begin{shiika}啓執や旅誘ひの友便り
家族旅行 土柱 阿波池田
\hfill{\rensuji*{54}・\rensuji*{3}・\rensuji*{0}}\end{shiika}
\begin{shiika}花の下城址碑ひそと休暇村
\hfill{\rensuji*{54}・\rensuji*{4}・\rensuji*{0}}\end{shiika}
\vspace{0.6cm}
さぬき白鳥黒川温泉に糸島さん 増田さんの案内で
\begin{shiika}山の温泉は音なく春蚊早出でし
\hfill{\rensuji*{54}・\rensuji*{4}・\rensuji*{20}}\end{shiika}
\vspace{0.6cm}
文友会西国三十三ケ所巡拝 長谷寺にて
\begin{shiika}草餅に門前町の賑へる
\hfill{\rensuji*{54}・\rensuji*{6}・\rensuji*{0}}\end{shiika}
\vspace{0.6cm}
高田さんに教えられ三年前栗を土に埋めた。何本か芽お出した
中の一本がすくすくと伸びた。
五十七年相川を去る時捨てていくのが惜しかった
\begin{shiika}実生栗初花咲けり吾も健
\hfill{\rensuji*{54}・\rensuji*{6}・\rensuji*{0}}\end{shiika}
\begin{shiika}冷奴遠き旅より帰り酌む
\hfill{\rensuji*{54}・\rensuji*{6}・\rensuji*{0}}\end{shiika}
\vspace{0.6cm}
小森田さんと上田城より別所温泉への旅
\begin{shiika}落ちるまま実梅の匂ひ城のみち
\hfill{\rensuji*{54}・\rensuji*{7}・\rensuji*{16}}\end{shiika}
\vspace{0.6cm}
小森田さんと郡上八幡 井波を訪ねて
\begin{shiika}城の灯のうるみ郡上の踊更く
\hfill{\rensuji*{54}・\rensuji*{8}・\rensuji*{23}}\end{shiika}
\vspace{0.6cm}
\begin{shiika}新秋や欄間彫る町木の香り
\hfill{\rensuji*{54}・\rensuji*{8}・\rensuji*{24}}\end{shiika}
\vspace{0.6cm}
\begin{shiika}谷底は見えずバス行く山の霧
\hfill{\rensuji*{54}・\rensuji*{8}・\rensuji*{24}大島醇子選}\end{shiika}
\vspace{0.6cm}
\begin{shiika}高原の駅コスモスの色極め
\hfill{\rensuji*{54}・\rensuji*{12}・\rensuji*{0}}\end{shiika}
\vspace{0.6cm}
文友会 西国三十三番 巡礼
\begin{shiika}結願の梵鐘ひびく峯の秋
\hfill{\rensuji*{54}・\rensuji*{12}・\rensuji*{0}}\end{shiika}
\vspace{0.6cm}
相川の家にて
\begin{shiika}太りゆく大根今日も抜き惜しみ
\hfill{\rensuji*{54}・\rensuji*{12}・\rensuji*{0}}\end{shiika}
\begin{shiika}実むらさき実生をたのむ土かぶせ
\hfill{\rensuji*{54}・\rensuji*{12}・\rensuji*{0}}\end{shiika}
\begin{shiika}青木の実名知らぬ鳥も枝くぐり
\hfill{\rensuji*{54}・\rensuji*{12}・\rensuji*{0}}\end{shiika}
\vspace{0.6cm}
新年謡の会
\begin{shiika}心地よき帯のしまりや謡ひ初め
\hfill{\rensuji*{55}・\rensuji*{1}・\rensuji*{0}}\end{shiika}
\vspace{0.6cm}
安藤さん青山さんと淡路島 健和荘で新年を過ごす 渡船のおり
\begin{shiika}新年の交す汽笛に群れ鴎
\hfill{\rensuji*{55}・\rensuji*{1}・\rensuji*{1}}\end{shiika}
\vspace{0.6cm}
村上ぬいさんの急逝
\begin{shiika}通夜の冷え遺作のばら絵明るきも
\hfill{\rensuji*{55}・\rensuji*{3}・\rensuji*{0}}\end{shiika}
\begin{shiika}出棺す白梅こぼる砂踏みて
\hfill{\rensuji*{55}・\rensuji*{3}・\rensuji*{0}}\end{shiika}
\vspace{0.6cm}
相川の家
\begin{shiika}雨戸くる朝なあさなを蕗育つ
\hfill{\rensuji*{55}・\rensuji*{4}・\rensuji*{0}}\end{shiika}
\begin{shiika}菜園の菊菜色よし久の子に
\hfill{\rensuji*{55}・\rensuji*{4}・\rensuji*{0}}\end{shiika}
\vspace{0.6cm}
浅野繁雄さんご他界 小森田さん入院
\begin{shiika}青葉して忌ごもる友と病める友
\hfill{\rensuji*{55}・\rensuji*{5}・\rensuji*{0}}\end{shiika}
\vspace{0.6cm}
小豆島国民宿舎(池田)に集まりて
\begin{shiika}明易し潮騒近き島の宿
\hfill{\rensuji*{55}・\rensuji*{6}・\rensuji*{0}}\end{shiika}
\vspace{0.6cm}
\begin{shiika}島の雷止みて翼船ましぐら
\hfill{\rensuji*{55}・\rensuji*{6}・\rensuji*{1}}\end{shiika}
\vspace{0.6cm}
竹四郎病む
\begin{shiika}梅雨嵐し離れ病む子をただ祈る
\hfill{\rensuji*{55}・\rensuji*{6}・\rensuji*{0}}\end{shiika}
\vspace{0.6cm}
海南 林満喜子さん宅を訪ねて
\begin{shiika}見送られ見返る薄暮白あやめ\\
海道先生が第一位にとってくださった
\hfill{\rensuji*{55}・\rensuji*{6}・\rensuji*{0}}\end{shiika}
\vspace{0.6cm}
整の昼寝 私のひるね
\begin{shiika}健やかな孫の寝息やプール焼け
\hfill{\rensuji*{55}・\rensuji*{8}・\rensuji*{0}}\end{shiika}
\begin{shiika}草引きて草の匂ひの手枕寝
\hfill{\rensuji*{55}・\rensuji*{8}・\rensuji*{0}}\end{shiika}
\vspace{0.6cm}
あわくら温泉に幡井さんと行く店の決算をすませて
\begin{shiika}水引の紅ぬれづめに水車
\hfill{\rensuji*{55}・\rensuji*{9}・\rensuji*{0}}\end{shiika}
\begin{shiika}みのり田の道登校のペダル踏む
\hfill{\rensuji*{55}・\rensuji*{9}・\rensuji*{0}}\end{shiika}
\begin{shiika}温泉涼し重き一事を成しとげて
\hfill{\rensuji*{55}・\rensuji*{9}・\rensuji*{0}}\end{shiika}
\vspace{0.6cm}
山下さんと退院した小森田さんを名古屋に訪ねて
\begin{shiika}退院の友いきいきと派手浴衣
\hfill{\rensuji*{55}・\rensuji*{7}・\rensuji*{17}}\end{shiika}
\vspace{0.6cm}
大川一善 安子さんの車で信穂高 木曽濁河温泉
\begin{shiika}ダム澄める揺れ映りいる合歓の花
\hfill{双適\rensuji*{55}・\rensuji*{8}・\rensuji*{2}}\end{shiika}
\begin{shiika}露天湯の一灯淡く月見草
\hfill{双適\rensuji*{55}・\rensuji*{0}・\rensuji*{3}}\end{shiika}
\begin{shiika}霊峰の碧に真向ひ秋ざくら
\hfill{\rensuji*{55}・\rensuji*{8}・\rensuji*{4}}\end{shiika}
\vspace{0.6cm}
私の誕生祝として大台ケ原へ一善安子さんがドライブしてくれた。
紅葉が盛りの山々プロ野球日本シリーズ広島優勝のラヂオをききつつ
\begin{shiika}先急ぎつつ仰ぎゆく峯紅葉
\hfill{\rensuji*{55}・\rensuji*{11}・\rensuji*{2}}\end{shiika}
\vspace{0.6cm}
相川の住居
\begin{shiika}しみじみと語らな白菊活けて待つ
\hfill{\rensuji*{55}・\rensuji*{12}・\rensuji*{0}}\end{shiika}
\begin{shiika}遠き旅はなやぎ帰り菊を焚く
\hfill{\rensuji*{55}・\rensuji*{12}・\rensuji*{0}}\end{shiika}
\begin{shiika}枯菊を焚きつつしばし物思ひ
\hfill{\rensuji*{55}・\rensuji*{12}・\rensuji*{0}}\end{shiika}
\begin{shiika}鉄橋を渡れば小駅片時雨
\hfill{\rensuji*{55}・\rensuji*{12}・\rensuji*{0}}\end{shiika}
\begin{shiika}黄の翅の止り色増す実むらさき
\hfill{\rensuji*{55}・\rensuji*{11}・\rensuji*{0}}\end{shiika}
\begin{shiika}天高し施肥よく効きし畑の色
\hfill{\rensuji*{55}・\rensuji*{11}・\rensuji*{0}}\end{shiika}
\vspace{0.6cm}
七草粥
\begin{shiika}七草の数揃はねど畑の菜を
\hfill{\rensuji*{56}・\rensuji*{1}・\rensuji*{0}}\end{shiika}
\vspace{0.6cm}
幡井さんと焼津 学保に庭からの一望焼津港
\begin{shiika}一望に漁港おさめて梅の丘
\hfill{\rensuji*{56}・\rensuji*{1}・\rensuji*{30}}\end{shiika}
\vspace{0.6cm}
浅野房子さんを訪ねて近くの温泉で一夜を
\begin{shiika}春炬燵尽きぬ話の果は伏し
\hfill{\rensuji*{56}・\rensuji*{3}・\rensuji*{0}}\end{shiika}
\vspace{0.6cm}
\begin{shiika}春の冷え別れて一人立つ小駅
\hfill{\rensuji*{56}・\rensuji*{3}・\rensuji*{0}}\end{shiika}
\vspace{0.6cm}
安子さんが井高野の手伝いを止めることについて一善の言い方処置に納得が
出来ない 筋の通らないことに妥協出来ない私の性
\begin{shiika}争ひてふと空しかり梅の闇
\hfill{\rensuji*{56}・\rensuji*{3}・\rensuji*{0}}\end{shiika}
\vspace{0.6cm}
飯田知子短大入学祝い
\begin{shiika}合格の祝袋は字も太く
\hfill{\rensuji*{56}・\rensuji*{3}・\rensuji*{0}}\end{shiika}
\vspace{0.6cm}
相川家
\begin{shiika}摘みし蕗独りの厨たのしかり
\hfill{\rensuji*{56}・\rensuji*{4}・\rensuji*{0}}\end{shiika}
\vspace{0.6cm}
\begin{shiika}散る桜庭の胸像ただ黙し
\hfill{\rensuji*{56}・\rensuji*{4}・\rensuji*{0}}\end{shiika}
\vspace{0.6cm}
\begin{shiika}武具飾る子は父となり遠くあり
\hfill{\rensuji*{56}・\rensuji*{4}・\rensuji*{0}}\end{shiika}
\vspace{0.6cm}
真鍋先生の鮎のこと 市原さんのご主人の釣りのこと
\begin{shiika}解禁の夕べたまはる吉野鮎
\hfill{\rensuji*{56}・\rensuji*{5}・\rensuji*{0}}\end{shiika}
\begin{shiika}釣りし鮒川に戻して春の風
\hfill{\rensuji*{56}・\rensuji*{5}・\rensuji*{0}}\end{shiika}
\vspace{0.6cm}
上京車中
\begin{shiika}冨士聳ゆ裾野の町の鯉のぼり
\hfill{\rensuji*{56}・\rensuji*{0}・\rensuji*{0}}\end{shiika}
\vspace{0.6cm}
養老の滝へ
\begin{shiika}滝水をコップに汲みて喉しまる
\hfill{\rensuji*{56}・\rensuji*{7}・\rensuji*{0}}\end{shiika}
\vspace{0.6cm}
相川地蔵まつり
\begin{shiika}御詠歌の流れへいそぐ地蔵盆
\hfill{\rensuji*{56}・\rensuji*{8}・\rensuji*{0}}\end{shiika}
\vspace{0.6cm}
児玉正志さん急の来客
\begin{shiika}枝豆に酌みて不意なる遠き客
\hfill{\rensuji*{56}・\rensuji*{9}・\rensuji*{0}}\end{shiika}
\vspace{0.6cm}
市原さんご夫妻の釣り
\begin{shiika}釣る夫の片辺に妻の秋日傘
\hfill{\rensuji*{56}・\rensuji*{10}・\rensuji*{0}}\end{shiika}
\vspace{0.6cm}
高松高女のクラス会 萩 津和野
\begin{shiika}武家屋敷崩れ土塀に石蕗盛り
\hfill{\rensuji*{56}・\rensuji*{10}・\rensuji*{22}}\end{shiika}
\begin{shiika}草子里時雨れる朝の大き虹
\hfill{\rensuji*{56}・\rensuji*{10}・\rensuji*{24}}\end{shiika}
\vspace{0.6cm}
遂に一善があやまりに来た 貞子の五十年忌法要が近ずいて
\begin{shiika}わだかまり解けて減りゆく盛みかん
\hfill{\rensuji*{56}・\rensuji*{10}・\rensuji*{0}}\end{shiika}
\vspace{0.6cm}
\begin{shiika}・
\hfill{\rensuji*{56}・\rensuji*{11}・\rensuji*{0}}\end{shiika}
\vspace{0.6cm}
相川の岩橋家近くの火事のあと
\begin{shiika}売地札草にかくれて秋暮るる
\hfill{\rensuji*{56}・\rensuji*{11}・\rensuji*{0}}\end{shiika}
\vspace{0.6cm}
相川の家 私の誕生日
\begin{shiika}栗おこわ我が誕生は頃もよく
\hfill{\rensuji*{56}・\rensuji*{11}・\rensuji*{0}}\end{shiika}
\begin{shiika}霜よけにレタス生々玉巻ける
\hfill{\rensuji*{56}・\rensuji*{11}・\rensuji*{0}}\end{shiika}
\begin{shiika}供華の菊剪りためらひぬ眠り蝶
\hfill{\rensuji*{56}・\rensuji*{11}・\rensuji*{0}}\end{shiika}
\vspace{0.6cm}
\begin{shiika}落葉炊く煙の中に思ふこと 
\hfill{\rensuji*{56}・\rensuji*{11}・\rensuji*{0}}\end{shiika}
\begin{shiika}新らしく菊きり供え旅に出る 
\hfill{\rensuji*{56}・\rensuji*{11}・\rensuji*{0}}\end{shiika}
\vspace{0.6cm}
鎌倉 お寺の名前を忘れたが
\begin{shiika}踏み惜しみつつ鎌倉の銀杏黄葉
\hfill{\rensuji*{56}・\rensuji*{11}・\rensuji*{24}}\end{shiika}
\vspace{0.6cm}
師走の姿
\begin{shiika}ウインドに背まるく映る師走町
\hfill{\rensuji*{56}・\rensuji*{12}・\rensuji*{0}}\end{shiika}
\vspace{0.6cm}
直紀 年末相川にきて手伝ってくれる
\begin{shiika}晦日そば孫の食べざま頼もしく
\hfill{\rensuji*{56}・\rensuji*{12}・\rensuji*{0}}\end{shiika}
\vspace{0.6cm}
上京 成城の家
\begin{shiika}窓の梅ほころびゆくをみるしじま
\hfill{\rensuji*{57}・\rensuji*{2}・\rensuji*{0}}\end{shiika}
\begin{shiika}散り梅のかかり濯ぎのもの乾く
\hfill{\rensuji*{57}・\rensuji*{2}・\rensuji*{0}}\end{shiika}
\vspace{0.6cm}
八百様を訪ねて
\begin{shiika}春遠しこもれる叔母に京の菓子
\hfill{\rensuji*{57}・\rensuji*{2}・\rensuji*{0}}\end{shiika}
\vspace{0.6cm}
海南の林さん受験(阪大)で泊まる
\begin{shiika}受験生泊めて祈りを同心に
\hfill{\rensuji*{57}・\rensuji*{3}・\rensuji*{0}}\end{shiika}
\vspace{0.6cm}
%-------------------------------------------------------------
相川の橋より
\begin{shiika}日脚伸ぶ中洲に群れる鳥の白
\hfill{\rensuji*{57}・\rensuji*{3}・\rensuji*{0}}\end{shiika}
\begin{shiika}蕗の薹焼みその香の朝厨
\hfill{\rensuji*{57}・\rensuji*{3}・\rensuji*{0}}\end{shiika}
\vspace{0.6cm}
仲塚の案内 垂水神社
\begin{shiika}散る花の流れゆくあり踏まるあり
\hfill{\rensuji*{57}・\rensuji*{4}・\rensuji*{0}}\end{shiika}
\vspace{0.6cm}
郷生と小田原城
\begin{shiika}天主より振る手呼ぶ声花の中
\hfill{\rensuji*{57}・\rensuji*{4}・\rensuji*{0}}\end{shiika}
\vspace{0.6cm}
相川の畑の垣超し中島さんのお嬢さん
\begin{shiika}葱坊主垣越しの子はよくしゃべる
\hfill{\rensuji*{57}・\rensuji*{5}・\rensuji*{0}}\end{shiika}
\begin{shiika}耳遠く笑顔で応ふ木の芽雨
\hfill{\rensuji*{57}・\rensuji*{5}・\rensuji*{0}}\end{shiika}
\vspace{0.6cm}
一善 安子さんと早発して青山高原にドライブ
それは伊賀上野方面への
再ドライブだった
その数日前 室生寺に之も早朝
出かけてたくさんの写真を撮ったつもりが、カメラはフイルムが
入っていなかった。 わざわざ伊賀上野 百合子宅まで訪れたのにい
 室生寺門前で草餅を買う 時間はまだまだ昼前 大野寺で昼弁当を
いただき相談は急に伊賀上野へ
\begin{shiika}草餅にふと道変へて娘に急ぐ
\hfill{\rensuji*{57}・\rensuji*{5}・\rensuji*{0}}\end{shiika}
\vspace{0.6cm}
小汐さん 増田さん 伊藤さん あわくら荘より鳥取砂丘 磨?寺へ
\begin{shiika}直ぐ消ゆる足跡砂に五月旅
\hfill{\rensuji*{57}・\rensuji*{5}・\rensuji*{0}}\end{shiika}
\begin{shiika}風光る砂丘を踏めば若返る
\hfill{\rensuji*{57}・\rensuji*{5}・\rensuji*{0}}\end{shiika}
\begin{shiika}石段のあえぎに著莪の花やさし
\hfill{\rensuji*{57}・\rensuji*{5}・\rensuji*{0}}\end{shiika}
\vspace{0.6cm}
岐阜羽島へ行ったとき
\begin{shiika}単線の停車は長し青田風
\hfill{\rensuji*{57}・\rensuji*{6}・\rensuji*{0}}\end{shiika}
\vspace{0.6cm}
思い出湖岸の旅
\begin{shiika}花栗の香に堂守の鍵開く
\hfill{\rensuji*{57}・\rensuji*{7}・\rensuji*{0}}\end{shiika}
\begin{shiika}老鴬や堂守力こめて説く
\hfill{\rensuji*{57}・\rensuji*{7}・\rensuji*{0}}\end{shiika}
\vspace{0.6cm}
北海道旅行
%======================================================
\begin{shiika}知床の大雪渓に昼の月
\hfill{\rensuji*{57}・\rensuji*{0}・\rensuji*{0}}\end{shiika}
\begin{shiika}雪渓を映し知床五湖寂と 
\hfill{\rensuji*{57}・\rensuji*{0}・\rensuji*{0}}\end{shiika}
\begin{shiika}えぞかんぞう岬はるかは異国なる
\hfill{\rensuji*{57}・\rensuji*{0}・\rensuji*{0}}\end{shiika}
\begin{shiika}昆布乾すさいはての島明易し 
\hfill{\rensuji*{57}・\rensuji*{0}・\rensuji*{0}}\end{shiika}
\begin{shiika}獅子独活の花眼の限り・
\hfill{\rensuji*{57}・\rensuji*{0}・\rensuji*{0}}\end{shiika}
\vspace{0.6cm}
成城の家 笹倉の庭に鷺草が
\begin{shiika}鷺草の鷺二羽となる娘に甘え
\hfill{双適\rensuji*{57}・\rensuji*{7}・\rensuji*{0}}\end{shiika}
\vspace{0.6cm}
相川の最後の夏
\begin{shiika}魂迎ふ一人となりて古家守る
\hfill{\rensuji*{57}・\rensuji*{8}・\rensuji*{0}}\end{shiika}
\vspace{0.6cm}
%==============================================
\begin{shiika}手ごなしで土をかぶせる秋の種
\hfill{\rensuji*{57}・\rensuji*{8}・\rensuji*{0}}\end{shiika}
\begin{shiika}十指もて土をかぶせる秋の種
\hfill{\rensuji*{57}・\rensuji*{8}・\rensuji*{0}}\end{shiika}
\begin{shiika}豪雷にいさかふ妹弟抱き合ふ
\hfill{\rensuji*{57}・\rensuji*{8}・\rensuji*{0}}\end{shiika}
%----------------------------furikana---------------------
\begin{shiika}亡娘ノート\Kana{紙,魚}{し,み}生きている悲しさよ
\hfill{\rensuji*{57}・\rensuji*{9}・\rensuji*{0}}\end{shiika}
\begin{shiika}秋立ちぬ束ねてさせり亡母の櫛
\hfill{\rensuji*{57}・\rensuji*{9}・\rensuji*{0}}\end{shiika}
\begin{shiika}晩菊の咲くや明日より他人の庭
\hfill{\rensuji*{57}・\rensuji*{10}・\rensuji*{0}}\end{shiika}
\begin{shiika}引き越しの荷隅にかばふ冬すみれ
\hfill{\rensuji*{57}・\rensuji*{10}・\rensuji*{0}}\end{shiika}
\begin{shiika}秋そゞろ引越荷物嵩む部屋
\hfill{\rensuji*{57}・\rensuji*{10}・\rensuji*{0}}\end{shiika}
\vspace{0.6cm}
%--------------------------------------------------
\chapter {水無瀬}
水無瀬に移り来て
\begin{shiika}秋風も他人もやさし移り住み
\hfill{\rensuji*{57}・\rensuji*{11}・\rensuji*{0}}\end{shiika}
\begin{shiika}見捨てかね新居に挿せり倒れ菊
\hfill{\rensuji*{57}・\rensuji*{11}・\rensuji*{0}}\end{shiika}
\vspace{0.6cm}
幡井さんと山代温泉国家公務員保養所
\begin{shiika}寛ぎて見る山荘の紅葉濃し
\hfill{\rensuji*{57}・\rensuji*{11}・\rensuji*{0}}\end{shiika}
\vspace{0.6cm}
水無瀬相川通勤 相川の駅のホーム
\begin{shiika}乗りおくれくやしき顔に冬の月
\hfill{\rensuji*{57}・\rensuji*{11}・\rensuji*{0}}\end{shiika}
\vspace{0.6cm}
水無瀬の日々
\begin{shiika}寒椿にぶる起ち居のすべもなく
\hfill{\rensuji*{57}・\rensuji*{12}・\rensuji*{0}}\end{shiika}
\vspace{0.6cm}
\begin{shiika}友呼ばむ一人に余る日向ぼこ
\hfill{\rensuji*{57}・\rensuji*{12}・\rensuji*{0}}\end{shiika}
\vspace{0.6cm}
相川の庭
\begin{shiika}転宅の迫りし庭の実むらさき
\hfill{\rensuji*{57}・\rensuji*{10}・\rensuji*{0}}\end{shiika}
\begin{shiika}移り住む名残の菊香衰えず
\hfill{\rensuji*{57}・\rensuji*{10}・\rensuji*{0}}\end{shiika}
\vspace{0.6cm}
伊勢への旅の時を思い出して
\begin{shiika}玉砂利に歩の乱れなし神の留守
\hfill{\rensuji*{57}・\rensuji*{12}・\rensuji*{0}}\end{shiika}
\vspace{0.6cm}
%----------------------------------
喜美子 聖子にはなさんと
\begin{shiika}大役の初旅冨士が雲間より
\hfill{\rensuji*{58}・\rensuji*{1}・\rensuji*{3}}\end{shiika}
\vspace{0.6cm}
日野百草園にて
\begin{shiika}梅日和白壁光る村一望
\hfill{\rensuji*{58}・\rensuji*{2}・\rensuji*{0}}\end{shiika}
\vspace{0.6cm}
水無瀬
\begin{shiika}しつけとる春立つ朝の装ひに
\hfill{\rensuji*{58}・\rensuji*{3}・\rensuji*{0}}\end{shiika}
\begin{shiika}水ぬるむ就職決り紅さす娘
\hfill{\rensuji*{58}・\rensuji*{3}・\rensuji*{0}}\end{shiika}
\begin{shiika}桜餅娘の訪ひくれし小半日
\hfill{\rensuji*{58}・\rensuji*{3}・\rensuji*{0}}\end{shiika}
\begin{shiika}目口なき紙の雛や掌になじむ
\hfill{\rensuji*{58}・\rensuji*{3}・\rensuji*{0}}\end{shiika}
\vspace{0.6cm}
高田さん弔問
\begin{shiika}裏の家の雨に堪へ咲く八重桜
\hfill{\rensuji*{58}・\rensuji*{4}・\rensuji*{0}}\end{shiika}
\begin{shiika}友の情雨に摘みきしわらび飯
\hfill{\rensuji*{58}・\rensuji*{4}・\rensuji*{0}}\end{shiika}
\begin{shiika}忌に集るしのぶ日がなを花の雨
\hfill{\rensuji*{58}・\rensuji*{4}・\rensuji*{0}}\end{shiika}
\vspace{0.6cm}
水無瀬楠公通の大楠像が学校庭に移し植え
\begin{shiika}除り去らる囀り包む街の樹が
\hfill{\rensuji*{58}・\rensuji*{4}・\rensuji*{0}}\end{shiika}
\begin{shiika}読むも憂し眺むも憂しや花の雨
\hfill{\rensuji*{58}・\rensuji*{4}・\rensuji*{0}}\end{shiika}
\begin{shiika}集ればお国訛よよもぎ餅
\hfill{\rensuji*{58}・\rensuji*{4}・\rensuji*{0}}\end{shiika}
\vspace{0.6cm}
%-------------------------------------
秩父路 高松高女の皆さんと
\begin{shiika}秩父路につづく芽桑の夕映えて
\hfill{\rensuji*{58}・\rensuji*{4}・\rensuji*{7}}\end{shiika}
\vspace{0.6cm}
一善と一言神社へ
\begin{shiika}万緑や一言神に願一つ
\hfill{\rensuji*{58}・\rensuji*{5}・\rensuji*{21}}\end{shiika}
\begin{shiika}田植機の若者帽子に赤い花
\hfill{\rensuji*{58}・\rensuji*{5}・\rensuji*{0}}\end{shiika}
\vspace{0.6cm}
%-------------------------------------
文友会 東北の旅
\begin{shiika}桜桃たわわの国へ喜寿の旅
\hfill{\rensuji*{58}・\rensuji*{5}・\rensuji*{0}}\end{shiika}
\vspace{0.6cm}
西川さん 水無瀬に迎えて
\begin{shiika}杖たよる友出迎へに梅雨はげし
\hfill{\rensuji*{58}・\rensuji*{6}・\rensuji*{11}}\end{shiika}
\vspace{0.6cm}
水無瀬
\begin{shiika}朝涼し咲きつぐ花を供華日記
\hfill{\rensuji*{58}・\rensuji*{0}・\rensuji*{0}}\end{shiika}
\begin{shiika}引き越して来たる浜木綿咲き安堵
\hfill{\rensuji*{58}・\rensuji*{0}・\rensuji*{0}}\end{shiika}
\begin{shiika}娘三人訪ひくれ風鈴よく鳴れり
\hfill{\rensuji*{58}・\rensuji*{0}・\rensuji*{0}}\end{shiika}
\begin{shiika}一族の年長となり魂まつる
\hfill{\rensuji*{58}・\rensuji*{8}・\rensuji*{0}}\end{shiika}
\vspace{0.6cm}
阪急\rensuji*{32}番街 皆美にて、竹四郎 喜美子と食事
\begin{shiika}動かぬ灯動く灯一望盆の果
\hfill{\rensuji*{58}・\rensuji*{8}・\rensuji*{0}}\end{shiika}
\begin{shiika}洗ひ髪立つベランダの風は秋
\hfill{\rensuji*{58}・\rensuji*{8}・\rensuji*{0}}\end{shiika}
\vspace{0.6cm}
山下さん 高田さん 駒ヶ根車山ペンsyングリーンスポット巡り」
\begin{shiika}蕎麦三日食べてさわやか信濃旅
\hfill{\rensuji*{58}・\rensuji*{9}・\rensuji*{4}}\end{shiika}
\vspace{0.6cm}
安藤さんと三方五湖
\begin{shiika}色鳥や岳に真向ふ湖の宿
\hfill{\rensuji*{58}・\rensuji*{9}・\rensuji*{0}}\end{shiika}
\begin{shiika}大き鳥湖上を舞ひて夏去れり
\hfill{\rensuji*{58}・\rensuji*{9}・\rensuji*{0}}\end{shiika}
\vspace{0.6cm}
箕面観光ホテル別館 桂 謡に会
\begin{shiika}庭紅葉もえて謡に力声
\hfill{\rensuji*{58}・\rensuji*{11}・\rensuji*{0}}\end{shiika}
\begin{shiika}謡ひ果て山荘黄葉をのこし暮る
\hfill{\rensuji*{58}・\rensuji*{11}・\rensuji*{0}}\end{shiika}
\vspace{0.6cm}
水無瀬折々
\begin{shiika}翅やすむ蝶もむらさき式部の実
\hfill{\rensuji*{58}・\rensuji*{11}・\rensuji*{0}}\end{shiika}
\begin{shiika}独り居のよき日淋し日菊挿して
\hfill{\rensuji*{58}・\rensuji*{1}・\rensuji*{0}}\end{shiika}
\begin{shiika}疎く住み安けき日々や杜鵤草
\hfill{\rensuji*{58}・\rensuji*{11}・\rensuji*{0}}\end{shiika}
\vspace{0.6cm}
成城の金魚
\begin{shiika}屑金魚育ち掬ひし児も少年
\hfill{\rensuji*{58}・\rensuji*{11}・\rensuji*{0}}\end{shiika}
\vspace{0.6cm}
伊藤さん八田さn清川さん 京都の紅葉案内
\begin{shiika}案内三日京の紅葉に酔ひ疲る
\hfill{\rensuji*{58}・\rensuji*{11}・\rensuji*{0}}\end{shiika}
\vspace{0.6cm}
\begin{shiika}照紅葉京一望の峯の寺
\hfill{\rensuji*{58}・\rensuji*{11}・\rensuji*{0}}\end{shiika}
\vspace{0.6cm}
高田さん宅に小森田さん 小田さんと 山荘和周庵 落成
\begin{shiika}山荘の集ひに菜飯冬ぬくし
\hfill{\rensuji*{58}・\rensuji*{12}・\rensuji*{9}}\end{shiika}
\begin{shiika}冬入日竹叢透し荘なごむ
\hfill{\rensuji*{58}・\rensuji*{12}・\rensuji*{9}}\end{shiika}
\vspace{0.6cm}
水無瀬元旦
\begin{shiika}一とせを会ひ得ぬ人の賀状増し
\hfill{\rensuji*{59}・\rensuji*{1}・\rensuji*{1}}\end{shiika}
\begin{shiika}しきたりをつづけて独り屠蘇機嫌
\hfill{\rensuji*{59}・\rensuji*{1}・\rensuji*{0}}\end{shiika}
\vspace{0.6cm}
安藤さんと三方五湖へ北陸線
\begin{shiika}トンネルを抜ける度雪深くなり
\hfill{\rensuji*{59}・\rensuji*{1}・\rensuji*{2}}\end{shiika}
\vspace{0.6cm}
水無瀬のシンビジュームがさく
\begin{shiika}ただいまと灯せば応ふ室の花
\hfill{\rensuji*{59}・\rensuji*{2}・\rensuji*{0}}\end{shiika}
\vspace{0.6cm}
水無瀬に石井晴美さんを迎え ?る枝の友
\begin{shiika}ちゃん呼びで遠き日戻る木の葉髪
\hfill{\rensuji*{59}・\rensuji*{2}・\rensuji*{0}}\end{shiika}
\vspace{0.6cm}
%-----------------------------------------------------------------
富田の駅で乗り換えの時 相川の古いお客様と出会う
\begin{shiika}春寒やぱったり出会ひ出ぬ名前
\hfill{\rensuji*{59}・\rensuji*{2}・\rensuji*{0}}\end{shiika}
\vspace{0.6cm}
直紀 郷生 一善に質問されて
\begin{shiika}争ひも夢よ首塚土筆の芽
\hfill{\rensuji*{59}・\rensuji*{3}・\rensuji*{0}}\end{shiika}
\vspace{0.6cm}
防府 藤本悦子さん宅 (藤本様とはこれが最后の出会いになる)
\begin{shiika}老夫婦夜をぼつぼつとひなあられ
\hfill{\rensuji*{59}・\rensuji*{3}・\rensuji*{3}}\end{shiika}
\vspace{0.6cm}
山下さんと湯布院 亀の井 別荘二泊
\begin{shiika}雪解風由布岳さして大鴉
\hfill{\rensuji*{59}・\rensuji*{3}・\rensuji*{5}}\end{shiika}
\vspace{0.6cm}
水無瀬折々
\begin{shiika}土を割る花芽それぞれ色ありて
\hfill{\rensuji*{59}・\rensuji*{3}・\rensuji*{0}}\end{shiika}
\begin{shiika}によきによきと花芽ラッシュの庭の土
\hfill{\rensuji*{59}・\rensuji*{3}・\rensuji*{0}}\end{shiika}
\begin{shiika}花苺児にしやがみ見す芯の粒
\hfill{\rensuji*{59}・\rensuji*{4}・\rensuji*{0}}\end{shiika}
\begin{shiika}朝毎の独りに足りる庭苺
\hfill{\rensuji*{59}・\rensuji*{5}・\rensuji*{0}}\end{shiika}
\begin{shiika}団地住みテレビの上の兜の威
\hfill{\rensuji*{59}・\rensuji*{5}・\rensuji*{0}}\end{shiika}
\begin{shiika}ホース先そらせばそこも青蛙
\hfill{\rensuji*{59}・\rensuji*{7}・\rensuji*{0}}\end{shiika}
\vspace{0.6cm}
水無瀬の庭の青蛙はなつかしい お隣佐藤さんに嬰誕生
\begin{shiika}花南天隣初嬰の襁褓干す
\hfill{\rensuji*{59}・\rensuji*{7}・\rensuji*{0}}\end{shiika}
\begin{shiika}待ちつつも一人を凉しと思ふ日も
\hfill{\rensuji*{59}・\rensuji*{8}・\rensuji*{0}}\end{shiika}
\begin{shiika}庭茂り払ふ枝にもある生命
\hfill{\rensuji*{59}・\rensuji*{8}・\rensuji*{0}}\end{shiika}
\begin{shiika}孫の名をとりちがえ呼ぶ盆家族
\hfill{\rensuji*{59}・\rensuji*{8}・\rensuji*{0}}\end{shiika}
\begin{shiika}夏萩に誰みくじ結ふ禁よそに
\hfill{\rensuji*{59}・\rensuji*{8}・\rensuji*{0}}\end{shiika}
\vspace{0.6cm}
悦子さん宅へ弔問
\begin{shiika}忌ごもりの友訪ひて汨つ戻り梅雨
\hfill{\rensuji*{59}・\rensuji*{7}・\rensuji*{0}}\end{shiika}
\vspace{0.6cm}
%-----------------------------------------------
山下さん 小森田さん と小海線から草津野友湖
\begin{shiika}夏書終へ東塔西塔仰ぐ朝
\hfill{\rensuji*{59}・\rensuji*{9}・\rensuji*{0}}\end{shiika}
\begin{shiika}空と無の多き夏書や朝鴉
\hfill{\rensuji*{59}・\rensuji*{9}・\rensuji*{0}}\end{shiika}
\begin{shiika}りんどうや標高識のたつ小駅
\hfill{\rensuji*{59}・\rensuji*{9}・\rensuji*{0}}\end{shiika}
\begin{shiika}高原列車おそしとゆれる花すすき
\hfill{\rensuji*{59}・\rensuji*{9}・\rensuji*{0}}\end{shiika}
\begin{shiika}紫の小波たてり松虫草
\hfill{\rensuji*{59}・\rensuji*{9}・\rensuji*{0}}\end{shiika}
\begin{shiika}思はざる遠冨士すゝきの小窓より
\hfill{\rensuji*{59}・\rensuji*{9}・\rensuji*{0}}\end{shiika}
\vspace{0.6cm}
滿藤さん宅 のうぜん花
\begin{shiika}朝風に彩をひろげてのうぜん花
\hfill{\rensuji*{59}・\rensuji*{9}・\rensuji*{0}}\end{shiika}
\vspace{0.6cm}
上野城 百合子出品を見に行く
\begin{shiika}風凉し天主の床の黒光り
\hfill{\rensuji*{59}・\rensuji*{8}・\rensuji*{0}}\end{shiika}
\vspace{0.6cm}
\begin{shiika}俳聖殿忍者屋敷も蝉しぐれ
\hfill{\rensuji*{59}・\rensuji*{8}・\rensuji*{0}}\end{shiika}
\vspace{0.6cm}
道成寺」白浜三段壁
\begin{shiika}秋凉し絵とき説法に笑ひあり
\hfill{\rensuji*{59}・\rensuji*{9}・\rensuji*{19}}\end{shiika}
\begin{shiika}水軍の洞の跡や秋の潮
\hfill{\rensuji*{59}・\rensuji*{9}・\rensuji*{19}}\end{shiika}
\vspace{0.6cm}
水無瀬盆踊り
%---------------------------------------------------------
\begin{shiika}青い眼の手ぶりに見入る踊の輪
\hfill{\rensuji*{59}・\rensuji*{8}・\rensuji*{0}}\end{shiika}
\begin{shiika}諷刺歌踊りの櫓は高調し
\hfill{\rensuji*{59}・\rensuji*{8}・\rensuji*{0}}\end{shiika}
\begin{shiika}送り火やもとの一人に戻る夜
\hfill{\rensuji*{59}・\rensuji*{8}・\rensuji*{0}}\end{shiika}
\vspace{0.6cm}
直紀の成人に感じたこと
\begin{shiika}帰省子の言葉大人ひふと淋し
\hfill{\rensuji*{59}・\rensuji*{8}・\rensuji*{0}}\end{shiika}
\begin{shiika}若者となるは別れか鳥雲に
\hfill{\rensuji*{59}・\rensuji*{8}・\rensuji*{0}}\end{shiika}
\vspace{0.6cm}
箱根?保にて
\begin{shiika}夏霧の湧きて流れて山の湖
\hfill{\rensuji*{59}・\rensuji*{7}・\rensuji*{0}}\end{shiika}
\vspace{0.6cm}
小川先生宅の山茶花
\begin{shiika}山茶花の垣咲き始めぬ謡声
\hfill{\rensuji*{59}・\rensuji*{11}・\rensuji*{0}}\end{shiika}
\vspace{0.6cm}
吉川三郎さんを高槻の病院に見舞う
\begin{shiika}冬の雲まこと知らせぬ人見舞ふ
\hfill{\rensuji*{59}・\rensuji*{11}・\rensuji*{0}}\end{shiika}
\vspace{0.6cm}
水無瀬年忘れ
\begin{shiika}年忘れ流す憂さなきワインの香
\hfill{\rensuji*{59}・\rensuji*{12}・\rensuji*{0}}\end{shiika}
\begin{shiika}賀状書く亡母の字に似る母の年令
\hfill{\rensuji*{59}・\rensuji*{12}・\rensuji*{0}}\end{shiika}
\begin{shiika}寄せ鍋の沸々はずむ故郷ことば
\hfill{\rensuji*{59}・\rensuji*{12}・\rensuji*{0}}\end{shiika}
\begin{shiika}するつと食ぶ熟柿に郷愁そぞろ湧く
\hfill{\rensuji*{59}・\rensuji*{12}・\rensuji*{0}}\end{shiika}
\vspace{0.6cm}
私の誕生日 水無瀬
\begin{shiika}吾が誕生秋刀魚で祝ひ心足る
\hfill{\rensuji*{59}・\rensuji*{11}・\rensuji*{0}}\end{shiika}
\vspace{0.6cm}
成城の新年
\begin{shiika}初冨士や大東京の隅に住み
\hfill{\rensuji*{60}・\rensuji*{1}・\rensuji*{0}}\end{shiika}
\vspace{0.6cm}
大阪への帰途
\begin{shiika}林立の煙突冨士に初煙
\hfill{\rensuji*{60}・\rensuji*{1}・\rensuji*{0}}\end{shiika}
\begin{shiika}初仕事裾野の町の白煙
\hfill{\rensuji*{60}・\rensuji*{1}・\rensuji*{0}}\end{shiika}
\vspace{0.6cm}
水無瀬
\begin{shiika}移し植え三年の梅に初つぼみ
\hfill{\rensuji*{60}・\rensuji*{2}・\rensuji*{0}}\end{shiika}
\begin{shiika}陽を集め日毎ふくらむ木瓜の花
\hfill{\rensuji*{60}・\rensuji*{2}・\rensuji*{0}}\end{shiika}
\begin{shiika}蘭匂ふ独りの部屋に惜しき程
\hfill{\rensuji*{60}・\rensuji*{3}・\rensuji*{0}}\end{shiika}
\vspace{0.6cm}
小田様のお嬢さま御他界 弔問
\begin{shiika}逆縁の香たく背なに春空し
\hfill{\rensuji*{60}・\rensuji*{2}・\rensuji*{0}}\end{shiika}
\vspace{0.6cm}
水無瀬
\begin{shiika}春や憂し着かえし裾の静電気
\hfill{\rensuji*{60}・\rensuji*{4}・\rensuji*{0}}\end{shiika}
\begin{shiika}割れ込まれ句心とぎれぬ春炬燵
\hfill{\rensuji*{60}・\rensuji*{3}・\rensuji*{0}}\end{shiika}
\begin{shiika}初蕨(わらび)雨に持ちくれ留守の扉に
\hfill{\rensuji*{60}・\rensuji*{4}・\rensuji*{0}}\end{shiika}
\begin{shiika}名にひかれ植え初花をひめ辛夷
\hfill{\rensuji*{60}・\rensuji*{4}・\rensuji*{0}}\end{shiika}
\vspace{0.6cm}
伊藤さん 清川さん と岩国城
\begin{shiika}天主より眺むる花の城下町
\hfill{\rensuji*{60}・\rensuji*{4}・\rensuji*{21}}\end{shiika}
\begin{shiika}階高し一打の鐘に花の散る
\hfill{\rensuji*{60}・\rensuji*{4}・\rensuji*{21}}\end{shiika}
\vspace{0.6cm}
小汐さん 伊藤さん 清川さんと鳳来寺
\begin{shiika}老鴬に耳あそばせて喜寿の足
\hfill{\rensuji*{60}・\rensuji*{5}・\rensuji*{9}}\end{shiika}
\vspace{0.6cm}
三日月
\begin{shiika}蝸牛わがもの顔に城跡の碑
\hfill{\rensuji*{60}・\rensuji*{5}・\rensuji*{8}}\end{shiika}
\vspace{0.6cm}
あわくら荘に集まりての帰り道 あわくら渓谷
\begin{shiika}ぷちぷちと峠に摘めり夏わらび
\hfill{\rensuji*{60}・\rensuji*{6}・\rensuji*{18}}\end{shiika}
\begin{shiika}木苺の酢っぱ甘さや渓流に
\hfill{\rensuji*{60}・\rensuji*{6}・\rensuji*{17}}\end{shiika}
\vspace{0.6cm}
水無瀬 庭に年々の青蛙
\begin{shiika}塗りかへて狭庭の客に青蛙
\hfill{\rensuji*{60}・\rensuji*{5}・\rensuji*{0}}\end{shiika}
\vspace{0.6cm}
成城の家より駅に出る道
\begin{shiika}花ざくろ・
\hfill{\rensuji*{60}・\rensuji*{6}・\rensuji*{0}}\end{shiika}
\vspace{0.6cm}
小田澄子さん逝く。小田さんからいただいた紫式部
\begin{shiika}御名のごと清らに生きて蓮花
\hfill{\rensuji*{60}・\rensuji*{6}・\rensuji*{0}}\end{shiika}
\begin{shiika}たまはりし紫式部さわ咲けど
\hfill{\rensuji*{60}・\rensuji*{8}・\rensuji*{0}}\end{shiika}
\begin{shiika}短夜や句机ならぶ夢の切れ
\hfill{\rensuji*{60}・\rensuji*{8}・\rensuji*{0}}\end{shiika}
\vspace{0.6cm}
水無瀬
\begin{shiika}夜濯ぎて一日終りぬ恙なく
\hfill{\rensuji*{60}・\rensuji*{8}・\rensuji*{0}}\end{shiika}
\vspace{0.6cm}
\begin{shiika}働けることの幸玉の汗
\hfill{\rensuji*{60}・\rensuji*{8}・\rensuji*{0}}\end{shiika}
\vspace{0.6cm}
\begin{shiika}言ふだけで気のすむ愚痴に団扇風
\hfill{\rensuji*{60}・\rensuji*{8}・\rensuji*{0}}\end{shiika}
\vspace{0.6cm}
\begin{shiika}階暑し団地こつこつセールスマン
\hfill{\rensuji*{60}・\rensuji*{9}・\rensuji*{0}}\end{shiika}
\vspace{0.6cm}
\rensuji*{60}年双適出句
\begin{shiika}梅雨しめる記帳簿将軍旧居訪ひ
\hfill{\rensuji*{60}・\rensuji*{6}・\rensuji*{25}}\end{shiika}
\begin{shiika}苔の花将軍愛馬の小さき塚
\hfill{\rensuji*{60}・\rensuji*{6}・\rensuji*{25}}\end{shiika}
\begin{shiika}将軍旧居もちの花
\hfill{\rensuji*{60}・\rensuji*{6}・\rensuji*{25}}\end{shiika}
\begin{shiika}意を通し過ぎし淋しさ夏の蝶
\hfill{入選\rensuji*{60}・\rensuji*{0}・\rensuji*{0}}\end{shiika}
\vspace{0.6cm}
小森田さんと山中温泉 和倉に
\begin{shiika}小駅の時計おそしと思ふ時雨来て
\hfill{\rensuji*{60}・\rensuji*{11}・\rensuji*{19}}\end{shiika}
\vspace{0.6cm}
一駅まちがえて芦原温泉にて下車
\begin{shiika}名もゆかしこほろぎ橋の渓紅葉
\hfill{\rensuji*{60}・\rensuji*{11}・\rensuji*{20}}\end{shiika}
\vspace{0.6cm}
\begin{shiika}冬の雷一発のみや・
\hfill{\rensuji*{60}・\rensuji*{11}・\rensuji*{20}}\end{shiika}
\vspace{0.6cm}
高田さん見舞い
\begin{shiika}冬ぬくし見舞ひし友にもてなされ
\hfill{\rensuji*{60}・\rensuji*{12}・\rensuji*{0}}\end{shiika}
\vspace{0.6cm}小川先生宅
\begin{shiika}謡声白山茶花の垣流れ
\hfill{\rensuji*{60}・\rensuji*{12}・\rensuji*{0}}\end{shiika}
\vspace{0.6cm}
落ち葉を眺めて
\begin{shiika}小説の終りのごとく落葉散る
\hfill{\rensuji*{60}・\rensuji*{12}・\rensuji*{0}}\end{shiika}
\vspace{0.6cm}
熱海伊豆山神社にて
\begin{shiika}愛語りし腰掛石や昼ちちろ
\hfill{\rensuji*{60}・\rensuji*{11}・\rensuji*{0}}\end{shiika}
\vspace{0.6cm}
\begin{shiika}曼茶羅に政子のむかし秋そぞろ
\hfill{\rensuji*{60}・\rensuji*{11}・\rensuji*{0}}\end{shiika}
\vspace{0.6cm}
\begin{shiika}露けくて墨のうすれしいわれ書
\hfill{\rensuji*{60}・\rensuji*{11}・\rensuji*{0}}\end{shiika}
\vspace{0.6cm}
水無瀬正月風景
%-----------------------------
\begin{shiika}輪飾りの小さきをかけ団地の扉
\hfill{\rensuji*{61}・\rensuji*{1}・\rensuji*{0}}\end{shiika}
\begin{shiika}寒木瓜の紅を深めて雨上る
\hfill{\rensuji*{61}・\rensuji*{1}・\rensuji*{0}}\end{shiika}
\begin{shiika}盆梅や鉢の木謡ひたき夜なり
\hfill{\rensuji*{61}・\rensuji*{1}・\rensuji*{0}}\end{shiika}
\vspace{0.6cm}
\begin{shiika}成人の日の背広着し子を見上ぐ
\hfill{\rensuji*{61}・\rensuji*{0}・\rensuji*{0}}\end{shiika}
\vspace{0.6cm}
\begin{shiika}試験子の窓に憂きほど春深雪
\hfill{\rensuji*{61}・\rensuji*{0}・\rensuji*{0}}\end{shiika}
\vspace{0.6cm}
\begin{shiika}弔ひて無口の帰り春吹雪
\hfill{\rensuji*{61}・\rensuji*{0}・\rensuji*{0}}\end{shiika}
\vspace{0.6cm}
\begin{shiika}ことなげに抜歯をされて春寒し
\hfill{\rensuji*{61}・\rensuji*{0}・\rensuji*{0}}\end{shiika}
\vspace{0.6cm}
\begin{shiika}白梅や三百年を語る幹
\hfill{\rensuji*{61}・\rensuji*{0}・\rensuji*{0}}\end{shiika}
\vspace{0.6cm}
\begin{shiika}ゆずり合ひつヽ空うばひ梅盛る
\hfill{\rensuji*{61}・\rensuji*{0}・\rensuji*{0}}\end{shiika}
\vspace{0.6cm}
\begin{shiika}春時雨急げば合はす鍵の鈴
\hfill{\rensuji*{61}・\rensuji*{0}・\rensuji*{0}}\end{shiika}
\vspace{0.6cm}
\begin{shiika}土を割る花芽それぞれ色ありて
\hfill{\rensuji*{61}・\rensuji*{0}・\rensuji*{0}}\end{shiika}
\vspace{0.6cm}
\begin{shiika}書き終えてほつと紅茶の浅き春
\hfill{\rensuji*{61}・\rensuji*{0}・\rensuji*{0}}\end{shiika}
\vspace{0.6cm}
\begin{shiika}庭隅に鈴蘭匂ひ旅ごころ
\hfill{\rensuji*{61}・\rensuji*{0}・\rensuji*{0}}\end{shiika}
\vspace{0.6cm}
\begin{shiika}屋根草もうすき緑に御寺春
\hfill{\rensuji*{61}・\rensuji*{0}・\rensuji*{0}}\end{shiika}
\vspace{0.6cm}
\begin{shiika}枝うつるりす生き生きと新樹光
\hfill{\rensuji*{61}・\rensuji*{0}・\rensuji*{0}}\end{shiika}
\vspace{0.6cm}
\begin{shiika}散るものは散らして

扇塚の春
\hfill{\rensuji*{61}・\rensuji*{0}・\rensuji*{0}}\end{shiika}
\vspace{0.6cm}
\begin{shiika}明日に咲く牡丹見よと泊めくれし
\hfill{\rensuji*{61}・\rensuji*{0}・\rensuji*{0}}\end{shiika}
\vspace{0.6cm}
\begin{shiika}牡丹の今開かむと息づかひ
\hfill{\rensuji*{61}・\rensuji*{0}・\rensuji*{0}}\end{shiika}
\vspace{0.6cm}
\begin{shiika}身も心青く染まりぬ宮若葉
\hfill{\rensuji*{61}・\rensuji*{0}・\rensuji*{0}}\end{shiika}
\vspace{0.6cm}
\begin{shiika}山越ゆるあの辺野崎か花曇
\hfill{\rensuji*{61}・\rensuji*{0}・\rensuji*{0}}\end{shiika}
\vspace{0.6cm}
\begin{shiika}バスの窓遠見を塞ぐ栗の花
\hfill{\rensuji*{61}・\rensuji*{0}・\rensuji*{0}}\end{shiika}
\vspace{0.6cm}
\begin{shiika}蛇の衣板一枚の城跡文
\hfill{\rensuji*{61}・\rensuji*{0}・\rensuji*{0}}\end{shiika}
\vspace{0.6cm}
\begin{shiika}アイスクリーム売の熱弁落城譜
\hfill{\rensuji*{61}・\rensuji*{0}・\rensuji*{0}}\end{shiika}
\vspace{0.6cm}
\begin{shiika}蔦青し城見ゆ坂のオランダ塀
\hfill{\rensuji*{61}・\rensuji*{0}・\rensuji*{0}}\end{shiika}
\vspace{0.6cm}
\begin{shiika}青葉冷え天主の跡の落城譜
\hfill{\rensuji*{61}・\rensuji*{0}・\rensuji*{0}}\end{shiika}
\vspace{0.6cm}
\begin{shiika}踊太鼓すぐそこにきき足を病む
\hfill{\rensuji*{61}・\rensuji*{0}・\rensuji*{0}}\end{shiika}
\vspace{0.6cm}
\begin{shiika}山男めきひげ面の帰省孫
\hfill{\rensuji*{61}・\rensuji*{0}・\rensuji*{0}}\end{shiika}
\vspace{0.6cm}
\begin{shiika}癒ゆること信じてきけり蝉の声
\hfill{\rensuji*{61}・\rensuji*{0}・\rensuji*{0}}\end{shiika}
\vspace{0.6cm}
\begin{shiika}癒ゆきざししかと凉しき今朝の風
\hfill{\rensuji*{61}・\rensuji*{0}・\rensuji*{0}}\end{shiika}
\vspace{0.6cm}
\begin{shiika}亡母の櫛ふとさしてみる盆支度
\hfill{\rensuji*{61}・\rensuji*{0}・\rensuji*{0}}\end{shiika}
\vspace{0.6cm}
\begin{shiika}杖に頼る試歩の足もと萩こぼる
\hfill{\rensuji*{61}・\rensuji*{0}・\rensuji*{0}}\end{shiika}
\vspace{0.6cm}
\begin{shiika}寝\Kana{団,扇}{うち,わ }にうちわどころの故郷のこと
\hfill{\rensuji*{61}・\rensuji*{0}・\rensuji*{0}}\end{shiika}
\vspace{0.6cm}
\begin{shiika}去ぬ燕便りとたよりすれちがひ
\hfill{\rensuji*{61}・\rensuji*{0}・\rensuji*{0}}\end{shiika}
\vspace{0.6cm}
\begin{shiika}鰯雲交しておかむ生き形見
\hfill{\rensuji*{61}・\rensuji*{0}・\rensuji*{0}}\end{shiika}
\vspace{0.6cm}
\begin{shiika}風に雲に秋の深みを知る夕べ
\hfill{\rensuji*{61}・\rensuji*{0}・\rensuji*{0}}\end{shiika}
\vspace{0.6cm}
\begin{shiika}カタカナ語事典にいどむ老夜長
\hfill{\rensuji*{61}・\rensuji*{0}・\rensuji*{0}}\end{shiika}
\vspace{0.6cm}
\begin{shiika}菊の香や来し方遠し五・
\hfill{\rensuji*{61}・\rensuji*{0}・\rensuji*{0}}\end{shiika}
\vspace{0.6cm}
\begin{shiika}雲を割り冬陽美し退職す
\hfill{\rensuji*{61}・\rensuji*{0}・\rensuji*{0}}\end{shiika}
\vspace{0.6cm}
\begin{shiika}むなしさも煙としたり菊を焚く
\hfill{\rensuji*{61}・\rensuji*{0}・\rensuji*{0}}\end{shiika}
\vspace{0.6cm}
\begin{shiika}年用意心のこもる故郷の荷
\hfill{\rensuji*{61}・\rensuji*{0}・\rensuji*{0}}\end{shiika}
\vspace{0.6cm}
\begin{shiika}満目の紅葉それぞれちがふ色
\hfill{\rensuji*{61}・\rensuji*{0}・\rensuji*{0}}\end{shiika}
\vspace{0.6cm}
\begin{shiika}静かなりいで湯娘と在り去年今年
\hfill{\rensuji*{62}・\rensuji*{0}・\rensuji*{0}}\end{shiika}
\vspace{0.6cm}
\begin{shiika}たまさかの晴着に帯と初芝居
\hfill{\rensuji*{62}・\rensuji*{0}・\rensuji*{0}}\end{shiika}
\vspace{0.6cm}
\begin{shiika}シテ謡ひ修めし安堵室の梅
\hfill{\rensuji*{62}・\rensuji*{0}・\rensuji*{0}}\end{shiika}
\vspace{0.6cm}
\begin{shiika}誰が為と笑はれもして初鏡
\hfill{\rensuji*{62}・\rensuji*{0}・\rensuji*{0}}\end{shiika}
\vspace{0.6cm}
\begin{shiika}梅白し陽ざしの居間の笑ひ声
\hfill{\rensuji*{62}・\rensuji*{0}・\rensuji*{0}}\end{shiika}
\vspace{0.6cm}
\begin{shiika}男子校女子校つづき芽ふく道
\hfill{\rensuji*{62}・\rensuji*{0}・\rensuji*{0}}\end{shiika}
\vspace{0.6cm}
\begin{shiika}庭の陽を占めて寒木瓜紅の濃し
\hfill{\rensuji*{62}・\rensuji*{0}・\rensuji*{0}}\end{shiika}
\vspace{0.6cm}
\begin{shiika}火廼要慎祀符の墨字に春ぼこり
\hfill{\rensuji*{62}・\rensuji*{0}・\rensuji*{0}}\end{shiika}
\vspace{0.6cm}
\begin{shiika}今日は憂し今日は美くし木の芽雨
\hfill{\rensuji*{62}・\rensuji*{0}・\rensuji*{0}}\end{shiika}
\vspace{0.6cm}
\begin{shiika}春愁を恥じて陶狸の腹を撫ず
\hfill{\rensuji*{62}・\rensuji*{0}・\rensuji*{0}}\end{shiika}
\vspace{0.6cm}
\begin{shiika}名桜につきぬ名残の里を去る
\hfill{\rensuji*{63}・\rensuji*{0}・\rensuji*{0}}\end{shiika}
\vspace{0.6cm}
\begin{shiika}山裾の梨の花園に白昼夢
\hfill{\rensuji*{62}・\rensuji*{0}・\rensuji*{0}}\end{shiika}
\vspace{0.6cm}
\begin{shiika}花クローバ終の棲家の地鎮祭
\hfill{\rensuji*{62}・\rensuji*{0}・\rensuji*{0}}\end{shiika}
\vspace{0.6cm}
\begin{shiika}松の花傘寿を集ふ公の庭
\hfill{\rensuji*{62}・\rensuji*{0}・\rensuji*{0}}\end{shiika}
\vspace{0.6cm}
\begin{shiika}文学館出でてまぶしき若葉光
\hfill{\rensuji*{62}・\rensuji*{0}・\rensuji*{0}}\end{shiika}
\vspace{0.6cm}
\begin{shiika}目礼がことばよ通院路の茂り
\hfill{\rensuji*{62}・\rensuji*{0}・\rensuji*{0}}\end{shiika}
\vspace{0.6cm}
\begin{shiika}青葉雨千人塚の匂ひ濃し
\hfill{\rensuji*{62}・\rensuji*{0}・\rensuji*{0}}\end{shiika}
\vspace{0.6cm}
\begin{shiika}土産店菖蒲と競ふ肥後名所
\hfill{\rensuji*{62}・\rensuji*{0}・\rensuji*{0}}\end{shiika}
\vspace{0.6cm}
\begin{shiika}五月晴阿蘇の寝釈迦に帰途祈り
\hfill{\rensuji*{62}・\rensuji*{0}・\rensuji*{0}}\end{shiika}
\vspace{0.6cm}
\begin{shiika}夏草に五百羅漢のかくれんぼ
\hfill{\rensuji*{62}・\rensuji*{0}・\rensuji*{0}}\end{shiika}
\vspace{0.6cm}
\begin{shiika}夏草にあそびつ羅漢の泣き笑ひ
\hfill{\rensuji*{62}・\rensuji*{0}・\rensuji*{0}}\end{shiika}
\vspace{0.6cm}
\begin{shiika}自転車で五日の旅の戻り梅雨
\hfill{\rensuji*{62}・\rensuji*{0}・\rensuji*{0}}\end{shiika}
\vspace{0.6cm}
\begin{shiika}初咲きの桔梗と供華に朝づとめ
\hfill{\rensuji*{62}・\rensuji*{0}・\rensuji*{0}}\end{shiika}
\vspace{0.6cm}
\begin{shiika}夜濯ぎの干場思はず下手な歌
\hfill{\rensuji*{62}・\rensuji*{0}・\rensuji*{0}}\end{shiika}
\vspace{0.6cm}
\begin{shiika}八階に住みて音なき遠花火
\hfill{\rensuji*{62}・\rensuji*{0}・\rensuji*{0}}\end{shiika}
\vspace{0.6cm}
\begin{shiika}早発ちてさかさ冨士みむ秋の湖
\hfill{\rensuji*{62}・\rensuji*{0}・\rensuji*{0}}\end{shiika}
\vspace{0.6cm}
\begin{shiika}霧晴れて小波が消すさかさ冨士
\hfill{\rensuji*{62}・\rensuji*{0}・\rensuji*{0}}\end{shiika}
\vspace{0.6cm}
\begin{shiika}文学碑たてる峠に秋の冨士
\hfill{\rensuji*{62}・\rensuji*{0}・\rensuji*{0}}\end{shiika}
\vspace{0.6cm}
\begin{shiika}花すゝき駅近かそうで遠かりし
\hfill{\rensuji*{62}・\rensuji*{0}・\rensuji*{0}}\end{shiika}
\vspace{0.6cm}
\begin{shiika}招くごとコスモス揺るる無人駅
\hfill{\rensuji*{62}・\rensuji*{0}・\rensuji*{0}}\end{shiika}
\vspace{0.6cm}
\begin{shiika}誰も来ずくつろぐ時の菊日和
\hfill{\rensuji*{62}・\rensuji*{0}・\rensuji*{0}}\end{shiika}
\vspace{0.6cm}
\begin{shiika}老夜長旅に集めし箸袋
\hfill{\rensuji*{62}・\rensuji*{0}・\rensuji*{0}}\end{shiika}
\vspace{0.6cm}
\begin{shiika}とっておきのワインもてなす良夜かな
\hfill{\rensuji*{62}・\rensuji*{0}・\rensuji*{0}}\end{shiika}
\vspace{0.6cm}
\begin{shiika}南洲を語る白髪月の部屋
\hfill{\rensuji*{62}・\rensuji*{0}・\rensuji*{0}}\end{shiika}
\vspace{0.6cm}
\begin{shiika}紅葉濃し峠二つを越えし温泉
\hfill{\rensuji*{62}・\rensuji*{0}・\rensuji*{0}}\end{shiika}
\vspace{0.6cm}
\begin{shiika}隣より争ひ声や秋の暮
\hfill{\rensuji*{62}・\rensuji*{0}・\rensuji*{0}}\end{shiika}
\vspace{0.6cm}
\begin{shiika}石蕗さかり先は稲荷の鳥居径
\hfill{\rensuji*{62}・\rensuji*{0}・\rensuji*{0}}\end{shiika}
\vspace{0.6cm}
\begin{shiika}海知らぬ犬を毎朝冬の浜
\hfill{\rensuji*{62}・\rensuji*{0}・\rensuji*{0}}\end{shiika}
\vspace{0.6cm}
\begin{shiika}新らしき木の香の中に賀状書く
\hfill{\rensuji*{62}・\rensuji*{0}・\rensuji*{0}}\end{shiika}
\vspace{0.6cm}
\begin{shiika}看とりつつ句帳かた辺に長き夜
\hfill{\rensuji*{62}・\rensuji*{0}・\rensuji*{0}}\end{shiika}
\vspace{0.6cm}
\begin{shiika}看とり女にある秋晴や特選句
\hfill{\rensuji*{62}・\rensuji*{0}・\rensuji*{0}}\end{shiika}
\vspace{0.6cm}
\begin{shiika}祭太鼓看とりの窓に遠くきく
\hfill{\rensuji*{62}・\rensuji*{0}・\rensuji*{0}}\end{shiika}
\vspace{0.6cm}
\begin{shiika}安眠なき看とりの夜々に虫親し
\hfill{\rensuji*{62}・\rensuji*{0}・\rensuji*{0}}\end{shiika}
\vspace{0.6cm}
\begin{shiika}愛語りし腰掛石や昼ちちろ
\hfill{\rensuji*{62}・\rensuji*{0}・\rensuji*{0}}\end{shiika}
\vspace{0.6cm}
\begin{shiika}露けしや墨のうすれしいわれ書
\hfill{\rensuji*{62}・\rensuji*{0}・\rensuji*{0}}\end{shiika}
\vspace{0.6cm}
\begin{shiika}曼茶羅に政子の昔秋そぞろ
\hfill{\rensuji*{62}・\rensuji*{0}・\rensuji*{0}}\end{shiika}
\vspace{0.6cm}
\begin{shiika}寒青空娘は頬染めて婚約を
\hfill{\rensuji*{63}・\rensuji*{0}・\rensuji*{0}}\end{shiika}
\vspace{0.6cm}
\begin{shiika}梅二月婚約成りし娘のまぶし
\hfill{\rensuji*{63}・\rensuji*{0}・\rensuji*{0}}\end{shiika}
\vspace{0.6cm}
\begin{shiika}婚近き娘と春いちご分ちあい
\hfill{\rensuji*{63}・\rensuji*{0}・\rensuji*{0}}\end{shiika}
\vspace{0.6cm}
\begin{shiika}列車徐行深雪のここに友住ふ
\hfill{\rensuji*{63}・\rensuji*{0}・\rensuji*{0}}\end{shiika}
\vspace{0.6cm}
\begin{shiika}たまわりし手造り味噌に蕗のとう
\hfill{\rensuji*{63}・\rensuji*{0}・\rensuji*{0}}\end{shiika}
\vspace{0.6cm}
\begin{shiika}枯芝にねてにらまるゝはらみ猫
\hfill{\rensuji*{63}・\rensuji*{0}・\rensuji*{0}}\end{shiika}
\vspace{0.6cm}
\begin{shiika}春寒や三日もつづく探しもの
\hfill{\rensuji*{63}・\rensuji*{0}・\rensuji*{0}}\end{shiika}
\vspace{0.6cm}
\begin{shiika}春灯失せものこゝに出て笑ふ
\hfill{\rensuji*{63}・\rensuji*{0}・\rensuji*{0}}\end{shiika}
\vspace{0.6cm}
\begin{shiika}椿落つ今日も名知らぬ鳥の来て
\hfill{\rensuji*{63}・\rensuji*{0}・\rensuji*{0}}\end{shiika}
\vspace{0.6cm}
\begin{shiika}ゆかし名ばかり揃えて盆梅展
\hfill{\rensuji*{63}・\rensuji*{0}・\rensuji*{0}}\end{shiika}
\vspace{0.6cm}
\begin{shiika}春潮に水尾ひく連絡船(ふね)のあと幾日
\hfill{\rensuji*{63}・\rensuji*{0}・\rensuji*{0}}\end{shiika}
\vspace{0.6cm}
\begin{shiika}終航の間近かき名残瀬戸の春
\hfill{\rensuji*{63}・\rensuji*{0}・\rensuji*{0}}\end{shiika}
\vspace{0.6cm}
\begin{shiika}花菜漬土産に訪ひくれ京言葉
\hfill{\rensuji*{63}・\rensuji*{0}・\rensuji*{0}}\end{shiika}
\vspace{0.6cm}
\begin{shiika}手染めとて淡き春着の京言葉
\hfill{\rensuji*{63}・\rensuji*{0}・\rensuji*{0}}\end{shiika}
\vspace{0.6cm}
\begin{shiika}花冷えて鬼女の棲みける巨き岩
\hfill{\rensuji*{63}・\rensuji*{0}・\rensuji*{0}}\end{shiika}
\vspace{0.6cm}
\begin{shiika}恐ろしき昔語りや花の里
\hfill{\rensuji*{63}・\rensuji*{0}・\rensuji*{0}}\end{shiika}
\vspace{0.6cm}
\begin{shiika}杉古りて黒塚ひそと花曇る
\hfill{\rensuji*{63}・\rensuji*{0}・\rensuji*{0}}\end{shiika}
\vspace{0.6cm}
\begin{shiika}若やぎて傘寿の集ひ牡丹園
\hfill{\rensuji*{63}・\rensuji*{0}・\rensuji*{0}}\end{shiika}
\vspace{0.6cm}
\begin{shiika}声低く僧が餅売る牡丹寺
\hfill{\rensuji*{63}・\rensuji*{0}・\rensuji*{0}}\end{shiika}
\vspace{0.6cm}
\begin{shiika}手をとりて笑む道祖神若葉光
\hfill{\rensuji*{63}・\rensuji*{0}・\rensuji*{0}}\end{shiika}
\vspace{0.6cm}
\begin{shiika}花の雨眠る山湖を去りがたく
\hfill{\rensuji*{63}・\rensuji*{0}・\rensuji*{0}}\end{shiika}
\vspace{0.6cm}
\begin{shiika}老鴬や奥へとたずね政子墓所
\hfill{\rensuji*{63}・\rensuji*{0}・\rensuji*{0}}\end{shiika}
\vspace{0.6cm}
\begin{shiika}旧姓で呼びあふ荘の明易し鎌倉荘)
\hfill{\rensuji*{63}・\rensuji*{0}・\rensuji*{0}}\end{shiika}
\vspace{0.6cm}
\begin{shiika}まぐなぎを払ひ百体地蔵訪ふ
\hfill{\rensuji*{63}・\rensuji*{0}・\rensuji*{0}}\end{shiika}
\vspace{0.6cm}
\begin{shiika}探ねゆく流れ涼しき渓いで湯(太閤の湯)
\hfill{\rensuji*{63}・\rensuji*{0}・\rensuji*{0}}\end{shiika}
\vspace{0.6cm}
\begin{shiika}カンナ燃えひしめきあえる養鶏舎
\hfill{\rensuji*{63}・\rensuji*{0}・\rensuji*{0}}\end{shiika}
\vspace{0.6cm}
\begin{shiika}雲走り峯にこま草這ひて咲く
\hfill{\rensuji*{63}・\rensuji*{0}・\rensuji*{0}}\end{shiika}
\vspace{0.6cm}
\begin{shiika}浜木綿にしばらくのこる夕茜
\hfill{\rensuji*{63}・\rensuji*{0}・\rensuji*{0}}\end{shiika}
\vspace{0.6cm}
\begin{shiika}故里の植田にうつす己が影
\hfill{\rensuji*{63}・\rensuji*{0}・\rensuji*{0}}\end{shiika}
\vspace{0.6cm}
\begin{shiika}錦飾る故郷ならずも茄子の花
\hfill{\rensuji*{63}・\rensuji*{0}・\rensuji*{0}}\end{shiika}
\vspace{0.6cm}
\begin{shiika}甚平着て今日も碁敵待つ
\hfill{\rensuji*{63}・\rensuji*{0}・\rensuji*{0}}\end{shiika}
\vspace{0.6cm}
\begin{shiika}叔父跡地ひまわり咲かす家五軒
\hfill{\rensuji*{63}・\rensuji*{0}・\rensuji*{0}}\end{shiika}
\vspace{0.6cm}
\begin{shiika}朝顔や一家は北に赴任して
\hfill{\rensuji*{63}・\rensuji*{0}・\rensuji*{0}}\end{shiika}
\vspace{0.6cm}
\begin{shiika}秋蝶が惜しむ別れの前よぎる
\hfill{\rensuji*{63}・\rensuji*{0}・\rensuji*{0}}\end{shiika}
\vspace{0.6cm}
\begin{shiika}見送りの垣根アベリア咲きこぼる
\hfill{\rensuji*{63}・\rensuji*{0}・\rensuji*{0}}\end{shiika}
\vspace{0.6cm}
\begin{shiika}滝二つ遠見の台に小手かざし
\hfill{\rensuji*{63}・\rensuji*{0}・\rensuji*{0}}\end{shiika}
\vspace{0.6cm}
\begin{shiika}穂すすきのみるみる刈られゆく売地
\hfill{\rensuji*{63}・\rensuji*{0}・\rensuji*{0}}\end{shiika}
\vspace{0.6cm}
\begin{shiika}吾が暮し覗いて聞いて青芒
\hfill{\rensuji*{63}・\rensuji*{0}・\rensuji*{0}}\end{shiika}
\vspace{0.6cm}
\begin{shiika}秋と思ふホームに目立つ黒い靴
\hfill{\rensuji*{63}・\rensuji*{0}・\rensuji*{0}}\end{shiika}
\vspace{0.6cm}
\begin{shiika}爽かや事終へて発つ旅の朝
\hfill{\rensuji*{63}・\rensuji*{0}・\rensuji*{0}}\end{shiika}
\vspace{0.6cm}
\begin{shiika}大秋晴善光寺平一望に
\hfill{\rensuji*{63}・\rensuji*{0}・\rensuji*{0}}\end{shiika}
\vspace{0.6cm}
\begin{shiika}歌声をのせて寄せ来る芒波
\hfill{\rensuji*{63}・\rensuji*{0}・\rensuji*{0}}\end{shiika}
\vspace{0.6cm}
\begin{shiika}コスモスのゆれる川沿ひ遊歩道
\hfill{\rensuji*{63}・\rensuji*{0}・\rensuji*{0}}\end{shiika}
\vspace{0.6cm}
\begin{shiika}母となる娘に寄す思ひ冬ぬくし
\hfill{\rensuji*{63}・\rensuji*{0}・\rensuji*{0}}\end{shiika}
\vspace{0.6cm}
\begin{shiika}実南天紅し娘は母となる
\hfill{\rensuji*{63}・\rensuji*{0}・\rensuji*{0}}\end{shiika}
\vspace{0.6cm}
水無瀬をたたむ決心
\begin{shiika}晩菊や終止符打たん独り住み
\hfill{\rensuji*{63}・\rensuji*{11}・\rensuji*{0}}\end{shiika}
\begin{shiika}息子と同居決めむ独りの湯豆腐鍋
\hfill{\rensuji*{63}・\rensuji*{11}・\rensuji*{0}}\end{shiika}
\vspace{0.6cm}
武生に仏壇を見に行く
\begin{shiika}トンネルを出て越前の雪景色
\hfill{\rensuji*{63}・\rensuji*{12}・\rensuji*{0}}\end{shiika}
\begin{shiika}仏壇を買ひに越路へ雪清し
\hfill{\rensuji*{63}・\rensuji*{12}・\rensuji*{0}}\end{shiika}
