水無瀬に移り来て
\begin{shiika}秋風も他人もやさし移り住み
\hfill{\rensuji*{57}・\rensuji*{11}・\rensuji*{0}}\end{shiika}
\begin{shiika}見捨てかね新居に挿せり倒れ菊
\hfill{\rensuji*{57}・\rensuji*{11}・\rensuji*{0}}\end{shiika}
\vspace{0.6cm}
幡井さんと山代温泉国家公務員保養所
\begin{shiika}寛ぎて見る山荘の紅葉濃し
\hfill{\rensuji*{57}・\rensuji*{11}・\rensuji*{0}}\end{shiika}
\vspace{0.6cm}
水無瀬相川通勤 相川の駅のホーム
\begin{shiika}乗りおくれくやしき顔に冬の月
\hfill{\rensuji*{57}・\rensuji*{11}・\rensuji*{0}}\end{shiika}
\vspace{0.6cm}
水無瀬の日々
\begin{shiika}寒椿にぶる起ち居のすべもなく
\hfill{\rensuji*{57}・\rensuji*{12}・\rensuji*{0}}\end{shiika}
\vspace{0.6cm}
\begin{shiika}友呼ばむ一人に余る日向ぼこ
\hfill{\rensuji*{57}・\rensuji*{12}・\rensuji*{0}}\end{shiika}
\vspace{0.6cm}
相川の庭
\begin{shiika}転宅の迫りし庭の実むらさき
\hfill{\rensuji*{57}・\rensuji*{10}・\rensuji*{0}}\end{shiika}
\begin{shiika}移り住む名残の菊香衰えず
\hfill{\rensuji*{57}・\rensuji*{10}・\rensuji*{0}}\end{shiika}
\vspace{0.6cm}
伊勢への旅の時を思い出して
\begin{shiika}玉砂利に歩の乱れなし神の留守
\hfill{\rensuji*{57}・\rensuji*{12}・\rensuji*{0}}\end{shiika}
\vspace{0.6cm}
%----------------------------------
喜美子 聖子にはなさんと
\begin{shiika}大役の初旅冨士が雲間より
\hfill{\rensuji*{58}・\rensuji*{1}・\rensuji*{3}}\end{shiika}
\vspace{0.6cm}
日野百草園にて
\begin{shiika}梅日和白壁光る村一望
\hfill{\rensuji*{58}・\rensuji*{2}・\rensuji*{0}}\end{shiika}
\vspace{0.6cm}
水無瀬
\begin{shiika}しつけとる春立つ朝の装ひに
\hfill{\rensuji*{58}・\rensuji*{3}・\rensuji*{0}}\end{shiika}
\begin{shiika}水ぬるむ就職決り紅さす娘
\hfill{\rensuji*{58}・\rensuji*{3}・\rensuji*{0}}\end{shiika}
\begin{shiika}桜餅娘の訪ひくれし小半日
\hfill{\rensuji*{58}・\rensuji*{3}・\rensuji*{0}}\end{shiika}
\begin{shiika}目口なき紙の雛や掌になじむ
\hfill{\rensuji*{58}・\rensuji*{3}・\rensuji*{0}}\end{shiika}
\vspace{0.6cm}
高田さん弔問
\begin{shiika}裏の家の雨に堪へ咲く八重桜
\hfill{\rensuji*{58}・\rensuji*{4}・\rensuji*{0}}\end{shiika}
\begin{shiika}友の情雨に摘みきしわらび飯
\hfill{\rensuji*{58}・\rensuji*{4}・\rensuji*{0}}\end{shiika}
\begin{shiika}忌に集るしのぶ日がなを花の雨
\hfill{\rensuji*{58}・\rensuji*{4}・\rensuji*{0}}\end{shiika}
\vspace{0.6cm}
水無瀬楠公通の大楠像が学校庭に移し植え
\begin{shiika}除り去らる囀り包む街の樹が
\hfill{\rensuji*{58}・\rensuji*{4}・\rensuji*{0}}\end{shiika}
\begin{shiika}読むも憂し眺むも憂しや花の雨
\hfill{\rensuji*{58}・\rensuji*{4}・\rensuji*{0}}\end{shiika}
\begin{shiika}集ればお国訛よよもぎ餅
\hfill{\rensuji*{58}・\rensuji*{4}・\rensuji*{0}}\end{shiika}
\vspace{0.6cm}
%-------------------------------------
秩父路 高松高女の皆さんと
\begin{shiika}秩父路につづく芽桑の夕映えて
\hfill{\rensuji*{58}・\rensuji*{4}・\rensuji*{7}}\end{shiika}
\vspace{0.6cm}
一善と一言神社へ
\begin{shiika}万緑や一言神に願一つ
\hfill{\rensuji*{58}・\rensuji*{5}・\rensuji*{21}}\end{shiika}
\begin{shiika}田植機の若者帽子に赤い花
\hfill{\rensuji*{58}・\rensuji*{5}・\rensuji*{0}}\end{shiika}
\vspace{0.6cm}
%-------------------------------------
文友会 東北の旅
\begin{shiika}桜桃たわわの国へ喜寿の旅
\hfill{\rensuji*{58}・\rensuji*{5}・\rensuji*{0}}\end{shiika}
\vspace{0.6cm}
西川さん 水無瀬に迎えて
\begin{shiika}杖たよる友出迎へに梅雨はげし
\hfill{\rensuji*{58}・\rensuji*{6}・\rensuji*{11}}\end{shiika}
\vspace{0.6cm}
水無瀬
\begin{shiika}朝涼し咲きつぐ花を供華日記
\hfill{\rensuji*{58}・\rensuji*{0}・\rensuji*{0}}\end{shiika}
\begin{shiika}引き越して来たる浜木綿咲き安堵
\hfill{\rensuji*{58}・\rensuji*{0}・\rensuji*{0}}\end{shiika}
\begin{shiika}娘三人訪ひくれ風鈴よく鳴れり
\hfill{\rensuji*{58}・\rensuji*{0}・\rensuji*{0}}\end{shiika}
\begin{shiika}一族の年長となり魂まつる
\hfill{\rensuji*{58}・\rensuji*{8}・\rensuji*{0}}\end{shiika}
\vspace{0.6cm}
阪急\rensuji*{32}番街 皆美にて、竹四郎 喜美子と食事
\begin{shiika}動かぬ灯動く灯一望盆の果
\hfill{\rensuji*{58}・\rensuji*{8}・\rensuji*{0}}\end{shiika}
\begin{shiika}洗ひ髪立つベランダの風は秋
\hfill{\rensuji*{58}・\rensuji*{8}・\rensuji*{0}}\end{shiika}
\vspace{0.6cm}
山下さん 高田さん 駒ヶ根車山ペンsyングリーンスポット巡り」
\begin{shiika}蕎麦三日食べてさわやか信濃旅
\hfill{\rensuji*{58}・\rensuji*{9}・\rensuji*{4}}\end{shiika}
\vspace{0.6cm}
安藤さんと三方五湖
\begin{shiika}色鳥や岳に真向ふ湖の宿
\hfill{\rensuji*{58}・\rensuji*{9}・\rensuji*{0}}\end{shiika}
\begin{shiika}大き鳥湖上を舞ひて夏去れり
\hfill{\rensuji*{58}・\rensuji*{9}・\rensuji*{0}}\end{shiika}
\vspace{0.6cm}
箕面観光ホテル別館 桂 謡に会
\begin{shiika}庭紅葉もえて謡に力声
\hfill{\rensuji*{58}・\rensuji*{11}・\rensuji*{0}}\end{shiika}
\begin{shiika}謡ひ果て山荘黄葉をのこし暮る
\hfill{\rensuji*{58}・\rensuji*{11}・\rensuji*{0}}\end{shiika}
\vspace{0.6cm}
水無瀬折々
\begin{shiika}翅やすむ蝶もむらさき式部の実
\hfill{\rensuji*{58}・\rensuji*{11}・\rensuji*{0}}\end{shiika}
\begin{shiika}独り居のよき日淋し日菊挿して
\hfill{\rensuji*{58}・\rensuji*{1}・\rensuji*{0}}\end{shiika}
\begin{shiika}疎く住み安けき日々や杜鵤草
\hfill{\rensuji*{58}・\rensuji*{11}・\rensuji*{0}}\end{shiika}
\vspace{0.6cm}
成城の金魚
\begin{shiika}屑金魚育ち掬ひし児も少年
\hfill{\rensuji*{58}・\rensuji*{11}・\rensuji*{0}}\end{shiika}
\vspace{0.6cm}
伊藤さん八田さn清川さん 京都の紅葉案内
\begin{shiika}案内三日京の紅葉に酔ひ疲る
\hfill{\rensuji*{58}・\rensuji*{11}・\rensuji*{0}}\end{shiika}
\vspace{0.6cm}
\begin{shiika}照紅葉京一望の峯の寺
\hfill{\rensuji*{58}・\rensuji*{11}・\rensuji*{0}}\end{shiika}
\vspace{0.6cm}
高田さん宅に小森田さん 小田さんと 山荘和周庵 落成
\begin{shiika}山荘の集ひに菜飯冬ぬくし
\hfill{\rensuji*{58}・\rensuji*{12}・\rensuji*{9}}\end{shiika}
\begin{shiika}冬入日竹叢透し荘なごむ
\hfill{\rensuji*{58}・\rensuji*{12}・\rensuji*{9}}\end{shiika}
\vspace{0.6cm}
水無瀬元旦
\begin{shiika}一とせを会ひ得ぬ人の賀状増し
\hfill{\rensuji*{59}・\rensuji*{1}・\rensuji*{1}}\end{shiika}
\begin{shiika}しきたりをつづけて独り屠蘇機嫌
\hfill{\rensuji*{59}・\rensuji*{1}・\rensuji*{0}}\end{shiika}
\vspace{0.6cm}
安藤さんと三方五湖へ北陸線
\begin{shiika}トンネルを抜ける度雪深くなり
\hfill{\rensuji*{59}・\rensuji*{1}・\rensuji*{2}}\end{shiika}
\vspace{0.6cm}
水無瀬のシンビジュームがさく
\begin{shiika}ただいまと灯せば応ふ室の花
\hfill{\rensuji*{59}・\rensuji*{2}・\rensuji*{0}}\end{shiika}
\vspace{0.6cm}
水無瀬に石井晴美さんを迎え ?る枝の友
\begin{shiika}ちゃん呼びで遠き日戻る木の葉髪
\hfill{\rensuji*{59}・\rensuji*{2}・\rensuji*{0}}\end{shiika}
\vspace{0.6cm}
%-----------------------------------------------------------------
富田の駅で乗り換えの時 相川の古いお客様と出会う
\begin{shiika}春寒やぱったり出会ひ出ぬ名前
\hfill{\rensuji*{59}・\rensuji*{2}・\rensuji*{0}}\end{shiika}
\vspace{0.6cm}
直紀 郷生 一善に質問されて
\begin{shiika}争ひも夢よ首塚土筆の芽
\hfill{\rensuji*{59}・\rensuji*{3}・\rensuji*{0}}\end{shiika}
\vspace{0.6cm}
防府 藤本悦子さん宅 (藤本様とはこれが最后の出会いになる)
\begin{shiika}老夫婦夜をぼつぼつとひなあられ
\hfill{\rensuji*{59}・\rensuji*{3}・\rensuji*{3}}\end{shiika}
\vspace{0.6cm}
山下さんと湯布院 亀の井 別荘二泊
\begin{shiika}雪解風由布岳さして大鴉
\hfill{\rensuji*{59}・\rensuji*{3}・\rensuji*{5}}\end{shiika}
\vspace{0.6cm}
水無瀬折々
\begin{shiika}土を割る花芽それぞれ色ありて
\hfill{\rensuji*{59}・\rensuji*{3}・\rensuji*{0}}\end{shiika}
\begin{shiika}によきによきと花芽ラッシュの庭の土
\hfill{\rensuji*{59}・\rensuji*{3}・\rensuji*{0}}\end{shiika}
\begin{shiika}花苺児にしやがみ見す芯の粒
\hfill{\rensuji*{59}・\rensuji*{4}・\rensuji*{0}}\end{shiika}
\begin{shiika}朝毎の独りに足りる庭苺
\hfill{\rensuji*{59}・\rensuji*{5}・\rensuji*{0}}\end{shiika}
\begin{shiika}団地住みテレビの上の兜の威
\hfill{\rensuji*{59}・\rensuji*{5}・\rensuji*{0}}\end{shiika}
\begin{shiika}ホース先そらせばそこも青蛙
\hfill{\rensuji*{59}・\rensuji*{7}・\rensuji*{0}}\end{shiika}
\vspace{0.6cm}
水無瀬の庭の青蛙はなつかしい お隣佐藤さんに嬰誕生
\begin{shiika}花南天隣初嬰の襁褓干す
\hfill{\rensuji*{59}・\rensuji*{7}・\rensuji*{0}}\end{shiika}
\begin{shiika}待ちつつも一人を凉しと思ふ日も
\hfill{\rensuji*{59}・\rensuji*{8}・\rensuji*{0}}\end{shiika}
\begin{shiika}庭茂り払ふ枝にもある生命
\hfill{\rensuji*{59}・\rensuji*{8}・\rensuji*{0}}\end{shiika}
\begin{shiika}孫の名をとりちがえ呼ぶ盆家族
\hfill{\rensuji*{59}・\rensuji*{8}・\rensuji*{0}}\end{shiika}
\begin{shiika}夏萩に誰みくじ結ふ禁よそに
\hfill{\rensuji*{59}・\rensuji*{8}・\rensuji*{0}}\end{shiika}
\vspace{0.6cm}
悦子さん宅へ弔問
\begin{shiika}忌ごもりの友訪ひて汨つ戻り梅雨
\hfill{\rensuji*{59}・\rensuji*{7}・\rensuji*{0}}\end{shiika}
\vspace{0.6cm}
%-----------------------------------------------
山下さん 小森田さん と小海線から草津野友湖
\begin{shiika}夏書終へ東塔西塔仰ぐ朝
\hfill{\rensuji*{59}・\rensuji*{9}・\rensuji*{0}}\end{shiika}
\begin{shiika}空と無の多き夏書や朝鴉
\hfill{\rensuji*{59}・\rensuji*{9}・\rensuji*{0}}\end{shiika}
\begin{shiika}りんどうや標高識のたつ小駅
\hfill{\rensuji*{59}・\rensuji*{9}・\rensuji*{0}}\end{shiika}
\begin{shiika}高原列車おそしとゆれる花すすき
\hfill{\rensuji*{59}・\rensuji*{9}・\rensuji*{0}}\end{shiika}
\begin{shiika}紫の小波たてり松虫草
\hfill{\rensuji*{59}・\rensuji*{9}・\rensuji*{0}}\end{shiika}
\begin{shiika}思はざる遠冨士すゝきの小窓より
\hfill{\rensuji*{59}・\rensuji*{9}・\rensuji*{0}}\end{shiika}
\vspace{0.6cm}
滿藤さん宅 のうぜん花
\begin{shiika}朝風に彩をひろげてのうぜん花
\hfill{\rensuji*{59}・\rensuji*{9}・\rensuji*{0}}\end{shiika}
\vspace{0.6cm}
上野城 百合子出品を見に行く
\begin{shiika}風凉し天主の床の黒光り
\hfill{\rensuji*{59}・\rensuji*{8}・\rensuji*{0}}\end{shiika}
\vspace{0.6cm}
\begin{shiika}俳聖殿忍者屋敷も蝉しぐれ
\hfill{\rensuji*{59}・\rensuji*{8}・\rensuji*{0}}\end{shiika}
\vspace{0.6cm}
道成寺」白浜三段壁
\begin{shiika}秋凉し絵とき説法に笑ひあり
\hfill{\rensuji*{59}・\rensuji*{9}・\rensuji*{19}}\end{shiika}
\begin{shiika}水軍の洞の跡や秋の潮
\hfill{\rensuji*{59}・\rensuji*{9}・\rensuji*{19}}\end{shiika}
\vspace{0.6cm}
水無瀬盆踊り
%---------------------------------------------------------
\begin{shiika}青い眼の手ぶりに見入る踊の輪
\hfill{\rensuji*{59}・\rensuji*{8}・\rensuji*{0}}\end{shiika}
\begin{shiika}諷刺歌踊りの櫓は高調し
\hfill{\rensuji*{59}・\rensuji*{8}・\rensuji*{0}}\end{shiika}
\begin{shiika}送り火やもとの一人に戻る夜
\hfill{\rensuji*{59}・\rensuji*{8}・\rensuji*{0}}\end{shiika}
\vspace{0.6cm}
直紀の成人に感じたこと
\begin{shiika}帰省子の言葉大人ひふと淋し
\hfill{\rensuji*{59}・\rensuji*{8}・\rensuji*{0}}\end{shiika}
\begin{shiika}若者となるは別れか鳥雲に
\hfill{\rensuji*{59}・\rensuji*{8}・\rensuji*{0}}\end{shiika}
\vspace{0.6cm}
箱根?保にて
\begin{shiika}夏霧の湧きて流れて山の湖
\hfill{\rensuji*{59}・\rensuji*{7}・\rensuji*{0}}\end{shiika}
\vspace{0.6cm}
小川先生宅の山茶花
\begin{shiika}山茶花の垣咲き始めぬ謡声
\hfill{\rensuji*{59}・\rensuji*{11}・\rensuji*{0}}\end{shiika}
\vspace{0.6cm}
吉川三郎さんを高槻の病院に見舞う
\begin{shiika}冬の雲まこと知らせぬ人見舞ふ
\hfill{\rensuji*{59}・\rensuji*{11}・\rensuji*{0}}\end{shiika}
\vspace{0.6cm}
水無瀬年忘れ
\begin{shiika}年忘れ流す憂さなきワインの香
\hfill{\rensuji*{59}・\rensuji*{12}・\rensuji*{0}}\end{shiika}
\begin{shiika}賀状書く亡母の字に似る母の年令
\hfill{\rensuji*{59}・\rensuji*{12}・\rensuji*{0}}\end{shiika}
\begin{shiika}寄せ鍋の沸々はずむ故郷ことば
\hfill{\rensuji*{59}・\rensuji*{12}・\rensuji*{0}}\end{shiika}
\begin{shiika}するつと食ぶ熟柿に郷愁そぞろ湧く
\hfill{\rensuji*{59}・\rensuji*{12}・\rensuji*{0}}\end{shiika}
\vspace{0.6cm}
私の誕生日 水無瀬
\begin{shiika}吾が誕生秋刀魚で祝ひ心足る
\hfill{\rensuji*{59}・\rensuji*{11}・\rensuji*{0}}\end{shiika}
\vspace{0.6cm}
成城の新年
\begin{shiika}初冨士や大東京の隅に住み
\hfill{\rensuji*{60}・\rensuji*{1}・\rensuji*{0}}\end{shiika}
\vspace{0.6cm}
大阪への帰途
\begin{shiika}林立の煙突冨士に初煙
\hfill{\rensuji*{60}・\rensuji*{1}・\rensuji*{0}}\end{shiika}
\begin{shiika}初仕事裾野の町の白煙
\hfill{\rensuji*{60}・\rensuji*{1}・\rensuji*{0}}\end{shiika}
\vspace{0.6cm}
水無瀬
\begin{shiika}移し植え三年の梅に初つぼみ
\hfill{\rensuji*{60}・\rensuji*{2}・\rensuji*{0}}\end{shiika}
\begin{shiika}陽を集め日毎ふくらむ木瓜の花
\hfill{\rensuji*{60}・\rensuji*{2}・\rensuji*{0}}\end{shiika}
\begin{shiika}蘭匂ふ独りの部屋に惜しき程
\hfill{\rensuji*{60}・\rensuji*{3}・\rensuji*{0}}\end{shiika}
\vspace{0.6cm}
小田様のお嬢さま御他界 弔問
\begin{shiika}逆縁の香たく背なに春空し
\hfill{\rensuji*{60}・\rensuji*{2}・\rensuji*{0}}\end{shiika}
\vspace{0.6cm}
水無瀬
\begin{shiika}春や憂し着かえし裾の静電気
\hfill{\rensuji*{60}・\rensuji*{4}・\rensuji*{0}}\end{shiika}
\begin{shiika}割れ込まれ句心とぎれぬ春炬燵
\hfill{\rensuji*{60}・\rensuji*{3}・\rensuji*{0}}\end{shiika}
\begin{shiika}初蕨(わらび)雨に持ちくれ留守の扉に
\hfill{\rensuji*{60}・\rensuji*{4}・\rensuji*{0}}\end{shiika}
\begin{shiika}名にひかれ植え初花をひめ辛夷
\hfill{\rensuji*{60}・\rensuji*{4}・\rensuji*{0}}\end{shiika}
\vspace{0.6cm}
伊藤さん 清川さん と岩国城
\begin{shiika}天主より眺むる花の城下町
\hfill{\rensuji*{60}・\rensuji*{4}・\rensuji*{21}}\end{shiika}
\begin{shiika}階高し一打の鐘に花の散る
\hfill{\rensuji*{60}・\rensuji*{4}・\rensuji*{21}}\end{shiika}
\vspace{0.6cm}
小汐さん 伊藤さん 清川さんと鳳来寺
\begin{shiika}老鴬に耳あそばせて喜寿の足
\hfill{\rensuji*{60}・\rensuji*{5}・\rensuji*{9}}\end{shiika}
\vspace{0.6cm}
三日月
\begin{shiika}蝸牛わがもの顔に城跡の碑
\hfill{\rensuji*{60}・\rensuji*{5}・\rensuji*{8}}\end{shiika}
\vspace{0.6cm}
あわくら荘に集まりての帰り道 あわくら渓谷
\begin{shiika}ぷちぷちと峠に摘めり夏わらび
\hfill{\rensuji*{60}・\rensuji*{6}・\rensuji*{18}}\end{shiika}
\begin{shiika}木苺の酢っぱ甘さや渓流に
\hfill{\rensuji*{60}・\rensuji*{6}・\rensuji*{17}}\end{shiika}
\vspace{0.6cm}
水無瀬 庭に年々の青蛙
\begin{shiika}塗りかへて狭庭の客に青蛙
\hfill{\rensuji*{60}・\rensuji*{5}・\rensuji*{0}}\end{shiika}
\vspace{0.6cm}
成城の家より駅に出る道
\begin{shiika}花ざくろ・
\hfill{\rensuji*{60}・\rensuji*{6}・\rensuji*{0}}\end{shiika}
\vspace{0.6cm}
小田澄子さん逝く。小田さんからいただいた紫式部
\begin{shiika}御名のごと清らに生きて蓮花
\hfill{\rensuji*{60}・\rensuji*{6}・\rensuji*{0}}\end{shiika}
\begin{shiika}たまはりし紫式部さわ咲けど
\hfill{\rensuji*{60}・\rensuji*{8}・\rensuji*{0}}\end{shiika}
\begin{shiika}短夜や句机ならぶ夢の切れ
\hfill{\rensuji*{60}・\rensuji*{8}・\rensuji*{0}}\end{shiika}
\vspace{0.6cm}
水無瀬
\begin{shiika}夜濯ぎて一日終りぬ恙なく
\hfill{\rensuji*{60}・\rensuji*{8}・\rensuji*{0}}\end{shiika}
\vspace{0.6cm}
\begin{shiika}働けることの幸玉の汗
\hfill{\rensuji*{60}・\rensuji*{8}・\rensuji*{0}}\end{shiika}
\vspace{0.6cm}
\begin{shiika}言ふだけで気のすむ愚痴に団扇風
\hfill{\rensuji*{60}・\rensuji*{8}・\rensuji*{0}}\end{shiika}
\vspace{0.6cm}
\begin{shiika}階暑し団地こつこつセールスマン
\hfill{\rensuji*{60}・\rensuji*{9}・\rensuji*{0}}\end{shiika}
\vspace{0.6cm}
\rensuji*{60}年双適出句
\begin{shiika}梅雨しめる記帳簿将軍旧居訪ひ
\hfill{\rensuji*{60}・\rensuji*{6}・\rensuji*{25}}\end{shiika}
\begin{shiika}苔の花将軍愛馬の小さき塚
\hfill{\rensuji*{60}・\rensuji*{6}・\rensuji*{25}}\end{shiika}
\begin{shiika}将軍旧居もちの花
\hfill{\rensuji*{60}・\rensuji*{6}・\rensuji*{25}}\end{shiika}
\begin{shiika}意を通し過ぎし淋しさ夏の蝶
\hfill{入選\rensuji*{60}・\rensuji*{0}・\rensuji*{0}}\end{shiika}
\vspace{0.6cm}
小森田さんと山中温泉 和倉に
\begin{shiika}小駅の時計おそしと思ふ時雨来て
\hfill{\rensuji*{60}・\rensuji*{11}・\rensuji*{19}}\end{shiika}
\vspace{0.6cm}
一駅まちがえて芦原温泉にて下車
\begin{shiika}名もゆかしこほろぎ橋の渓紅葉
\hfill{\rensuji*{60}・\rensuji*{11}・\rensuji*{20}}\end{shiika}
\vspace{0.6cm}
\begin{shiika}冬の雷一発のみや・
\hfill{\rensuji*{60}・\rensuji*{11}・\rensuji*{20}}\end{shiika}
\vspace{0.6cm}
高田さん見舞い
\begin{shiika}冬ぬくし見舞ひし友にもてなされ
\hfill{\rensuji*{60}・\rensuji*{12}・\rensuji*{0}}\end{shiika}
\vspace{0.6cm}小川先生宅
\begin{shiika}謡声白山茶花の垣流れ
\hfill{\rensuji*{60}・\rensuji*{12}・\rensuji*{0}}\end{shiika}
\vspace{0.6cm}
落ち葉を眺めて
\begin{shiika}小説の終りのごとく落葉散る
\hfill{\rensuji*{60}・\rensuji*{12}・\rensuji*{0}}\end{shiika}
\vspace{0.6cm}
熱海伊豆山神社にて
\begin{shiika}愛語りし腰掛石や昼ちちろ
\hfill{\rensuji*{60}・\rensuji*{11}・\rensuji*{0}}\end{shiika}
\vspace{0.6cm}
\begin{shiika}曼茶羅に政子のむかし秋そぞろ
\hfill{\rensuji*{60}・\rensuji*{11}・\rensuji*{0}}\end{shiika}
\vspace{0.6cm}
\begin{shiika}露けくて墨のうすれしいわれ書
\hfill{\rensuji*{60}・\rensuji*{11}・\rensuji*{0}}\end{shiika}
\vspace{0.6cm}
水無瀬正月風景
%-----------------------------
\begin{shiika}輪飾りの小さきをかけ団地の扉
\hfill{\rensuji*{61}・\rensuji*{1}・\rensuji*{0}}\end{shiika}
\begin{shiika}寒木瓜の紅を深めて雨上る
\hfill{\rensuji*{61}・\rensuji*{1}・\rensuji*{0}}\end{shiika}
\begin{shiika}盆梅や鉢の木謡ひたき夜なり
\hfill{\rensuji*{61}・\rensuji*{1}・\rensuji*{0}}\end{shiika}
\vspace{0.6cm}
成城にて 直紀背広 成人の日ではなかったが、くにの入試日
\begin{shiika}成人の日の背広着し子を見上ぐ
\hfill{\rensuji*{61}・\rensuji*{2}・\rensuji*{0}}\end{shiika}
\begin{shiika}試験子の窓に憂きほど春深雪
\hfill{\rensuji*{61}・\rensuji*{3}・\rensuji*{0}}\end{shiika}
\vspace{0.6cm}
伊藤さんの長男様御他界
\begin{shiika}弔ひて無口の帰り春吹雪
\hfill{\rensuji*{61}・\rensuji*{2}・\rensuji*{0}}\end{shiika}
\vspace{0.6cm}
田辺歯科
\begin{shiika}ことなげに抜歯をされて春寒し
\hfill{\rensuji*{61}・\rensuji*{3}・\rensuji*{0}}\end{shiika}
\vspace{0.6cm}
%------------------------------------------------
藤沢 中島さんに石川の梅案内していただく
\begin{shiika}白梅や三百年を語る幹
\hfill{\rensuji*{61}・\rensuji*{3}・\rensuji*{0}}\end{shiika}
\begin{shiika}ゆずり合ひつヽ空うばひ梅盛る
\hfill{\rensuji*{61}・\rensuji*{3}・\rensuji*{0}}\end{shiika}
\vspace{0.6cm}
水無瀬折ふし
\begin{shiika}春時雨急げば合はす鍵の鈴
\hfill{\rensuji*{61}・\rensuji*{3}・\rensuji*{0}}\end{shiika}
\begin{shiika}土を割る花芽それぞれ色ありて
\hfill{\rensuji*{61}・\rensuji*{3}・\rensuji*{0}}\end{shiika}
\begin{shiika}書き終えてほつと紅茶の浅き春
\hfill{\rensuji*{61}・\rensuji*{3}・\rensuji*{0}}\end{shiika}
\begin{shiika}庭隅に鈴蘭匂ひ旅ごころ
\hfill{\rensuji*{61}・\rensuji*{4}・\rensuji*{0}}\end{shiika}
\vspace{0.6cm}
中島さんと鎌倉苔寺(松葉谷妙法寺)
\begin{shiika}屋根草もうすき緑に御寺春
\hfill{\rensuji*{61}・\rensuji*{4}・\rensuji*{0}}\end{shiika}
\vspace{0.6cm}
\begin{shiika}枝うつるりす生き生きと新樹光
\hfill{\rensuji*{61}・\rensuji*{4}・\rensuji*{0}}\end{shiika}
\vspace{0.6cm}
\begin{shiika}散るものは散らして扇塚の春
\hfill{\rensuji*{61}・\rensuji*{4}・\rensuji*{0}}\end{shiika}
\vspace{0.6cm}
生駒大川の牡丹
\begin{shiika}明日に咲く牡丹見よと泊めくれし
\hfill{\rensuji*{61}・\rensuji*{5}・\rensuji*{0}}\end{shiika}
\begin{shiika}牡丹の今開かむと息づかひ
\hfill{\rensuji*{61}・\rensuji*{5}・\rensuji*{0}}\end{shiika}
\vspace{0.6cm}
\begin{shiika}身も心青く染まりぬ宮若葉
\hfill{\rensuji*{61}・\rensuji*{5}・\rensuji*{0}}\end{shiika}
\begin{shiika}山越ゆるあの辺野崎か花曇
\hfill{\rensuji*{61}・\rensuji*{4}・\rensuji*{0}}\end{shiika}
\vspace{0.6cm}
山下さん小森田さん悦子さんと島原 雲仙 平戸
\begin{shiika}バスの窓遠見を塞ぐ栗の花
\hfill{\rensuji*{61}・\rensuji*{6}・\rensuji*{13}}\end{shiika}
\begin{shiika}蛇の衣板一枚の城跡文
\hfill{\rensuji*{61}・\rensuji*{6}・\rensuji*{14}}\end{shiika}
\begin{shiika}アイスクリーム売の熱弁落城譜
\hfill{\rensuji*{61}・\rensuji*{6}・\rensuji*{14}}\end{shiika}
\begin{shiika}蔦青し城見ゆ坂のオランダ塀
\hfill{\rensuji*{61}・\rensuji*{6}・\rensuji*{15}}\end{shiika}
\begin{shiika}青葉冷え天主の跡の落城譜
\hfill{\rensuji*{61}・\rensuji*{0}・\rensuji*{0}}\end{shiika}
\vspace{0.6cm}
足の痛みが始まって 水無瀬
\begin{shiika}踊太鼓すぐそこにきき足を病む
\hfill{\rensuji*{61}・\rensuji*{8}・\rensuji*{0}}\end{shiika}
\begin{shiika}山男めきひげ面の帰省孫
\hfill{\rensuji*{61}・\rensuji*{8}・\rensuji*{0}}\end{shiika}
\begin{shiika}癒ゆること信じてきけり蝉の声
\hfill{\rensuji*{61}・\rensuji*{8}・\rensuji*{0}}\end{shiika}
\begin{shiika}癒ゆきざししかと凉しき今朝の風
\hfill{\rensuji*{61}・\rensuji*{9}・\rensuji*{0}}\end{shiika}
\begin{shiika}亡母の櫛ふとさしてみる盆支度
\hfill{\rensuji*{61}・\rensuji*{8}・\rensuji*{0}}\end{shiika}
\begin{shiika}杖に頼る試歩の足もと萩こぼる
\hfill{\rensuji*{61}・\rensuji*{9}・\rensuji*{0}}\end{shiika}
\vspace{0.6cm}
井高野で泊って
\begin{shiika}寝\Kana{団,扇}{うち,わ }にうちわどころの故郷のこと
\hfill{\rensuji*{61}・\rensuji*{9}・\rensuji*{0}}\end{shiika}
\vspace{0.6cm}
遠藤さんちの手紙が行きちがいになること三度
\begin{shiika}去ぬ燕便りとたよりすれちがひ
\hfill{\rensuji*{61}・\rensuji*{9}・\rensuji*{0}}\end{shiika}
\vspace{0.6cm}
山下さんと形見の交換 木目込ひな 日本の国立公園」
\begin{shiika}鰯雲交しておかむ生き形見
\hfill{\rensuji*{61}・\rensuji*{10}・\rensuji*{0}}\end{shiika}
\vspace{0.6cm}
水無瀬
\begin{shiika}風に雲に秋の深みを知る夕べ
\hfill{\rensuji*{61}・\rensuji*{10}・\rensuji*{0}}\end{shiika}
\begin{shiika}カタカナ語事典にいどむ老夜長
\hfill{\rensuji*{61}・\rensuji*{10}・\rensuji*{0}}\end{shiika}
\begin{shiika}菊の香や来し方遠し五・
\hfill{\rensuji*{61}・\rensuji*{9}・\rensuji*{0}}\end{shiika}
\begin{shiika}雲を割り冬陽美し退職す
\hfill{\rensuji*{61}・\rensuji*{11}・\rensuji*{0}}\end{shiika}
\begin{shiika}むなしさも煙としたり菊を焚く
\hfill{\rensuji*{61}・\rensuji*{11}・\rensuji*{0}}\end{shiika}
\begin{shiika}年用意心のこもる故郷の荷
\hfill{\rensuji*{61}・\rensuji*{12}・\rensuji*{0}}\end{shiika}
\vspace{0.6cm}
伊藤さんと花の寺へ
\begin{shiika}満目の紅葉それぞれちがふ色
\hfill{\rensuji*{61}・\rensuji*{11}・\rensuji*{15}}\end{shiika}
\vspace{0.6cm}
井上直子さんと箕面観光ホテルにて越年
\begin{shiika}静かなりいで湯娘と在り去年今年
\hfill{\rensuji*{62}・\rensuji*{1}・\rensuji*{1}}\end{shiika}
\vspace{0.6cm}
安藤さんと文楽 謡新年の会 堀田様宅
\begin{shiika}たまさかの晴着に帯と初芝居
\hfill{\rensuji*{62}・\rensuji*{1}・\rensuji*{0}}\end{shiika}
\begin{shiika}シテ謡ひ修めし安堵室の梅
\hfill{\rensuji*{62}・\rensuji*{1}・\rensuji*{0}}\end{shiika}
\begin{shiika}誰が為と笑はれもして初鏡
\hfill{\rensuji*{62}・\rensuji*{1}・\rensuji*{0}}\end{shiika}
\vspace{0.6cm}
成城の家相川より移した梅開く
\begin{shiika}梅白し陽ざしの居間の笑ひ声
\hfill{\rensuji*{62}・\rensuji*{2}・\rensuji*{0}}\end{shiika}
\vspace{0.6cm}
相川
\begin{shiika}男子校女子校つづき芽ふく道
\hfill{\rensuji*{62}・\rensuji*{2}・\rensuji*{0}}\end{shiika}
\vspace{0.6cm}
水無瀬
\begin{shiika}庭の陽を占めて寒木瓜紅の濃し
\hfill{\rensuji*{62}・\rensuji*{2}・\rensuji*{0}}\end{shiika}
%厨(くりや)祀符(きふ」)
\begin{shiika}火廼要慎\Kana{祀,符}{き,ふ}の墨字に春ぼこり
\hfill{\rensuji*{62}・\rensuji*{3}・\rensuji*{0}}\end{shiika}
\begin{shiika}今日は憂し今日は美くし木の芽雨
\hfill{\rensuji*{62}・\rensuji*{30}・\rensuji*{0}}\end{shiika}
\begin{shiika}春愁を恥じて陶狸の腹を撫ず
\hfill{\rensuji*{62}・\rensuji*{3}・\rensuji*{0}}\end{shiika}
\vspace{0.6cm}
一善と常照皇寺
\begin{shiika}名桜につきぬ名残の里を去る
\hfill{\rensuji*{63}・\rensuji*{4}・\rensuji*{19}}\end{shiika}
\vspace{0.6cm}
安藤さんと春日観光農園
\begin{shiika}山裾の梨の花園に白昼夢
\hfill{\rensuji*{62}・\rensuji*{4}・\rensuji*{15}}\end{shiika}
\vspace{0.6cm}
鵠沼地鎮祭
\begin{shiika}花クローバ終の棲家の地鎮祭
\hfill{\rensuji*{62}・\rensuji*{5}・\rensuji*{0}}\end{shiika}
\vspace{0.6cm}
鎌倉文学館
\begin{shiika}松の花傘寿を集ふ公の庭
\hfill{\rensuji*{62}・\rensuji*{5}・\rensuji*{13}}\end{shiika}
\begin{shiika}文学館出でてまぶしき若葉光
\hfill{\rensuji*{62}・\rensuji*{5}・\rensuji*{13}}\end{shiika}
\vspace{0.6cm}
相川 三国さんへの日々
\begin{shiika}目礼がことばよ通院路の茂り
\hfill{\rensuji*{62}・\rensuji*{6}・\rensuji*{0}}\end{shiika}
\vspace{0.6cm}
山下さん伊藤さん悦子さんと長崎 天草 熊本 阿蘇
\begin{shiika}青葉雨千人塚の匂ひ濃し
\hfill{\rensuji*{62}・\rensuji*{5}・\rensuji*{27}}\end{shiika}
\begin{shiika}土産店菖蒲と競ふ肥後名所
\hfill{\rensuji*{62}・\rensuji*{5}・\rensuji*{28}}\end{shiika}
\begin{shiika}五月晴阿蘇の寝釈迦に帰途祈り
\hfill{\rensuji*{62}・\rensuji*{5}・\rensuji*{26}}\end{shiika}
\vspace{0.6cm}
伊藤さん清川さん」と寄居少林寺五百羅漢
\begin{shiika}夏草に五百羅漢のかくれんぼ
\hfill{\rensuji*{62}・\rensuji*{7}・\rensuji*{9}}\end{shiika}
\begin{shiika}夏草にあそびつ羅漢の泣き笑ひ
\hfill{\rensuji*{62}・\rensuji*{7}・\rensuji*{9}}\end{shiika}
\vspace{0.6cm}
くに 自転車信州の旅を
\begin{shiika}自転車で五日の旅の戻り梅雨
\hfill{\rensuji*{62}・\rensuji*{7}・\rensuji*{0}}\end{shiika}
\vspace{0.6cm}
水無瀬
\begin{shiika}初咲きの桔梗と供華に朝づとめ
\hfill{\rensuji*{62}・\rensuji*{8}・\rensuji*{0}}\end{shiika}
\begin{shiika}夜濯ぎの干場思はず下手な歌
\hfill{\rensuji*{62}・\rensuji*{8}・\rensuji*{0}}\end{shiika}
\vspace{0.6cm}
大和桜ケ丘のマンション
\begin{shiika}八階に住みて音なき遠花火
\hfill{\rensuji*{62}・\rensuji*{8}・\rensuji*{0}}\end{shiika}
\vspace{0.6cm}
山中湖健保に泊りて 山下さん清川さんと
\begin{shiika}早発ちてさかさ冨士みむ秋の湖
\hfill{\rensuji*{62}・\rensuji*{9}・\rensuji*{5}}\end{shiika}
\begin{shiika}霧晴れて小波が消すさかさ冨士
\hfill{\rensuji*{62}・\rensuji*{9}・\rensuji*{0}}\end{shiika}
\begin{shiika}文学碑たてる峠に秋の冨士
\hfill{\rensuji*{62}・\rensuji*{9}・\rensuji*{0}}\end{shiika}
\vspace{0.6cm}
下呂禅昌寺駅
\begin{shiika}花すゝき駅近かそうで遠かりし
\hfill{\rensuji*{62}・\rensuji*{9}・\rensuji*{4}}\end{shiika}
\begin{shiika}招くごとコスモス揺るる無人駅
\hfill{\rensuji*{62}・\rensuji*{9}・\rensuji*{4}}\end{shiika}
\vspace{0.6cm}
水無瀬
\begin{shiika}誰も来ずくつろぐ時の菊日和
\hfill{\rensuji*{62}・\rensuji*{0}・\rensuji*{0}}\end{shiika}
\vspace{0.6cm}
\begin{shiika}老夜長旅に集めし箸袋
\hfill{\rensuji*{62}・\rensuji*{0}・\rensuji*{0}}\end{shiika}
\vspace{0.6cm}
水無瀬に児玉正志さんを迎えて
\begin{shiika}とっておきのワインもてなす良夜かな
\hfill{\rensuji*{62}・\rensuji*{0}・\rensuji*{0}}\end{shiika}
\vspace{0.6cm}
\begin{shiika}南洲を語る白髪月の部屋
\hfill{\rensuji*{62}・\rensuji*{10}・\rensuji*{0}}\end{shiika}
\vspace{0.6cm}
鹿教湯温泉へ
\begin{shiika}紅葉濃し峠二つを越えし温泉
\hfill{\rensuji*{62}・\rensuji*{11}・\rensuji*{19}}\end{shiika}
\vspace{0.6cm}
鵠沼にて\\
我が家と隣家を置き換えてみた
\begin{shiika}隣より争ひ声や秋の暮
\hfill{\rensuji*{62}・\rensuji*{11}・\rensuji*{0}}\end{shiika}
\vspace{0.6cm}
鵠沼稲荷に沿って裏へ
\begin{shiika}石蕗さかり先は稲荷の鳥居径
\hfill{\rensuji*{62}・\rensuji*{11}・\rensuji*{0}}\end{shiika}
\vspace{0.6cm}
竹四郎チロとの散歩
\begin{shiika}海知らぬ犬を毎朝冬の浜
\hfill{\rensuji*{62}・\rensuji*{12}・\rensuji*{0}}\end{shiika}
\vspace{0.6cm}
\begin{shiika}新らしき木の香の中に賀状書く
\hfill{\rensuji*{62}・\rensuji*{12}・\rensuji*{0}}\end{shiika}
\vspace{0.6cm}
百合子の看病の日を思ひ
\begin{shiika}看とりつつ句帳かた辺に長き夜
\hfill{\rensuji*{62}・\rensuji*{10}・\rensuji*{0}}\end{shiika}
\begin{shiika}看とり女にある秋晴や特選句
\hfill{\rensuji*{62}・\rensuji*{10}・\rensuji*{0}}\end{shiika}
\begin{shiika}祭太鼓看とりの窓に遠くきく
\hfill{\rensuji*{62}・\rensuji*{10}・\rensuji*{0}}\end{shiika}
\begin{shiika}安眠なき看とりの夜々に虫親し
\hfill{\rensuji*{62}・\rensuji*{10}・\rensuji*{0}}\end{shiika}
\vspace{0.6cm}
伊豆山神社
\begin{shiika}愛語りし腰掛石や昼ちちろ
\hfill{\rensuji*{62}・\rensuji*{10}・\rensuji*{0}}\end{shiika}
\begin{shiika}露けしや墨のうすれしいわれ書
\hfill{\rensuji*{62}・\rensuji*{10}・\rensuji*{0}}\end{shiika}
\begin{shiika}曼茶羅に政子の昔秋そぞろ
\hfill{\rensuji*{62}・\rensuji*{10}・\rensuji*{0}}\end{shiika}
\vspace{0.6cm}
飯田知子婚約
\begin{shiika}寒青空娘は頬染めて婚約を
\hfill{\rensuji*{63}・\rensuji*{1}・\rensuji*{0}}\end{shiika}
\begin{shiika}梅二月婚約成りし娘のまぶし
\hfill{\rensuji*{63}・\rensuji*{2}・\rensuji*{0}}\end{shiika}
\begin{shiika}婚近き娘と春いちご分ちあい
\hfill{\rensuji*{63}・\rensuji*{3}・\rensuji*{0}}\end{shiika}
\vspace{0.6cm}
新刊線車中 小林ふじさん思
\begin{shiika}列車徐行深雪のここに友住ふ
\hfill{\rensuji*{63}・\rensuji*{3}・\rensuji*{0}}\end{shiika}
\vspace{0.6cm}
水無瀬
\begin{shiika}たまわりし手造り味噌に蕗のとう
\hfill{\rensuji*{63}・\rensuji*{0}・\rensuji*{0}}\end{shiika}
\vspace{0.6cm}
三号棟福井へのマンションの路
\begin{shiika}枯芝にねてにらまるゝはらみ猫
\hfill{\rensuji*{63}・\rensuji*{2}・\rensuji*{0}}\end{shiika}
\vspace{0.6cm}
\begin{shiika}春寒や三日もつづく探しもの
\hfill{\rensuji*{63}・\rensuji*{2}・\rensuji*{0}}\end{shiika}
\vspace{0.6cm}
手袋紛失 カーペットの下に隠れていた
\begin{shiika}春灯失せものこゝに出て笑ふ
\hfill{\rensuji*{63}・\rensuji*{2}・\rensuji*{0}}\end{shiika}
\vspace{0.6cm}
\begin{shiika}椿落つ今日も名知らぬ鳥の来て
\hfill{\rensuji*{63}・\rensuji*{3}・\rensuji*{0}}\end{shiika}
\vspace{0.6cm}
大川夫妻と長浜盆梅展
\begin{shiika}ゆかし名ばかり揃えて盆梅展
\hfill{\rensuji*{63}・\rensuji*{2}・\rensuji*{0}}\end{shiika}
\vspace{0.6cm}
宇高連絡船の名残
\begin{shiika}春潮に水尾ひく連絡船(ふね)のあと幾日
\hfill{\rensuji*{63}・\rensuji*{3}・\rensuji*{0}}\end{shiika}
\begin{shiika}終航の間近かき名残瀬戸の春
\hfill{\rensuji*{63}・\rensuji*{3}・\rensuji*{0}}\end{shiika}
\vspace{0.6cm}
美佐さん 西田さん 水無瀬に
\begin{shiika}花菜漬土産に訪ひくれ京言葉
\hfill{\rensuji*{63}・\rensuji*{3}・\rensuji*{0}}\end{shiika}
\begin{shiika}手染めとて淡き春着の京言葉
\hfill{\rensuji*{63}・\rensuji*{3}・\rensuji*{0}}\end{shiika}
\vspace{0.6cm}
黒塚
\begin{shiika}花冷えて鬼女の棲みける巨き岩
\hfill{\rensuji*{63}・\rensuji*{4}・\rensuji*{22}}\end{shiika}
\begin{shiika}恐ろしき昔語りや花の里
\hfill{\rensuji*{63}・\rensuji*{4}・\rensuji*{22}}\end{shiika}
\begin{shiika}杉古りて黒塚ひそと花曇る
\hfill{\rensuji*{63}・\rensuji*{4}・\rensuji*{22}}\end{shiika}
\vspace{0.6cm}
鹿教湯より美ヶ原 美鈴湖
\begin{shiika}若やぎて傘寿の集ひ牡丹園
\hfill{\rensuji*{63}・\rensuji*{0}・\rensuji*{0}}\end{shiika}
\begin{shiika}声低く僧が餅売る牡丹寺
\hfill{\rensuji*{63}・\rensuji*{0}・\rensuji*{0}}\end{shiika}
\begin{shiika}手をとりて笑む道祖神若葉光
\hfill{\rensuji*{63}・\rensuji*{0}・\rensuji*{0}}\end{shiika}
\begin{shiika}花の雨眠る山湖を去りがたく
\hfill{\rensuji*{63}・\rensuji*{0}・\rensuji*{0}}\end{shiika}
\vspace{0.6cm}
小汐さん迎え鎌倉へ
\begin{shiika}老鴬や奥へとたずね政子墓所
\hfill{\rensuji*{63}・\rensuji*{6}・\rensuji*{1}}\end{shiika}
\vspace{0.6cm}
\begin{shiika}旧姓で呼びあふ荘の明易し鎌倉荘)
\hfill{\rensuji*{63}・\rensuji*{6}・\rensuji*{0}}\end{shiika}
\vspace{0.6cm}
日光百体地蔵
\begin{shiika}まぐなぎを払ひ百体地蔵訪ふ
\hfill{\rensuji*{63}・\rensuji*{6}・\rensuji*{0}}\end{shiika}
\vspace{0.6cm}
箱根
\begin{shiika}探ねゆく流れ涼しき渓いで湯(太閤の湯)
\hfill{\rensuji*{63}・\rensuji*{7}・\rensuji*{0}}\end{shiika}
\begin{shiika}カンナ燃えひしめきあえる養鶏舎
\hfill{\rensuji*{63}・\rensuji*{7}・\rensuji*{0}}\end{shiika}
\vspace{0.6cm}
志賀高原 発哺温泉より東館山
\begin{shiika}雲走り峯にこま草這ひて咲く
\hfill{\rensuji*{63}・\rensuji*{7}・\rensuji*{0}}\end{shiika}
\vspace{0.6cm}
光簡保 山下 山脇 藤本さんと
\begin{shiika}浜木綿にしばらくのこる夕茜
\hfill{\rensuji*{63}・\rensuji*{7}・\rensuji*{0}}\end{shiika}
\vspace{0.6cm}
故郷さぬき国分
\begin{shiika}故里の植田にうつす己が影
\hfill{\rensuji*{63}・\rensuji*{8}・\rensuji*{0}}\end{shiika}
\begin{shiika}錦飾る故郷ならずも茄子の花
\hfill{\rensuji*{63}・\rensuji*{8}・\rensuji*{0}}\end{shiika}
\begin{shiika}甚平着て今日も碁敵待つ
\hfill{\rensuji*{63}・\rensuji*{8}・\rensuji*{0}}\end{shiika}
\begin{shiika}叔父跡地ひまわり咲かす家五軒
\hfill{\rensuji*{63}・\rensuji*{8}・\rensuji*{0}}\end{shiika}
\vspace{0.6cm}
水無瀬 福井さん北海道に転勤
\begin{shiika}朝顔や一家は北に赴任して
\hfill{\rensuji*{63}・\rensuji*{8}・\rensuji*{0}}\end{shiika}
\vspace{0.6cm}
水無瀬で山下さんと九月旅の終わりにてお別れ
\begin{shiika}秋蝶が惜しむ別れの前よぎる
\hfill{\rensuji*{63}・\rensuji*{9}・\rensuji*{0}}\end{shiika}
\vspace{0.6cm}
\begin{shiika}見送りの垣根アベリア咲きこぼる
\hfill{\rensuji*{63}・\rensuji*{9}・\rensuji*{0}}\end{shiika}
\vspace{0.6cm}
日光明智平ロープウエイにて
\begin{shiika}滝二つ遠見の台に小手かざし
\hfill{\rensuji*{63}・\rensuji*{9}・\rensuji*{0}}\end{shiika}
\vspace{0.6cm}
鵠沼の空き地
\begin{shiika}穂すすきのみるみる刈られゆく売地
\hfill{\rensuji*{63}・\rensuji*{9}・\rensuji*{0}}\end{shiika}
\begin{shiika}吾が暮し覗いて聞いて青芒
\hfill{\rensuji*{63}・\rensuji*{9}・\rensuji*{0}}\end{shiika}
\vspace{0.6cm}
初秋水無瀬駅のホーム
\begin{shiika}秋と思ふホームに目立つ黒い靴
\hfill{\rensuji*{63}・\rensuji*{9}・\rensuji*{0}}\end{shiika}
\vspace{0.6cm}
山下さんとの九月旅
\begin{shiika}爽かや事終へて発つ旅の朝
\hfill{\rensuji*{63}・\rensuji*{9}・\rensuji*{0}}\end{shiika}
\begin{shiika}大秋晴善光寺平一望に
\hfill{\rensuji*{63}・\rensuji*{9}・\rensuji*{0}}\end{shiika}
\begin{shiika}歌声をのせて寄せ来る芒波
\hfill{\rensuji*{63}・\rensuji*{9}・\rensuji*{0}}\end{shiika}
\vspace{0.6cm}
鵠沼引地川遊歩道コスモスの新名所と新聞に出る
\begin{shiika}コスモスのゆれる川沿ひ遊歩道
\hfill{\rensuji*{63}・\rensuji*{9}・\rensuji*{0}}\end{shiika}
\vspace{0.6cm}
知子母となる知らせ
\begin{shiika}母となる娘に寄す思ひ冬ぬくし
\hfill{\rensuji*{63}・\rensuji*{11}・\rensuji*{0}}\end{shiika}
\begin{shiika}実南天紅し娘は母となる
\hfill{\rensuji*{63}・\rensuji*{11}・\rensuji*{0}}\end{shiika}
\vspace{0.6cm}
水無瀬をたたむ決心
\begin{shiika}晩菊や終止符打たん独り住み
\hfill{\rensuji*{63}・\rensuji*{11}・\rensuji*{0}}\end{shiika}
\begin{shiika}息子と同居決めむ独りの湯豆腐鍋
\hfill{\rensuji*{63}・\rensuji*{11}・\rensuji*{0}}\end{shiika}
\vspace{0.6cm}
武生に仏壇を見に行く
\begin{shiika}トンネルを出て越前の雪景色
\hfill{\rensuji*{63}・\rensuji*{12}・\rensuji*{0}}\end{shiika}
\begin{shiika}仏壇を買ひに越路へ雪清し
\hfill{\rensuji*{63}・\rensuji*{12}・\rensuji*{0}}\end{shiika}
\vspace{0.6cm}
昭和六十四年 初詣日向薬師
\begin{shiika}山ふところに香煙みちて初薬師
\hfill{\rensuji*{1}・\rensuji*{1}・\rensuji*{1}}\end{shiika}
\begin{shiika}初護摩の煙いただき肩かるし
\hfill{\rensuji*{1}・\rensuji*{1}・\rensuji*{1}}\end{shiika}
