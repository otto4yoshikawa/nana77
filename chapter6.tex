
\begin{shiika}春の夕餉釣りし一尾を母の前    
\hfill{\rensuji*{8}・\rensuji*{0}・\rensuji*{0}}\end{shiika}

\begin{shiika}快気とはかくもうれしき春の朝   
\hfill{\rensuji*{8}・\rensuji*{0}・\rensuji*{0}}\end{shiika}

\begin{shiika}春光やを拝み浴びをり癒え兆    
\hfill{\rensuji*{8}・\rensuji*{0}・\rensuji*{0}}\end{shiika}

\begin{shiika}径端の小さき笑顔犬ふぐり     
\hfill{\rensuji*{8}・\rensuji*{0}・\rensuji*{0}}\end{shiika}

\begin{shiika}鯉のぼりたーかく揚げて待つ帰国
\hfill{\rensuji*{8}・\rensuji*{0}・\rensuji*{0}}\end{shiika}

\begin{shiika}日本を知らぬ児を待つ武者飾り   
\begin{shiika}いつまでも御元気でねてふ賀状の数 
\hfill{\rensuji*{8}・\rensuji*{0}・\rensuji*{0}}\end{shiika}

\begin{shiika}退職と一筆添へし賀状かな     
\hfill{\rensuji*{8}・\rensuji*{0}・\rensuji*{0}}\end{shiika}

\begin{shiika}初入日三六六の一を呑み
\hfill{\rensuji*{8}・\rensuji*{0}・\rensuji*{0}}\end{shiika}

\begin{shiika}ページくる吾が音寒し影寒し
\hfill{\rensuji*{8}・\rensuji*{0}・\rensuji*{0}}\end{shiika}

\begin{shiika}小豆粥老ひてすこやか姉弟
\hfill{\rensuji*{8}・\rensuji*{0}・\rensuji*{0}}\end{shiika}

\begin{shiika}春寒し言はでききをり二度話    
\hfill{\rensuji*{8}・\rensuji*{0}・\rensuji*{0}}\end{shiika}

\begin{shiika}鳥は雲に二度行くスーパー買いわすれ 
\hfill{\rensuji*{8}・\rensuji*{0}・\rensuji*{0}}\end{shiika}

\begin{shiika}梅二月八・
\hfill{\rensuji*{8}・\rensuji*{0}・\rensuji*{0}}\end{shiika}

\begin{shiika}娘等去にてかろき疲れに窓の梅   
\hfill{\rensuji*{8}・\rensuji*{0}・\rensuji*{0}}\end{shiika}

\begin{shiika}よきことを知らす娘の声梅紅し   
\hfill{\rensuji*{8}・\rensuji*{0}・\rensuji*{0}}\end{shiika}

\begin{shiika}芽吹く庭健かと木々に呼びかけて  
\hfill{\rensuji*{8}・\rensuji*{0}・\rensuji*{0}}\end{shiika}

\begin{shiika}鳥雲に謝しつつ辛き車椅子     
\hfill{\rensuji*{8}・\rensuji*{0}・\rensuji*{0}}\end{shiika}

\begin{shiika}鶯やに車椅子停めくれ息子よ    
\hfill{\rensuji*{8}・\rensuji*{0}・\rensuji*{0}}\end{shiika}

\begin{shiika}とてせめて電話は春の声    
\hfill{\rensuji*{8}・\rensuji*{0}・\rensuji*{0}}\end{shiika}

\begin{shiika}春彼岸弟訪ひくれ仏顔に
\hfill{\rensuji*{8}・\rensuji*{0}・\rensuji*{0}}\end{shiika}

\begin{shiika}岬うらら成果一尾の小半日     
\hfill{\rensuji*{8}・\rensuji*{0}・\rensuji*{0}}\end{shiika}

\hfill{\rensuji*{8}・\rensuji*{0}・\rensuji*{0}}\end{shiika}

\begin{shiika}薔薇咲かせ迎え明るき指圧院    
\hfill{\rensuji*{8}・\rensuji*{0}・\rensuji*{0}}\end{shiika}

\begin{shiika}土産地蕗香りひろげて国言葉    
\hfill{\rensuji*{8}・\rensuji*{0}・\rensuji*{0}}\end{shiika}

\begin{shiika}木の芽雨偲び草とて届く茶器    
\hfill{\rensuji*{8}・\rensuji*{0}・\rensuji*{0}}\end{shiika}

\begin{shiika}片隅に生きる幸せ額の花      
\hfill{\rensuji*{8}・\rensuji*{0}・\rensuji*{0}}\end{shiika}

\begin{shiika}新茶くみほめ言葉待つ母の顔    
\hfill{\rensuji*{8}・\rensuji*{0}・\rensuji*{0}}\end{shiika}

\begin{shiika}草茂る逆らはぬこと牙につきて   
\hfill{\rensuji*{8}・\rensuji*{0}・\rensuji*{0}}\end{shiika}

\begin{shiika}明易やドイツ転勤ききしより    
\hfill{\rensuji*{8}・\rensuji*{0}・\rensuji*{0}}\end{shiika}

\begin{shiika}泰山木朽ちてすがれる花かなし   
\hfill{\rensuji*{8}・\rensuji*{0}・\rensuji*{0}}\end{shiika}

\begin{shiika}朝涼やからっぽ頭にからっ腹    
\hfill{\rensuji*{8}・\rensuji*{0}・\rensuji*{0}}\end{shiika}

\begin{shiika}いざ昼寝今日はいづこへ夢の旅   
\hfill{\rensuji*{8}・\rensuji*{0}・\rensuji*{0}}\end{shiika}

\begin{shiika}夕涼し肌になじみし藍の服     
\hfill{\rensuji*{8}・\rensuji*{0}・\rensuji*{0}}\end{shiika}

\begin{shiika}暑からむ遅れて浴びる百視線    
\hfill{\rensuji*{8}・\rensuji*{0}・\rensuji*{0}}\end{shiika}

\begin{shiika}端居して出世無縁の長寿眉     
\hfill{\rensuji*{8}・\rensuji*{0}・\rensuji*{0}}\end{shiika}

\begin{shiika}暑に耐えし頬なでてみる今朝の風  
\hfill{\rensuji*{8}・\rensuji*{0}・\rensuji*{0}}\end{shiika}

\begin{shiika}秋暑し訪問販売二度のブザー    
\hfill{\rensuji*{8}・\rensuji*{0}・\rensuji*{0}}\end{shiika}

\begin{shiika}夜々うれし子の友に賜ぶ古梅酒   
\hfill{\rensuji*{8}・\rensuji*{0}・\rensuji*{0}}\end{shiika}

\begin{shiika}花火見に橋へ子が押す車椅子    
\hfill{\rensuji*{8}・\rensuji*{0}・\rensuji*{0}}\end{shiika}

\begin{shiika}癒へてつくる迎え送りの盆団子   
\hfill{\rensuji*{8}・\rensuji*{0}・\rensuji*{0}}\end{shiika}

\begin{shiika}白萩や見知らぬ同志笑みかわし   
\hfill{\rensuji*{8}・\rensuji*{0}・\rensuji*{0}}\end{shiika}

\begin{shiika}寺育ち白曼珠沙華燃え知らず    
\hfill{\rensuji*{8}・\rensuji*{0}・\rensuji*{0}}\end{shiika}

\begin{shiika}風やさしコスモスやさし車椅子   
\hfill{\rensuji*{8}・\rensuji*{0}・\rensuji*{0}}\end{shiika}

\begin{shiika}思はざる花つけにけり秋の草    
\hfill{\rensuji*{8}・\rensuji*{0}・\rensuji*{0}}\end{shiika}

\begin{shiika}秋冷ゆる友の情の京しるこ     
\hfill{\rensuji*{8}・\rensuji*{0}・\rensuji*{0}}\end{shiika}

\begin{shiika}故里や出会ふたれかれ野菊晴    
\hfill{\rensuji*{8}・\rensuji*{0}・\rensuji*{0}}\end{shiika}

\begin{shiika}栗むきつ老ひて姉弟郷言葉     
\hfill{\rensuji*{8}・\rensuji*{0}・\rensuji*{0}}\end{shiika}

\begin{shiika}風のまま吾も白髪穂亡や
\hfill{\rensuji*{8}・\rensuji*{0}・\rensuji*{0}}\end{shiika}

\begin{shiika}花は実に色増す石榴日々親し    
\hfill{\rensuji*{8}・\rensuji*{0}・\rensuji*{0}}\end{shiika}

\begin{shiika}急げともあわてるなとも虫の鳴く  
\hfill{\rensuji*{8}・\rensuji*{0}・\rensuji*{0}}\end{shiika}

\begin{shiika}天高し卒寿見上ぐる明治晴     
\hfill{\rensuji*{8}・\rensuji*{0}・\rensuji*{0}}\end{shiika}

\begin{shiika}秋深き豆煮る母のひとり言
\hfill{\rensuji*{8}・\rensuji*{0}・\rensuji*{0}}\end{shiika}

\begin{shiika}冬に入る病上手に附き合わす   
\hfill{\rensuji*{8}・\rensuji*{0}・\rensuji*{0}}\end{shiika}

\begin{shiika}いつまでも娘は子こたつの母苦言  
\hfill{\rensuji*{8}・\rensuji*{0}・\rensuji*{0}}\end{shiika}

\begin{shiika}よろこびにふとある怖さ夕紅葉   
\hfill{\rensuji*{8}・\rensuji*{0}・\rensuji*{0}}\end{shiika}

\begin{shiika}熟柿つるっと食べばふるさと近く来る
\hfill{\rensuji*{8}・\rensuji*{0}・\rensuji*{0}}\end{shiika}

\begin{shiika}枝桜紅葉に告ぐ別れ
\hfill{\rensuji*{8}・\rensuji*{0}・\rensuji*{0}}\end{shiika}

\begin{shiika}落葉掃きつい長くなる隣同志   
\hfill{\rensuji*{8}・\rensuji*{0}・\rensuji*{0}}\end{shiika}

\begin{shiika}やがてこの娘が孫の嫁冬いちご  
\hfill{\rensuji*{8}・\rensuji*{0}・\rensuji*{0}}\end{shiika}

\begin{shiika}雲を割る冬日や老のねがふこと  
\hfill{\rensuji*{8}・\rensuji*{0}・\rensuji*{0}}\end{shiika}

\begin{shiika}お元旦老母くり返すありがたや  
\hfill{\rensuji*{9}・\rensuji*{0}・\rensuji*{0}}\end{shiika}

\begin{shiika}しわのなき黒豆に老母初お箸
\hfill{\rensuji*{9}・\rensuji*{0}・\rensuji*{0}}\end{shiika}

\begin{shiika}初写真嫁孫の笑み三代
\hfill{\rensuji*{9}・\rensuji*{0}・\rensuji*{0}}\end{shiika}

\begin{shiika}愛犬と話す日日あり寒日和
\hfill{\rensuji*{9}・\rensuji*{0}・\rensuji*{0}}\end{shiika}

\begin{shiika}翔ばたいて大きなおまへ初からす 
\hfill{\rensuji*{9}・\rensuji*{0}・\rensuji*{0}}\end{shiika}

\begin{shiika}五・
\hfill{\rensuji*{9}・\rensuji*{0}・\rensuji*{0}}\end{shiika}

\begin{shiika}孫嫁のもうすぐ二人梅紅し
\hfill{\rensuji*{9}・\rensuji*{0}・\rensuji*{0}}\end{shiika}

\begin{shiika}お化粧で他人顔なり春写真    
\hfill{\rensuji*{9}・\rensuji*{0}・\rensuji*{0}}\end{shiika}

\begin{shiika}春障子四畳半の城明るし
\hfill{\rensuji*{9}・\rensuji*{0}・\rensuji*{0}}\end{shiika}

\begin{shiika}下萌に煎餅分ける愛犬に
\hfill{\rensuji*{9}・\rensuji*{0}・\rensuji*{0}}\end{shiika}

\begin{shiika}春耕をまぶしく見をりホーム窓
\hfill{\rensuji*{9}・\rensuji*{0}・\rensuji*{0}}\end{shiika}

\begin{shiika}啓窒やシルバーホームの預け解け
\hfill{\rensuji*{9}・\rensuji*{0}・\rensuji*{0}}\end{shiika}

\begin{shiika}春暁の正夢なれや初ひ孫
\hfill{\rensuji*{9}・\rensuji*{0}・\rensuji*{0}}\end{shiika}

\begin{shiika}向ひ合うパ・
\hfill{\rensuji*{9}・\rensuji*{0}・\rensuji*{0}}\end{shiika}

\begin{shiika}おばさんと呼びくれ三人桜餅
\hfill{\rensuji*{9}・\rensuji*{0}・\rensuji*{0}}\end{shiika}

\begin{shiika}浮雲に名付けあそびや春の風
\hfill{\rensuji*{9}・\rensuji*{0}・\rensuji*{0}}\end{shiika}

\begin{shiika}こちら向くラッパ水仙こんにちは
\hfill{\rensuji*{9}・\rensuji*{0}・\rensuji*{0}}\end{shiika}

\begin{shiika}花衣車椅子にも湧くはずみ
\hfill{\rensuji*{9}・\rensuji*{0}・\rensuji*{0}}\end{shiika}

\begin{shiika}思い桜樹齢二百を恋う卒寿
\hfill{\rensuji*{9}・\rensuji*{0}・\rensuji*{0}}\end{shiika}

\begin{shiika}花の雨ワインケーキの香に和む
\hfill{\rensuji*{9}・\rensuji*{0}・\rensuji*{0}}\end{shiika}

\begin{shiika}初咲きの大勺や句や婚の朝    
\hfill{\rensuji*{9}・\rensuji*{0}・\rensuji*{0}}\end{shiika}

\begin{shiika}桜湯のぱーつとひらけり控室
\hfill{\rensuji*{9}・\rensuji*{0}・\rensuji*{0}}\end{shiika}

\begin{shiika}純白の花嫁孫となる五月
\hfill{\rensuji*{9}・\rensuji*{0}・\rensuji*{0}}\end{shiika}

\begin{shiika}柿若葉秘仏開扉めぐり会い
\hfill{\rensuji*{9}・\rensuji*{0}・\rensuji*{0}}\end{shiika}

\begin{shiika}来し道の険しさ言はず余花仰ぐ
\hfill{\rensuji*{9}・\rensuji*{0}・\rensuji*{0}}\end{shiika}

\begin{shiika}御幣上る薫風にのる上棟歌
\hfill{\rensuji*{9}・\rensuji*{0}・\rensuji*{0}}\end{shiika}

\begin{shiika}目つむりて青汁ぐっとばら真紅
\hfill{\rensuji*{9}・\rensuji*{0}・\rensuji*{0}}\end{shiika}

\begin{shiika}痛いとは生ける証しか梅雨の膝
\hfill{\rensuji*{9}・\rensuji*{0}・\rensuji*{0}}\end{shiika}

\begin{shiika}梅雨鏡拭けば亡母にとれほどに
\hfill{\rensuji*{9}・\rensuji*{0}・\rensuji*{0}}\end{shiika}

\begin{shiika}都忘れ咲かせ老いけり京遠く
\hfill{\rensuji*{9}・\rensuji*{0}・\rensuji*{0}}\end{shiika}

\begin{shiika}今年また梅酒たまわる命かな
\hfill{\rensuji*{9}・\rensuji*{0}・\rensuji*{0}}\end{shiika}

\begin{shiika}子つばめの翔つを見送る車椅子
\hfill{\rensuji*{9}・\rensuji*{0}・\rensuji*{0}}\end{shiika}

\begin{shiika}ナイターに興じる老母の片辺して
\hfill{\rensuji*{9}・\rensuji*{0}・\rensuji*{0}}\end{shiika}

\begin{shiika}白髪といていのちあるもの髪洗ふ
\hfill{\rensuji*{9}・\rensuji*{0}・\rensuji*{0}}\end{shiika}

\begin{shiika}ぎょうさんな娘の悲鳴蜘蛛の糸
\hfill{\rensuji*{9}・\rensuji*{0}・\rensuji*{0}}\end{shiika}

\begin{shiika}郷ばなしつきずやさしき団扇かぜ
\hfill{\rensuji*{9}・\rensuji*{0}・\rensuji*{0}}\end{shiika}

\begin{shiika}夏服の派手を鏡に息子の土産
\hfill{\rensuji*{9}・\rensuji*{0}・\rensuji*{0}}\end{shiika}

\begin{shiika}きれし夢惜しや貴船のはも料理
\hfill{\rensuji*{9}・\rensuji*{0}・\rensuji*{0}}\end{shiika}

\begin{shiika}迎はるる仏とならで魂迎ふ
\hfill{\rensuji*{9}・\rensuji*{0}・\rensuji*{0}}\end{shiika}

\begin{shiika}仏めく盆僧の額黒光り
\hfill{\rensuji*{9}・\rensuji*{0}・\rensuji*{0}}\end{shiika}

\begin{shiika}赤とんぼヘルパーと唄う車椅子
\hfill{\rensuji*{9}・\rensuji*{0}・\rensuji*{0}}\end{shiika}

\begin{shiika}星月夜シルバーホーム消灯はやき
\hfill{\rensuji*{9}・\rensuji*{0}・\rensuji*{0}}\end{shiika}

\begin{shiika}誰似かと爽やかろんぎ初曽孫
\hfill{\rensuji*{9}・\rensuji*{0}・\rensuji*{0}}\end{shiika}

\begin{shiika}白桔梗時には欲しい母小言
\hfill{\rensuji*{9}・\rensuji*{0}・\rensuji*{0}}\end{shiika}

\begin{shiika}おきし手を又も引きよす枝豆を
\hfill{\rensuji*{9}・\rensuji*{0}・\rensuji*{0}}\end{shiika}
