
\documentclass[b5paper]{tbook}
\usepackage{ascmac}
\usepackage{shiika}

\begin{document}
\title{ふみこ句日記}
\date{2000/5/51}
\maketitle}

\chapter*{はじめに}
昭和四十八年九月浅野房子さんと三朝温泉への車中、
山下光子に出会ひ三朝の病院に療養中の大塚さんを見舞う
旅だったが 話は吉川美佐姉のすすめにより
京鹿子火曜教室に浅野さん 小田澄子さんが入会


九月初句会に出席した様子だった。
私も一か月おくれて 十月よりともかく出句した。

造る書くと言うことには全々自信のない出発だから
あまり進んだ気持ちでは」なっかった。
以来 もう止めるを繰り返した。
美佐さんへの義理を続けていると言った。

そして十八年の年月が過ぎた。納得のいく自分の句
句は殆んど無い。

個人で句集を作られた句友も何人かあるが 火曜火鏡 合同句集の
仲間入りが精一杯のこと、それ以上自分の句を活字にのこすことは
考えてもいなかった。けれどここ数年前から句日記として 整理
してみようと思い立った。
下手、句になっていない句 それでよい。思うばかりでなかなか
とりかかれないで 二、三年は過ぎた。

今回 玉造温泉 厚生年金会館 保養ホームに入所 山下さん 悦子さん
と合流するまでの一週間 一人の機を得て漸く一頁をかき出し始める。
振り返り見る十八年 記憶確かでないもももあるが
思い出は楽しい;


\hfill {  \rensuji*{3} ・ \rensuji* {8} ・  \rensuji*[1]{26}}


\chapter{野仏}
吉祥会で大森先生 池永先生に 一緒して当尾の石仏をめぐりて s48.9
\begin{shiika}
野仏の笑ひ在せり
\\
曼珠沙華
\end{shiika}

\vspace{0.5cm}

「草紅葉」兼題 幼き日の思い出 s48.10
	

\begin{shiika}
日を浴びてままごとの子や草紅葉
\end{shiika}

\vspace{0.5cm}

	

顔見世の名残を夢に見しも去年	19731200

髪結ひて寝ず娘は待つ初詣	19740100

猫の恋根笹の乱れ昨日今日	19740200

山の色幾重の果の雪解光	19740200
\end{document}
\makeatletter
\newenvironment{shiika}
{\let\\\@normalcr\par
 \list{}{kanjiskip=0.25zw plus 0.01zw minus 0,01zw
  \xkanjiskip=\kanjiskip
 \utemsepz@ \topsep=\z@ \paesep=\z@
 \leftmergin=4zw \itemindeby=-2zw
\listparindentitemindeny \rightmargin=-zw }
 \iterekax}{\par\endlist}
\makeatother



==========================================================================
ななは俳句の達人でもある
              	 年月日  
野仏の笑ひ在せり曼珠沙華	19730900
日を浴びてままごとの子や草紅葉	19731000
顔見世の名残を夢に見しも去年	19731200
髪結ひて寝ず娘は待つ初詣	19740100
猫の恋根笹の乱れ昨日今日	19740200
山の色幾重の果の雪解光	19740200
陵の薄陽の濠も水草生ふ	19740300
娘の縁談又もこわれぬ春の雪	19740300
花過ぎぬいづこともなき旅心	19740400
山裾の雨に煙れる桐の花	19740500
夜神東の明りに映ゆる銀杏黄葉	19741100
野仏の顔かくすまで草の花	19740900
置炬燵向ふ人なきあで蒲団	19741100
年用意丹波男の荷は売れ早き	19741200
友待つに暮色刻々粉雪舞ふ	19750100
風ぬくき末黒野烏群をなし	19750200
化粧水掌に冷えのなし春隣	19750300
綿菓子も売れて野崎の花曇	19750400
花曇年甲斐もなき物忘れ	19750400
若やぎて夏来る歌口ずさむ	19750500
梅雨曇出入せはしき軒雀	19750600
花葵露地の家々箱咲きに	19750600
あらはなるちくり根洗ひ大夕立	19750700
看る夜の心もとなき星の飛ぶ	19750826
子等去りぬ礎石にならぶ蝉の殻	19750800
大月夜唐招提寺の庭に彳つ	
色鳥や朝の湖の小桟橋	19751000
秋惜しむほほ紅少こしさしてみむ	19751000
新鮮と我から言ひて冬菜売	19751200
独り居の朝茶の香り笹に来る	19760100
家長の座に心しまりて大福茶	19760100
新らしき命を呼びて野火勢ふ	19760200
春泥の径つき寺の小門あり	19760300
黄帽子水筒どの児の靴も春の泥	19760300
花の奥雨に煙れる塔のあり	19760400
老鶯や御手の茶壺のかたむける	19760517
老鴬に唐松林行きにゆく	19760516
湖見ゆる古戦場道落し文	19760700
病妹の欲りし日とあり梨供ふ	19760900
鐘楼に屋根草のびて露ふかし	19761017
四つ手網死魚の乾けり秋の声	19761017
晩菊のうつろいはじむ白きより	19761100
晩菊やなほ美くしき謡の師	19761100
秋冷ゆる赤きストビラ散る舗道	19761100
綿虫の籬越え来て雨を呼ぶ	19761100
蛤の潮のしたたり出船待つ	19770305
河原なる飛球の行方風光る	19770300
吉野山春蘭の店は客呼ばず	19770405
花弁ゆれ奥より出でし虻の貌	19770400
燕の子黄ならびの嘴花のごと	19770500
木苺や山の佛の唇あせて	19770625
寝冷え子のうつろの瞳絵本散る	19770700
蜜豆に唇さみし嘘を言ふ	19770700
湖の色北より深み秋きざす	19770800
竹生島真向ふ宿の洗鯉	19770800
登るほど尾花は細し高野道	19770900
行けど行けど穂芒波や夕茜	19770900
天高し隠岐の草原牛肥えて	19770900
霊場の鐘にも和さずけらつつき	19771000
下枝より褪せて小庭の実むらさき	19771000
庭雀床払ひせしふとん干す	19771200
白寿祝ぐ願いをこめて羽根蒲団	19771200
若水や心新らたに栓開く	19780100
句友の訃夜を沈丁の香のせまり	19780300
春潮に群れ飛ぶかもめ水尾追ひて	19780300
門かたく喪の家ひそと花ゆすら	19780400
潮騒の丘の花冷学徒眠る	19780300
城跡の古井戸涸れず苔の花	19780605
桑の実に郷愁ありて札所径	19780600
焼香待つ黒幕裾の蟻地獄	19780700
葉鶏頭一筋町の故郷晴れ	19781000
結願の杖納め得し鵙日和	19781000
花売の残す菊の香路地の朝	19781200
口ませし孫の電話や冬すみれ	19781200
曼珠沙華島の陵人稀に	19780900
出張のしげかれ疾かれ牡蠣土産	19781000
寄れば逃ぐ子に獅子舞の昂りて	
寒餅を切る夜のまど  とろり	
旅立ちの鏡に向ふ夏帽子	
久々の子に浴衣着せ今宵酌む	
草の花名を問ひ問はれ三輪の径	
三代が屠蘇なみなみと三つの盃	19790100
冬萠や繃帯の足歩を試す	19790100
昂りぬ沈丁の雨音もなく	19790300
啓執や旅誘ひの友便り	19790300
山の温泉は音なく春蚊早出でし	19790420
草餅に門前町の賑へる	19790600
実生栗初花咲けり吾も健	19790600
冷奴遠き旅より帰り酌む
落ちるまま実梅の匂ひ城のみち	19790716
城の灯のうるみ郡上の踊更く	19790823
新秋や欄間彫る町木の香り	19790824
谷底は見えずバス行く山の霧	19790824
高原の駅コスモスの色極め	
結願の梵鐘ひびく峯の秋	19791200
太りゆく大根今日も抜き惜しみ	19791200
実むらさき実生をたのむ土かぶせ	19791200
青木の実名知らぬ鳥も枝くぐり	19791200
心地よき帯のしまりや謡ひ初め	19800100
新年の交す汽笛に群れ鴎	19800101
通夜の冷え遺作のばら絵明るきも	19800000
出棺す白梅こぼる砂踏みて	
雨戸くる朝なあさなを蕗育つ	19800400

ななの俳句はまだまだあるよ
              	 年月日  
菜園の菊菜色よし久の子に	19800400
青葉して忌ごもる友と病める友	19800500
明易し潮騒近き島の宿	19800531
島の雷止みて翼船ましぐら	19800601
梅雨嵐し離れ病む子をただ祈る	19800600
見送られ見返る薄暮白あやめ	19800600
健やかな孫の寝息やプール焼け	19800800
草引きて草の匂ひの手枕寝	19800800
水引の紅ぬれづめに水車	19800900
みのり田の道登校のペダル踏む	19800900
温泉涼し重き一事を成しとげて	19800900
退院の友いきいきと派手浴衣	19800717
ダム澄める揺れ映りいる合歓の花	19800802
露天湯の一灯淡く月見草	19800803
霊峰の碧に真向ひ秋ざくら	19800804
先急ぎつつ仰ぎゆく峯紅葉	19801102
しみじみと語らな白菊活けて待つ
遠き旅はなやぎ帰り菊を焚く
枯菊を焚きつつしばし物思ひ
鉄橋を渡れば小駅片時雨	19801200
黄の翅の止り色増す実むらさき	19801100
天高し施肥よく効きし畑の色	19801100
七草の数揃はねど畑の菜を	19810100
一望に漁港おさめて梅の丘	19810130
春炬燵尽きぬ話の果は伏し	19810300
春の冷え別れて一人立つ小駅
争ひてふと空しかり梅の闇	19810300
合格の祝袋は字も太く	19810300
摘みし蕗独りの厨たのしかり	19810400
散る桜庭の胸像ただ黙し	19810400
武具飾る子は父となり遠くあり	19810500
解禁の夕べたまはる吉野鮎	19810500
釣りし鮒川に戻して春の風	19810400
冨士聳ゆ裾野の町の鯉のぼり	19810500
滝水をコップに汲みて喉しまる	19810700
御詠歌の流れへいそぐ地蔵盆	19810800
枝豆に酌みて不意なる遠き客	19810900
釣る夫の片辺に妻の秋日傘	19811000
武家屋敷崩れ土塀に石蕗盛り	19811022
草子里時雨れる朝の大き虹	19811024
わだかまり解けて減りゆく盛みかん	19811000
噂消え火事場に茂る泡立草	19811100
売地札草にかくれて秋暮るる	19811100
栗おこわ我が誕生は頃もよく	19811100
霜よけにレタス生々玉巻ける
供華の菊剪りためらひぬ眠り蝶
落葉炊く煙の中に思ふこと
新らしく菊きり供え旅に出る
踏み惜しみつつ鎌倉の銀杏黄葉	19811124
ウインドに背まるく映る師走町	19811200
晦日そば孫の食べざま頼もしく	19811200
窓の梅ほころびゆくをみるしじま	19820200
散り梅のかかり濯ぎのもの乾く	19820200
春遠しこもれる叔母に京の菓子	19820200
受験生泊めて祈りを同心に	19820300
日脚伸ぶ中洲に群れる鳥の白	19820300
蕗の薹焼みその香の朝厨	19820300
散る花の流れゆくあり踏まるあり	19820407
天主より振る手呼ぶ声花の中	19820400
葱坊主垣越しの子はよくしゃべる	19820500
耳遠く笑顔で応ふ木の芽雨	19820500
草餅にふと道変へて娘に急ぐ	19820500
直ぐ消ゆる足跡砂に五月旅	19820511
風光る砂丘を踏めば若返る	19820511
石段のあえぎに著莪の花やさし	19820512
単線の停車は長し青田風	19820600
花栗の香に堂守の鍵開く	19820700
老鴬や堂守力こめて説く
知床の大雪渓に昼の月	19820629
雪渓を映し知床五湖寂と
えぞかんぞう岬はるかは異国なる
昆布乾すさいはての島明易し
獅子独活の花眼の限り能取岬	19820706
鷺草の鷺二羽となる娘に甘え
魂迎ふ一人となりて古家守る	19820800
手ごなしで土をかぶせる秋の種
豪雷にいさかふ妹弟抱き合ふ
亡娘ノート紙魚生きている悲しさよ
秋立ちぬ束ねてさせり亡母の櫛
晩菊の咲くや明日より他人の庭	19821000
引き越しの荷隅にかばふ冬すみれ	19821000
秋そゞろ引越荷物嵩む部屋	19821000
秋風も他人もやさし移り住み	19821100
見捨てかね新居に挿せり倒れ菊	19821100
寛ぎて見る山荘の紅葉濃し	19821100
乗りおくれくやしき顔に冬の月	19821100
寒椿にぶる起ち居のすべもなく	19821200
友呼ばむ一人に余る日向ぼこ	19821200
転宅の迫りし庭の実むらさき	19821000
移り住む名残の菊香衰えず	19821000
玉砂利に歩の乱れなし神の留守	19821000
大役の初旅冨士が雲間より	19830103
梅日和白壁光る村一望	19830200
しつけとる春立つ朝の装ひに	19830300
水ぬるむ就職決り紅さす娘	19830300
桜餅娘の訪ひくれし小半日	19830300
目口なき紙の雛や掌になじむ	19830300
裏の家の雨に堪へ咲く八重桜	19830400
友の情雨に摘みきしわらび飯	19830400
忌に集るしのぶ日がなを花の雨	19830400
楠公通の大楠学校庭に移し植え
除り去らる囀り包む街の樹が
読むも憂し眺むも憂しや花の雨
集ればお国訛よよもぎ餅
秩父路につづく芽桑の夕映えて	19830407
万緑や一言神に願一つ	19830521
田植機の若者帽子に赤い花	19830521
桜桃たわわの国へ喜寿の旅	19830611
杖たよる友出迎へに梅雨はげし	19830700
朝涼し咲きつぐ花を供華日記	19830700
引き越して来たる浜木綿咲き安堵	19830700
娘三人訪ひくれ風鈴よく鳴れり	19830700
一族の年長となり魂まつる	19830800
動かぬ灯動く灯一望盆の果	19830800
洗ひ髪立つベランダの風は秋	19830800
蕎麦三日食べてさわやか信濃旅	19830904
色鳥や岳に真向ふ湖の宿	19830900
大き鳥湖上を舞ひて夏去れり	19830900
庭紅葉もえて謡に力声	19831100
謡ひ果て山荘黄葉をのこし暮る	19831100
翅やすむ蝶もむらさき式部の実	19831100
独り居のよき日淋し日菊挿して	19831100
疎く住み安けき日々や杜鵤草	19831100
屑金魚育ち掬ひし児も少年	19831100
案内三日京の紅葉に酔ひ疲る	19831100
照紅葉京一望の峯の寺	19831100
山荘の集ひに菜飯冬ぬくし	19831207
冬入日竹叢透し荘なごむ	19831207
一とせを会ひ得ぬ人の賀状増し	19840100
しきたりをつづけて独り屠蘇機嫌	19840100
トンネルを抜ける度雪深くなり	19840102
ただいまと灯せば応ふ室の花	19840200
ちゃん呼びで遠き日戻る木の葉髪	19840200
春寒やぱったり出会ひ出ぬ名前	19840200
争ひも夢よ首塚土筆の芽	19840300
老夫婦夜をぼつぼつとひなあられ	19840303
雪解風由布岳さして大鴉	19840305
土を割る花芽それぞれ色ありて	19840300
によきによきと花芽ラッシュの庭の土	19840300
花苺児にしやがみ見す芯の粒	19840400
朝毎の独りに足りる庭苺	19840500
団地住みテレビの上の兜の威	19840500
ホース先そらせばそこも青蛙	19840700
花南天隣初嬰の襁褓干す	19840700
待ちつつも一人を凉しと思ふ日も	19840800
庭茂り払ふ枝にもある生命	19840800
孫の名をとりちがえ呼ぶ盆家族
夏萩に誰みくじ結ふ禁よそに
忌ごもりの友訪ひて汨つ戻り梅雨	19840700
夏書終へ東塔西塔仰ぐ朝	19840600
空と無の多き夏書や朝鴉	19840600
りんどうや標高識のたつ小駅	19840900
高原列車おそしとゆれる花すすき	19840900
紫の小波たてり松虫草
思はざる遠冨士すゝきの小窓より
朝風に彩をひろげてのうぜん花	19840700
風凉し天主の床の黒光り	19840700
俳聖殿忍者屋敷も蝉しぐれ
秋凉し絵とき説法に笑ひあり	19840917
水軍の洞の跡や秋の潮	19840917
青い眼の手ぶりに見入る踊の輪	19840800
諷刺歌踊りの櫓は高調し	19840800
送り火やもとの一人に戻る夜	19840800
帰省子の言葉大人ひふと淋し	19840800
若者となるは別れか鳥雲に	19840800
夏霧の湧きて流れて山の湖	19840700
山茶花の垣咲き始めぬ謡声	19841100
冬の雲まこと知らせぬ人見舞ふ	19840000
年忘れ流す憂さなきワインの香	19841200
賀状書く亡母の字に似る母の年令	19841200
寄せ鍋の沸々はずむ故郷ことば	19841200
するつと食ぶ熟柿に郷愁そぞろ湧く	19841200
吾が誕生秋刀魚で祝ひ心足る	19841100
初冨士や大東京の隅に住み	19850100
林立の煙突冨士に初煙	19850100
初仕事裾野の町の白煙	19850100
移し植え三年の梅に初つぼみ	19850200
陽を集め日毎ふくらむ木瓜の花	19850200
蘭匂ふ独りの部屋に惜しき程	19850300
逆縁の香たく背なに春空し	19850200
春や憂し着かえし裾の静電気	19850400
割れ込まれ句心とぎれぬ春炬燵	19850300
初蕨(わらび)雨に持ちくれ留守の扉に	19850400
名にひかれ植え初花をひめ辛夷	19850400
天主より眺むる花の城下町	19850421
階高し一打の鐘に花の散る	19850421
老鴬に耳あそばせて喜寿の足	19850509
蝸牛わがもの顔に城跡の碑	19850509
ぷちぷちと峠に摘めり夏わらび	19850618
木苺の酢っぱ甘さや渓流に	19850617
塗りかへて狭庭の客に青蛙	19850600
花ざくろ觸れて硬しや朱の色	19850600
御名のごと清らに生きて蓮花	19850600
たまはりし紫式部さわ咲けど	19850800
短夜や句机ならぶ夢の切れ	19850800
夜濯ぎて一日終りぬ恙なく	19850800
働けることの幸玉の汗	19850800
言ふだけで気のすむ愚痴に団扇風	19850800
階暑し団地こつこつセールスマン	19850900
梅雨しめる記帳簿将軍旧居訪ひ	19850625
苔の花将軍愛馬の小さき塚	19850625
    将軍旧居もちの花	19850625
意を通し過ぎし淋しさ夏の蝶	19850625
小駅の時計おそしと思ふ時雨来て	19851119
名もゆかしこほろぎ橋の渓紅葉	19851120
冬の雷一発のみや能登に泊つ	19851120
冬ぬくし見舞ひし友にもてなされ	19851200
謡声白山茶花の垣流れ	19851200
小説の終りのごとく落葉散る	19851200
愛語りし腰掛石や昼ちちろ	19850000
曼茶羅に政子のむかし秋そぞろ	19850000
露けくて墨のうすれしいわれ書	19850000
輪飾りの小さきをかけ団地の扉	19860100
寒木瓜の紅を深めて雨上る	19860100
盆梅や鉢の木謡ひたき夜なり	19860100
成人の日の背広着し子を見上ぐ	19860200
試験子の窓に憂きほど春深雪	19860300
弔ひて無口の帰り春吹雪	19860200
ことなげに抜歯をされて春寒し	19860300
白梅や三百年を語る幹	19860300
ゆずり合ひつヽ空うばひ梅盛る	19860300
春時雨急げば合はす鍵の鈴	19860300
土を割る花芽それぞれ色ありて	19860300
書き終えてほつと紅茶の浅き春	19860300
庭隅に鈴蘭匂ひ旅ごころ	19860400
屋根草もうすき緑に御寺春	19860400
枝うつるりす生き生きと新樹光	19860400
散るものは散らして扇塚の春	19860400
明日に咲く牡丹見よと泊めくれし	19860500
牡丹の今開かむと息づかひ	19860500
身も心青く染まりぬ宮若葉	19860500
山越ゆるあの辺野崎か花曇	19860400
バスの窓遠見を塞ぐ栗の花	19860613
蛇の衣板一枚の城跡文	19860614
アイスクリーム売の熱弁落城譜	19860614
蔦青し城見ゆ坂のオランダ塀	19860615
青葉冷え天主の跡の落城譜
踊太鼓すぐそこにきき足を病む	19860800
山男めきひげ面の帰省孫	19860800
癒ゆること信じてきけり蝉の声	19860800
癒ゆきざししかと凉しき今朝の風	19860900
亡母の櫛ふとさしてみる盆支度	19860800
杖に頼る試歩の足もと萩こぼる	19860900
寝団扇にうちわどころの故郷のこと	19860900
去ぬ燕便りとたよりすれちがひ	19860900
鰯雲交しておかむ生き形見	19861000
風に雲に秋の深みを知る夕べ	19861000
カタカナ語事典にいどむ老夜長	19861000
菊の香や来し方遠し五十年忌	19860900
雲を割り冬陽美し退職す	19861100
むなしさも煙としたり菊を焚く	19861100
年用意心のこもる故郷の荷	19861200
満目の紅葉それぞれちがふ色	19861115
静かなりいで湯娘と在り去年今年	19870101
たまさかの晴着に帯と初芝居	19870100
シテ謡ひ修めし安堵室の梅	19870100
誰が為と笑はれもして初鏡
梅白し陽ざしの居間の笑ひ声	19870200
男子校女子校つづき芽ふく道	19870200
庭の陽を占めて寒木瓜紅の濃し	19870200
火廼要慎祀符の墨字に春ぼこり	19870300
今日は憂し今日は美くし木の芽雨	19870300
春愁を恥じて陶狸の腹を撫ず	19870300
名桜につきぬ名残の里を去る	198870419
山裾の梨の花園に白昼夢	19870415
花クローバ終の棲家の地鎮祭	19870500
松の花傘寿を集ふ公の庭	19870513
文学館出でてまぶしき若葉光	19870513
目礼がことばよ通院路の茂り	19870600
青葉雨千人塚の匂ひ濃し	19870527
土産店菖蒲と競ふ肥後名所	19870428
五月晴阿蘇の寝釈迦に帰途祈り	19870529
夏草に五百羅漢のかくれんぼ	19870709
夏草にあそびつ羅漢の泣き笑ひ	19870709
自転車で五日の旅の戻り梅雨	19870700
初咲きの桔梗と供華に朝づとめ	19870800
夜濯ぎの干場思はず下手な歌	19870800
八階に住みて音なき遠花火	19870800
早発ちてさかさ冨士みむ秋の湖	19870915
霧晴れて小波が消すさかさ冨士	19870915
文学碑たてる峠に秋の冨士	19870915
花すゝき駅近かそうで遠かりし	19870904
招くごとコスモス揺るる無人駅	19870904
誰も来ずくつろぐ時の菊日和	19871100
老夜長旅に集めし箸袋	19871100
とっておきのワインもてなす良夜かな	19871000
南洲を語る白髪月の部屋	19871000
紅葉濃し峠二つを越えし温泉	19871119
隣より争ひ声や秋の暮	19871100
石蕗さかり先は稲荷の鳥居径	19871100
海知らぬ犬を毎朝冬の浜	19871200
新らしき木の香の中に賀状書く	19871200
看とりつつ句帳かた辺に長き夜	19871000
看とり女にある秋晴や特選句	19871000
祭太鼓看とりの窓に遠くきく	19871000
安眠なき看とりの夜々に虫親し	19871000
愛語りし腰掛石や昼ちちろ	19871000
露けしや墨のうすれしいわれ書	19871000
曼茶羅に政子の昔秋そぞろ	19871000
寒青空娘は頬染めて婚約を	19880100
梅二月婚約成りし娘のまぶし	19880200
婚近き娘と春いちご分ちあい	19880300
列車徐行深雪のここに友住ふ	19880200
たまわりし手造り味噌に蕗のとう	19880200
枯芝にねてにらまるゝはらみ猫	19880200
春寒や三日もつづく探しもの
春灯失せものこゝに出て笑ふ
椿落つ今日も名知らぬ鳥の来て	19880300
ゆかし名ばかり揃えて盆梅展	19880200
春潮に水尾ひく連絡船(ふね)のあと幾日	19880300
終航の間近かき名残瀬戸の春	19880300
花菜漬土産に訪ひくれ京言葉	19880300
手染めとて淡き春着の京言葉	19880300
花冷えて鬼女の棲みける巨き岩	19880423
恐ろしき昔語りや花の里	19880423
杉古りて黒塚ひそと花曇る	19880423
若やぎて傘寿の集ひ牡丹園	19880516
声低く僧が餅売る牡丹寺	19880516
手をとりて笑む道祖神若葉光	19880516
花の雨眠る山湖を去りがたく	19880517
老鴬や奥へとたずね政子墓所	19880601
旧姓で呼びあふ荘の明易し(鎌倉荘)
まぐなぎを払ひ百体地蔵訪ふ	19880600
探ねゆく流れ涼しき渓いで湯(太閤の湯)	19880700
カンナ燃えひしめきあえる養鶏舎	19880700
雲走り峯にこま草這ひて咲く	19880700
浜木綿にしばらくのこる夕茜	19880700
故里の植田にうつす己が影	19880800
錦飾る故郷ならずも茄子の花	19880800
甚平着て今日も碁敵待つ	19880800
叔父跡地ひまわり咲かす家五軒	19880800
朝顔や一家は北に赴任して	19880800
秋蝶が惜しむ別れの前よぎる	19880900
見送りの垣根アベリア咲きこぼる	19880900
滝二つ遠見の台に小手かざし	19880900
穂すすきのみるみる刈られゆく売地	19880900
吾が暮し覗いて聞いて青芒	19880900
秋と思ふホームに目立つ黒い靴	19880900
爽かや事終へて発つ旅の朝	19880900
大秋晴善光寺平一望に	19880900
歌声をのせて寄せ来る芒波	19880900
コスモスのゆれる川沿ひ遊歩道	19880900
母となる娘に寄す思ひ冬ぬくし	19881100
実南天紅し娘は母となる	19881100
晩菊や終止符打たん独り住み	19881100
息子と同居決めむ独りの湯豆腐鍋	19881100
トンネルを出て越前の雪景色	19881200
仏壇を買ひに越路へ雪清し	19881200
山ふところに香煙みちて初薬師	19890102
初護摩の煙いただき肩かるし	19890102
紅梅のふふみしことも友へ書く	19890000
大茶盛廻す茶碗に和気あふれ	19890100
寒木瓜の紅流れそう雨つづく	19890200
春寒し故なく心のとがる今日	19890200
契約のとれてマフラー忘れ去ぬ	19890200
雪ごもり写経の日々と紙便り	19890200
春風や繰り上げ帰国のよき知らせ	19890200
引き越しの迫り咲きつぐ春の彩	19890300
転宅の別れの集ひ鰆すし	19890300
すましたる貴婦人めける柴木蓮	19890400
昼顔や島にたづねる古き墓	19890430
夕明りのこる卯波や島に泊つ	19890430
城下町一望にほふ栗の花	19890425
お天主へ石垣高し松の花	19890425
天主閣仰ぐ茶店の藤こぼる	19890425
紫陽花の彩拡げゆく遊歩道	19890500
夏三つ葉雨の小やみに摘む留守居	19890600
母も娘もショートカットにさくらんぼ	19890600
窓開き大向日葵に見つめらる	19890700
驕りても向日葵は好き美くしき	19890700
留守居して一人に惜しき風凉し	19890700
水撒きて陶狸うれしき顔となる	19890700
思ひきり水撒き散らす重きもの	19890700
賞め言葉裏に返さず花クローバ	19890700
水撒きて木々と話をする留守居	19890800
白粉花空家となりし垣に満つ	19890800
病葉のこの量踏みて医に通ふ	19890800
鳶舞ふ高野の夏の深き空	19890700
野猿乗り夏の河原の若者等	19890700
グラヂオラス店の娘明るく迎へくれ	19890700
ポンポンダリヤ活けて村営コーヒー館
漁火に想ひそれぞれ宿浴衣	19890800
盆列車着席までを送らるる	19890800
伝説の湖ははるかに芒原	19890900
湖も山もみるみる消えて霧の海	19890900
山の霧流れて速し湖生る	19890900
のぼり来て賽の河原の細芒	19890900
旅に訪ふドラマ舞台の町も秋	19890900
久の出会ひ杖目じるしと言ふも秋	19890900
秋釣の成果に夕餉賑へり	19891000
秋雨のやまず留守居の夕仕度	19891000
コスモスの身丈を埋めてはるか冨士	19891000
湧き水の秋澄む池に冨士の影	19891000
天高し誕生釈迦の細き指	19891029
落葉かき風に根気の作務の僧	19891029
柿届く家なき故郷の友も老ひ	19891100
郷言葉の電話果なし老夜長	19891100
命延ぶ泉いただき峯を越す	19891106
野仏の膝にさい銭紅葉散る	19891106
冬濤の音きヽ紀伊の朝茶粥	19891200
娘が立てし枕屏風に安眠して	19891200
晩菊に名残水やり旅に出る	19891200
報恩講善女となりてしる粉賜ぶ	19891029
花車たがへず来たり年用意	19891200
心ゆくまで謡ひけり年忘れ	19891200
娘の忌日となりて年経る小つもごり	19891200
旅立ちを止めて眺むる強吹雪	19900100
おくれ咲く紅山茶花の雪化粧	19900100
潮の香をはこび来る風春近し	19900200
水温みあひる天国てふ川辺	19900200
指圧効きかろき足もと蕗のとう	19900200
桃ふふみ声出し笑ふと嬰便り	19900300
初雛に招かれ曾孫しかと抱く	19900300
亡母の忌や弟としのぶ春炬燵
高々と辛夷咲きみつ城跡園	19900300
もてなさる小さき土鍋に土筆煮て	19900300
こんがりと焼味噌蕗のとうほのと	19900300

ななの俳句はまだまだあるよ
              	 年月日  
蕗摘みて老の自慢のちらしずし	19900400
一心の白夕闇にほのと浮く	19900400
陶狸の背出で入る鳥の巣づくりか	19900400
葉桜や友のギブスはまだ除れず	19900400
露座観音見おろす里の柿若葉	19900500
柿若葉光る白壁つづく里	19900500
風薫る河童出そうな筑後川	19900500
老鴬に迎えられけり峡の宿	19900500
鱚一尾釣りて得意の帰宅ベル	19900600
釣りし鱚ほめて一箸づつ廻し	19900600
ご協力と酢い甘夏を嫁出し来	19900600
紫陽花や登山電車は幾曲がり	19900600
お世辞とも思ひつつ買ふ夏帽子	19900700
夏帽子鏡の顔はヤヤすまし	19900700
のびて寝る猫のかたへに端居して	19900700
待つ荷物おそし木樺はしぼみ初む	19900700
鎌倉の御寺凉やか友葬る	19900700
母として慕はれ甥とビールくむ	19900800
風鈴や父母知らぬ甥よき父に	19900800
五十年忌修すあの日も秋暑く	19900800
巨寺にみちのくらしき萩まつり	19900900
雨上がり紅たわヽなるりんご園	19900900
子に孫にりんご送りて津軽旅	19900900
台風もよしといで湯にやり過ごし	19900900
久に来し皇居のお濠曼珠沙華	19901000
コスモスの風に流せるほどの些事	19901000
ただ声をききたく夜長の遠電話
バスを待つこわれベンチに秋の蝶
茫々の芒の中や美人塚	19901110
神在月とガイド熱あり出雲路よ	19901110
濃紅葉座禅堂の扉はかたく閉じ	19901119
寄進瓦に筆持つひまも紅葉散る	19901119
庭小春鳩来て犬が少し吠え	19901200
晩菊や顔見ぬ電話言ひ過ぎし	19901200
枯木してはるか冨士見る道となる	19901200
数の子の歯音うれしや八十路三つ	19910101
初詣極楽寺てふ名にひかれ	19910102
初旅や全き冨士に真向へり	19910100
立春の陽に勇気湧きトレーニング	19910200
足鍛え眠り覚めたる山のぼる	19910200
人波に流されてみる梅まつり	19910200
指呼の山みるみるかくす春吹雪	19910219
舞へ狂へいで湯ごもりの春吹雪	19910219
ほの酔ひや孫つぎくれしお白酒	19910300
ひなの前老も交りて撮る今宵	19910300
梅林へ少しの坂も手を引かれ	19910310
白梅の古木に希ふ吾が余生	19910310
湖見ゆる観音堂の大桜	19910400
芽柳の日々に大ゆれ風青し	19910400
花散るや石州瓦の光る村	19910400
初蝶や癒えて佇つ庭彩ふえて	19910400
初蝶やふっつり切れし思ひごと	19910400
新茶賜ぶ少年今は病院長	19910500
芍薬や三度の転居共にして
染め止めて白髪軽し青葉風	19910600
年令らしく白髪でおしゃれ夏帽子	19910600
釣り土産べらとはうれし瀬戸育ち	19910600
早苗田の日毎濃くなる療の窓	19910600
山の湖万緑の中遠くあり	19910600
山間の夏霧深き駅に着く	19910700
立葵彩を揃えて山の駅	19910700
薬草湯の香りのこりて宿浴衣	19910700
大寸の宿衣たぐりて岩魚膳	19910700
億の土地我がもの顔に青すすき	19910800
通院の道は川沿ひ月見草	19910800
時計おそし独り留守居の小粒ぶどう	19910800
秋暑しビルの掃除夫見上ぐ窓	19910800
保養所のヴェランダ踊りの列を見る	19910800
踊りうちわよべの土産と保養友	19910800
秋の湖哀話流して遊覧船	19910900
温泉の町にお湯かけ地蔵秋うらら	19910900
敬老日ほの酔はされて若返る	19910900
誰が家ぞ芒刈られて地鎮祭	19910900
秋場所の終り落ちつき夕支度	19910900
ゆかしさに秋七草の寺巡り	19910900
尊氏も正成も美男菊衣	19911000
天高し八十路二人が峯に彳つ	19911000
穂芒の波うねうねと芒山	19911000
秋茄子を嫁にすすめて共笑ひ
神有りの出雲の湖はかもめ舞ふ	19911100
宍道湖の大橋たもと柳散る	19911100
宍道湖の秋の入日に出合ひけり	19911100
名菓舗の近くに石焼芋の声	19911100
鳴き砂を踏めば聞えし秋の声	19911100
白髪を少しのぞかせ冬帽子	19911200
もう一度鏡をのぞく冬帽子	19911200
久に会ふ少しおしゃれに冬帽子	19911200
諦めもした犬癒えて冬ぬくし	19911200
独言ならずチロとの話始め	19911200
愛犬のチロも淑気の尾をふれり	19920100
年の夜吾より古き茶棚拭く	19911200
立春大吉吾より古き茶棚拭く	19911200
名水へ凍ての渓路手をひかれ	19920103
謡初帯山小さく装ふ同志	19920100
謡初足のねぢりを許し合ひ	19920100
保養所で看る東京の雪ニュース	19920200
お返しを気にする老や冬いちご	19920200
大山ははるか田に群る白鳥かな	19920200
旅帰り待ちくれ紅梅咲き満つる	19920200
紅梅や吾が色にせむと言ひし亡友	
梅の闇逢ふ日約せし友逝きぬ	
旅はずむ卒業進学祝ぎ二つ	19920300
たまさかの母と息子の旅春の虹	19920300
春眠の十指ほぐしつ今日へ覚む	19920300
春セーター鏡に肩のうすきこと	19920300
美くしく老いたきものよ柴木蓮	19920400
シクラメン茶の間笑ひ溢れさす	19920400
ふる里はすみれたんぽぽ墓の径	19920400
桃の花さら前かけの辻地蔵	19920400
お遍路の憩なる礎石大伽藍	19920400
菜の花を手いつぱい摘み日毎漬け	19920400
日々摘めど菜の花畑の黄は濃ゆく	19920400
花杏真白従妹に甘え気味	19920400
芍薬の蕾ふくらむ庭の日々	19920500
発つ朝にうす紅ほのと花水木	19920500
いそいそと半袖えらび旅立てり	19920500
山迫る車窓次々藤の花	19920500
若葉風亡妹の友とめぐり逢ひ	19920500
短か夜や亡妹の友と泊つ出雲	19920500
ビール酌むかちんとグラス若やぎて	19920600
ビール酌むドラマのように共鳴し	19920600
ビール乾し少し多弁に刻忘る	19920600
向日葵が君臨空地の草いくさ	19920700
木樺咲く一日の花の教えごと	19920700
垣根ばら互の無事を老犬と	19920700
夕仕度水の出細き大暑かな	19920700
開け放つ窓に早起き木樺かな	19920700
酌みもして婿の気配り凉しき餉	19920700
倒産の去りゆく一家百日紅	19920800
一言がちくりと秋の草に棘	19920800
遠冨士の景ある売地草茂る	19920800
芝生踏む素足に伝ふ今朝の秋	19920800
新凉や試歩の芝生に笑み交す	19920800
高階に寝て眺め居り雲の峰	19920800
霧にまだ眠る町並試歩はげむ	19920900
夏霧の深し湯の町まだ覚めず	19920900
回廊に沿ふ白萩に清めらる	19920900
水攻めの城跡や蓮の実の大粒	19920900
苗木より三年無花果三つ熟れる	19920900
長生きに想ひいろいろ敬老日	19920900
秋灯下親しきものは虫眼鏡	19921000
保養所の昼餉にぎやか大秋刀魚	19921000
露芝生試歩の目標果し得て	19921000
秋日和木椅子に一病話し合ふ	19921000
シャッターを頼む一会や寺紅葉	19921000
庭園灯淡きに和せぬ木犀の香	19921000
実梅の香まこと顔して嘘をきく	19920700
夜の仏間大蜘蛛打ちて逃がしけり	19920700
耳遠く独りもよしと新茶汲む	19920700
魂迎ふやがては迎えらるる吾	19920700
帰省子に一夜越し方きかれけり	19920700
山荘の冨士見ゆ窓に姫りんご	19921100
夜霧匂ふ同郷なりし荘の主	19921100
天高し無傷の紺を飛機が割る	19921100
セーターの赤を鏡に問ふ八十路	19921100
声高や桜紅葉の女子校道	19921100
迎えられ娘の柚子風呂の香りかな	19921200
いさかひが笑ひに母と娘の冬至	19921200
年用意母と娘の声いづれとも	19921200
部屋に冷ゆ胸像の夫に独り言	19921200
行く年へ刻む時計に息つめて	19921200
我が城と正月飾り四畳半	19930100
繰るほどに夢ふくらみ来初暦	19930100
二日早帰る子送る母の背	19930100
好物で老犬はげます寒の入	19930100
居候の老に朝毎寒玉子	19930100
老犬と共に留守居す梅日和	19930200
老犬の背に紅梅の一片が	19930200
一跳ねに広がる水輪水ぬるむ	19930200
春立ちぬ川面は白き雲浮かべ	19930200
白き雲浮かべ川面は春立ちぬ	19930200
倖せは歯音にありし年の豆	19930200
今日よりはチロ居ぬ生活春寒し	19930330
姫こぶし一輪樹下にチロは死す	19930330
春嵐おさまる朝にチロは死す	19930330
春寒しピンクの布に巻く屍	19930330
窓開けばおやつ待つチロ無き余寒	19930330
従姉妹どち幼な呼びして桃の郷	19930400
故里や摘みてたちまち木の芽和え	19930400
故里はお遍路の鈴あわあわと	19930400
朧夜や骨までしゃぶる瀬戸の味	19930400
短夜やはらから集ふ郷言葉	19930400
老鴬に迎え送られ札所寺	19930400
仁王門くぐりて見上ぐ余花やさし	19930400
牡丹や余生つぎこむ花づくり	19930400
新背広卒業の子を見上げけり	19930400
祝背広就職といふ巣立かな	19930400
就職は別れの一つ鳥雲に	19930400
散華とも霊園しとど花吹雪	19930400
咲き競ひし源平桃も葉となりぬ	19930500
藤娘出そう藤房ととのへり
三代の旅信濃路を青葉風	19930500
大手まり真白湯の香の中にゆれ	19930500
まじり気のなきみどり嶺よ露天風呂	19930500
峯八分疲れは軽し藤の花	19930500
からみ合ひ花房乱る深山藤	19930500
子に植えし桜桃熟るる少女有美	19930400
遍路憩ふ礎石千年語りつぐ	19930400
点滴の紫班をさする梅雨の窓	19930605
明易すや退院といふ別れかな	19930513
濃紫陽花点滴の染みうすれゆく	19930513
錠剤をならべ数えて夕薄暑	19930700
負け相撲少し頭痛の戻り梅雨	19930700
連れだちていそいそ母娘浴衣買ひ	19930700
連れだちて母娘の購む派手浴衣	19930700
浴衣茶会立居気になる娘を送る	
月下美人迎へ車で御対面	
月下美人息を弛めず咲き拡ぐ	
手伝ひ娘不満あるげに水を打つ	
咲きましたとて嫁が見す鷺草鉢	19930800
鷺草の飛びさる舞ひよう目離せず	19930800
水撒けば陶狸がうれし涙する	19930800
これはまあ皿をはみ出る初秋刀魚	19930900
倉裡裏の鬼灯赤し妻若し	19930900
猫難の子雀放つ秋彼岸	19930900
雀獲りしかり猫抱く秋彼岸	19930900
映る影流るる音も水の秋	19931000
秋晴やいそいそ釣に碁敵と	19931000
秋晴や碁敵はまた釣がたき	19931000
釣りし沙魚はねる厨にはや碁音	19931000
雁渡る双手で握手する別れ	19931000
口釜へ増ゆる孫との日向ぼこ	19931000
柿送る案内電話の郷言葉	19931000
柳散る入日に染まる湖のほとり	19931100
五指ほぐすなだむ節おし今朝の秋	19931100
夜逃げとや閉ざせる窓に満月光	19931000
人恋ふかに垣越し延び来青き蔦	19931000
猫舌は母似亡母恋ふ湯豆腐鍋	19931200
物言はず一日留守居の師走呆け	19931200
冬日向売れぬ空地は猫のもの	19931200
カレンダーも庭も山茶花日々惜しむ	19931200
柚子ほめてつい佇ち話いただけり	19931200
留守居して米研ぐ窓に寒宵月	19931200
大晴れや蒲団干す家干せぬ家	19931200
爪切りて指美しや賀状書く	19931200
吹き溜る枯葉の中の紅一葉	19931200
宵戎押さへ揉まれて娘はきげん	19940100
ただいまの娘の声弾む宵戎	19940100
初釜へ晴着見送る母も美し	19940100
はよ来ませ郷言うれし初電話	19940100
寒玉子盛りあがる黄身老もまた	19940100
春寒やもう夢でしか逢へぬ人	19940109
頑張れよ愛犬館も初日さす	19940100
受験子に買ふ知恵袋文殊さま	19940116
春寒し起ち居いちいち声あげて	19940200
中古車群旗はたはたと春を呼ぶ	19940200
猫柳活ける娘もまたつやつやし	19940200
花葉挿しふと京の友思ひけり	19940200
再会や土を割り出る花芽たち	19940300
分葱和へおふくろ味の老自慢	19940300
名もゆかし若草豆腐のうすみどり	19940300
点心に一口ほどのたらの芽よ	19940300
茄子胡瓜畑銀座と故里便り	19940600
額の花一人で居たき時もあり	19940600
夏帽子のぞく白髪も好しとして	19940600
夏帽子年齢をきかれて逆に問ひ	19940600
山梔子の真白につらき雨つづく	19940600
青葉風入れてもきれぬ愚痴話	19940600
言ひたきをたたむくちなし真白なる	19940600
辻地蔵朝取りトマトにお眼細く	19940700
暑に耐える白前掛の辻地蔵	19940700
青田風通し一睡の浄土かな	19940700
喉走る名水冷えの心太	19940700
空暗し呼べば遠退く夕立雲	19940700
今日も亦他所夕立とそれにけり	19940700
花合歓や渓の音きく温泉の窓	19940700
含羞草いで湯泊りの老四人	19940700
故里は金比羅歌舞伎花の山	19940400
岐れ道ミモザ盛りの島巡り	19940400
一言の棘のいたみや夏薊	19940700
一言の棘に猛暑の雲みあぐ	19940700
風鈴や窓辺に母と娘の笑顔	19940700
昼寝覚めまだ侍り猫伸びきって	19940700
シルバーホーム笑ち会釈して廊凉し	19940800
お元気ねきれいに食べし夏料理	19940800
西瓜割漢につづく娘が果す	19940800
踊の輪みるみる三重に炭坑節	19940800
高階に眼覚めてわっと雲の峰	19940800
熱帯夜慣れて別れのなにとなう	19940800
朝凉や肩まで掛けてふと淋し	19940800
雲の峰息子は太平洋の空ならん	19940800
満月や仰ぎし友はいま筑紫	19940900
月白やせり上り待つ大舞台	19940900
手折り来て芒挿しくれホーム友	19940900
敬老日過ぎて忘れを詫ぶ息子かな	19940900
夕木槿一日思案し言ふまじと	19940900
傷つけしことに気附かず青芒	19940900
押し分けも背伸びもなくて草の花	19940900
侘びて住むごと庭隅の時鳥草	19941000
住むは誰隣の芒刈られけり	19941000
息子に目立ちきし白きもの柿をむく	19941000
高階に泊つ霧ぬれの大夜景	19941000
秋灯に左傾ぎの寿百の字	19941000
ふる里や菜飯に小芋の煮ころがし	19941100
大根抜く厨に待つはおろしがね	19941100
木あがりの茄子見落さず芥子漬	19941100
木あがりの茄子と思へぬ芥子漬	19941100
そつと出る夫追ふ妻や露の畑	19941100
医と寺の娘が幼な友木の葉髪	19941100
秋風や札所の寺の大礎石	19941100
木犀匂ふ金銀並びし故里の庭	19941100
着ぶくれて椅子のくぼみに孫自慢	19941200
ほほえみで答ふ遠耳冬すみれ	19941200
言ふだけを言ふてコートの忘れ物	19941200
爪切りて指美くしく賀状書く	19941200
保養所の握手の別れ紅葉散る	19941200
晩菊にそとさよならをしばし旅	19941200
物忘れめつきり増えて年の暮	19941200
晩菊の一本供花とし剪りにけり	19941200
補聴器を切りて一人の冬の夜	19941200
ほんのりと米寿の頬に屠蘇の紅	19950100
倖せは初夢もなき深眠り	19950100
住連飾りドアーにかけて十二階	19950100
開かんと冬薔薇秘めし力かな	19950100
梅一輪いちりん日々を留守居して	19950200
倖せや日々の留守居に梅一輪	19950200
紅梅や白磁揃ひの朝餉の膳	19950200
話す日々米寿祝の冬ばらに	199502000
毛糸解く編み直されぬ過去てふもの	19950200
春寒し幼なに戻るおないどし	19950200
空地占め空の青吸ひ犬ふぐり	19950300
椀に浮くさみどりを吸い春一番	19950300
朝桜夢のあと追ふ思慕の人	19950300
聞くだけで事情を愚痴の春炬燵	19950300
躓きて掌をつくところ土筆んぼ	19950300
躓きて土筆三本折りて詫ぶ	19950300
雪柳白壁拒み闇寄せず	19950400
白壁の汚れはじらふ雪柳	19950400
ワインの栓ぼんに拍手や夜はおぼろ	19950400
花は葉に母の素直は息子の憂ひ	19950400
応えなく平寝落ちしよ花疲れ	19950400
落ち椿さつさと主掃きにけり	19950400
兄弟が初鯉のぼり揚げにけり	19950500
母の日に娘二人の遠電話	19950500
母の日や六十年を母の道	19950500
岐れ道えらべば険し果の余花	19950500
試歩のばす思ひたがわず藤の花	19950500
絵タイルの道若やぎて地球の日	19950500
高きほど大揺れてをり夾竹桃	19950600
雑草の茂りたくまし子もたくまし	19950600
草いくさ陣地広げし青芒	19950600
葉を研ぎて陣地広げむ青芒	19950600
職退くも余生と言へぬ梅青し	19950600
娘名で忌の案内状梅雨じめり	19950600
海の風山の風入れ夏座敷	19950700
夕木槿汚れなき白閉じにけり	19950700
春秋を裾にひろげて讃岐冨士	19950700
はいはいと重ねてさびし含羞草	19950700
眠り草ねむらぬ葉あり反抗期	19950700
装ひし遠き日のあり薄衣	19950700
咲き満つもなほあわあわと花みずき	19950700
花水木乙女の恋の物語	19950700
故郷発つ朝採りトマト重すぎて	19950700
傷つけしこと気付かずや青芒	19950800
やさしくも棘ある言葉夏薊	19950800
夏痩せを知らずに生きて米寿かな	19950800
掌中の珠とはこれよ白桃むく	19950800
無花果を鳥につつかれ犬叱る	19950800
新凉や又取り出して読む佳信	19950800
爽やかや返書のペンのよくすべり	19950800
鳥わたる返書に三色ボールペン	19950800
露けしや二人の友の新佛	19950800
コスモスに手をふる急行待避駅	19951000
秋夕焼こつくりさんの道標	19951000
出ぬ電話そうか今宵は月の句座	19951000
家の味継ぎて伝えて祭ずし	19951000
貰ふなら遠慮はすまじ秋茄子	19951000
栗むくや消えぬ弟の国訛	19951100
故郷もつ倖せしかと柿をむく	19951100
文化の日遠き明治の今日生れ	19951100
透きとおる秋や少年ハーモニカ吹く	19951100
鰯雲告げたき人は遠く住み	19951100
いま倖障子をよぎる鳥の影	19951200
山茶花や豆腐屋を待つ留守居役	19951200
冬桜口紅うすくひく米寿	19951200
騙されてをれば事なし枯尾花	19951200
梅ケ枝の終の一葉の散る別れ	19951200
いつまでも御元気でねてふ賀状の数 	199601
退職と一筆添へし賀状かな     	199601
初入日三六六の一を呑み	 	199601
ページくる吾が音寒し影寒し	 	199601
小豆粥老ひてすこやか姉弟	   	199601
春寒し言はでききをり二度話    	199602
鳥は雲に二度行くスーパー買いわすれ 	199602
梅二月八十路わきまふ笑顔よき   	199602
娘等去にてかろき疲れに窓の梅   	199602
よきことを知らす娘の声梅紅し   	199602
芽吹く庭健かと木々に呼びかけて  	199603
鳥雲に謝しつつ辛き車椅子     	199603
鶯やに車椅子停めくれ息子よ    	199603
とてせめて電話は春の声    	199603
春彼岸弟訪ひくれ仏顔に      	199603
岬うらら成果一尾の小半日     	199604
春の夕餉釣りし一尾を母の前    	199604
快気とはかくもうれしき春の朝   	199604
春光やを拝み浴びをり癒え兆    	199604
径端の小さき笑顔犬ふぐり     	199604
鯉のぼりたーかく揚げて待つ帰国  	199605
日本を知らぬ児を待つ武者飾り   	199605
薔薇咲かせ迎え明るき指圧院    	199605
土産地蕗香りひろげて国言葉    	199605
木の芽雨偲び草とて届く茶器    	199605
片隅に生きる幸せ額の花      	199606
新茶くみほめ言葉待つ母の顔    	199606
草茂る逆らはぬこと牙につきて   	199606
明易やドイツ転勤ききしより    	199606
泰山木朽ちてすがれる花かなし   	199606
朝涼やからっぽ頭にからっ腹    	199607
いざ昼寝今日はいづこへ夢の旅   	199607
夕涼し肌になじみし藍の服     	199607
暑からむ遅れて浴びる百視線    	199607
端居して出世無縁の長寿眉     	199607
暑に耐えし頬なでてみる今朝の風  	199608
秋暑し訪問販売二度のブザー    	199608
夜々うれし子の友に賜ぶ古梅酒   	199608
   花火見に橋へ子が押す車椅子    	199608
癒へてつくる迎え送りの盆団子   	199608
白萩や見知らぬ同志笑みかわし   	199609
  寺育ち白曼珠沙華燃え知らず    	199609
風やさしコスモスやさし車椅子   	199609
思はざる花つけにけり秋の草    	199609
秋冷ゆる友の情の京しるこ     	199609
故里や出会ふたれかれ野菊晴    	199610
栗むきつ老ひて姉弟郷言葉     	199610
風のまま吾も白髪穂亡や
花は実に色増す石榴日々親し    	199610
急げともあわてるなとも虫の鳴く  	199610
天高し卒寿見上ぐる明治晴     	199611
秋深き豆煮る母のひとり言         	199611
冬に入る病上手に附き合わす   	
いつまでも娘は子こたつの母苦言  	199611
よろこびにふとある怖さ夕紅葉   	199611
熟柿つるっと食べばふるさと近く来る	199612
枝 桜紅葉に告ぐ別れ	
落葉掃きつい長くなる隣同志   	199612
やがてこの娘が孫の嫁冬いちご  	199612
雲を割る冬日や老のねがふこと  	199612
お元旦老母くり返すありがたや  	1997/01
しわのなき黒豆に老母初お箸	1997/01
初写真嫁孫の笑み三代	1997/01
愛犬と話す日日あり寒日和	1997/01
翔ばたいて大きなおまへ初からす 	1997/01  ??
五十年忌白梅古りし月日かな   	1997/02 
孫嫁のもうすぐ二人梅紅し	1997/02
お化粧で他人顔なり春写真    	1997/02 
春障子四畳半の城明るし	1997/02
下萌に煎餅分ける愛犬に	1997/02
春耕をまぶしく見をりホーム窓	1997/03
啓窒やシルバーホームの預け解け  	1997/03
春暁の正夢なれや初ひ孫      	1997/03
向ひ合うパソコン句帖春炬燵    	1997/03 
「おばさん」と呼びくれ三人桜餅  	1997/03
浮雲に名付けあそびや春の風    	1997/04
こちら向くラッパ水仙こんにちは	1997/04
花衣車椅子にも湧くはずみ	1997/04
思い桜樹齢二百を恋う卒寿	1997/04
花の雨ワインケーキの香に和む	1997/04
初咲きの大勺や句や婚の朝    	1997/05
桜湯のぱーつとひらけり控室	1997/05
純白の花嫁孫となる五月	1997/05
柿若葉秘仏開扉めぐり会い	1997/05
来し道の険しさ言はず余花仰ぐ	1997/05
御幣上る薫風にのる上棟歌	1997/06
目つむりて青汁ぐっとばら真紅	1997/06
痛いとは生ける証しか梅雨の膝	1997/06
梅雨鏡拭けば亡母にとれほどに	1997/06
都忘れ咲かせ老いけり京遠く	1997/06
今年また梅酒たまわる命かな	1997/07
子つばめの翔つを見送る車椅子	1997/07
ナイターに興じる老母の片辺して	1997/07
白髪といていのちあるもの髪洗ふ	1997/07
ぎょうさんな娘の悲鳴蜘蛛の糸	1997/07
郷ばなしつきずやさしき団扇かぜ	1997/08
夏服の派手を鏡に息子の土産	1997/08
きれし夢惜しや貴船のはも料理  	1997/08 ??
迎はるる仏とならで魂迎ふ	1997/08
仏めく盆僧の額黒光り	1997/08
赤とんぼヘルパーと唄う車椅子	1997/09
星月夜シルバーホーム消灯はやき	1997/09
誰似かと爽やかろんぎ初曽孫	1997/09
白桔梗時には欲しい母小言	1997/09
おきし手を又も引きよす枝豆を	1997/09
これでおわり、どうもありがとう、またきてね。
