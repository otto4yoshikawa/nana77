
\documentclass[b5paper]{tbook}[tombow]
\usepackage{ascmac}
\usepackage{shiika}
\usepackage{kyakuchu}
\usepackage{multicol}
\usepackage{furikana}
\usepackage[dvipdfmx]{graphicx}
\begin{document}
\chapter*{はじめに}
%\input{hajimeni2}
\chapter{笹倉の庭}
成城の家 笹倉の庭に鷺草が
\begin{shiika}鷺草の鷺二羽となる\Kana{娘}{こ}に甘え
\hfill{双適\rensuji*{57}・\rensuji*{7}・\rensuji*{0}}\end{shiika}
\vspace{0.4cm}
相川の最後の夏
\begin{shiika}魂迎ふ一人となりて古家守る
\hfill{\rensuji*{57}・\rensuji*{8}・\rensuji*{0}}\end{shiika}
\vspace{0.4cm}
%==============================================
\begin{shiika}手ごなしで土をかぶせる秋の種
\hfill{\rensuji*{57}・\rensuji*{8}・\rensuji*{0}}\end{shiika}
\begin{shiika}十指もて土をかぶせる秋の種
\hfill{\rensuji*{57}・\rensuji*{8}・\rensuji*{0}}\end{shiika}
\begin{shiika}豪雷にいさかふ妹弟抱き合ふ
\hfill{\rensuji*{57}・\rensuji*{8}・\rensuji*{0}}\end{shiika}
%----------------------------furikana---------------------
\vspace{0.4cm}
飯田知子短大入学祝い
\begin{shiika}合格の祝袋は字も太く
\hfill{\rensuji*{56}・\rensuji*{3}・\rensuji*{0}}\end{shiika}
\vspace{0.4cm}
不二子のノート
\begin{shiika}亡娘ノート\Kana{紙,魚}{し,み}生きている悲しさよ
\hfill{\rensuji*{57}・\rensuji*{9}・\rensuji*{0}}\end{shiika}
上野城 百合子出品を見に行く
\begin{shiika}風凉し天主の床の黒光り
\hfill{\rensuji*{59}・\rensuji*{8}・\rensuji*{0}}\end{shiika}
\vspace{ 0.4cm}
\begin{shiika}俳聖殿忍者屋敷も蝉しぐれ
\hfill{\rensuji*{59}・\rensuji*{8}・\rensuji*{0}}\end{shiika}
\vspace{ 0.4cm}
\vspace{ 0.4cm}
百合子の看病の日を思ひ
\begin{shiika}看とりつつ句帳かた辺に長き夜
\hfill{\rensuji*{62}・\rensuji*{10}・\rensuji*{0}}\end{shiika}
\begin{shiika}看とり女にある秋晴や特選句
\hfill{\rensuji*{62}・\rensuji*{10}・\rensuji*{0}}\end{shiika}
\qquad\qquad\qquad\fbox{編者注} 
 百合子が夫栄介の看病で\\
\qquad\qquad\qquad「点滴の窓を祭りの鉾過ぎる」\\
\qquad\qquad\qquad が伊賀上野の句会で特賞に選ばれた\\
\begin{shiika}祭太鼓看とりの窓に遠くきく
\hfill{\rensuji*{62}・\rensuji*{10}・\rensuji*{0}}\end{shiika}
\begin{shiika}安眠なき看とりの夜々に虫親し
\hfill{\rensuji*{62}・\rensuji*{10}・\rensuji*{0}}\end{shiika}
\vspace{ 0.4cm}

笹倉光雄さんと食事 新宿「かも川」で
\begin{shiika}酌みもして婿の気配り凉しき餉
\hfill{\rensuji*{4}・\rensuji*{7}・\rensuji*{0}}\end{shiika}
\vspace{0.4cm}


成城笹倉にて
\begin{shiika}中古車群旗はたはたと春を呼ぶ
\hfill{\rensuji*{6}・\rensuji*{2}・\rensuji*{0}}\end{shiika}
\begin{shiika}猫柳活ける娘もまたつやつやし
\hfill{\rensuji*{6}・\rensuji*{2}・\rensuji*{0}}\end{shiika}
\begin{shiika}花葉挿しふと京の友思ひけり
\hfill{\rensuji*{6}・\rensuji*{2}・\rensuji*{0}}\end{shiika}
\vspace{0.4cm}

\chapter{母お気に入りの句}
\input{chapter5}

%---------------------------------------------------------------
%\chapter*{あとがき}
%��͋�W�̏o�ł�]��ł��Ȃ������̂ŁA���R���K���ɕ��u�����܂܂��������A
\verb|http://www.geocities.jp/takefumi1604/index.html|
���R���K���ւ� ���܂ł��u���R���K�� �����v�œ����
���q�b�g�����̂ɂ�
���̐g�Ӑ����Ɉ�‚Ƃ��ā@���̃m�[�g�̓Y�������������@TEX�t�@�C���ɂ��Ă݂��B
�����@����L���M���q�b�g�����̂ɂ͋������B
�����Ắu�e�v�Ō�������Ɓu�匎�铂���񎛂̒�ɜe�v
�����O�\�N�l������n�߂ā@�R�P���@��������

���̖{���������‚���͂Ȃ����Apdf�@�Ŕz�z�ł���悤�ɂ����̂�
���̖�ڂ�����

�P�O�O�O��̂Ȃ��Ł@�ꂨ���ɂ���̋���@��3��
�ɂ܂Ƃ߂Ă݂��B���̂Ȃ���
\begin{shiika}
�[�����ďo�������̒�����
\end{shiika}
���\��Ƃ������B

�����O�\�N����
\hfill{�g��|�l�Y}



\end{document}
%\chapter{小倉百人一首作者}
%
1  天智天皇     \begin{shiika}秋の田のかりほの盧のとまをあらみ\\ 我ころも手は露にぬれつゝ 
\end{shiika}2  持統天皇     \begin{shiika}春過て夏来にけらし白妙の\\ 衣ほすてふあまの香来山 
\end{shiika}3  柿本人丸     \begin{shiika}あし引の山鳥のおのしたり尾の\\ なかゝゝし夜を独かもねん 
\end{shiika}4  山辺赤人     \begin{shiika}田子のうらにうち出てみれはしろ妙の\\ 不二の高根にゆきは降つゝ 
\end{shiika}5  猿丸大夫     \begin{shiika}おく山に紅葉ふみわけ鳴くしかの\\ 聲きくときそ秋はかなしき 
\end{shiika}6  中納言家持   \begin{shiika}鵲の渡せるはしにをく霜の\\しろきをみれはよそ更にける 
\end{shiika}7  安部仲麿     \begin{shiika}天の原ふりさけみれは春日なる\\三笠のやまに出し月かも 
\end{shiika}8  喜撰法師     \begin{shiika}我盧はみやこのたつみしかそ住\\よを宇治山と人はいふなり 
\end{shiika}9  小野小町     \begin{shiika}花の色はうつりにけりないたつらに\\わか身よにふるなかめせしまに
\end{shiika}10 蝉丸         \begin{shiika}是や此行もかへるも別ては\\しるもしらぬも相坂のせき 
\end{shiika}11 参議篁       \begin{shiika}和田の原八十嶋かけてこき出ぬと \\人にはつけよあまの釣舟 
\end{shiika}12 僧正遍昭     \begin{shiika}天つ風雲のかよひち吹とちよ\\をとめのすかたしはしとゝめん 
\end{shiika}13 陽成院       \begin{shiika}つくはねのみねよりおつるみなの川\\恋そつもりてふちとなりぬる 
\end{shiika}14 河原左大臣   \begin{shiika}みちのくの忍ふ文字すり誰ゆへに\\乱れ初にしわれならなくに 
\end{shiika}15 光孝天皇     \begin{shiika}君かためはるの野に出てわかなつむ\\わか衣手に雪はふりつゝ 
\end{shiika}16 中納言行平   \begin{shiika}立わかれいなはの山の嶺に生る\\まつとしきかはいまかへりこん 
\end{shiika}17 在原業平朝臣 \begin{shiika}千早振神代もきかす立田川\\からくれなゐに水くゝるとは 
\end{shiika}18 藤原敏行朝臣 \begin{shiika}住の江のきしによる波よるさへや\\夢のかよひち人めよくらん 
\end{shiika}19 伊勢         \begin{shiika}なには潟みちかきあしのふしのまも\\あはてこのよを過してよとや 
\end{shiika}20 元良親王     \begin{shiika}侘ぬれは今はたおなし難波なる\\身をつくしてもあはんとそ思ふ 
\end{shiika}21 素性法師     \begin{shiika}今こんといひしはかりに長月の\\有明の月をまちいてつるかな 
\end{shiika}22 文屋康秀     \begin{shiika}吹からに秋の草木のしほるれは\\むへ山風をあらしといふらん 
\end{shiika}23 大江千里     \begin{shiika}月みれは千々にものこそかなしけれ\\我身ひとつの秋にはあらねと 
\end{shiika}24 菅家         \begin{shiika}この度はぬさも取あへす手向山\\もみちのにしき神のまにゝゝ 
\end{shiika}25 三条右大臣   \begin{shiika}なにしおははあふ坂山のさねかつら\\人にしられて来るよしも哉 
\end{shiika}26 貞信公       \begin{shiika}をくら山嶺のもみち葉心あらは\\今一度のみゆきまたなん 
\end{shiika}27 中納言兼輔   \begin{shiika}みかの原わきてなかるゝ和泉川\\いつみきとてか恋しかるらん 
\end{shiika}28 源宗于朝臣   \begin{shiika}山里は冬そさひしさ増りける\\人めも草もかれぬとおもへは 
\end{shiika}29 凡河内躬恒   \begin{shiika}心あてに折はやおらむ初しもの\\をきまとはせるしら菊の花 
\end{shiika}30 壬生忠峯     \begin{shiika}有明のつれなく見えし別れより\\暁計うきものはなし 
\end{shiika}31 坂上是則     \begin{shiika}朝ほらけ在明の月とみるまてに\\よし野ゝさとにふれるしら雪 
\end{shiika}32 春道列樹     \begin{shiika}山川に風の懸たるしからみは\\なかれもあへぬ紅葉なりけり 
\end{shiika}33 紀友則       \begin{shiika}久方の光のとけき春の日に\\しつ心なくはなの散らん 
\end{shiika}34 藤原興風     \begin{shiika}誰をかも知人にせん高砂の\\松も昔の友ならなくに 
\end{shiika}35 紀貫之       \begin{shiika}人はいさ心もしらす古郷は\\花そむかしの香ににほひける 
\end{shiika}36 清原深養父   \begin{shiika}夏のよはまたよひなから明ぬるを\\雲のいつこに月やとるらん 
\end{shiika}37 文屋朝康     \begin{shiika}しら露に風のふきしく秋のゝは\\つらぬきとめぬたまそ散ける 
\end{shiika}38 右近         \begin{shiika}わすらるゝ身をは思はす誓ひてし\\人のいのちのおしくも有かな 
\end{shiika}39 参議等       \begin{shiika}浅ちふのをのゝしの原忍ふれと\\あまりてなとか人のこひしき 
\end{shiika}40 平兼盛       \begin{shiika}忍ふれと色に出にけり我こひは\\ものやおもふとひとのとふまて 
\end{shiika}41 壬生忠見     \begin{shiika}恋すてふ我名はまたき立にけり\\人しれすこそおもひそめしか 
\end{shiika}42 清原元輔     \begin{shiika}契きなかたみにそてをしほりつゝ\\すゑのまつ山波こさしとは 
\end{shiika}43 権中納言敦忠 \begin{shiika}あひみての後の心にくらふれは\\むかしはものをおもはさりけり 
\end{shiika}44 中納言朝忠   \begin{shiika}逢事のたえてしなくは中ゝゝに\\人をも身をもうらみさらまし 
\end{shiika}45 謙徳公       \begin{shiika}哀ともいふへき人はおもほえて\\身の徒になりぬへき哉 
\end{shiika}46 曽禰好忠     \begin{shiika}ゆらのとを渡る舟人かちを絶\\行ゑもしらぬこひのみち哉 
\end{shiika}47 恵慶法師     \begin{shiika}八重葎しけれる宿のさひしきに\\人社見えね秋は来にけり 
\end{shiika}48 源重之       \begin{shiika}風を痛み岩うつ波のをのれのみ\\碎て物をおもふころかな 
\end{shiika}49 大中臣能宣   \begin{shiika}みかき守ゑしのたく火の夜はもえて\\ひるは消つゝものをこそおもへ
\end{shiika}50 藤原義孝     \begin{shiika}君かためおしからさりしいのちさへ\\永くもかなとおもひけるかな 
\end{shiika}51 藤原実方朝臣 \begin{shiika}かくとたにえやはいふきのさしも草\\ さしもしらしな燃るおもひを 
\end{shiika}52 藤原道信     \begin{shiika}明ぬれはくるゝものとは知なから\\猶うらめしき朝朗かな 
\end{shiika}53 右大将道綱母 \begin{shiika}なけきつゝ独ぬるよの明るまは\\いかに久しきものとかはしる 
\end{shiika}54 儀同三司母   \begin{shiika}わすれしの行すゑまては難けれは\\けふをかきりのいのちとも哉 
\end{shiika}55 大納言公任   \begin{shiika}瀧の音はたえて久しく成ぬれと\\名こそなかれて尚聞えけれ 
\end{shiika}56 和泉式部     \begin{shiika}あらさらん此よの外のおもひ出に\\いま一度のあふ事も哉 
\end{shiika}57 紫式部       \begin{shiika}めくりあひてみしやそれとも分ぬまに\\雲かくれにしよはの月哉 
\end{shiika}58 大弐三位     \begin{shiika}有馬山猪名のさゝ原風ふけは\\いてそよ人をわすれやはする 
\end{shiika}59 赤添衛門     \begin{shiika}やすらはてねなましものをさよ更て\\片ふくまての月を見しかな 
\end{shiika}60 小式部内侍   \begin{shiika}大江山生野ゝみちの遠けれは\\またふみも見すあまのはしたて 
\end{shiika}61 伊瀬大輔     \begin{shiika}いにしへの奈良のみやこの八重桜\\けふこゝのへに匂ひぬるかな 
\end{shiika}62 清少納言     \begin{shiika}よをこめて鳥のそらねははかるとも\\世にあふさかの関はゆるさし 
\end{shiika}63 左京大夫道雅 \begin{shiika}今はたゝおもひたえなんとはかりを\\人つてならていふよしも哉 
\end{shiika}64 権中納言定頼 \begin{shiika}朝朗うちの川霧たえゝゝに\\顕はれ渡る瀬ゝのあしろ木 
\end{shiika}65 相模         \begin{shiika}うらみ侘ほさぬ袖たにある物を\\恋に朽なむ名こそおしけれ 
\end{shiika}66 大僧正行尊   \begin{shiika}もろ共に哀とおもへ山さくら\\はなより外にしる人もなし 
\end{shiika}67 周防内侍     \begin{shiika}春の夜の夢はかりなる手枕に\\甲斐なくたゝん名こそおしけれ 
\end{shiika}68 三条院       \begin{shiika}心にもあらてうきよになからへは\\こひしかるへき夜半の月哉 
\end{shiika}69 能因法師     \begin{shiika}あらしふく三室の山のもみちはゝ\\たつ田の川のにしき成けり 
\end{shiika}70 良暹法師     \begin{shiika}さひしさに宿をたち出てなかむれは\\いつくもおなし秋の夕暮 
\end{shiika}71 大納言経信   \begin{shiika}夕されは門田のいなは音つれて\\芦のまろやにあき風そふく 
\end{shiika}72 内親王紀伊   \begin{shiika}音にきくたかしのはまの化波は\\かけしやそてのぬれもこそすれ 
\end{shiika}73 権中納言匡房 \begin{shiika}高砂のおのへのさくら咲にけり\\とやまの霞みたゝすもあらなん 
\end{shiika}74 源俊頼朝臣   \begin{shiika}うかりける人を初瀬の山おろし\\はけしかれとはいのらぬものを 
\end{shiika}75 藤原基俊     \begin{shiika}契りをきしさせもかつゆをいのちにて\\哀ことしの秋もいぬめり 
\end{shiika}76 法性寺入道   \begin{shiika}和田の原こき出てみれは久方の\\雲井にまかふおきつしら波 
\end{shiika}77 崇徳院       \begin{shiika}瀬をはやみ岩にせかるゝたき川の\\われてもすゑにあはむとそおもふ
\end{shiika}78 源兼昌      \begin{shiika}あはち嶋かよふ千鳥の鳴こゑに\\幾夜ねさめぬすまのせきもり 
\end{shiika}79 左京大夫顕輔 \begin{shiika}秋風に棚引雲のたえまより\\もれいつる月のかけのさやけさ 
\end{shiika}80 待賢門院堀河 \begin{shiika}長からん心もしらすくろ髮の\\みたれて今朝はものをこそ思へ 
\end{shiika}81 後徳大寺左大 \begin{shiika}ほとゝきす鳴つる方を眺むれは\\唯有明の月そのこれる 
\end{shiika}82 道因法師     \begin{shiika}思ひわひさてもいのちは有ものを\\うきに堪ぬはなみた成けり 
\end{shiika}83 俊成         \begin{shiika}世中よ道こそなけれおもひ入 \\山のおくにも鹿そ鳴なる 
\end{shiika}84 藤原清輔朝臣 \begin{shiika}なからへはまたこの比や忍はれん\\うしと見しよそいまはこひしき 
\end{shiika}85 俊恵法師     \begin{shiika}よもすから物思ふころは明やらて\\閨の隙さへつれなかりけり 
\end{shiika}86 西行法師     \begin{shiika}歎けとて月やはものを思はする\\かこち顔なるわかなみたかな 
\end{shiika}87 寂蓮法師     \begin{shiika}村雨の露もまたひぬ槇のはに\\霧たちのほるあきのゆふ暮 
\end{shiika}88 皇嘉門院別当 \begin{shiika}難波江のあしのかりねの一夜ゆへ\\身をつくしてやこひ渡るへき 
\end{shiika}89 式子内親王   \begin{shiika}玉のをよ絶なはたえねなからへは\\しのふる事のよはりもそする 
\end{shiika}90 殷富門院大輔 \begin{shiika}見せはやなをしまのあまの袖たにも\\ぬれにそぬれし色はかはらす 
\end{shiika}91 後京極大臣   \begin{shiika}きりゝゝす鳴やしもよのさむしろに\\ころもかたしきひとりかもねん
\end{shiika}92 二条院讃岐   \begin{shiika}わか袖はしほひに見えぬおきの石の\\人こそしらねかはくまもなし 
\end{shiika}93 鎌倉右大臣   \begin{shiika}世中は常にもかもな渚こく\\海人のをふねの綱手かなしも 
\end{shiika}94 参議雅経     \begin{shiika}みよし野ゝ山の秋風さよ更て\\故郷さむくころもうつ也 
\end{shiika}95 前大僧正慈円 \begin{shiika}おほけなくうきよの民におほふ哉\\我たつ杣にすみそめの袖 
\end{shiika}96 入道太政大臣 \begin{shiika}花さそふあらしの庭の雪ならて\\ふり行ものはわか身成けり 
\end{shiika}97 権中納言定家 \begin{shiika}来ぬ人をまつほのうらの夕なきに\\やくや藻しほの身もこかれつゝ 
\end{shiika}98 従二位家隆   \begin{shiika}風そよくならの小川の夕暮は\\御秡そなつのしるし成ける 
\end{shiika}99 後鳥羽院     \begin{shiika}人もおしひともうらめしあちきなく\\よをおもふゆへに物思ふ身は 
\end{shiika}100 順徳院      \begin{shiika}百敷やふるき軒端の忍ふにも\\なを餘りあるむかし成けり 
\end{shiika}


\end{document}

\includegraphics[height=9cm, angle=90, bb=0 0 640 640]{cat.jpg}
\begin{minipage}<y>[t]{6cm}
           福岡山下写真館
\end{minipage}
\includegraphics[height=9cm, angle=90, bb=0 0 640 640]{fumiko2.jpg}]\begin{minipage}<y>[t]{6cm}
           山下光子さんと
\end{minipage}




\end{document}

\end{document}
---------------------------------------------------------------------
\makeatletter
\newenvironment{shiika}
{\let\\\@normalcr\par
 \list{}{kanjiskip=0.25zw plus 0.01zw minus 0,01zw
  \xkanjiskip=\kanjiskip
 \utemsepz@ \topsep=\z@ \paesep=\z@
 \leftmergin=4zw \itemindeby=-2zw
\listparindentitemindeny \rightmargin=-zw }
 \iterekax}{\par\endlist}
\makeatother



==========================================================================
