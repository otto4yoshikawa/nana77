\noindent 
吉祥会で大森先生 池永先生に一緒に当尾の石仏を巡りて 
\begin{shiika}
野仏の笑ひ在せり 曼珠沙華\hfill { \rensuji*{48} ・ \rensuji* {8} }
\end{shiika}
\vspace{0.6cm}
「草紅葉」兼題 幼き日の思い出 
\begin{shiika}
 日を浴びてままごとの子や草紅葉\hfill { \rensuji*{48} ・ \rensuji* {10} }
\end{shiika}
\vspace{0.6cm}
「顔見世」 去年は文友会で顔もせに。今年はただ思い出のみ
\begin{shiika}
顔見世の名残を夢に見しも去年\hfill { \rensuji*{48} ・ \rensuji* {12} ・}
\end{shiika}
\vspace{0.5cm}
\vspace{0.6cm}
お隣の浅野まゆみさんかわいい日本髪で
\begin{shiika}
髪結ひて寝ず娘は待つ初詣\hfill { \rensuji*{49} ・ \rensuji* {1} ・}
\end{shiika}
\vspace{0.6cm}
相川北通りの家根笹の中で狂い猫
\begin{shiika}
猫の恋根笹の乱れ昨日今日\hfill { \rensuji*{49} ・ \rensuji* {2} ・}
\end{shiika}
\vspace{0.6cm}
上京の車中 
浜松あたりで遠連山をみて
\begin{shiika}
山の色幾重の果の雪解光\hfill { \rensuji*{49} ・ \rensuji* {2} ・ }	
\end{shiika}
\vspace{0.6cm}
\begin{shiika}野仏の笑ひ在せり曼珠沙華
\hfill{\rensuji*{48}・\rensuji*{9}・\rensuji*{0}}\end{shiika}
\vspace{0.6cm}
「水草生まふ」 兼題 日浅い私には大変むつかしい。ふと一善の
車で探梅につれてもらった時\\賀名生 だったかそして仁徳陵ところを走ったことを
思い出す。
\begin{shiika}陵の薄陽の濠も水草生ふ
\hfill{\rensuji*{49}・\rensuji*{3}・\rensuji*{0}}\end{shiika}
\vspace{0.6cm}
「春の雪」兼題 直子さんの縁談がまた立ち消えた。
\begin{shiika}娘の縁談又もこわれぬ春の雪
\hfill{\rensuji*{49}・\rensuji*{3}・\rensuji*{0}}\end{shiika}
\vspace{0.6cm}
一つの旅を終えるとまた次に心は走る。
\begin{shiika}花過ぎぬいづこともなき旅心
\hfill{\rensuji*{49}・\rensuji*{4}・\rensuji*{0}}\end{shiika}
\vspace{0.6cm}
「桐の花」兼題 小森田さんとあわくら荘に 帰りは姫路までバスにした。
\begin{shiika}山裾の雨に煙れる桐の花
\hfill{\rensuji*{49}・\rensuji*{5}・\rensuji*{0}}\end{shiika}
\newpage
\fbox{編者のコメント}

母ふみ子は昭和二十五年から、阪急京都線 相川駅前で文房具
の店を始めた。その後雑誌 書籍も扱うようになった。二十年頑張ったころは
店員に任せて旅行できる余裕ができた。

旅行は、高松女学校のクラスメート、京都女専のクラスメート、文具商の
組合からの誘いだった。

寝起きは 相川北通りの家で 家の半分は貸していた。

昭和十九年に長柄から強制疎開で 相川に来た当時は、母。姉三人、私 そして 
居候が三人、女中さんの大所帯だったが、姉達はかたづき、私は東京に就職
で、母は一人暮らしになった。

私の東京での就職に関しては、母は行動範囲が増えるといって、賛成してくれた。
昭和五十年頃は 私はソフトウエア会社に勤めて、妻と子供二人で、世田谷の
マンション暮らしだった。

\newpage
\vspace{0.6cm}
「草の花」兼題どこで得た句かはっきりしない。
\begin{shiika}野仏の顔かくすまで草の花
\hfill{\rensuji*{49}・\rensuji*{9}・\rensuji*{0}}\end{shiika}
\vspace{0.6cm}
山下さん 小森田さん 青山さん 四人連れ 児玉東洋さんの
車で佐多岬 桜島 霧島と廻っていただく。\\別れて高千穂の
国民宿舎に泊った夜 高千穂神社の夜神楽をみに行く。
\begin{shiika}夜神東の明りに映ゆる銀杏黄葉
\hfill{\rensuji*{49}・\rensuji*{11}・\rensuji*{0}}\end{shiika}
\vspace{0.6cm}
「炬燵」兼題 一人暮らしの私の句だと浅野さんの御主人がはやす
\begin{shiika}置炬燵向ふ人なきあで蒲団
\hfill{\rensuji*{49}・\rensuji*{11}・\rensuji*{0}}\end{shiika}
\vspace{0.6cm}
「年用意」丹波から週二回野菜その他を積んで車が来る大塚「きく」
の前でとまる。\\ 大塚ののぶ子さんが電話で「丹波よ」と相川の店へしらせてくれる。
\begin{shiika}年用意丹波男の荷は売れ早き
\hfill{\rensuji*{49}・\rensuji*{12}・\rensuji*{0}}\end{shiika}
\vspace{0.6cm}
\vspace{0.6cm}
小森田さんが名古屋から夕方までに相川へ着く筈になっているのに
遅い
\begin{shiika}友待つに暮色刻々粉雪舞ふ
\hfill{\rensuji*{50}・\rensuji*{1}・\rensuji*{0}}\end{shiika}
\vspace{0.6cm}
上京車窓より。
\begin{shiika}風ぬくき末黒野烏群をなし
\hfill{\rensuji*{50}・\rensuji*{2}・\rensuji*{0}}\end{shiika}
\vspace{0.6cm}
私は化粧水は使っていないが ふと出来た句
\begin{shiika}化粧水掌に冷えのなし春隣
\hfill{\rensuji*{50}・\rensuji*{3}・\rensuji*{0}}\end{shiika}
\vspace{0.6cm}
「花曇」野崎詣りをしらのは去年だったかと思う。
\begin{shiika}綿菓子も売れて野崎の花曇
\hfill{\rensuji*{50}・\rensuji*{4}・\rensuji*{0}}\end{shiika}
\begin{shiika}花曇年甲斐もなき物忘れ
\hfill{\rensuji*{50}・\rensuji*{4}・\rensuji*{0}}\end{shiika}
\vspace{0.6cm}
この様な軽やかな心に時もある
\begin{shiika}若やぎて夏来る歌口ずさむ
\hfill{\rensuji*{50}・\rensuji*{0}・\rensuji*{5}}\end{shiika}
\vspace{0.6cm}
相川の家の軒に雀がいそかしげに出入りする
\begin{shiika}梅雨曇出入せはしき軒雀
\hfill{\rensuji*{50}・\rensuji*{6}・\rensuji*{0}}\end{shiika}
\vspace{0.6cm}
相川の町の露地風景
\begin{shiika}花曇年甲斐もなき物忘れ
\hfill{\rensuji*{50}・\rensuji*{6}・\rensuji*{0}}\end{shiika}
\vspace{0.6cm}
どこの寺院だったかなー
\begin{shiika}あらはなるちくり根洗ひ大夕立
\hfill{\rensuji*{50}・\rensuji*{7}・\rensuji*{0}}\end{shiika}
\vspace{0.6cm}
「流れ星」この頃誰かが病気をして心にかかっていた
\begin{shiika}看る夜の心もとなき星の飛ぶ
\hfill{\rensuji*{50}・\rensuji*{8}・\rensuji*{26}}\end{shiika}
\vspace{0.6cm}
「空蝉」故かんげつ国分寺境内の礎石で遊んだ日をおもいだして
\begin{shiika}子等去りぬ礎石にならぶ蝉の殻
\hfill{\rensuji*{50}・\rensuji*{8}・\rensuji*{0}}\end{shiika}
\vspace{0.6cm}
唐招提寺 観月の夜
\begin{shiika}大月夜唐招提寺の庭に彳つ
\hfill{\rensuji*{50}・\rensuji*{9}・\rensuji*{0}}\end{shiika}
\vspace{0.6cm}
「色鳥」山下さん青山さんと越前賤ケ岳 長浜竹生島の旅
\begin{shiika}色鳥や朝の湖の小桟橋
\hfill{\rensuji*{50}・\rensuji*{10}・\rensuji*{0}}\end{shiika}
\vspace{0.6cm}
「秋惜しむ」小森田さんと笑い乍らの出来たもの
\begin{shiika}秋惜しむほほ紅少こしさしてみむ
\hfill{\rensuji*{50}・\rensuji*{10}・\rensuji*{0}}\end{shiika}
\vspace{0.6cm}
大塚さん「きく」の前に荷をおろす「丹波」のこと
\begin{shiika}新鮮と我から言ひて冬菜売
\hfill{\rensuji*{50}・\rensuji*{12}・\rensuji*{0}}\end{shiika}
\vspace{0.6cm}
相川の座敷の庭に笹子
の声がと井上さんからきく
\begin{shiika}独り居の朝茶の香り笹に来る
\hfill{\rensuji*{51}・\rensuji*{1}・\rensuji*{0}}\end{shiika}
\vspace{0.6cm}
「大福茶」我が家は梅昆布茶が毎年のこと大福茶と
思っている。
\begin{shiika}家長の座に心しまりて大福茶
\hfill{\rensuji*{51}・\rensuji*{1}・\rensuji*{0}}\end{shiika}
\vspace{0.6cm}
「野焼き」 あちこちに見る野火に次の命の芽生えを思った。
\begin{shiika}新らしき命を呼びて野火勢ふ
\hfill{\rensuji*{51}・\rensuji*{2}・\rensuji*{0}}\end{shiika}
\vspace{0.6cm}
「春泥」 浄瑠璃寺への柊が浮かんできた。 そして遠足の
列が眼に入る。
\begin{shiika}春泥の径つき寺の小門あり
\hfill{\rensuji*{51}・\rensuji*{3}・\rensuji*{0}}\end{shiika}
\begin{shiika}黄帽子水筒どの児の靴も春の泥
\hfill{\rensuji*{51}・\rensuji*{3}・\rensuji*{0}}\end{shiika}
\vspace{0.6cm}
高山祭をめざして小森田さん 美佐さん 宮川ひでさんと下呂
へ行く。折り悪し雨で宵の「曳別れ」は
みることができなかったが車窓より禅昌寺の塔を眺めて
\begin{shiika}花の奥雨に煙れる塔のあり
\hfill{\rensuji*{51}・\rensuji*{4}・\rensuji*{0}}\end{shiika}
\vspace{0.6cm}
小森田」さん 高田さんと妙高々原 穂高 と旅して 穂高の
有明松尾寺にて、妙高々原にて
\begin{shiika}老鶯や御手の茶壺のかたむける
\hfill{\rensuji*{51}・\rensuji*{5}・\rensuji*{0}}\end{shiika}
\begin{shiika}老鴬に唐松林行きにゆく
\hfill{\rensuji*{51}・\rensuji*{5}・\rensuji*{0}}\end{shiika}
「落し文」 むつかしい兼題にふと昨年の賤ケ岳を思い出して
\begin{shiika}湖見ゆる古戦場道落し文
\hfill{\rensuji*{51}・\rensuji*{7}・\rensuji*{0}}\end{shiika}
\vspace{0.6cm}
亡妹貞子が死の近くなった頃
梨をしきりにほしがった。梨の頃がくると思い出す。
\begin{shiika}病妹の欲りし日とあり梨供ふ
\hfill{\rensuji*{51}・\rensuji*{9}・\rensuji*{0}}\end{shiika}
\vspace{0.6cm}
京都女専クラス会 九州志賀島 大宰府 柳川巡りにて
\begin{shiika}鐘楼に屋根草のびて露ふかし
\hfill{\rensuji*{51}・\rensuji*{10}・\rensuji*{17}}\end{shiika}
\begin{shiika}四つ手網死魚の乾けり秋の声
\hfill{\rensuji*{51}・\rensuji*{10}・\rensuji*{17}}\end{shiika}
\vspace{0.6cm}
「晩菊」相川の庭の菊 謡の小川先生のこと。
\begin{shiika}晩菊のうつろいはじむ白きより
\hfill{\rensuji*{51}・\rensuji*{11}・\rensuji*{0}}\end{shiika}
\begin{shiika}晩菊やなほ美くしき謡の師
\hfill{\rensuji*{51}・\rensuji*{11}・\rensuji*{0}}\end{shiika}
\vspace{0.6cm}
耳の治療で大手町病院に通っていた頃
\begin{shiika}晩菊やなほ美くしき謡の師
\hfill{\rensuji*{51}・\rensuji*{11}・\rensuji*{0}}\end{shiika}
\vspace{0.6cm}
天満マーチャンダイズあたりにて
\begin{shiika}秋冷ゆる赤きストビラ散る舗道
\hfill{\rensuji*{51}・\rensuji*{11}・\rensuji*{0}}\end{shiika}
\vspace{0.6cm}
相川の庭の垣をみて。
\begin{shiika}綿虫の籬越え来て雨を呼ぶ
\hfill{\rensuji*{51}・\rensuji*{11}・\rensuji*{0}}\end{shiika}
\vspace{0.6cm}
西川さん 増田さん と淡路島健和荘泊り 灘水仙郷 若人も森など巡る。\\
帰途乗船場にて浅利貝を買う。
\begin{shiika}蛤の潮のしたたり出船待つ
\hfill{\rensuji*{52}・\rensuji*{3}・\rensuji*{0}}\end{shiika}
\vspace{0.6cm}
東横線多摩川鉄橋通過
\begin{shiika}河原なる飛球の行方風光る
\hfill{\rensuji*{52}・\rensuji*{3}・\rensuji*{0}}\end{shiika}
\vspace{0.6cm}
小田さんの案内で山下さんと三人で吉野山へ
\begin{shiika}吉野山春蘭の店は客呼ばず
\hfill{\rensuji*{52}・\rensuji*{4}・\rensuji*{5}}\end{shiika}
\vspace{0.6cm}
相川の畑にて
\begin{shiika}花弁ゆれ奥より出でし虻の貌
\hfill{\rensuji*{52}・\rensuji*{4}・\rensuji*{0}}\end{shiika}
\vspace{0.6cm}
相川の店二階の軒先に燕巣をつくる
\begin{shiika}燕の子黄ならびの嘴花のごと
\hfill{\rensuji*{52}・\rensuji*{5}・\rensuji*{0}}\end{shiika}
\vspace{0.6cm}
あわくら荘に青山さん 西川さん 増田さん と。自然林のほうへ
\begin{shiika}木苺や山の佛の唇あせて
\hfill{\rensuji*{52}・\rensuji*{6}・\rensuji*{25}}\end{shiika}
\vspace{0.6cm}
整くんが寝冷えしていた時
\begin{shiika}寝冷え子のうつろの瞳絵本散る
\hfill{\rensuji*{52}・\rensuji*{7}・\rensuji*{0}}\end{shiika}
\vspace{0.6cm}
「蜜豆」ふとこんなこともあったかな
\begin{shiika}蜜豆に唇さみし嘘を言ふ
\hfill{\rensuji*{52}・\rensuji*{7}・\rensuji*{0}}\end{shiika}
\vspace{0.6cm}
%------------------------------------
一家の旅今津 海津大崎 竹生島 つづら荘泊り\\
八月も終わりに近い つづら荘の前の湖辺にて得た句
\begin{shiika}湖の色北より深み秋きざす
\hfill{双適入選\rensuji*{52}・\rensuji*{8}・\rensuji*{0}}\end{shiika}
\begin{shiika}竹生島真向ふ宿の洗鯉
\hfill{\rensuji*{52}・\rensuji*{8}・\rensuji*{0}}\end{shiika}
\vspace{0.6cm}
高野山登山ケーブルカーの窓より芒を眺めて
\begin{shiika}登るほど尾花は細し高野道
\hfill{\rensuji*{52}・\rensuji*{9}・\rensuji*{0}}\end{shiika}
\vspace{0.6cm}
芒むらの眺めはあちこちに得られた。それに秋吉台の景を重ねて
\begin{shiika}行けど行けど穂芒波や夕茜
\hfill{\rensuji*{52}・\rensuji*{9}・\rensuji*{0}}\end{shiika}
\vspace{0.6cm}
\begin{shiika}天高し隠岐の草原牛肥えて
\hfill{\rensuji*{52}・\rensuji*{9}・\rensuji*{0}}\end{shiika}
\begin{shiika}霊場の鐘にも和さずけらつつき
\hfill{\rensuji*{52}・\rensuji*{10}・\rensuji*{0}}\end{shiika}
%----------------------------
\vspace{0.6cm}
小田から頂戴した紫しきぶが大きくなって美しい実をたくさんに。
\begin{shiika}下枝より褪せて小庭の実むらさき
\hfill{\rensuji*{52}・\rensuji*{10}・\rensuji*{0}}\end{shiika}
\vspace{0.6cm}
相川の家で お謡の小川先生御母堂白寿祝い
\begin{shiika}庭雀床払ひせしふとん干す
\hfill{\rensuji*{52}・\rensuji*{12}・\rensuji*{0}}\end{shiika}
\begin{shiika}白寿祝ぐ願いをこめて羽根蒲団
\hfill{\rensuji*{52}・\rensuji*{12}・\rensuji*{0}}\end{shiika}
相川の家元旦の水。若水を汲むにはあらねど。
\begin{shiika}若水や心新らたに栓開く
\hfill{\rensuji*{53}・\rensuji*{1}・\rensuji*{0}}\end{shiika}
\vspace{0.6cm}
小田澄子さんの御親類 句友 藤田みや様の訃。
\begin{shiika}句友の訃夜を沈丁の香のせまり
\hfill{\rensuji*{53}・\rensuji*{3}・\rensuji*{0}}\end{shiika}
%-----------------------
\vspace{0.6cm}
淡路島への船中よりの景を思い出して
\begin{shiika}春潮に群れ飛ぶかもめ水尾追ひて
\hfill{\rensuji*{53}・\rensuji*{3}・\rensuji*{0}}\end{shiika}
\vspace{0.6cm}
大森先生御他界 城陽大森家を訪ねる
\\中を開かない門のうちには花ゆらす
\vspace{0.6cm}
\begin{shiika}門かたく喪の家ひそと花ゆすら
\hfill{\rensuji*{53}・\rensuji*{4}・\rensuji*{0}}\end{shiika}
\vspace{0.6cm}
\begin{shiika}潮騒の丘の花冷学徒眠る
\hfill{\rensuji*{53}・\rensuji*{5}・\rensuji*{0}}\end{shiika}
\vspace{0.6cm}
小森田 美佐さんと淡路島行く
\begin{shiika}城跡の古井戸涸れず苔の花
\hfill{\rensuji*{53}・\rensuji*{6}・\rensuji*{5}}\end{shiika}
\vspace{0.6cm}
四国八十八ケ所札どころ巡拝
\begin{shiika}桑の実に郷愁ありて札所径
\hfill{\rensuji*{53}・\rensuji*{6}・\rensuji*{0}}\end{shiika}
\vspace{0.6cm}
相川蒔田家の告別式だったか
\begin{shiika}焼香待つ黒幕裾の蟻地獄
\hfill{\rensuji*{53}・\rensuji*{7}・\rensuji*{0}}\end{shiika}
\vspace{0.6cm}
%-------------------------------------------
八十八ケ所霊場巡り(文友会) 最終回さぬき路\\杖は本当に持ち帰り
\begin{shiika}葉鶏頭一筋町の故郷晴れ
\hfill{\rensuji*{53}・\rensuji*{10}・\rensuji*{0}}\end{shiika}
\begin{shiika}結願の杖納め得し鵙日和
\hfill{\rensuji*{53}・\rensuji*{10}・\rensuji*{0}}\end{shiika}
\vspace{0.6cm}
相川風景 よく花屋さん狭い路にも立ち入る
\begin{shiika}花売の残す菊の香路地の朝
\hfill{\rensuji*{53}・\rensuji*{12}・\rensuji*{0}}\end{shiika}
\vspace{0.6cm}
郷生の電話だったかなー
\begin{shiika}口ませし孫の電話や冬すみれ
\hfill{\rensuji*{53}・\rensuji*{12}・\rensuji*{0}}\end{shiika}
\vspace{0.6cm}
クラス会佐渡
\begin{shiika}曼珠沙華島の陵人稀に
\hfill{\rensuji*{53}・\rensuji*{9}・\rensuji*{0}}\end{shiika}
\vspace{0.6cm}
一善広島より出張大阪に来て泊る
\begin{shiika}出張のしげかれ疾かれ牡蠣土産
\hfill{\rensuji*{53}・\rensuji*{10}・\rensuji*{0}}\end{shiika}
\begin{shiika}寄れば逃ぐ子に獅子舞の昂りて
\hfill{\rensuji*{53}・\rensuji*{0}・\rensuji*{0}}\end{shiika}
\begin{shiika}寒餅を切る夜のまど?文とろり
\hfill{\rensuji*{53}・\rensuji*{10}・\rensuji*{0}}\end{shiika}
\begin{shiika}旅立ちの鏡に向ふ夏帽子
\hfill{\rensuji*{53}・\rensuji*{10}・\rensuji*{0}}\end{shiika}
\begin{shiika}久々の子に浴衣着せ今宵酌む
\hfill{\rensuji*{53}・\rensuji*{10}・\rensuji*{0}}\end{shiika}
\begin{shiika}菜の花名を問ひ問はれ三輪の径
\hfill{\rensuji*{53}・\rensuji*{10}・\rensuji*{0}}\end{shiika}
\vspace{0.6cm}
%----------------------------------------------
元旦のお祝い
\begin{shiika}三代が屠蘇なみなみと三つの盃
\hfill{\rensuji*{54}・\rensuji*{1}・\rensuji*{1}}\end{shiika}
\vspace{0.6cm}
年末相川の店より北通りの家へ帰宅の途中走り出た猫に足元狂い捻挫して
佐古整形院で治療
\begin{shiika}冬萠や繃帯の足歩を試す
\hfill{\rensuji*{54}・\rensuji*{1}・\rensuji*{0}}\end{shiika}
\vspace{0.6cm}

楽しんで相川の家えは沈丁花を挿し木いた。\\
すくすく成長したかと思うと突然枯れもした。私はその香りがあまり
好きでなかった、気になる匂ひだから何とか句材にした。
\begin{shiika}昂りぬ沈丁の雨音もなく
\hfill{\rensuji*{54}・\rensuji*{3}・\rensuji*{0}}\end{shiika}
\begin{shiika}啓執や旅誘ひの友便り
家族旅行 土柱 阿波池田
\hfill{\rensuji*{54}・\rensuji*{3}・\rensuji*{0}}\end{shiika}
\begin{shiika}花の下城址碑ひそと休暇村
\hfill{\rensuji*{54}・\rensuji*{4}・\rensuji*{0}}\end{shiika}
\vspace{0.6cm}
さぬき白鳥黒川温泉に糸島さん 増田さんの案内で
\begin{shiika}山の温泉は音なく春蚊早出でし
\hfill{\rensuji*{54}・\rensuji*{4}・\rensuji*{20}}\end{shiika}
\vspace{0.6cm}
文友会西国三十三ケ所巡拝 長谷寺にて
\begin{shiika}草餅に門前町の賑へる
\hfill{\rensuji*{54}・\rensuji*{6}・\rensuji*{0}}\end{shiika}
\vspace{0.6cm}
高田さんに教えられ三年前栗を土に埋めた。何本か芽お出した
中の一本がすくすくと伸びた。
五十七年相川を去る時捨てていくのが惜しかった
\begin{shiika}実生栗初花咲けり吾も健
\hfill{\rensuji*{54}・\rensuji*{6}・\rensuji*{0}}\end{shiika}
\begin{shiika}冷奴遠き旅より帰り酌む
\hfill{\rensuji*{54}・\rensuji*{6}・\rensuji*{0}}\end{shiika}
\vspace{0.6cm}
小森田さんと上田城より別所温泉への旅
\begin{shiika}落ちるまま実梅の匂ひ城のみち
\hfill{\rensuji*{54}・\rensuji*{7}・\rensuji*{16}}\end{shiika}
\vspace{0.6cm}
小森田さんと郡上八幡 井波を訪ねて
\begin{shiika}城の灯のうるみ郡上の踊更く
\hfill{\rensuji*{54}・\rensuji*{8}・\rensuji*{23}}\end{shiika}
\vspace{0.6cm}
\begin{shiika}新秋や欄間彫る町木の香り
\hfill{\rensuji*{54}・\rensuji*{8}・\rensuji*{24}}\end{shiika}
\vspace{0.6cm}
\begin{shiika}谷底は見えずバス行く山の霧
\hfill{\rensuji*{54}・\rensuji*{8}・\rensuji*{24}大島醇子選}\end{shiika}
\vspace{0.6cm}
\begin{shiika}高原の駅コスモスの色極め
\hfill{\rensuji*{54}・\rensuji*{12}・\rensuji*{0}}\end{shiika}
\vspace{0.6cm}
文友会 西国三十三番 巡礼
\begin{shiika}結願の梵鐘ひびく峯の秋
\hfill{\rensuji*{54}・\rensuji*{12}・\rensuji*{0}}\end{shiika}
\vspace{0.6cm}
相川の家にて
\begin{shiika}太りゆく大根今日も抜き惜しみ
\hfill{\rensuji*{54}・\rensuji*{12}・\rensuji*{0}}\end{shiika}
\begin{shiika}実むらさき実生をたのむ土かぶせ
\hfill{\rensuji*{54}・\rensuji*{12}・\rensuji*{0}}\end{shiika}
\begin{shiika}青木の実名知らぬ鳥も枝くぐり
\hfill{\rensuji*{54}・\rensuji*{12}・\rensuji*{0}}\end{shiika}
\vspace{0.6cm}
新年謡の会
\begin{shiika}心地よき帯のしまりや謡ひ初め
\hfill{\rensuji*{55}・\rensuji*{1}・\rensuji*{0}}\end{shiika}
\vspace{0.6cm}
安藤さん青山さんと淡路島 健和荘で新年を過ごす 渡船のおり
\begin{shiika}新年の交す汽笛に群れ鴎
\hfill{\rensuji*{55}・\rensuji*{1}・\rensuji*{1}}\end{shiika}
\vspace{0.6cm}
村上ぬいさんの急逝
\begin{shiika}通夜の冷え遺作のばら絵明るきも
\hfill{\rensuji*{55}・\rensuji*{3}・\rensuji*{0}}\end{shiika}
\begin{shiika}出棺す白梅こぼる砂踏みて
\hfill{\rensuji*{55}・\rensuji*{3}・\rensuji*{0}}\end{shiika}
\vspace{0.6cm}
相川の家
\begin{shiika}雨戸くる朝なあさなを蕗育つ
\hfill{\rensuji*{55}・\rensuji*{4}・\rensuji*{0}}\end{shiika}
\begin{shiika}菜園の菊菜色よし久の子に
\hfill{\rensuji*{55}・\rensuji*{4}・\rensuji*{0}}\end{shiika}
\vspace{0.6cm}
浅野繁雄さんご他界 小森田さん入院
\begin{shiika}青葉して忌ごもる友と病める友
\hfill{\rensuji*{55}・\rensuji*{5}・\rensuji*{0}}\end{shiika}
\vspace{0.6cm}
小豆島国民宿舎(池田)に集まりて
\begin{shiika}明易し潮騒近き島の宿
\hfill{\rensuji*{55}・\rensuji*{6}・\rensuji*{0}}\end{shiika}
\vspace{0.6cm}
\begin{shiika}島の雷止みて翼船ましぐら
\hfill{\rensuji*{55}・\rensuji*{6}・\rensuji*{1}}\end{shiika}
\vspace{0.6cm}
竹四郎病む
\begin{shiika}梅雨嵐し離れ病む子をただ祈る
\hfill{\rensuji*{55}・\rensuji*{6}・\rensuji*{0}}\end{shiika}
\vspace{0.6cm}
海南 林満喜子さん宅を訪ねて
\begin{shiika}見送られ見返る薄暮白あやめ\\
海道先生が第一位にとってくださった
\hfill{\rensuji*{55}・\rensuji*{6}・\rensuji*{0}}\end{shiika}
\vspace{0.6cm}
整の昼寝 私のひるね
\begin{shiika}健やかな孫の寝息やプール焼け
\hfill{\rensuji*{55}・\rensuji*{8}・\rensuji*{0}}\end{shiika}
\begin{shiika}草引きて草の匂ひの手枕寝
\hfill{\rensuji*{55}・\rensuji*{8}・\rensuji*{0}}\end{shiika}
\vspace{0.6cm}
あわくら温泉に幡井さんと行く店の決算をすませて
\begin{shiika}水引の紅ぬれづめに水車
\hfill{\rensuji*{55}・\rensuji*{9}・\rensuji*{0}}\end{shiika}
\begin{shiika}みのり田の道登校のペダル踏む
\hfill{\rensuji*{55}・\rensuji*{9}・\rensuji*{0}}\end{shiika}
\begin{shiika}温泉涼し重き一事を成しとげて
\hfill{\rensuji*{55}・\rensuji*{9}・\rensuji*{0}}\end{shiika}
\vspace{0.6cm}
山下さんと退院した小森田さんを名古屋に訪ねて
\begin{shiika}退院の友いきいきと派手浴衣
\hfill{\rensuji*{55}・\rensuji*{7}・\rensuji*{17}}\end{shiika}
\vspace{0.6cm}
大川一善 安子さんの車で信穂高 木曽濁河温泉
\begin{shiika}ダム澄める揺れ映りいる合歓の花
\hfill{双適\rensuji*{55}・\rensuji*{8}・\rensuji*{2}}\end{shiika}
\begin{shiika}露天湯の一灯淡く月見草
\hfill{双適\rensuji*{55}・\rensuji*{0}・\rensuji*{3}}\end{shiika}
\begin{shiika}霊峰の碧に真向ひ秋ざくら
\hfill{\rensuji*{55}・\rensuji*{8}・\rensuji*{4}}\end{shiika}
\vspace{0.6cm}
私の誕生祝として大台ケ原へ一善安子さんがドライブしてくれた。
紅葉が盛りの山々プロ野球日本シリーズ広島優勝のラヂオをききつつ
\begin{shiika}先急ぎつつ仰ぎゆく峯紅葉
\hfill{\rensuji*{55}・\rensuji*{11}・\rensuji*{2}}\end{shiika}
\vspace{0.6cm}
相川の住居
\begin{shiika}しみじみと語らな白菊活けて待つ
\hfill{\rensuji*{55}・\rensuji*{12}・\rensuji*{0}}\end{shiika}
\begin{shiika}遠き旅はなやぎ帰り菊を焚く
\hfill{\rensuji*{55}・\rensuji*{12}・\rensuji*{0}}\end{shiika}
\begin{shiika}枯菊を焚きつつしばし物思ひ
\hfill{\rensuji*{55}・\rensuji*{12}・\rensuji*{0}}\end{shiika}
\begin{shiika}鉄橋を渡れば小駅片時雨
\hfill{\rensuji*{55}・\rensuji*{12}・\rensuji*{0}}\end{shiika}
\begin{shiika}黄の翅の止り色増す実むらさき
\hfill{\rensuji*{55}・\rensuji*{11}・\rensuji*{0}}\end{shiika}
\begin{shiika}天高し施肥よく効きし畑の色
\hfill{\rensuji*{55}・\rensuji*{11}・\rensuji*{0}}\end{shiika}
\vspace{0.6cm}
七草粥
\begin{shiika}七草の数揃はねど畑の菜を
\hfill{\rensuji*{56}・\rensuji*{1}・\rensuji*{0}}\end{shiika}
\vspace{0.6cm}
幡井さんと焼津 学保に庭からの一望焼津港
\begin{shiika}一望に漁港おさめて梅の丘
\hfill{\rensuji*{56}・\rensuji*{1}・\rensuji*{30}}\end{shiika}
\vspace{0.6cm}
浅野房子さんを訪ねて近くの温泉で一夜を
\begin{shiika}春炬燵尽きぬ話の果は伏し
\hfill{\rensuji*{56}・\rensuji*{3}・\rensuji*{0}}\end{shiika}
\vspace{0.6cm}
\begin{shiika}春の冷え別れて一人立つ小駅
\hfill{\rensuji*{56}・\rensuji*{3}・\rensuji*{0}}\end{shiika}
\vspace{0.6cm}
安子さんが井高野の手伝いを止めることについて一善の言い方処置に納得が
出来ない 筋の通らないことに妥協出来ない私の性
\begin{shiika}争ひてふと空しかり梅の闇
\hfill{\rensuji*{56}・\rensuji*{3}・\rensuji*{0}}\end{shiika}
\vspace{0.6cm}
飯田知子短大入学祝い
\begin{shiika}合格の祝袋は字も太く
\hfill{\rensuji*{56}・\rensuji*{3}・\rensuji*{0}}\end{shiika}
\vspace{0.6cm}
相川家
\begin{shiika}摘みし蕗独りの厨たのしかり
\hfill{\rensuji*{56}・\rensuji*{4}・\rensuji*{0}}\end{shiika}
\vspace{0.6cm}
\begin{shiika}散る桜庭の胸像ただ黙し
\hfill{\rensuji*{56}・\rensuji*{4}・\rensuji*{0}}\end{shiika}
\vspace{0.6cm}
\begin{shiika}武具飾る子は父となり遠くあり
\hfill{\rensuji*{56}・\rensuji*{4}・\rensuji*{0}}\end{shiika}
\vspace{0.6cm}
真鍋先生の鮎のこと 市原さんのご主人の釣りのこと
\begin{shiika}解禁の夕べたまはる吉野鮎
\hfill{\rensuji*{56}・\rensuji*{5}・\rensuji*{0}}\end{shiika}
\begin{shiika}釣りし鮒川に戻して春の風
\hfill{\rensuji*{56}・\rensuji*{5}・\rensuji*{0}}\end{shiika}
\vspace{0.6cm}
上京車中
\begin{shiika}冨士聳ゆ裾野の町の鯉のぼり
\hfill{\rensuji*{56}・\rensuji*{0}・\rensuji*{0}}\end{shiika}
\vspace{0.6cm}
養老の滝へ
\begin{shiika}滝水をコップに汲みて喉しまる
\hfill{\rensuji*{56}・\rensuji*{7}・\rensuji*{0}}\end{shiika}
\vspace{0.6cm}
相川地蔵まつり
\begin{shiika}御詠歌の流れへいそぐ地蔵盆
\hfill{\rensuji*{56}・\rensuji*{8}・\rensuji*{0}}\end{shiika}
\vspace{0.6cm}
児玉正志さん急の来客
\begin{shiika}枝豆に酌みて不意なる遠き客
\hfill{\rensuji*{56}・\rensuji*{9}・\rensuji*{0}}\end{shiika}
\vspace{0.6cm}
市原さんご夫妻の釣り
\begin{shiika}釣る夫の片辺に妻の秋日傘
\hfill{\rensuji*{56}・\rensuji*{10}・\rensuji*{0}}\end{shiika}
\vspace{0.6cm}
高松高女のクラス会 萩 津和野
\begin{shiika}武家屋敷崩れ土塀に石蕗盛り
\hfill{\rensuji*{56}・\rensuji*{10}・\rensuji*{22}}\end{shiika}
\begin{shiika}草子里時雨れる朝の大き虹
\hfill{\rensuji*{56}・\rensuji*{10}・\rensuji*{24}}\end{shiika}
\vspace{0.6cm}
遂に一善があやまりに来た 貞子の五十年忌法要が近ずいて
\begin{shiika}わだかまり解けて減りゆく盛みかん
\hfill{\rensuji*{56}・\rensuji*{10}・\rensuji*{0}}\end{shiika}
\vspace{0.6cm}
\begin{shiika}・
\hfill{\rensuji*{56}・\rensuji*{11}・\rensuji*{0}}\end{shiika}
\vspace{0.6cm}
相川の岩橋家近くの火事のあと
\begin{shiika}売地札草にかくれて秋暮るる
\hfill{\rensuji*{56}・\rensuji*{11}・\rensuji*{0}}\end{shiika}
\vspace{0.6cm}
相川の家 私の誕生日
\begin{shiika}栗おこわ我が誕生は頃もよく
\hfill{\rensuji*{56}・\rensuji*{11}・\rensuji*{0}}\end{shiika}
\begin{shiika}霜よけにレタス生々玉巻ける
\hfill{\rensuji*{56}・\rensuji*{11}・\rensuji*{0}}\end{shiika}
\begin{shiika}供華の菊剪りためらひぬ眠り蝶
\hfill{\rensuji*{56}・\rensuji*{11}・\rensuji*{0}}\end{shiika}
\vspace{0.6cm}
\begin{shiika}落葉炊く煙の中に思ふこと 
\hfill{\rensuji*{56}・\rensuji*{11}・\rensuji*{0}}\end{shiika}
\begin{shiika}新らしく菊きり供え旅に出る 
\hfill{\rensuji*{56}・\rensuji*{11}・\rensuji*{0}}\end{shiika}
\vspace{0.6cm}
鎌倉 お寺の名前を忘れたが
\begin{shiika}踏み惜しみつつ鎌倉の銀杏黄葉
\hfill{\rensuji*{56}・\rensuji*{11}・\rensuji*{24}}\end{shiika}
\vspace{0.6cm}
師走の姿
\begin{shiika}ウインドに背まるく映る師走町
\hfill{\rensuji*{56}・\rensuji*{12}・\rensuji*{0}}\end{shiika}
\vspace{0.6cm}
直紀 年末相川にきて手伝ってくれる
\begin{shiika}晦日そば孫の食べざま頼もしく
\hfill{\rensuji*{56}・\rensuji*{12}・\rensuji*{0}}\end{shiika}
\vspace{0.6cm}
上京 成城の家
\begin{shiika}窓の梅ほころびゆくをみるしじま
\hfill{\rensuji*{57}・\rensuji*{2}・\rensuji*{0}}\end{shiika}
\begin{shiika}散り梅のかかり濯ぎのもの乾く
\hfill{\rensuji*{57}・\rensuji*{2}・\rensuji*{0}}\end{shiika}
\vspace{0.6cm}
八百様を訪ねて
\begin{shiika}春遠しこもれる叔母に京の菓子
\hfill{\rensuji*{57}・\rensuji*{2}・\rensuji*{0}}\end{shiika}
\vspace{0.6cm}
海南の林さん受験(阪大)で泊まる
\begin{shiika}受験生泊めて祈りを同心に
\hfill{\rensuji*{57}・\rensuji*{3}・\rensuji*{0}}\end{shiika}
\vspace{0.6cm}
%-------------------------------------------------------------
相川の橋より
\begin{shiika}日脚伸ぶ中洲に群れる鳥の白
\hfill{\rensuji*{57}・\rensuji*{3}・\rensuji*{0}}\end{shiika}
\begin{shiika}蕗の薹焼みその香の朝厨
\hfill{\rensuji*{57}・\rensuji*{3}・\rensuji*{0}}\end{shiika}
\vspace{0.6cm}
仲塚の案内 垂水神社
\begin{shiika}散る花の流れゆくあり踏まるあり
\hfill{\rensuji*{57}・\rensuji*{4}・\rensuji*{0}}\end{shiika}
\vspace{0.6cm}
郷生と小田原城
\begin{shiika}天主より振る手呼ぶ声花の中
\hfill{\rensuji*{57}・\rensuji*{4}・\rensuji*{0}}\end{shiika}
\vspace{0.6cm}
相川の畑の垣超し中島さんのお嬢さん
\begin{shiika}葱坊主垣越しの子はよくしゃべる
\hfill{\rensuji*{57}・\rensuji*{5}・\rensuji*{0}}\end{shiika}
\begin{shiika}耳遠く笑顔で応ふ木の芽雨
\hfill{\rensuji*{57}・\rensuji*{5}・\rensuji*{0}}\end{shiika}
\vspace{0.6cm}
一善 安子さんと早発して青山高原にドライブ
それは伊賀上野方面への
再ドライブだった
その数日前 室生寺に之も早朝
出かけてたくさんの写真を撮ったつもりが、カメラはフイルムが
入っていなかった。 わざわざ伊賀上野 百合子宅まで訪れたのにい
 室生寺門前で草餅を買う 時間はまだまだ昼前 大野寺で昼弁当を
いただき相談は急に伊賀上野へ
\begin{shiika}草餅にふと道変へて娘に急ぐ
\hfill{\rensuji*{57}・\rensuji*{5}・\rensuji*{0}}\end{shiika}
\vspace{0.6cm}
小汐さん 増田さん 伊藤さん あわくら荘より鳥取砂丘 磨?寺へ
\begin{shiika}直ぐ消ゆる足跡砂に五月旅
\hfill{\rensuji*{57}・\rensuji*{5}・\rensuji*{0}}\end{shiika}
\begin{shiika}風光る砂丘を踏めば若返る
\hfill{\rensuji*{57}・\rensuji*{5}・\rensuji*{0}}\end{shiika}
\begin{shiika}石段のあえぎに著莪の花やさし
\hfill{\rensuji*{57}・\rensuji*{5}・\rensuji*{0}}\end{shiika}
\vspace{0.6cm}
岐阜羽島へ行ったとき
\begin{shiika}単線の停車は長し青田風
\hfill{\rensuji*{57}・\rensuji*{6}・\rensuji*{0}}\end{shiika}
\vspace{0.6cm}
思い出湖岸の旅
\begin{shiika}花栗の香に堂守の鍵開く
\hfill{\rensuji*{57}・\rensuji*{7}・\rensuji*{0}}\end{shiika}
\begin{shiika}老鴬や堂守力こめて説く
\hfill{\rensuji*{57}・\rensuji*{7}・\rensuji*{0}}\end{shiika}
\vspace{0.6cm}
北海道旅行
%======================================================
\begin{shiika}知床の大雪渓に昼の月
\hfill{\rensuji*{57}・\rensuji*{0}・\rensuji*{0}}\end{shiika}
\begin{shiika}雪渓を映し知床五湖寂と 
\hfill{\rensuji*{57}・\rensuji*{0}・\rensuji*{0}}\end{shiika}
\begin{shiika}えぞかんぞう岬はるかは異国なる
\hfill{\rensuji*{57}・\rensuji*{0}・\rensuji*{0}}\end{shiika}
\begin{shiika}昆布乾すさいはての島明易し 
\hfill{\rensuji*{57}・\rensuji*{0}・\rensuji*{0}}\end{shiika}
\begin{shiika}獅子独活の花眼の限り・
\hfill{\rensuji*{57}・\rensuji*{0}・\rensuji*{0}}\end{shiika}
\vspace{0.6cm}
成城の家 笹倉の庭に鷺草が
\begin{shiika}鷺草の鷺二羽となる\Kana{娘}{こ}に甘え
\hfill{双適\rensuji*{57}・\rensuji*{7}・\rensuji*{0}}\end{shiika}
\vspace{0.6cm}
相川の最後の夏
\begin{shiika}魂迎ふ一人となりて古家守る
\hfill{\rensuji*{57}・\rensuji*{8}・\rensuji*{0}}\end{shiika}
\vspace{0.6cm}
%==============================================
\begin{shiika}手ごなしで土をかぶせる秋の種
\hfill{\rensuji*{57}・\rensuji*{8}・\rensuji*{0}}\end{shiika}
\begin{shiika}十指もて土をかぶせる秋の種
\hfill{\rensuji*{57}・\rensuji*{8}・\rensuji*{0}}\end{shiika}
\begin{shiika}豪雷にいさかふ妹弟抱き合ふ
\hfill{\rensuji*{57}・\rensuji*{8}・\rensuji*{0}}\end{shiika}
%----------------------------furikana---------------------
\begin{shiika}亡娘ノート\Kana{紙,魚}{し,み}生きている悲しさよ
\hfill{\rensuji*{57}・\rensuji*{9}・\rensuji*{0}}\end{shiika}
\begin{shiika}秋立ちぬ束ねてさせり亡母の櫛
\hfill{\rensuji*{57}・\rensuji*{9}・\rensuji*{0}}\end{shiika}
\begin{shiika}晩菊の咲くや明日より他人の庭
\hfill{\rensuji*{57}・\rensuji*{10}・\rensuji*{0}}\end{shiika}
\begin{shiika}引き越しの荷隅にかばふ冬すみれ
\hfill{\rensuji*{57}・\rensuji*{10}・\rensuji*{0}}\end{shiika}
\begin{shiika}秋そゞろ引越荷物嵩む部屋
\hfill{\rensuji*{57}・\rensuji*{10}・\rensuji*{0}}\end{shiika}

