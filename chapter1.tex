 
吉祥会で大森先生 池永先生に一緒に当尾の石仏を巡りて 

\begin{shiika}
野仏の笑ひ在せり 曼珠沙華\hfill { \rensuji*{48} ・ \rensuji* {8} }
\end{shiika}
\vspace{0.6cm}
「草紅葉」兼題 幼き日の思い出 
\begin{shiika}
 日を浴びてままごとの子や草紅葉\hfill { \rensuji*{48} ・ \rensuji* {10} }
\end{shiika}
\vspace{0.6cm}
「顔見世」 去年は文友会で顔もせに。今年はただ思い出のみ
\begin{shiika}
顔見世の名残を夢に見しも去年\hfill { \rensuji*{48} ・ \rensuji* {12} ・}
\end{shiika}
\vspace{0.6cm}
お隣の浅野まゆみさんかわいい日本髪で
\begin{shiika}
髪結ひて寝ず娘は待つ初詣\hfill { \rensuji*{49} ・ \rensuji* {1} ・}
\end{shiika}
\vspace{0.6cm}
相川北通りの家根笹の中で狂い猫
\begin{shiika}
猫の恋根笹の乱れ昨日今日\hfill { \rensuji*{49} ・ \rensuji* {2} ・}
\end{shiika}
\vspace{0.6cm}
上京の車中 
浜松あたりで遠連山をみて
\begin{shiika}
山の色幾重の果の雪解光\hfill { \rensuji*{49} ・ \rensuji* {2} ・ }	
\end{shiika}
\vspace{0.6cm}
\begin{shiika}野仏の笑ひ在せり曼珠沙華
\hfill{\rensuji*{48}・\rensuji*{9}・\rensuji*{0}}\end{shiika}
\vspace{0.6cm}
「水草生まふ」 兼題 日浅い私には大変むつかしい。ふと一善の
車で探梅につれてもらった時\\賀名生 だったかそして仁徳陵ところを走ったことを
思い出す。
\begin{shiika}陵の薄陽の濠も水草生ふ
\hfill{\rensuji*{49}・\rensuji*{3}・\rensuji*{0}}\end{shiika}



\vspace{0.6cm}
「春の雪」兼題 直子さんの縁談がまた立ち消えた。
\begin{shiika}娘の縁談又もこわれぬ春の雪
\hfill{\rensuji*{49}・\rensuji*{3}・\rensuji*{0}}\end{shiika}

\vspace{0.6cm}
一つの旅を終えるとまた次に心は走る。
\begin{shiika}花過ぎぬいづこともなき旅心
\hfill{\rensuji*{49}・\rensuji*{4}・\rensuji*{0}}\end{shiika}
\vspace{0.6cm}
「桐の花」兼題 小森田さんとあわくら荘に 帰りは姫路までバスにした。
\begin{shiika}山裾の雨に煙れる桐の花
\hfill{\rensuji*{49}・\rensuji*{5}・\rensuji*{0}}\end{shiika}
\vspace{0.6cm}
「草の花」兼題どこで得た句かはっきりしない。
\begin{shiika}野仏の顔かくすまで草の花
\hfill{\rensuji*{49}・\rensuji*{9}・\rensuji*{0}}\end{shiika}
\vspace{0.6cm}
山下さん 小森田さん 青山さん 四人連れ 児玉東洋さんの
車で佐多岬 桜島 霧島と廻っていただく。\\別れて高千穂の
国民宿舎に泊った夜 高千穂神社の夜神楽をみに行く。
\begin{shiika}夜神東の明りに映ゆる銀杏黄葉
\hfill{\rensuji*{49}・\rensuji*{11}・\rensuji*{0}}\end{shiika}
\vspace{0.6cm}
「炬燵」兼題 一人暮らしの私の句だと浅野さんの御主人がはやす
\begin{shiika}置炬燵向ふ人なきあで蒲団
\hfill{\rensuji*{49}・\rensuji*{11}・\rensuji*{0}}\end{shiika}
\vspace{0.6cm}
「年用意」丹波から週二回野菜その他を積んで車が来る大塚「きく」
の前でとまる。\\ 大塚ののぶ子さんが電話で「丹波よ」と相川の店へしらせてくれる。
\begin{shiika}年用意丹波男の荷は売れ早き
\hfill{\rensuji*{49}・\rensuji*{12}・\rensuji*{0}}\end{shiika}
\vspace{0.6cm}
\vspace{0.6cm}
小森田さんが名古屋から夕方までに相川へ着く筈になっているのに
遅い
\begin{shiika}友待つに暮色刻々粉雪舞ふ
\hfill{\rensuji*{50}・\rensuji*{1}・\rensuji*{0}}\end{shiika}
\vspace{0.6cm}
上京車窓より。
\begin{shiika}風ぬくき末黒野烏群をなし
\hfill{\rensuji*{50}・\rensuji*{2}・\rensuji*{0}}\end{shiika}
\vspace{0.6cm}
私は化粧水は使っていないが ふと出来た句
\begin{shiika}化粧水掌に冷えのなし春隣
\hfill{\rensuji*{50}・\rensuji*{3}・\rensuji*{0}}\end{shiika}
\vspace{0.6cm}
「花曇」野崎詣りをしらのは去年だったかと思う。
\begin{shiika}綿菓子も売れて野崎の花曇
\hfill{\rensuji*{50}・\rensuji*{4}・\rensuji*{0}}\end{shiika}
\begin{shiika}花曇年甲斐もなき物忘れ
\hfill{\rensuji*{50}・\rensuji*{4}・\rensuji*{0}}\end{shiika}
\vspace{0.6cm}
この様な軽やかな心に時もある
\begin{shiika}若やぎて夏来る歌口ずさむ
\hfill{\rensuji*{50}・\rensuji*{0}・\rensuji*{5}}\end{shiika}
\vspace{0.6cm}
相川の家の軒に雀がいそかしげに出入りする
\begin{shiika}梅雨曇出入せはしき軒雀
\hfill{\rensuji*{50}・\rensuji*{6}・\rensuji*{0}}\end{shiika}
\vspace{0.6cm}
相川の町の露地風景
\begin{shiika}花曇年甲斐もなき物忘れ
\hfill{\rensuji*{50}・\rensuji*{6}・\rensuji*{0}}\end{shiika}
\vspace{0.6cm}
どこの寺院だったかなー
\begin{shiika}あらはなるちくり根洗ひ大夕立
\hfill{\rensuji*{50}・\rensuji*{7}・\rensuji*{0}}\end{shiika}
\vspace{0.6cm}
「流れ星」この頃誰かが病気をして心にかかっていた
\begin{shiika}看る夜の心もとなき星の飛ぶ
\hfill{\rensuji*{50}・\rensuji*{8}・\rensuji*{26}}\end{shiika}
\vspace{0.6cm}
「空蝉」故かんげつ国分寺境内の礎石で遊んだ日をおもいだして
\begin{shiika}子等去りぬ礎石にならぶ蝉の殻
\hfill{\rensuji*{50}・\rensuji*{8}・\rensuji*{0}}\end{shiika}
\vspace{0.6cm}
唐招提寺 観月の夜
\begin{shiika}大月夜唐招提寺の庭に彳つ
\hfill{\rensuji*{50}・\rensuji*{9}・\rensuji*{0}}\end{shiika}
\vspace{0.6cm}
「色鳥」山下さん青山さんと越前賤ケ岳 長浜竹生島の旅
\begin{shiika}色鳥や朝の湖の小桟橋
\hfill{\rensuji*{50}・\rensuji*{10}・\rensuji*{0}}\end{shiika}
\vspace{0.6cm}
「秋惜しむ」小森田さんと笑い乍らの出来たもの
\begin{shiika}秋惜しむほほ紅少こしさしてみむ
\hfill{\rensuji*{50}・\rensuji*{10}・\rensuji*{0}}\end{shiika}
\vspace{0.6cm}
大塚さん「きく」の前に荷をおろす「丹波」のこと
\begin{shiika}新鮮と我から言ひて冬菜売
\hfill{\rensuji*{50}・\rensuji*{12}・\rensuji*{0}}\end{shiika}
\vspace{0.6cm}
相川の座敷の庭に笹子
の声がと井上さんからきく
\begin{shiika}独り居の朝茶の香り笹に来る
\hfill{\rensuji*{51}・\rensuji*{1}・\rensuji*{0}}\end{shiika}
\vspace{0.6cm}
「大福茶」我が家は梅昆布茶が毎年のこと大福茶と
思っている。
\begin{shiika}家長の座に心しまりて大福茶
\hfill{\rensuji*{51}・\rensuji*{1}・\rensuji*{0}}\end{shiika}
\vspace{0.6cm}
「野焼き」 あちこちに見る野火に次の命の芽生えを思った。
\begin{shiika}新らしき命を呼びて野火勢ふ
\hfill{\rensuji*{51}・\rensuji*{2}・\rensuji*{0}}\end{shiika}
\vspace{0.6cm}
「春泥」 浄瑠璃寺への柊が浮かんできた。 そして遠足の
列が眼に入る。
\begin{shiika}春泥の径つき寺の小門あり
\hfill{\rensuji*{51}・\rensuji*{3}・\rensuji*{0}}\end{shiika}
\begin{shiika}黄帽子水筒どの児の靴も春の泥
\hfill{\rensuji*{51}・\rensuji*{3}・\rensuji*{0}}\end{shiika}
\vspace{0.6cm}
高山祭をめざして小森田さん 美佐さん 宮川ひでさんと下呂
へ行く。折り悪し雨で宵の「曳別れ」は
みることができなかったが車窓より禅昌寺の塔を眺めて
\begin{shiika}花の奥雨に煙れる塔のあり
\hfill{\rensuji*{51}・\rensuji*{4}・\rensuji*{0}}\end{shiika}
\vspace{0.6cm}
小森田」さん 高田さんと妙高々原 穂高 と旅して 穂高の
有明松尾寺にて、妙高々原にて
\begin{shiika}老鶯や御手の茶壺のかたむける
\hfill{\rensuji*{51}・\rensuji*{5}・\rensuji*{0}}\end{shiika}
\begin{shiika}老鴬に唐松林行きにゆく
\hfill{\rensuji*{51}・\rensuji*{5}・\rensuji*{0}}\end{shiika}
「落し文」 むつかしい兼題にふと昨年の賤ケ岳を思い出して
\begin{shiika}湖見ゆる古戦場道落し文
\hfill{\rensuji*{51}・\rensuji*{7}・\rensuji*{0}}\end{shiika}
\vspace{0.6cm}
亡妹貞子が死の近くなった頃
梨をしきりにほしがった。梨の頃がくると思い出す。
\begin{shiika}病妹の欲りし日とあり梨供ふ
\hfill{\rensuji*{51}・\rensuji*{9}・\rensuji*{0}}\end{shiika}
\vspace{0.6cm}
京都女専クラス会 九州志賀島 大宰府 柳川巡りにて
\begin{shiika}鐘楼に屋根草のびて露ふかし
\hfill{\rensuji*{51}・\rensuji*{10}・\rensuji*{17}}\end{shiika}
\begin{shiika}四つ手網死魚の乾けり秋の声
\hfill{\rensuji*{51}・\rensuji*{10}・\rensuji*{17}}\end{shiika}
\vspace{0.6cm}
「晩菊」相川の庭の菊 謡の小川先生のこと。
\begin{shiika}晩菊のうつろいはじむ白きより
\hfill{\rensuji*{51}・\rensuji*{11}・\rensuji*{0}}\end{shiika}
\begin{shiika}晩菊やなほ美くしき謡の師
\hfill{\rensuji*{51}・\rensuji*{11}・\rensuji*{0}}\end{shiika}
\vspace{0.6cm}
耳の治療で大手町病院に通っていた頃
\begin{shiika}晩菊やなほ美くしき謡の師
\hfill{\rensuji*{51}・\rensuji*{11}・\rensuji*{0}}\end{shiika}
\vspace{0.6cm}
天満マーチャンダイズあたりにて
\begin{shiika}秋冷ゆる赤きストビラ散る舗道
\hfill{\rensuji*{51}・\rensuji*{11}・\rensuji*{0}}\end{shiika}
\vspace{0.6cm}
相川の庭の垣をみて。
\begin{shiika}綿虫の籬越え来て雨を呼ぶ
\hfill{\rensuji*{51}・\rensuji*{11}・\rensuji*{0}}\end{shiika}
\vspace{0.6cm}
