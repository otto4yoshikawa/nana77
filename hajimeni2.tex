昭和四十八年九月浅野房子さんと三朝温泉への車中、
山下光子に出会ひ三朝の病院に療養中の大塚さんを見舞う
旅だったが 話は吉川美佐姉のすすめにより
京鹿子火曜教室に浅野さん 小田澄子さんが入会


九月初句会に出席した様子だった。
私も一か月おくれて 十月よりともかく出句した。

造る書くと言うことには全々自信のない出発だから
あまり進んだ気持ちでは」なっかった。
以来 もう止めるを繰り返した。
美佐さんへの義理を続けていると言った。

そして十八年の年月が過ぎた。納得のいく自分の句
句は殆んど無い。

個人で句集を作られた句友も何人かあるが 火曜火鏡 合同句集の
仲間入りが精一杯のこと、それ以上自分の句を活字にのこすことは
考えてもいなかった。けれどここ数年前から句日記として 整理
してみようと思い立った。
下手、句になっていない句 それでよい。思うばかりでなかなか
とりかかれないで 二、三年は過ぎた。

今回 玉造温泉 厚生年金会館 保養ホームに入所 山下さん 悦子さん
と合流するまでの一週間 一人の機を得て漸く一頁をかき出し始める。
振り返り見る十八年 記憶確かでないもももあるが
思い出は楽しい。
\hfill {  \rensuji*{3} ・\rensuji* {8}・\rensuji*[1]{26}}
